\documentclass[11pt,a4paper]{article}
\usepackage[utf8]{inputenc}
\usepackage{amsfonts}
\usepackage{amssymb}
\usepackage{mdframed}
\usepackage{tikz}
\usetikzlibrary{calc}
\usepackage{tkz-tab}
\usepackage{pgfplots}
\usepackage{xcolor}
\usepackage{fancyhdr}
\usepackage{lastpage}
\usepackage[fleqn]{amsmath}
\usepackage{mathrsfs}
\setlength{\mathindent}{0pt}

\newcommand{\pdt}{\mathbin{\vcenter{\hbox{\scalebox{0.6}{\textbullet}}}}}

% Spécifications du document
\newcommand{\doctitre}{Géométrie repérée} % Ex: Le second degré
\newcommand{\docniveau}{$1^{\text{re}}$ Spécialité mathématiques} % Ex: $1^{\text{re}}$ Spécialité mathématiques
\newcommand{\doctheme}{Géométrie} %Ex: Algèbre
\newcommand{\doctype}{Démonstrations} % Ex: Démonstrations
\newcommand{\docshorttype}{Démo} % Démo

% Couleurs pour les graphiques
\definecolor{dark_green}{HTML}{008000}

% Paramètres du document
\RequirePackage{geometry}
\geometry{tmargin=1cm,bmargin=1.9cm,lmargin=1.9cm,rmargin=1.9cm}
\renewcommand{\familydefault}{\sfdefault}
\setlength{\parindent}{0pt}
\title{\doctitre}
\author{\docniveau \\ \doctheme\text{ - }\doctype}
\date{}
\fancypagestyle{custom}{
  \fancyhf{}
  \renewcommand{\headrulewidth}{0pt}
  \lfoot{\doctheme\text{ - }\docshorttype}
  \cfoot{\doctitre} % Change \titre to \doctitre
  \rfoot{\thepage/\pageref{LastPage}}
}

% Styles pour les mdframed
\mdfdefinestyle{definitionStyle}{
    leftline=true,
    rightline=false,
    topline=false,
    bottomline=false,
    linewidth=2pt,
    linecolor=black,
    innertopmargin=0pt,
    innerbottommargin=0pt,
    innerrightmargin=0pt,
    innerleftmargin=5pt,
}

\mdfdefinestyle{proprieteStyle}{
    linewidth=1pt,
    linecolor=black,
    innertopmargin=5pt,
    innerbottommargin=5pt,
    innerrightmargin=5pt,
    innerleftmargin=5pt,
}
% ----- DEBUT DU DOCUMENT -----
\begin{document}

% Style et numérotation
\maketitle
\pagestyle{custom}
\thispagestyle{custom}

\section*{I. Équation cartésienne d'une droite et vecteur normal}

\underline{Démonstration :} \\
Soit $(d)$ la droite passant par un point $A(x_A;y_A)$ et dont on connait le vecteur normal $\displaystyle\vec{n}\binom{a}{b}$.
\vspace*{-8pt}
\begin{equation*}
  \begin{split}
    M(x;y)\in (d)&\Leftrightarrow \overrightarrow{AM}\binom{x-x_A}{y-y_A} \text{ et } \vec{n}\binom{a}{b} \text{ sont orthogonaux} \\
    &\Leftrightarrow \overrightarrow{AM}\pdt\vec{n} = 0 \\
    &\Leftrightarrow a(x-x_A)+b(y-y_B)=0 \\
    &\Leftrightarrow ax+bx\underbrace{-ax_A-by_A}_{c}=0
  \end{split}
\end{equation*}

\section*{II. Équation cartésienne d'un cercle}

\underline{Démonstration \emph{(equation d'un cercle connaissant son center et son rayon)}:}
\begin{equation*}
  \begin{split}
    \text{Soit } M(x;y) \in \mathscr{C}&\Leftrightarrow\Omega M=R \\
    &\Leftrightarrow\Omega M^2=R^2\text{ avec } \Omega M=\sqrt{(x-x_0)^2+(y-y_0)^2} \\
    &\Leftrightarrow\Omega (x-x_0)^2+(y-y_0)^2=R^2\\
  \end{split}
\end{equation*}

\underline{Démonstration \emph{(equation d'un cercle connaissant son diamètre)}:}
\begin{equation*}
  \begin{split}
    M(x;y)\in\mathscr{C}&\Leftrightarrow\text{le triangle $MAB$ est rectangle en $M$} \\
    &\Leftrightarrow\overrightarrow{AM}\pdt\overrightarrow{BM}=0 \text{ avec } \overrightarrow{AM}\binom{x-x_A}{y-y_A} \text{ et } \overrightarrow{BM}\binom{x-x_B}{y-y_B} \\
    &\Leftrightarrow (x-x_A)(x-x_B)+(y-y_A)(y-y_B)=0
  \end{split}
\end{equation*}

\section*{III. Équation cartésienne d'une parabole}

\underline{Démonstration \emph{(d'après Mathématiques 1$^{\text{re}}$ spé Collection Barbazo)} :} \\

\begin{itemize}
  \item On va montrer que pour tout point $A$ appartenant à $\mathscr{P}$ et distinct de son sommet, il existe un point $B$ distinct de $A$ appartenant à $\mathscr{P}$ et ayant la même ordonnée que $A$. \\
  Soient deux points $A(x_A;y_A)$ et $B(x_B;y_B)$ appartenant à $\mathscr{P}$. \\
  
  On a donc $y_A=ax{_A}^2+bx_A+c$ et $y_B=ax{_B}^2+bx_B+c$.
  \vspace*{-8pt}
  \begin{equation*}
    \begin{split}
      y_A=y_B&\Leftrightarrow ax{_A}^2+bx_A+c = ax{_B}^2+bx_B+c \\
      &\Leftrightarrow a(x{_A}^2-x{_B}^2) + b(x_A-x_B)=0 \\
      &\Leftrightarrow (x{_A}-x{_B})\left[ a(x_A-x_B) +b \right]=0 \\ 
      &\Leftrightarrow (x{_A}-x{_B})=0 \text{ ou }  a(x_A-x_B) +b=0 \\
      &\Leftrightarrow (x{_A}=x{_B}) \text{ ou }  x_A+x_B=\frac{-b}{a}
    \end{split}
  \end{equation*}
  Donc, si un point $A(x_A;y_A)$ distinct du sommet de la parabole appartient à $\mathscr{P}$, le point $\displaystyle B\left(-\frac{b}{a}-x_A;y_A\right)$ appartient à $\mathscr{P}$, est distinct de $A$ et a la même ordonnée que $A$. \\
  La courbe admet donc bien deux points distincts d'ordonnée $y_A$.
  
  \item On va déterminer les coordonnées du milieur $I$ de $[AB]$. \\
  Les points $A$ et $B$ ont la même ordonnée donc $\displaystyle y_I=\frac{y_A+y_B}{2}=\frac{2y_A}{2}=y_A$. \\
  De plus, d'après ce qui précède, $\displaystyle x_A+x_B=-\frac{b}{a}$, donc $\displaystyle\frac{x_A+x_B}{2}=-\frac{b}{2a}$, et donc $\displaystyle x_I=\frac{x_A+x_B}{2}=-\frac{b}{2a}$. \\
  On en déduit donc que $\displaystyle I\left(-\frac{b}{2a};y_A\right)$.

  \item Soit $\Delta$ la droite d'équation $\displaystyle x=-\frac{b}{2a}$. Les points $A$ et $B$ ont la même ordonnée, donc le vecteur $\overrightarrow{AB}$ a pour coordonnées $\displaystyle\binom{x_B-x_A}{0}$.
  Un vecteur directeur de $\Delta$ est $\displaystyle\vec{j}\binom{0}{1}$. Or $\overrightarrow{AB}\pdt\vec{j}=(x_B-x_A)\times0+0\times1=0$.
  Les vecteurs sont donc orthogonaux, donc la droite $\Delta$ est orthogonale au segment $[AB]$.
  De plus, le point $I$, milieur de $[AB]$, appartient à $\Delta$. Donc $\Delta$ est la médiatrice du segment $[AB]$.
  $B$ est donc le symétrique de $A$ par rapport à $\Delta$.

  \item Si $A$ est le point de la courbe d'abscisse $-\frac{b}{2a}$, alors $A$ appartient à $\Delta$, c'est le sommet de la parabole.
  Donc $A$ invariant par symétrie par rapport à $\Delta$. On a alors $x_A=x_B$. \\
  Tout point de la parabole admet ainsi un symétrique par rapport à $\Delta$, ce qui signifie que $\Delta$ est axe de symétrie de la parabole. 
\end{itemize}
  
\end{document}
% ----- FIN DU DOCUMENT -----