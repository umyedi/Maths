\documentclass[11pt,a4paper]{article}
\usepackage[utf8]{inputenc}
\usepackage[T1]{fontenc}
\usepackage{amsfonts}
\usepackage{amssymb}
\usepackage{mdframed}
\usepackage{tikz}
\usepackage{tkz-tab}
\usepackage{pgfplots}
\usepackage{xcolor}
\usepackage{fancyhdr}
\usepackage{lastpage}
\usepackage[fleqn]{amsmath}
\setlength{\mathindent}{0pt}

% Défini les couleurs pour les graphiques
\definecolor{dark_green}{HTML}{008000}

% Extensions
\RequirePackage{geometry} 
\geometry{tmargin=1cm,bmargin=1.9cm,lmargin=1.9cm,rmargin=1.9cm}

% Paramètres du titre
\def\classe{$1^{\text{re}}$ Spécialité mathématiques}
\def\titre{Le second degré}
\def\theme{Algèbre - Démonstrations}

% Paramètres de numérotation des pages
\fancypagestyle{custom}{
  \fancyhf{}
  \renewcommand{\headrulewidth}{0pt}
  \lfoot{Algèbre - Démos}
  \cfoot{Le second degré}
  \rfoot{\thepage/\pageref{LastPage}}
}

\usepackage{titlesec} % Pour personnaliser les titres
\usepackage{xcolor} % Pour définir des couleurs

\title{\titre}
\author{\classe \\ \theme}
\date{}

\renewcommand{\familydefault}{\sfdefault}

% Styles pour les mdframed
\mdfdefinestyle{definitionStyle}{
    leftline=true,
    rightline=false,
    topline=false,
    bottomline=false,
    linewidth=2pt,
    linecolor=black,
    innertopmargin=0pt,
    innerbottommargin=0pt,
    innerrightmargin=0pt,
    innerleftmargin=5pt,
}

\mdfdefinestyle{proprieteStyle}{
    linewidth=1pt,
    linecolor=black,
    innertopmargin=5pt,
    innerbottommargin=5pt,
    innerrightmargin=5pt,
    innerleftmargin=5pt,
}

% Supprime l'indentation des paragraphes
\setlength{\parindent}{0pt}

\begin{document}

\maketitle
\pagestyle{custom}
\thispagestyle{custom}

\section*{I. Les fonctions polynômes du second degré}
\subsection*{1. Forme canonique}

\underline{Démonstration :}
\begin{alignat*}{2}
     & \text{Soit } & f(x) & = ax^2+bx+c                                                                                     \\
     &              &      & = a\left[x^2+\frac{b}{a}x\right]+c                                                              \\
     &              &      & = a\left[x^2+\frac{2b}{2a}x + \left(\frac{b}{2a}\right)^2- \left(\frac{b}{2a}\right)^2\right]+c \\
     &              &      & = a\left[\left(x+\frac{b}{2a}\right)^2-\frac{b^2}{4a^2}\right]+c                                \\
     &              &      & = a\left(x+\frac{b}{2a}\right)^2-\frac{b^2}{4a}+c                                               \\
     &              &      & = a(x-\alpha)+\beta \text{ avec $\alpha=\frac{-b}{2a}$ et $\beta=\frac{-b^2}{4a}+c=f(\alpha)$}
\end{alignat*}

\subsection*{2. Sens de variation}

\underline{Démonstration de la propriété sur le sens de variation :}

\begin{itemize}
    \item 1$^{\text{er}}$ cas : $a>0$ ~\\
        Soient $x_1$ et $x_2$ deux réels de l'intervalle $]-\infty;\alpha[$ tels que $x_1<x_2<\alpha$.
        \begin{align*}
            x_1-\alpha            & <  x_2-\alpha < 0                                                                    \\
            (x_1-\alpha)^2        & >  (x_2-\alpha)^2 \text{ car }x_1-\alpha \text{ et }x_2-\alpha \text{ sont négatifs} \\
            a(x_1-\alpha)^2       & > a(x_2-\alpha)^2 \text{ car } a>0                                                   \\
            a(x_1-\alpha)^2+\beta & > a(x_2-\alpha)^2+\beta                                                              \\
            f(x_1)                & > f(x_2)
        \end{align*}
        Donc $f$ est strictement décroissante sur $]-\infty;\alpha[$.

        Soient $x_1$ et $x_2$ deux réels de l'intervalle $[\alpha;\infty[$ tels que $\alpha<x_1<x_2$.
        \begin{align*}
            0\leq x_1-\alpha      & <  x_2-\alpha                                                                        \\
            (x_1-\alpha)^2        & <  (x_2-\alpha)^2 \text{ car }x_1-\alpha \text{ et }x_2-\alpha \text{ sont positifs} \\
            a(x_1-\alpha)^2       & < a(x_2-\alpha)^2 \text{ car } a>0                                                   \\
            a(x_1-\alpha)^2+\beta & < a(x_2-\alpha)^2+\beta                                                              \\
            f(x_1)                & < f(x_2)
        \end{align*}
        Donc $f$ est strictement croissante sur $[\alpha;\infty[$.

    \item 2$^{\text{ème}}$ cas : $a<0$ ~\\
        Soient $x_1$ et $x_2$ deux réels de l'intervalle $]-\infty;\alpha[$ tels que $x_1<x_2<\alpha$.
        \begin{align*}
            x_1-\alpha            & <  x_2-\alpha < 0                                                                    \\
            (x_1-\alpha)^2        & >  (x_2-\alpha)^2 \text{ car }x_1-\alpha \text{ et }x_2-\alpha \text{ sont positifs} \\
            a(x_1-\alpha)^2       & < a(x_2-\alpha)^2 \text{ car } a<0                                                   \\
            a(x_1-\alpha)^2+\beta & < a(x_2-\alpha)^2+\beta                                                              \\
            f(x_1)                & < f(x_2)
        \end{align*}
        Donc $f$ est strictement croissante sur $]-\infty;\alpha[$.

        Soient $x_1$ et $x_2$ deux réels de l'intervalle $[\alpha;\infty[$ tels que $\alpha<x_1<x_2$.
        \begin{align*}
            0\leq x_1-\alpha          & <  x_2-\alpha                                                                        \\
            (x_1-\alpha)^2        & <  (x_2-\alpha)^2 \text{ car }x_1-\alpha \text{ et }x_2-\alpha \text{ sont positifs} \\
            a(x_1-\alpha)^2       & > a(x_2-\alpha)^2 \text{ car } a<0                                                   \\
            a(x_1-\alpha)^2+\beta & > a(x_2-\alpha)^2+\beta                                                              \\
            f(x_1)                & > f(x_2)
        \end{align*}
        Donc $f$ est strictement décroissante sur $[\alpha;\infty[$.


\end{itemize}

\section*{II. Factorisation d'une fonction du second degré et équation du second degré}
\subsection*{1-2. Factorisation - Résolution des équation du second degré}

\underline{Démonstration :} \\

Pour $f(x)=ax^2+bx+c$, on a vu que $f(x)$ peut s'écrire sous la forme $f(x)=a(x+\frac{b}{2a})+\frac{-b}{4a}+c$.
\begin{alignat*}{2}
     & \text{Donc } & f(x) & = a\left(x+\frac{b}{2a}\right)+\frac{-b}{4a}+\frac{4ac}{4a}                                         \\
     &              &      & = a\left[\left(x+\frac{b}{2a}\right)^2+\frac{-b^2}{4a^2}+\frac{4ac}{4a^2}\right]                    \\
     &              &      & = a\left[\left(x+\frac{b}{2a}\right)^2-\frac{b^2-4ac}{4a^2}\right] \text{On pose $\Delta=-b^2-4ac$} \\
     &              &      & = a\left[\left(x+\frac{b}{2a}\right)^2-\frac{\Delta}{4a^2}\right]
\end{alignat*}

\begin{itemize}
    \item 1$^{\text{er}}$ cas : $\Delta<0$ ~\\
          Si $\Delta<0$ alors $f(x)>0$ pour tout $x\in\mathbb{R}$ (si $a>0$) ou $f(x)<0$ pour tout $x\in\mathbb{R}$ (si $a<0$). \\
          Donc $f(x)=0$ n'admet pas de solution et $f(x)$ n'est pas factorisable.

    \item 2$^{\text{ème}}$ cas : $\Delta=0$ \\
          Si $\Delta=0$ alors $f(x)=a( x+\frac{b}{2a})^2$ ou $f(x)=a(x-\alpha)^2$ avec $\alpha=\frac{-b}{2a}$. \\
          Donc l'équation $f(x)=0$ admet une solution (double) $\alpha$ et $f(x)$ est factorisable ou $f(x)=a(x-\alpha)^2$.

    \item 3$^{\text{ème}}$ cas : $\Delta>0$
          \begin{alignat*}{2}
               & \text{Si $\Delta>0$ alors } & f(x) & = a\left[\left(x+\frac{b}{2a}\right)^2-\left(\frac{\sqrt{\Delta}}{2a}\right)^2\right]                      \\
               &                                   &      & =a\left(x+\frac{b}{2a}-\frac{\sqrt{\Delta}}{2a}\right)\left(x+\frac{b}{2a}+\frac{\sqrt{\Delta}}{2a}\right) \\
               &                                   &      & =a\left(x-\frac{-b-\sqrt{\Delta}}{2a}\right)\left(x-\frac{-b+\sqrt{\Delta}}{2a}\right)                     \\
          \end{alignat*}
          Donc $f(x)$ est factorisable en $f(x)=a(x-x_1)(x-x_2)$ avec $x_1=\frac{-b-\sqrt{\Delta}}{2a}$ et $x_2=\frac{-b+\sqrt{\Delta}}{2a}$. \\
          L'équation $f(x)=0$ admet deux solutions $x_1$ et $x_2$.
\end{itemize}

\subsection*{3. Somme et produit des racines}
\underline{Démonstration :}
\begin{alignat*}{2}
     & \text{Calculons } & x_1+x_2 & = \frac{-b-\sqrt{\Delta}}{2a} + \frac{-b+\sqrt{\Delta}}{2a} \\
     &                   &         & = \frac{-b-\sqrt{\Delta}-b+\sqrt{\Delta}}{2a}               \\
     &                   &         & = \frac{-2b}{2a}=\frac{-b}{a}
\end{alignat*}
\begin{alignat*}{2}
     & \text{Calculons } & x_1\times x_2 & = \frac{-b-\sqrt{\Delta}}{2a} \times \frac{-b+\sqrt{\Delta}}{2a} \\
     &                   &               & = \frac{(-b-\sqrt{\Delta})(-b+\sqrt{\Delta})}{4a^2}              \\
     &                   &               & = \frac{(-b)^2-(\sqrt{\Delta})^2}{4a}                            \\
     &                   &               & = \frac{b^2-\Delta}{4a^2}                                        \\
     &                   &               & = \frac{b^2-(b^2-4ac)}{4a^2}                                     \\
     &                   &               & = \frac{b^2-b^2+4ac}{4a^2}                                       \\
     &                   &               & = \frac{c}{a}
\end{alignat*}

\end{document}