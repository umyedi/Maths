\documentclass[10pt,a4paper]{article}



% Import des extensions

\RequirePackage[french]{babel}
\RequirePackage[T1]{fontenc}
\RequirePackage[utf8]{inputenc}
\RequirePackage{ae}
\RequirePackage{aecompl}
\RequirePackage{zefonts}
\RequirePackage{multicol}
\RequirePackage{array, multirow, tabularx}
\RequirePackage{makecell}
\RequirePackage{graphicx}
% \RequirePackage{picins}
\RequirePackage{amsfonts,amsmath,amssymb}
\RequirePackage{eurosym}
% \RequirePackage{ulem} %permet de barrer sdu texte avec \sout{} , hachurer avec \xout , souligner avec vaguelette \uwave
\RequirePackage{textcomp}
\RequirePackage{graphicx}
\RequirePackage{yhmath} %arc de cercle avec $\wideparen{AB}$
\RequirePackage[np]{numprint} %espacement grands nombres
\RequirePackage{fdsymbol} % symbole calculatrice casio
\RequirePackage{enumerate}
\RequirePackage{fancybox} % shadowbox etc...
\RequirePackage{pifont}
\RequirePackage{tabularx} 
\RequirePackage{boxedminipage}
\RequirePackage[thmmarks,framed]{ntheorem}   %fonction  \newtheorem améliorée incompatible avec amsthm
\RequirePackage{framed}
\RequirePackage{titlesec}
\RequirePackage{colortbl} %charge xcolor , à mettre avant tikz sinon conflit...
\RequirePackage{pgfkeys} % repères avec tikz
\RequirePackage{tikz}
\RequirePackage{tkz-tab} %tableaux de variation
\RequirePackage{esvect} %vecteurs
\RequirePackage{pgf} %exporter figures geogebra
\RequirePackage{mathrsfs} %exporter figures geogebra
\RequirePackage{slashbox} %sépare une cellule en 2 dans un tableau \backslashbox{Texte dessous}{Texte dessus}
%\RequirePackage{diagbox} %\diagbox{}{}
\usetikzlibrary{arrows} %exporter figures geogebra
\RequirePackage[french,frenchkw,algoruled]{algorithm2e} %algo 
%\RequirePackage{algorithm} %algo style algobox
% \RequirePackage{algpseudocode}% algo style algobox
% \input algtolatex %algo style algobox - nécessite le fichier algolatex.
\RequirePackage{calrsfs} % plus belles majuscules avec \mathcal
% \RequirePackage{sesamanuel} % pour les commandes spéciales du manuel 2de
% \RequirePackage{sesamanuelTIKZ} % pour les commandes spéciales de la figure
%\RequirePackage{stmaryrd} %double crochet avec \llbracket et rrbracket$
\RequirePackage{qrcode} %qrcode cliquable \qrcode[height=taille]{adresse}
\RequirePackage{hyperref} %lien hypertexte \href{adresse}{texte}
%\RequirePackage[xcas]{pro-graphes} %graphe and co
%\RequirePackage{dijkstra}
\RequirePackage{tcolorbox} %cadres plus jolis
\RequirePackage{karnaugh-map} %permet de tracer des tableaux de Karnaugh
%\RequirePackage{graphicx}
\RequirePackage[export]{adjustbox} % Alignement vertical images avec b,t,c \includegraphics[scale=1,valign=t]{Image.png}
\RequirePackage{fancyvrb} % verbatim amélioré \begin{Verbatim}[frame=single,label=,numbers=left] etc [frame=single/leftline/topline/bottomline/lines] framerule=1mm framesep=5mm rulecolor=\color{red} fillcolor=\color{yellow}
\usepackage[load-configurations = abbreviations]{siunitx}
%\sisetup{locale = FR,detect-all,inter-unit-product= \cdot} %norme SI pour les nombres et unités 
%\SI{25000}{mm} \SI{6.022e23}{\per\mol}  \SI{300}{\watt\per\square\meter} \SI{300}{W / m^2} \SI[per-mode=symbol]{210}{\km\per\hour}
% unité seule \si{\newton\meter} nombre seul \num{24415.15625}

\RequirePackage{witharrows} %permet de mattre des flèches entre lignes d'équations \begin{WithArrows}@ \Arrow{@}\\@\end{WithArrows}

\RequirePackage{xlop} % pose les oprétaions "à la main" \opadd{@}{@}; opsub ; opmul ; opdiv (eucl) ; opidiv
%\RequirePackage{minted}{Python} % permet d'écrire en Python avec \begin{minted}{Python}@\end{minted}
\RequirePackage{stmaryrd} % crochets intervalles entiers \llbracket ; \rrbracket ; parall \sslash ; contradiction \lightning

\RequirePackage{mathtools}

%%%%%%%Compteurs%%%%%%%%%%%%%%%%%%%%%%%%%%%%%%%%%%%%%%%%%%%%%%%%%%%%%%%%%%%%%%%%%%%%%
% \newcounter{defcompt}   %Défintion d'un compteur
% \newcounter{exocompt}
% \newcounter{theocompt}
% \newcounter{propcompt}
% \newcounter{regcompt}
\newcounter{numexos}

\setcounter{numexos}{0} %initialisation du compteur





% Nouvelles commandes
\definecolor{cqcqcq}{rgb}{0.7529411764705882,0.7529411764705882,0.7529411764705882} %gris geogebra

\newcommand{\paral}{~\mathbin{\!/\mkern-5mu/\!}~} % symbole parallèle

\newcommand{\R}{\mathbb{R}}
\newcommand{\N}{\mathbb{N}}
\newcommand{\D}{\mathbb{D}}
\newcommand{\Z}{\mathbb{Z}}
\newcommand{\Q}{\mathbb{Q}}
\newcommand{\C}{\mathbb{C}}
\newcommand{\U}{\mathbb{U}}

\newcommand{\rel}{\mathcal{R}} % relation binaire

\newcommand{\e}{\text{e}} % exponentielle en lettre droite dans $$ 

\newcolumntype{R}[1]{>{\raggedleft\arraybackslash }b{#1}}
\newcolumntype{L}[1]{>{\raggedright\arraybackslash }b{#1}}
\newcolumntype{C}[1]{>{\centering\arraybackslash }b{#1}}

\newcolumntype{G}[1]{>{\raggedright\arraybackslash }X{#1}}
\newcolumntype{D}[1]{>{\raggedleft\arraybackslash }X{#1}}
\newcolumntype{M}[1]{>{\centering\arraybackslash }X{#1}}


\newcommand{\exe}{\textbf{Exemple : }}
\newcommand{\exes}{\textbf{Exemples : }}
\newcommand{\rema}{\textbf{Remarque : }}
\newcommand{\rems}{\textbf{Remarques : }}
\newcommand{\rap}{\textbf{Rappel : }}
\newcommand{\raps}{\textbf{Rappels : }}
\newcommand{\dem}{\textbf{Démonstration : }}
\newcommand{\dems}{\textbf{Démonstrations : }}
\newcommand{\csq}{\textbf{Conséquence : }}







\newcommand{\Ex}[1]{{\sc{Exercice #1:}}}
\newcommand{\defi}[1]{\begin{leftbar}
\textbf{Définition :} {#1}
 \end{leftbar}}
\newcommand{\defis}[1]{\begin{leftbar}
\textbf{Définitions :} {#1}
 \end{leftbar}}
 
\newcommand{\prop}[1]{\begin{framed}
\textbf{Propriété :} {#1}
 \end{framed}} 
 
\newcommand{\props}[1]{\begin{framed}
\textbf{Propriétés :} {#1}
 \end{framed}} 
 
\newcommand{\theo}[1]{\begin{framed}
\textbf{Théorème :} {#1}
 \end{framed}} 
\newcommand{\theos}[1]{\begin{framed}
\textbf{Théorèmes :} {#1}
 \end{framed}}  
 
\newcommand{\reg}[1]{\begin{framed}
\textbf{Règle :} {#1}
 \end{framed}} 
\newcommand{\regs}[1]{\begin{framed}
\textbf{Règles :} {#1}
 \end{framed}}  
 
\newcommand{\propdef}[1]{\begin{framed}
\textbf{Propriété - Définition :} {#1}
 \end{framed}} 
\newcommand{\propsdef}[1]{\begin{framed}
\textbf{Propriétés - Définition :} {#1}
 \end{framed}}  
 
 
 \newcommand{\csqs}[1]{\begin{framed}
 \textbf{Conséquences :} {#1}
 \end{framed}} 
 
\newcommand{\cor}[1]{\begin{framed}
\textbf{Corollaire :} {#1}
 \end{framed}}     
     
     
%\newcommand{\cadre}[2]{
%\begin{tcolorbox}[colback=red!5!white,
%                  colframe=red!75!black,
%                  title={#1}]
%{#2}
%\end{tcolorbox}  }

\newcommand{\cadre}[3]{
\begin{tcolorbox}[colback=#1!5!white,
                  colframe=#1!75!black,
                  title={#2}]
{#3}
\end{tcolorbox}  }




\newcommand*{\Coord}[3]{% 
\ensuremath{\vv{#1}\, 
    \begin{pmatrix} 
      #2\\ 
      #3 
    \end{pmatrix}}} % coordonnées de vecteurs.
    
\newcommand{\fonction}[5]{\begin{tabular}[t]{cccc}
$#1 :$ & $#2$ & $\longrightarrow$ & $#3$ \\
 & $#4$ & $\longmapsto$ & $#5$
\end{tabular}}
    
\newcommand{\att}{{\fontencoding{U}\fontfamily{futs}\selectfont\char 66\relax \quad}}    % symbole attention

\newcommand{\exo}{%Création d'une macro ayant un paramètre
\addtocounter{numexos}{1}%chaque fois que cette macro est appelée, elle ajoute 1 au compteur numexos
{\sc{Exercice\,\thenumexos\,:}}\,%la valeur du compteur appelée par \thenumeexos
}

\newcommand{\bint}{\displaystyle \int\limits} %signe intégral plus grand

\newcommand{\bsum}{\displaystyle \sum} %signe somme plus grand

\newcommand{\bprod}{\displaystyle \prod\limits} %signe produit plus grand

\newcommand*{\norme}[1]{\left\lVert\vv{#1}\right\rVert}  %norme de vecteur

\newcommand{\ps}[2]{\ensuremath{\vv{#1}.\vv{#2}}} %produit scalaire

\newcommand{\x}{\times} %produit

\newcommand{\modulo}{\text{ modulo }} 

\renewcommand{\Im}[1]{\text{Im} (#1)}

\renewcommand{\Re}[1]{\text{Re} (#1)}

\newcommand{\modu}[3]{{#1} \equiv {#2} ~ [{#3}]}


\newcommand*{\sep}[1]{
\dotfill
\vspace*{-0.3cm}
\begin{center}
\textbf{#1}
\end{center}
\vspace*{-0.5cm}
\dotfill}


%Flèches avec commentaire : exemple $\xRightarrow[test1]{test1}$
%\makeatletter
%\newcommand{\xRightarrow}[2][]{\ext@arrow 0359\Rightarrowfill@{#1}{#2}}
%\makeatother
%
%\makeatletter
%\newcommand{\xLeftarrow}[2][]{\ext@arrow 0359\Leftarrowfill@{#1}{#2}}
%\makeatother
%
%\makeatletter
%\newcommand{\xLeftrightarrow}[2][]{\ext@arrow 0359\Leftrightarrowfill@{#1}{#2}}
%\makeatother


% Renouvellement commandes

\renewcommand{\leq}{\leqslant}
\renewcommand{\geq}{\geqslant}
\renewcommand{\thesection}{\Roman{section} )  \hspace{-4mm}}
\renewcommand{\thesubsection}{\quad \arabic{subsection}\hspace{-0.9mm} ) \hspace{-5mm}}
\renewcommand{\thesubsubsection}{\qquad \alph{subsubsection}\hspace{0.5mm} ) \hspace{-5mm}}


% Présentation générale

\newcommand{\titre}[1]{\begin{center} \Large \sc \fbox{{#1}}\end{center}}

\newcommand{\entete}[1]{\begin{center} \large \underline{#1} \end{center}}

\titleformat*{\section}{\large\bfseries}
\titleformat*{\subsection}{\large\bfseries}
\titleformat*{\subsubsection}{\large\bfseries}
\titleformat*{\paragraph}{\large\bfseries}
\titleformat*{\subparagraph}{\large\bfseries}

%\newcommand{\DS}[3]{\begin{tabular}{|L{6cm}|C{5cm}|C{6cm}|}
%\hline 
%Nom : & Devoir Surveillé #1 & Classe \quad :\quad #2 \\ 
%Prénom : &  & #3 \\ 
%\hline 
%\end{tabular} }

\newcommand{\DS}[3]{\begin{tabularx}{1 \linewidth}{|G|M|M|}
\hline 
Nom : & Devoir Surveillé #1 & Classe \quad :\quad #2 \\ 
Prénom : &  & #3 \\ 
\hline 
\end{tabularx} }

\newcommand{\DSBTS}[4]{\begin{tabularx}{1 \linewidth}{|G|M|M|}
\hline 
Nom : & Devoir Surveillé #2 & Classe \quad :\quad #3 \\ 
Prénom : & #1 & #4 \\ 
\hline 
\end{tabularx} }

\newcommand{\DM}[3]{\begin{tabular}{|L{6.1cm}|C{5.6cm}|C{6.1cm}|}
\hline 
Nom : & Devoir à la maison~ #1 & Classe \quad :\quad #2 \\ 
Prénom : &  & Pour le #3 \\ 
\hline 
\end{tabular} }

\newcommand{\ct}[3]{\begin{tabular}{|L{6.1cm}|C{5.6cm}|C{6.1cm}|}
\hline 
Nom : & Contrôle~ #1 & Classe \quad :\quad #2 \\ 
Prénom : &  & Le #3 \\ 
\hline 
\end{tabular} }

% Repères automatisés avec Tikz 
% Définition des nouvelles options xmin, xmax, ymin, ymax
% Valeurs par défaut : -3, 3, -3, 3
\tikzset{
xmin/.store in=\xmin, xmin/.default=-3, xmin=-3,
xmax/.store in=\xmax, xmax/.default=3, xmax=3,
ymin/.store in=\ymin, ymin/.default=-3, ymin=-3,
ymax/.store in=\ymax, ymax/.default=3, ymax=3,
}
% Commande qui trace la grille entre (xmin,ymin) et (xmax,ymax)
\newcommand {\grille}
{\draw[help lines] (\xmin,\ymin) grid (\xmax,\ymax);}
% Commande \axes
\newcommand {\axes} {
\draw[->] (\xmin,0) -- (\xmax,0);
\draw[->] (0,\ymin) -- (0,\ymax);
}
% Commande qui limite l’affichage à (xmin,ymin) et (xmax,ymax)
\newcommand {\fenetre}
{\clip (\xmin,\ymin) rectangle (\xmax,\ymax);}

% Exemple 
%\begin{center}
%\begin{tikzpicture} [xmin=@,xmax=@,ymin=@,ymax=@]
%\grille \axes \fenetre
%\draw plot[smooth] (\x,@);
%\end{tikzpicture}
%\end{center}

%%%%%%%\newtheorem{}{×}%%%%%%%%%%%%%%%%%%%%%%%%%%%%%%%%%%%%%%%%%%%%%%%%%%%%%%%%%%%%%
%\theoremseparator{\hspace{0.8mm} :}
%{\theoremseparator{~:}
%\newtheorem*{rem}{Remarque}
%\newtheorem*{rems}{Remarques}
%\newframedtheorem{theo}[theocompt]{Théorème}
%\newtheorem{prop}[propcompt]{Propriété}
%\newtheorem{regle}[regcompt]{Règle}
%\newtheorem{defi}[defcompt]{Définition}
%\newtheorem*{voc}{Vocabulaire}}


%algo
%VARIABLES : 
%\Variables
%Début et fin de bloc :
%\DebutAlgo 
%\FinAlgo 
%\DebutPour
%\FinPour
%\DebutTantQue
%\FinTantQue
%\DebutSi
%\FinSi
%\DebutSinon
%\FinSinon
%SI...ALORS :
%\Si{(...)}
%SINON :
%\Sinon
%POUR ... ALLANT_DE ... A ...
%\Pour{...}{...}{...}
%TANT_QUE(...)
%\Tantque{(...)}
%Pour toutes les autres instructions (y compris les déclarations de variable)
%\Ligne ...


% Mise en page
\RequirePackage[left=1cm, right=1cm, top=1cm , bottom=1cm]{geometry}
\pagestyle{empty} % pas de numéro de page
\renewcommand{\arraystretch}{1.5} %hauteur des lignes dans un tableau
\setlength{\parindent}{0pt} % pas de retrait de paragraphe

%\RequirePackage[left=1cm, right=1cm, top=-0.5cm , bottom=1cm,includeheadfoot]{geometry}
%\usepackage{fancyhdr}
%\pagestyle{fancy}

%\renewcommand{\headrulewidth}{1pt}
%\fancyhead[C]{\textbf{page \thepage}} 
%\fancyhead[L]{\leftmark}
%\fancyhead[R]{machin}

%\renewcommand{\footrulewidth}{1pt}
%\fancyfoot[C]{\textbf{page \thepage}} 
%\fancyfoot[L]{truc}
%\fancyfoot[R]{\leftmark}

% \tikz[baseline=(letter.base)]\node[draw,circle,inner sep=1pt] (letter) {B} %lettre entourée



\begin{document}


\titre{Matrices : quelques applications}

\section{Résolution d'un système linéaire}


\exe On considère le système $(S)$ suivant $\left\lbrace \begin{array}{l}5x+2y=16  \\ 4x+3y=17 \end{array} \right.$

On pose $A=\begin{pmatrix} 5 & 2\\ 4 & 3 \end{pmatrix}, X=\begin{pmatrix} x\\ y \end{pmatrix}$ et $ B=\begin{pmatrix} 16\\ 
17 \end{pmatrix}$.


On a alors $A \x X = \begin{pmatrix} 5x+2y\\ 4x+3y \end{pmatrix}.$

Ainsi le système peut s'écrire $AX=B$.


\prop{Un système linéaire de la forme $\left\lbrace \begin{array}{l} a_{11}x_1+a_{12}x_2+ \dots  + a_{1n}x_n=b_1\\

a_{21}x_1+a_{22}x_2+ \dots  + a_{2n}x_n=b_2\\
 
 \dots \\

a_{n1}x_1+a_{n2}x_2+ \dots  + a_{nn}x_n=b_n\\
 
\end{array} \right.$

est équivalent à l'équation $AX = B$ où $A = (a_{ij })$ est une matrice carrée d'ordre $n$, $X = (x_j)$ et $B = (b_j)$ sont deux matrices colonnes.

Si $A$ est inversible, alors l'équation $AX = B$ admet pour unique solution $X = A^{-1}B$.

}

\dem $AX=B$ d'où $A^{-1}AX=A^{-1}B \iff I_n X=A^{-1}B \iff X = A^{-1}B$.

\rema Dans le contexte de la propriété précédente, si $A$ n'est pas inversible alors le système correspondant possède une infinité de solutions ou aucune solution.

\exes
\begin{enumerate}[1.]
\item Résoudre, à l'aide des matrices et sans utiliser sa calculatrice, le système $S_1 : \left\lbrace \begin{array}{l}5x+2y=16  \\ 4x+3y=17 \end{array} \right.$
\item Résoudre, si possible, à l'aide des matrices et de sa calculatrice le système 
$S_2 : \left\lbrace \begin{array}{l}x+y+z=125  \\ 2x+3y+z=136 \\ x+2y+3z=143 \end{array} \right.$
\end{enumerate}

\section{Suites de matrices}

\defi{Soit $(U_n)$ une suite de matrices colonnes à $p$ lignes.

La suite $(U_n)$ est définie par son premier terme (en général la matrice colonne $U_0$), et par la relation de récurrence $U_{n+1}=AU_n+C$, où $A$ est une matrice carrée d'ordre $p$, et $C$ une matrice colonne à $p$ lignes.

Dans ce cas, l'état stable, s'il existe, est la matrice colonne à $p$ lignes $S$ qui vérifie $S=AS+C$.

}

\defi{Soit $U_n$ une suite de matrices colonnes à $p$ lignes. Soit $L$ une matrice colonne à $p$ lignes.

La suite $(U_n)$ tend vers $L$ lorsque la limite quand $n$ tend vers $+\infty$ de chaque coefficient de $U_n$ est égale au coefficient de $L$ correspondant.}


\prop{On considère une suite $(U_n)$ de matrices colonnes telle que $U_{n+1} = AU_n + B$ pour tout entier $n$.

\begin{enumerate}[$(i)$]
\item S'il existe une matrice $X$ telle que $AX + B = X$ alors la suite $(V_n)$ définie par $V_n = U_n - X$ vérifie :

$V_{n+1} = AV_n$ pour tout entier $n$.

Dans ce cas on a $U_n = A^n(U_0 - X) + X$ pour tout entier $n$.
\item Si $(U_n)$ est une suite convergente, alors elle converge vers une matrice $U$ vérifiant $AU + B = U$.

\end{enumerate}
}


\dem

Soit une suite $(U_n)$ de matrices colonnes telles que $U_{n+1} = AU_n + B$ pour tout entier $n$.

$(i)$ S'il existe une matrice $X$ telle que $AX + B = X$, posons $(V_n)$ définie par $V_n = U_n - X$.

Alors $V_{n+1} = U_{n+1} - X = AU_n + B - (AX + B) = AU_n + B - AX - B = AU_n - AX = A(U_n - X) = AVn$.

La suite $(V_n)$ est donc une suite géométrique de raison $A$, donc $V_n = A^nV_0$ pour tout entier $n$.

(La démonstration de cette propriété peut se faire par récurrence).

\newpage

Or $V_0 = U_0 - X$ donc $V_n = A^n(U_0 - X)$ pour tout entier $n$.

$V_n = U_n - X$ donc $U_n = V_n + X = A^n (U_0 - X) + X$ pour tout entier $n$.


$(ii)$ Si $(U_n)$ est une suite convergente, soit $U$ sa limite.

D'une part $\lim\limits_{n \rightarrow +\infty} U_{n+1}=\lim\limits_{n \rightarrow +\infty} U_{n}=U$.

D'autre part $\lim\limits_{n \rightarrow +\infty} U_{n+1}=\lim\limits_{n \rightarrow +\infty} (Au_n+B)=
A(\lim\limits_{n \rightarrow +\infty} U_n) +B=AU+B$.

Donc $U$ vérifie bien $AU+B=U$.

\medskip

\exe 


Soit $(a_n)$ et $(b_n)$ deux suites définies par : $a_0=10,b_0=20,a_{n+1}=0,9a_n-0,7b_n+4$ et $b_{n+1}=0,2b_n+3$

On pose $U_n=\begin{pmatrix} a_n\\b_n \end{pmatrix}$.


La suite $(U_n)$ est définie par son premier terme $U_0=\begin{pmatrix} 10\\20 \end{pmatrix}$ et par la relation de récurrence $U_{n+1}=AU_n+C$.


\begin{enumerate}[a.]
\item Donner les matrices $A$ et $C$.
\item Déterminer, s'il existe, l'état stable $S$.
\item On considère la suite $(V_n)$ vérifiant $V_n=U_n-S$. Montrer que $V_{n+1}=AV_n$.
\item Montrer par récurrence que, pour tout entier $n$ on a $V_n=A^nV_0$.
\item Montrer par récurrence que $A^n=\begin{pmatrix} 0,9^n & 0,2^n-0,9^n\\0 & 0,2 ^n \end{pmatrix}$.
\item Déterminer une formule explicite pour $U_n$ en fonction de $A^n,U_0$ et $S$.

Puis en déduire que $\left\lbrace \begin{array}{l} a_n=-20 \x 0,9^n + 16,25 \x 0,2^n + 13,75 \\ b_n= 16,25 \x 0,2^n + 3,75 \end{array} \right.$

\item Montrer que, si la suite $(U_n)$ tend vers une matrice $L$, alors $L=S$.
\item Montrer que la suite $(U_n)$ tend effectivement vers $S$.
\end{enumerate}


\section{Quelques transformations géométriques}

\prop { On se place dans un repère $(O,\vv{i},\vv{j})$. Soient $A(x_A;y_A),B(x_B;y_B)$ et $\Coord{u}{x_{\vv{u}}}{y_{\vv{u}}}$.

$B$ est l'image de $A$ par la translation de vecteur $\vv{u}$ si, et seulement si $\begin{pmatrix} x_B\\y_B \end{pmatrix}=
\begin{pmatrix} x_A\\y_A \end{pmatrix} + \begin{pmatrix} x_{\vv{u}}\\y_{\vv{u}}\end{pmatrix} $.

%\item $B$ est l'image de $A$ par la rotation de centre $O$ et d'angle $\theta$ si, et seulement si $\begin{pmatrix} x_B\\y_B \end{pmatrix}= \begin{pmatrix} \cos(\theta) & -\sin(\theta) \\ \sin(\theta) & \cos(\theta) \end{pmatrix} \x \begin{pmatrix} x_A\\y_A \end{pmatrix}$.
%
%On appelle matrice de rotation de centre $O$ et d'angle $\theta$ la matrice \raisebox{-7.5mm}{\shadowbox{ $R_{(O;\theta)}=\begin{pmatrix} \cos(\theta) & -\sin(\theta) \\ \sin(\theta) & \cos(\theta) \end{pmatrix}$}}
%
%\item $\vv{v}$ est l'image de $\vv{u}$ par la rotation de centre $O$ et d'angle $\theta$ si, et seulement si $\begin{pmatrix} x_{\vv{v}} \\ y_{\vv{v}} \end{pmatrix} = \begin{pmatrix} \cos(\theta) & -\sin(\theta) \\ \sin(\theta) & \cos(\theta) \end{pmatrix} \x \begin{pmatrix} x_{\vv{u}}\\y_{\vv{u}}\end{pmatrix}$.

%\end{enumerate}
}




\props{ On peut définir pour les transformations géométriques suivantes des matrices de transformation $T=\begin{pmatrix} a & b \\c & d  \end{pmatrix}$ qui à tout point $M(x;y)$ du plan, associe son point image $M'(x';y')$ tel que  $\begin{pmatrix} x'\\y' \end{pmatrix} = \begin{pmatrix} a & b \\ c & d  \end{pmatrix} \x \begin{pmatrix} x\\y \end{pmatrix}$.

\begin{enumerate}[$(i)$]
\item  Matrice de rotation de centre $O$ et d'angle $\theta$ : $R_{(O;\theta)}=\begin{pmatrix} \cos(\theta) & -\sin(\theta) \\ \sin(\theta) & \cos(\theta) \end{pmatrix}$
\item Matrice d'homothétie de centre $O$ et de rapport $k \in \R$ : $H_{(O;k)}=\begin{pmatrix} k &  0 \\ 0 & k \end{pmatrix}$.

\item Matrice de la symétrie axiale par rapport à l'axe des abscisses : $S_{((Ox))}=\begin{pmatrix} 1 &  0 \\ 0 & -1 \end{pmatrix}$.

\item Matrice de la symétrie axiale par rapport à l'axe des ordonnées : $S_{((Oy))}=\begin{pmatrix} -1 &  0 \\ 0 & 1 \end{pmatrix}$.

\item Matrice de la symétrie centrale de centre O : $S_{0}=\begin{pmatrix} -1 &  0 \\ 0 & -1 \end{pmatrix}$.

\end{enumerate}


 
 
}



\dem Admises.


\medskip

\exe Soit $\Coord{u}{-1}{3},A(-2;3)$ et $ \theta = \dfrac{\pi}{6}$. Déterminer par calcul matriciel :

\begin{enumerate}[a.]
\item Les coordonnées de $B$ image de $A$ par la translation de vecteur $\vv{u}$
\item Les coordonnées de $C$ image de $B$ par la rotation de centre $O$ et d'angle $\theta$. 
\item Les coordonnées de $D$ image de $C$ par l'homothétie de centre $O$ et rapport $k=2$.
\item Les coordonnées de $E$ image de $D$ par la symétrie axiale par rapport à l'axe des ordonnées.
\end{enumerate}

\end{document}	





