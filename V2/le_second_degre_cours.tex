\documentclass[10pt]{article} % Sets the document class to 'article' with a font size of 12pt

\newcommand{\DocumentTitle}{Le second degré}
\newcommand{\DocumentTheme}{Algebre}
\newcommand{\DocumentType}{Cours}

% Preamble Section
\usepackage[a4paper, margin=1in]{geometry} % Sets the paper size to A4 and all margins to 1 inch
\usepackage[utf8]{inputenc}   % Allows for input of international characters
\usepackage[T1]{fontenc}      % Utilisation de l'encodage T1
\usepackage{lmodern}          % Support des polices avec le module 'french'
\usepackage[french]{babel}    % Ajout du support pour le français
\usepackage[fleqn]{amsmath}   % Imports the AMS math package for advanced math formatting ([fleqn] -> align alignat to the left)
\usepackage{amsfonts}         % Imports AMS fonts for math fonts
\usepackage{amssymb}          % Imports AMS symbols for math symbols
\usepackage{amsthm}           % Imports AMS theorems
\usepackage{fancyhdr}         % Import to custom the page footer
\usepackage{mdframed}         % Imports styles
\usepackage{enumerate}        % Imports styles for enumerate
\usepackage{multicol}         
\usepackage{xcolor}           % Couleurs
\usepackage{tikz}             % Package général pour les graphiques
\usepackage{pgfplots}         % Compléments pour les graphiques
\usepackage{tkz-tab}          % Tableaux de variations
\usepackage{lastpage}         % Pour avoir le total de page dans le footer

\pgfplotsset{compat=newest}   % Active les dernières fonctionnalités de pgfplots

% Enlève l'indentation des bloc de paragraphes et d'équation
\setlength{\parindent}{0pt}
\setlength{\mathindent}{0pt}

% ========== Title Section ==========

\title{\DocumentTitle} % Title of the document
\author{\DocumentTheme\space-\space\DocumentType} % Author's name
\date{} % Date

\pagestyle{fancy}
\fancyhf{} % Clear all header and footer fields

% ========== Footer Section ==========

\newcommand{\customfooter}{
    \fancyfoot[L]{\DocumentTheme\space-\space\DocumentType}
    \fancyfoot[C]{\DocumentTitle}
    \fancyfoot[R]{\thepage/\pageref{LastPage}}
    % Remove the line below the header and above the footer
    \renewcommand{\headrulewidth}{0pt}
    \renewcommand{\footrulewidth}{0pt}
}

% Apply the custom footer to the fancy style
\customfooter

% Apply the custom footer to the plain style (used on the first page)
\fancypagestyle{plain}{\customfooter}

% ========== Sections ==========

% Sections : I. / II. / III. ...
\renewcommand{\thesection}{\Roman{section}}
% Subsection : I.1 / I.2 / I.3 ...
\renewcommand{\thesubsection}{\thesection. \arabic{subsection}}
% Subsubsection : I.1.a / I.1.b / I.1.c ...
\renewcommand{\thesubsubsection}{\thesubsection. \alph{subsubsection}}

% ========== Colors (from Geogebra) ==========

\definecolor{green}{HTML}{006400}
\definecolor{red}{HTML}{CC0000}
\definecolor{blue}{HTML}{0000FF}
\definecolor{orange}{HTML}{FF5500}
\definecolor{purple}{HTML}{9933FF}
\definecolor{gray}{HTML}{666666}
\definecolor{brown}{HTML}{993300}

\newcommand{\green}[1]{\color{green}{#1}\color{black}}
\newcommand{\red}[1]{\color{red}{#1}\color{black}}
\newcommand{\blue}[1]{\color{blue}{#1}\color{black}}
\newcommand{\orange}[1]{\color{orange}{#1}\color{black}}
\newcommand{\purple}[1]{\color{purple}{#1}\color{black}}
\newcommand{\gray}[1]{\color{gray}{#1}\color{black}}
\newcommand{\brown}[1]{\color{brown}{#1}\color{black}}


% ========== Définition(s) ==========

% Style de la boîte de définition(s)
\mdfdefinestyle{DefinitionStyle}{
    leftmargin=0cm,
    rightmargin=0cm,
    linecolor=black,
    linewidth=2pt,
    topline=false,
    bottomline=false,
    rightline=false
}

% Définition de la commande '\definition'
\newcommand{\definition}[2]{
    \begin{mdframed}[style=DefinitionStyle]
        \textbf{Définition : #1}\\
        #2
    \end{mdframed}
}

% Définition de la commande '\definitions'
\newcommand{\definitions}[2]{
    \begin{mdframed}[style=DefinitionStyle]
        \textbf{Définitions : #1}\\
        #2
    \end{mdframed}
}

% ========== Propriété(s) et Théorème(s) ==========

% Style de la boîte de propriété(s) / théorème(s)
\mdfdefinestyle{ProprieteStyle}{
    leftmargin=0cm,
    rightmargin=0cm,
    linecolor=black,
    linewidth=1pt
}

% Création d'un compteur pour les propriétés
\newcounter{propriete}

% Définition de la commande '\propriete'
\newcommand{\propriete}[2]{
    \refstepcounter{propriete}
    \begin{mdframed}[style=ProprieteStyle] % ou un style différent si vous en avez créé un
        \textbf{Propriété \thepropriete\ :\space#1}\\
        #2
    \end{mdframed}
}

% Définition de la commande 'proprietes'
\newcommand{\proprietes}[2]{
    \refstepcounter{propriete}
    \begin{mdframed}[style=ProprieteStyle] % ou un style différent si vous en avez créé un
        \textbf{Propriétés \thepropriete\ :\space#1}\\
        #2
    \end{mdframed}
}

% Création d'un compteur pour les théorèmes
\newcounter{theoreme}

% Définition de la commande '\theoreme'
\newcommand{\theoreme}[2]{
    \refstepcounter{theoreme}
    \begin{mdframed}[style=ProprieteStyle] % ou un style différent si vous en avez créé un
        \textbf{Théorème \thetheoreme\ :\space#1}\\
        #2
    \end{mdframed}
}

% Définition de la commande '\theoremes'
\newcommand{\theoremes}[2]{
    \refstepcounter{theoreme}
    \begin{mdframed}[style=ProprieteStyle] % ou un style différent si vous en avez créé un
        \textbf{Théorèmes \thetheoreme\ :\space#1}\\
        #2
    \end{mdframed}
}

% ========== Exemple(s) ==========

% Définition de la commande 'exemple'
\newcommand{\exemple}[2]{
    \textbf{Exemple :} #1
    \begin{quote}
        #2
    \end{quote}
}

% Définition de la commande 'exemples'
\newcommand{\exemples}[2]{
    \textbf{Exemples :} #1
    \begin{quote}
        #2
    \end{quote}
}

% ========== Remarques ==========

% Définition de la commande 'remarque'
\newcommand{\remarque}[1]{
    \textbf{Remarque :} #1
}

% Définition de la commande 'remarques'
\newcommand{\remarques}[1]{
    \textbf{Remarques :} #1
}

% ========== Démonstration ==========

\newcommand{\demonstration}[2]{
    \textbf{Démonstration :} #1
    \begin{quote}
        #2
    \end{quote}
}

% ========== Notations ==========

% Ensembles
\newcommand{\C}{\mathbb{C}}
\newcommand{\R}{\mathbb{R}}
\newcommand{\Q}{\mathbb{Q}}
\newcommand{\D}{\mathbb{D}}
\newcommand{\Z}{\mathbb{Z}}
\newcommand{\N}{\mathbb{N}}


\begin{document}

\maketitle % Generates the title

\section{Les fonctions polynômes du second degré}

\subsection{Forme développée}

\definitions{}{
	On appelle fonction polynôme (ou trinôme) du second degré toute fonction $f$ définie sur $\R$ par
	$f(x)=ax^2+bx+c$ où $a$, $b$ et $c$ sont trois réel avec $a\not=0$. \\
	Les réels $a$, $b$ et $c$ sont appelés coefficients de la fonction.
}

\remarque{L'expression $ax^2+bx+c$ est dit forme développée de $f(x)$.}

\subsection{Forme canonique}

\theoreme{}{
	Toute fonction trinôme du second degré définie par $f(x)=ax^2+bx+c$ peut s'écrire sous une forme appelée
	canonique $f(x)=a(x-\alpha)^2+\beta$, avec $\alpha=-\dfrac{b}{2a}$ et $\beta=f(\alpha)$.
}

\subsection{Sens de variation}

\propriete{}{
	Soit $f$ une fonction définie sur $\mathbb{R}$ par $f(x)=a(x-\alpha)^2+\beta$.
	\begin{enumerate}[(i)]
		\item Cas où $a>0$ : la fonction $f$ est strictement décroissante sur $]-\infty;\alpha]$ puis strictement croissante sur $[\alpha;+\infty[$. La fonction $f$ admet un minimum égal à $\beta$ atteint en $x=\alpha$.
		\item Cas où $a<0$ : la fonction $f$ est strictement croissante sur $]-\infty;\alpha]$ puis strictement décroissante sur $[\alpha;+\infty[$. La fonction $f$ admet un maximum égal à $\beta$ atteint en $x=\alpha$.
	\end{enumerate}
}

\textbf{On retient :}
\vspace{0.5em}

\begin{minipage}{0.5\textwidth}
	\begin{tikzpicture}[baseline, scale=1, transform shape]
		% Configuration des styles de ligne et de texte
		\tkzTabInit[
			lgt = 1.5, 	% Largeur de la première colonne
			espcl = 2,	% Largeur de la deuxième colonne
		]{$x$/1, $f(x)$/1.5}{$-\infty$, $\alpha$, $+\infty$}
		% Contenu du tableau de variations
		\tkzTabVar{+/, -/$\beta$, +/}
	\end{tikzpicture}
	\vspace{0.5em} % Ajoute de l'espace avant le tikzpicture
	\par car $a > 0$ % Text on a new line 
\end{minipage}% <- This '%' sign is crucial to avoid unintended space
\begin{minipage}{0.5\textwidth}
	\begin{tikzpicture}[baseline, scale=1, transform shape]
		% Configuration des styles de ligne et de texte
		\tkzTabInit[
			lgt=1.5, % largeur entre les lignes
			espcl=2, % espacement des colonnes
		]{$x$/1, $f(x)$/1.5}{$-\infty$, $\alpha$, $+\infty$}
		% Contenu du tableau de variations
		\tkzTabVar{-/, +/$\beta$, -/}
	\end{tikzpicture}
	\vspace{0.5em} % Ajoute de l'espace avant le tikzpicture
	\par car $a < 0$ % Text on a new line 
\end{minipage}

\subsection{Représentation graphique}

\propriete{conséquence}{
	Soit $f$ une  fonction définie par $f(x)=a(x-\alpha)^2+\beta$. \\
	Dans un repère orthogonal d'origine $O$, la représentation graphique de la fonction $f$ est une parabole de sommet
	$S(\alpha;\beta)$ qui admet pour axe de symétrie la droite d'équation $x=\alpha$.
}

\newpage

\textbf{On retient :}
\vspace{0.5em}

\begin{minipage}{0.5\textwidth}
	\centering

	\begin{tikzpicture}
		\begin{axis}[
				axis lines=middle,
				xlabel={$x$}, ylabel={$y$},
				xmin=0, xmax=4,
				ymin=0, ymax=4,
				xtick={2},
				xticklabels={$\color{green}\alpha$},
				ytick={1},
				yticklabels={$\color{blue}\beta$},
				samples = 100
			]

			\addplot[green, densely dashed, thin] coordinates {(2,0) (2,5)}; % Courbe de x=2
			\addplot[blue, densely dashed, thin] {1}; % Courbe de y=1

			\addplot[red, thick] {(x - 2)^2 + 1}; % Courbe de f(x)
			\node[red] at (axis cs:3.25,3.75) {$f(x)$}; % Nom de f(x)

			\addplot[mark=*, black, mark size = 1.5, thick] coordinates {(2,1)};
			\node[black] at (axis cs:2.5,0.75) {$S(\green{\alpha};\blue{\beta})$}; % Nom de S(alpha;beta)
		\end{axis}
	\end{tikzpicture}

	\vspace{0.5em} % Ajoute de l'espace avant le tikzpicture
	\par car $a > 0$ % Text on a new line 
\end{minipage}% <- This '%' sign is crucial to avoid unintended space
\begin{minipage}{0.5\textwidth}
	\centering
	\begin{tikzpicture}
		\begin{axis}[
				axis lines=middle,
				xlabel={$x$}, ylabel={$y$},
				xmin=0, xmax=4,
				ymin=0, ymax=4,
				xtick={2},
				xticklabels={$\color{green}\alpha$},
				ytick={1},
				yticklabels={$\color{blue}\beta$},
				samples = 100
			]

			\addplot[green, densely dashed, thin] coordinates {(2,0) (2,5)}; % Courbe de x=2
			\addplot[blue, densely dashed, thin] {3}; % Courbe de y=1

			\addplot[red, thick] {-(x - 2)^2 + 3}; % Courbe de f(x)
			\node[red] at (axis cs:3.25,0.5) {$f(x)$}; % Nom de f(x)

			\addplot[mark=*, black, mark size = 1.5, thick] coordinates {(2,3)};
			\node[black] at (axis cs:2.5,3.25) {$S(\green{\alpha};\blue{\beta})$}; % Nom de S(alpha;beta)
		\end{axis}
	\end{tikzpicture}

	\vspace{0.5em} % Ajoute de l'espace avant le tikzpicture
	\par car $a < 0$ % Text on a new line 
\end{minipage}


\section{Factorisation d'une fonction du second degré et résolution d'équation du second degré}

\subsection{Factorisation}

\definition{discriminant}{
	On appelle discriminant de la fonction trinôme $f(x)=ax^2 + bx + c$ ou de l'équation $ax^2+bx+c=0$ le réel $\Delta$ défini
	par $\Delta=b^2-4ac$.
}

\theoreme{factorisation d'un trinôme du second degré}{
	Soit $f$ définie sur $\R$ par $f(x)=ax^2+bx+c$.
	\begin{enumerate}[(i)]
		\item Si $\Delta<0$, alors $f(x)=ax^2+bx+c$ n'est pas factorisable.
		\item Si $\Delta=0$, alors $f(x)=a(x-\alpha)^2$ où $\alpha=-\dfrac{b}{2a}$.
		\item Si $\Delta>0$, alors $f(x)=a(x-x_1)(x-x_2)$ où  $x_1=\dfrac{-b-\sqrt{\Delta}}{2a}$ et $x_2=\dfrac{-b+\sqrt{\Delta}}{2a}$.
	\end{enumerate}
}

\subsection{Résolution des équation du second degré}

\theoreme{}{
	Soit l'équation $ax^2+bx+c=0$ avec $a\not=0$.
	\begin{enumerate}[(i)]
		\item Si $\Delta<0$, l'équation $ax^2+bx+c=0$ n'admet aucune solution.
		\item Si $\Delta=0$, l'équation $ax^2+bx+c=0$ admet une unique solution $\alpha=\dfrac{-b}{2a}$.
		\item Si $\Delta>0$, l'équation $ax^2+bx+c=0$ admet deux solutions distinctes : $x_1=\dfrac{-b-\sqrt{\Delta}}{2a}$ et $x_2=\dfrac{-b+\sqrt{\Delta}}{2a}$.
	\end{enumerate}
}

\subsection{Somme et produit des racines}

\propriete{}{
	Soit $x_1$ et $x_2$ les racines d'une fonction polynôme du second degré $f(x)=ax^2+bx+c$, avec $a\not=0$. \\
	On a alors $x_1+x_2=-\dfrac{b}{a}$ et $x_1\times x_2=\dfrac{c}{a}$
}

\section{Signe d'une fonction du second degré et inéquations}

\propriete{}{
	Soit $f$ définie sur $\mathbb{R}$ par $f(x)=ax^2+bx+c$.
	\begin{enumerate}[(i)]
		\item Si $\Delta<0$, alors pour tout réel $x$, $f(x)$ est du signe de $a$.
		\item Si $\Delta=0$, alors pour tout réel $x$, $f(x)$ est du signe de $a$ sauf en $\alpha$ où $f(x)=0$.
		\item Si $\Delta>0$, alors pour tout réel $x$, $f(x)$ s'annule en $x_1$ et $x_2$ et est du signe de $a$ pour
		      tout $x\in]-\infty;x_1[\cup]x_2;+\infty[$ avec $x_1<x_2$ et du signe opposé à celui de $a$ pour tout $x\in]x_1;x_2[$.
	\end{enumerate}
}

\remarque{On peut retenir que $f(x)$ est du signe de $a$ sauf entre les racines lorsqu'elles existent.}



\end{document}
