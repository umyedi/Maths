\documentclass[10pt]{article}

\newcommand{\DocumentTitle}{Probabilités et variables aléatoires}
\newcommand{\DocumentTheme}{Probabilités}
\newcommand{\DocumentType}{Cours}

% Preamble Section
\usepackage[a4paper, margin=1in]{geometry} % Sets the paper size to A4 and all margins to 1 inch
\usepackage[utf8]{inputenc}   % Allows for input of international characters
\usepackage[T1]{fontenc}      % Utilisation de l'encodage T1
\usepackage{lmodern}          % Support des polices avec le module 'french'
\usepackage[french]{babel}    % Ajout du support pour le français
\usepackage[fleqn]{amsmath}   % Imports the AMS math package for advanced math formatting ([fleqn] -> align alignat to the left)
\usepackage{amsfonts}         % Imports AMS fonts for math fonts
\usepackage{amssymb}          % Imports AMS symbols for math symbols
\usepackage{amsthm}           % Imports AMS theorems
\usepackage{fancyhdr}         % Import to custom the page footer
\usepackage{mdframed}         % Imports styles
\usepackage{enumerate}        % Imports styles for enumerate
\usepackage{multicol}
\usepackage{xcolor}           % Couleurs
\usepackage{tikz}             % Package général pour les graphiques
\usepackage{pgfplots}         % Compléments pour les graphiques
\usepackage{tkz-tab}          % Tableaux de variations
\usepackage{lastpage}         % Pour avoir le total de page dans le footer
\usepackage{makecell}         % Retour à la ligne dans une case d'un tabular
\usepackage{stmaryrd}         % Intervalles entières : \llbracket et \rrbracket

\pgfplotsset{compat=newest}   % Active les dernières fonctionnalités de pgfplots

% Enlève l'indentation des bloc de paragraphes et d'équation
\setlength{\parindent}{0pt}
\setlength{\mathindent}{0pt}

% ========== Title Section ==========

\title{\DocumentTitle} % Title of the document
\author{\DocumentTheme\space-\space\DocumentType} % Author's name
\date{} % Date

\pagestyle{fancy}
\fancyhf{} % Clear all header and footer fields

% ========== Footer Section ==========

\newcommand{\customfooter}{
    \fancyfoot[L]{\DocumentTheme\space-\space\DocumentType}
    \fancyfoot[C]{\DocumentTitle}
    \fancyfoot[R]{\thepage/\pageref{LastPage}}
    % Remove the line below the header and above the footer
    \renewcommand{\headrulewidth}{0pt}
    \renewcommand{\footrulewidth}{0pt}
}

% Apply the custom footer to the fancy style
\customfooter

% Apply the custom footer to the plain style (used on the first page)
\fancypagestyle{plain}{\customfooter}

% ========== Sections ==========

% Sections : I. / II. / III. ...
\renewcommand{\thesection}{\Roman{section}}
% Subsection : I.1 / I.2 / I.3 ...
\renewcommand{\thesubsection}{\thesection. \arabic{subsection}}
% Subsubsection : I.1.a / I.1.b / I.1.c ...
\renewcommand{\thesubsubsection}{\thesubsection. \alph{subsubsection}}

% ========== Colors (from Geogebra) ==========

\definecolor{green}{HTML}{006400}
\definecolor{red}{HTML}{CC0000}
\definecolor{blue}{HTML}{0000FF}
\definecolor{orange}{HTML}{FF5500}
\definecolor{purple}{HTML}{9933FF}
\definecolor{gray}{HTML}{666666}
\definecolor{brown}{HTML}{993300}

\newcommand{\green}[1]{\color{green}{#1}\color{black}}
\newcommand{\red}[1]{\color{red}{#1}\color{black}}
\newcommand{\blue}[1]{\color{blue}{#1}\color{black}}
\newcommand{\orange}[1]{\color{orange}{#1}\color{black}}
\newcommand{\purple}[1]{\color{purple}{#1}\color{black}}
\newcommand{\gray}[1]{\color{gray}{#1}\color{black}}
\newcommand{\brown}[1]{\color{brown}{#1}\color{black}}


% ========== Définition(s) ==========

% Style de la boîte de définition(s)
\mdfdefinestyle{DefinitionStyle}{
    leftmargin=0cm,
    rightmargin=0cm,
    linecolor=black,
    linewidth=2pt,
    topline=false,
    bottomline=false,
    rightline=false
}

% Définition de la commande '\definition'
\newcommand{\definition}[2]{%
    \begin{mdframed}[style=DefinitionStyle]
        \ifstrempty{#1}{% Teste si le premier argument est vide
            \textbf{Définition :}\ % Si vide, n'affiche pas de titre de théorème
        }{%
            \textbf{Définition #1 :}\ % Si non vide, affiche le titre avec le premier argument
        }\\#2
    \end{mdframed}
}

% Définition de la commande '\definitions'
\newcommand{\definitions}[2]{%
    \begin{mdframed}[style=DefinitionStyle]
        \ifstrempty{#1}{% Teste si le premier argument est vide
            \textbf{Définitions :}\ % Si vide, n'affiche pas de titre de théorème
        }{%
            \textbf{Définitions #1 :}\ % Si non vide, affiche le titre avec le premier argument
        }\\#2
    \end{mdframed}
}

% ========== Propriété(s) et Théorème(s) ==========

% Style de la boîte de propriété(s) / théorème(s)
\mdfdefinestyle{ProprieteStyle}{
    leftmargin=0cm,
    rightmargin=0cm,
    linecolor=black,
    linewidth=1pt
}

% Création d'un compteur pour les propriétés
\newcounter{propriete}

% Définition de la commande '\propriete'
\newcommand{\propriete}[2]{%
    \refstepcounter{propriete}% Incrémente le compteur de théorème
    \begin{mdframed}[style=ProprieteStyle]
        \ifstrempty{#1}{% Teste si le premier argument est vide
            \textbf{\thepropriete. Propriété :}\ % Si vide, n'affiche pas de titre de théorème
        }{%
            \textbf{\thepropriete. Propriété #1 :}\ % Si non vide, affiche le titre avec le premier argument
        }\\#2
    \end{mdframed}
}

% Définition de la commande 'proprietes'
\newcommand{\proprietes}[2]{%
    \refstepcounter{propriete}% Incrémente le compteur de théorème
    \begin{mdframed}[style=ProprieteStyle]
        \ifstrempty{#1}{% Teste si le premier argument est vide
            \textbf{\thepropriete. Propriétés :}\ % Si vide, n'affiche pas de titre de théorème
        }{%
            \textbf{\thepropriete. Propriétés #1 :}\ % Si non vide, affiche le titre avec le premier argument
        }\\#2
    \end{mdframed}
}



% Création d'un compteur pour les théorèmes
\newcounter{theoreme}

% Définition de la commande '\theoreme'
\newcommand{\theoreme}[2]{%
    \refstepcounter{theoreme}% Incrémente le compteur de théorème
    \begin{mdframed}[style=ProprieteStyle]
        \ifstrempty{#1}{% Teste si le premier argument est vide
            \textbf{\thetheoreme. Théorème :}\ % Si vide, n'affiche pas de titre de théorème
        }{%
            \textbf{\thetheoreme. Théorème #1 :}\ % Si non vide, affiche le titre avec le premier argument
        }\\#2
    \end{mdframed}
}

% Définition de la commande '\theoremes'
\newcommand{\theoremes}[2]{%
    \refstepcounter{theoreme}% Incrémente le compteur de théorème
    \begin{mdframed}[style=ProprieteStyle]
        \ifstrempty{#1}{% Teste si le premier argument est vide
            \textbf{\thetheoreme. Théorèmes :}\ % Si vide, n'affiche pas de titre de théorème
        }{%
            \textbf{\thetheoreme. Théorèmes #1 :}\ % Si non vide, affiche le titre avec le premier argument
        }\\#2
    \end{mdframed}
}

% ========== Exemple(s) ==========

% Définition de la commande 'exemple'
\newcommand{\exemple}[2]{
    \textbf{Exemple :} #1
    \begin{quote}
        #2
    \end{quote}
}

% Définition de la commande 'exemples'
\newcommand{\exemples}[2]{
    \textbf{Exemples :} #1
    \begin{quote}
        #2
    \end{quote}
}

% ========== Remarques ==========

% Définition de la commande 'remarque'
\newcommand{\remarque}[1]{
    \textbf{Remarque :} #1
}

% Définition de la commande 'remarques'
\newcommand{\remarques}[1]{
    \textbf{Remarques :} #1
}

% ========== Démonstration ==========

\newcommand{\demonstration}[2]{
    \textbf{Démonstration :} #1
    \begin{quote}
        #2
    \end{quote}
}

% ========== Notations ==========

% Ensembles
\newcommand{\C}{\mathbb{C}}
\newcommand{\R}{\mathbb{R}}
\newcommand{\Q}{\mathbb{Q}}
\newcommand{\D}{\mathbb{D}}
\newcommand{\Z}{\mathbb{Z}}
\newcommand{\N}{\mathbb{N}}

% Limites
\let\oldlim\lim
\renewcommand{\lim}[1]{\mathop{\displaystyle\oldlim}\limits_{#1}}

% Opérateurs
\newcommand{\x}{\times}
\newcommand{\equival}{\Leftrightarrow}

% Vecteurs
\renewcommand{\Vec}[1]{\overrightarrow{#1}}
\newcommand*{\norme}[1]{\left\lVert\vv{#1}\right\rVert}
\newcommand{\ps}[2]{\ensuremath{\vv{#1}.\vv{#2}}}

\newcommand{\vcoord}[2]{\begin{pmatrix} #1 \\ #2 \end{pmatrix}}
\newcommand{\Vcoord}[3]{\begin{pmatrix} #1 \\ #2 \\ #3 \end{pmatrix}}

% ========== Tableaux ==========
\newcommand{\boldhline}{\hline\noalign{\vskip 0pt}\hline}

\renewcommand{\arraystretch}{1.6} % Increase cell height in tables

\begin{document}

\maketitle

\section{Vocabulaire et notations ensemblistes}

\subsection{Modes de générations des ensembles}

\definition{}{
	Lorsqu'on définit un ensemble en extension, on écrit la liste complête de ses éléments entre deux accolades. L'ordre
	et les répétitions ne sont pas pris en compte.
}

\exemple{}{
	La notation $\{1; 2; 3\}$ désigne le même ensemble que la notation $\{1; 3; 2\}$ ou encore de la notation
	$\{1; 2; 2; 3\}$.
}

\definition{}{
	Soit $E$ un ensemble. Lorsqu'on définit un sous-ensemble $F$ de $E$ en compréhension, on donne une proposition
	$P(x)$ qui caractérise les éléments de $F$. L'ensemble des éléments de $E$ qui vérifient $P(x)$ est noté
	$\{x\in E / P(x)\}$.
}

\remarque{On peut aussi utiliser les notations $\{x\in E | P(x)\}$ et $\{x\in E ; P(x)\}$.}


\exemples{}{
	\begin{itemize}
		\item Soit $E=\{1; 2; 3; 4; 5\}$. La notation $\{x\in E/x\text{ est impair}\}$ désigne l'ensemble $\{1; 3; 5\}$.
		\item On définit en compréhension l'intervalle $[2, 5]$ par $\{ x\in\R /2\leq x \leq 5 \}$.
		\item En compréhension, l'ensemble des multiples de $3$ se note $\{ n\in\Z / \exists k\in\N, n=3k\}$.
	\end{itemize}
}

\subsection{Couples et produits cartésiens}

\definition{}{
	Soient $x$ et $y$ deux objets (nombres, points\dots). On définit un nouveau type d'objet, que l'on note $(x,y)$ et
	que l'on appelle le couple $(x,y)$.
}

\remarque{Deux couples $(x,y)$ et $(a,b)$ sont égaux si $x=a$ et $y=b$. Attention à ne pas confondre le couple $(1,2)$
	avec l'ensemble $\{1;2\}$ qui lui, est égal à l'ensemble $\{ 2;1 \}$.}

\definition{}{
	Soient $E$ et $F$ deux ensembles. On appelle produit cartésien de $E$ et $F$ l'ensemble des couples $(x,y)$ avec
	$x\in E$ et $y\in F$. On le note $E\times F$.
}

\remarques{
	\begin{itemize}
		\item $E\times E = E^2$, $E\times E\times E = E^3$\dots
		\item La notion de produit cartésien peut aussi s'étendre à plus de deux ensembles. Par exemple, si $E$, $F$ et
		      $G$ sont trois ensembles, le produit cartésien $E\times F\times G$ est l'ensemble des triplets $(x, y, z)$ avec
		      $x\in E$, $y\in F$ et $z\in G$.
	\end{itemize}
}

\exemples{}{
	\begin{itemize}
		\item Soient $E=\{1;2\}$ et $F=\{ 7;8;9 \}$. On a alors $E\times F = \{(1,7);(1,8);(1,9);(2,7);(2,8);(2,9)\}$.
		\item Soit $E=\{1;2\}$. On a alors $E^3 = \{(1,1,1);(1,1,2);(1,2,1);(2,1,1);(1,2,2);(2,1,2);(2,2,1);(2,2,2)\}$
	\end{itemize}
}

\newpage

\subsection{Inclusion}

\definition{}{
	Pour deux ensembles $A$ et $B$, on dit que $A$ est inclus dans $B$ (ou que $A$ est un sous-ensemble de $B$) lorsque
	tous les éléments appartenant à $A$ appartiennent aussi à $B$. On le note $A \subset B$.
}

\exemple{}{
	\begin{itemize}
		\item $\{3;9\} \subset \{3;6;9\}$
		\item $\{1\} \subset \{0;1;1;2;3;5\}$
	\end{itemize}
}

\subsection{Intersection et réunion}


\definition{}{
	Soit $\Omega$ un ensemble et $A$, $B$ deux sous-ensembles de $\Omega$.
	\begin{itemize}
		\item On appelle l'intersection de $A$ et $B$ l'ensemble des éléments de $\Omega$ qui appartiennent à la fois à
		      $A$ et à $B$. On la note $A \cup B$.
		\item On appelle la réunion (ou l'union) de $A$ et $B$ l'ensemble des éléments de $\Omega$ qui appartiennent à au
		      moins l'un des deux ensembles $A$ et $B$. On la note $A \cap B$.
	\end{itemize}
}

\exemple{}{
	\begin{itemize}
		\item $ ]-\infty;3]\cap]2;+\infty[ \text{ $=$ } ]2;3]$
		\item $ [1;3] \cup ]2;+\infty[ \text{ $=$ } [1;+\infty[ $
	\end{itemize}
}

\subsection{Complémentaire}

\definition{}{
	Soit $\Omega$ un ensemble et $A$ un sous-ensemble de $\Omega$. On appelle complémentaire de $A$ dans $\Omega$
	l'ensemble des éléments de $\Omega$ qui n'appartiennent pas à $A$. On le note $\bar{A}$.
}

\exemple{}{
	Soit $\Omega=\{ 1; 2; 3; 4; 5; 6 \}$ et $A=\{1; 3; 5; 6\}$. On a alors $\bar{A}=\{2; 4\}$
}

\subsection{Notations générales}

\begin{tabular}{|c|c|c|}
	\hline
	Notation               & Vocabulaire ensembliste    & Vocabulaire probabiliste   \\
	\hline
	$\Omega$               & ensemble plein             & évènement certain          \\
	\hline
	$\emptyset$            & ensemble vide              & évènement impossible       \\
	\hline
	$\omega$               & élément de $\Omega$        & évènement élémentaire      \\
	\hline
	$A$                    & sous-ensemble de $\Omega$  & évènement                  \\
	\hline
	$\omega \in A$         & $\omega$ appartient à $A$  & $\omega$ réalise $A$       \\
	\hline
	$A \subset B$          & $A$ inclus dans $B$        & $A$ implique $B$           \\
	\hline
	$A \cup B$             & réunion de $A$ et $B$      & $A$ ou (inclusif) $B$      \\
	\hline
	$A \cap B$             & intersection de $A$ et $B$ & $A$ et $B$                 \\
	\hline
	$\bar{A}$              & complémentaire de $A$      & évènement contraire de $A$ \\
	\hline
	$A \cap B = \emptyset$ & $A$ et $B$ disjoints       & $A$ et $B$ incompatibles   \\
	\hline
\end{tabular}

\newpage

\section{Probabilité conditionnelle et Indépendance}

\subsection{Univers et évènement}

\definitions{}{
	\vspace{-10pt}
	\begin{itemize}
		\item On appelle univers l'ensemble des issues possibles d'une expérience aléatoire. On le note souvent $\Omega$.
		\item On appelle évènement un sous-ensemble de l'univers $\Omega$, c'est à dire un ensemble d'issues.
	\end{itemize}
}

\subsection{Probabilité d'un évènement}

\definition{}{
	Soient $\Omega$ un unviers muni d'une loi de probabilité et $A$ un évènement non vide. On appelle probabilité de $A$
	la somme des probabilités des issues appartenant à $A$. On la note $P(A)$.
}

\remarque{On peut également utiliser la notation $P(A)$.}

\subsection{Équiprobabilité}

\definition{}{
	Soient $A$ et $B$ deux évènements d'un univers $\Omega$ tels que $P(A)\not= 0$. La probabilité de $B$ sachant que
	$A$ est réalisé (ou de $B$ sachant $A$) est le nombre, noté $p_A(B)$ défini par $p_A(B)=\dfrac{P(A\cap B)}{P(A)}$.
}

\propriete{}{
	La probabilité $p_A(B)$ vérifie bien $0\leq p_A(B)\leq1$ et $p_A(B)+p_A(\bar B)=1$.
}

\subsection{Probabilité de l'intersection}

\propriete{}{
	Si $A$ et $B$ sont deux évènements avec $P(A)\not=0$ alors $P(A\cap B)=p_A(B)\times P(A)$.
}

\remarques{
	\begin{itemize}
		\item Si $P(B)\not=0$ alors on a aussi $P(A\cap B)=p_B(A)\times P(B)$.
		\item Dans toutes les formules, les rôles de $A$ et $B$ peuvent être inversés.
	\end{itemize}
}

\subsection{Représentation à l'aide d'un arbre pondéré}

\begin{forest}
	for tree={
	grow'=east,
	align=center,
	edge={thick},
	l sep+=2cm,
	s sep+=0.2cm
	}
	[
	[$A$, edge label={node[midway,above left]{$P(A)$}}
				[$B$, edge label={node[midway,above]{$p_A(B)$}}]
				[$\bar B$, edge label={node[midway,below]{$p_A(\bar B)$}}]
		]
		[$\bar A$, edge label={node[midway,below left]{$P(\bar A)$}}
				[$B$, edge label={node[midway,above]{$p_{\bar A}(B)$}}]
				[$\bar B$, edge label={node[midway,below]{$p_{\bar A}(\bar B)$}}]
		]
	]
\end{forest}

\proprietes{}{
	\vspace{-10pt}
	\begin{enumerate}[(i)]
		\item La somme des probabilités inscrites sur les branches partant d'un même nœud est égale à $1$.
		\item La probabilité d'un chemin est le produit des probabilités inscrites sur ses branches.
	\end{enumerate}
}

\subsection{Partition d'un univers}

\definition{}{
	Soit $\Omega$ un univers muni d'une loi de probabilité
	\begin{itemize}
		\item On dit que deux évènements $A_1$ et $A_2$ forment une partition de $\Omega$ si $A_1\cap A_2=\emptyset$ et
		      $A_1\cup A_2=\Omega$.
		\item Plus généralement, on dit que des évènements forment une partition de $\Omega$ s'ils sont deux à deux
		      incompatibles de leur réunion est l'univers tout entier.
	\end{itemize}
}

\subsection{Formule des probabilités totales}

\theoreme{}{
	Soit $A_1,A_2,\dots,A_n$ un système complet d'évènements de l'univers $\Omega$. Alors la probabilité d'un évènement
	quelconque $B$ est donné par :
	\begin{equation*}
		\begin{split}
			P(B) & = P(B\cap A_1)+P(B\cap A_2)+\dots+P(B\cap A_n)                                \\
			& = P(A_1)\times p_{A_1}(B)+P(A_2)\times p_{A_2}(B)+...+P(A_n)\times p_{A_n}(B)
		\end{split}
	\end{equation*}
}

\propriete{}{
	La probabilité d'un évènement correspondant à plusieurs chemins est la sommes des probabilités de ces chemins.
}

\subsection{Indépendance de deux évènements}

\definition{}{
	On dit que deux évènements $A$ et $B$ sont indépendants lorsque $P(A\cap B)=P(A)\times P(B)$.
}

\propriete{}{
	On suppose que $P(A)\not=0$. $A$ et $B$ sont indépendants si et seulement si $p_A(B)=P(B)$.
}

\propriete{}{
	Si $A$ et $B$ sont deux évènements indépendants, alors $\bar A$ et $B$ le sont aussi.
}


\subsection{Succession de deux épreuves indépendantes}

\definition{}{
	Soit $n$ un entier naturel. On dit qu'une expérience aléatoire est la succession des $n$ épreuves indépendante si
	l'univers qui lui est associé est un produit cartésien $\Omega_1\times\Omega_2\times\dots\times\Omega_n$ muni d'une
	loi de probabilité vérifiant que la probabilité d'une issue $(x_1,x_2,\dots,x_n)$ est égal au produit des
	probabilités de ses composantes $x_i$.
}

\remarque{Autrement dit, c'est lorsque deux expériences aléatoires se succèdent et que les résultats de la première
	n'ont aucune influence sur les résultats de la seconde.}


\exemple{}{
On lance deux pièces à pile ou face. On suppose que la première pièce est déséquilibrée de sorte que « pile »
apparait deux fois plus souvent que « face ». La deuxième pièce est supposée bien équilibrée.

Le premier lancer est modélisé par $\Omega_1={P_1,F_1}$ muni de la loi $P(P_1)=\dfrac{2}{3}$ et
$P(F_1)=\dfrac{1}{2}$.

Le deuxième lancer est modélisé par $\Omega_2={P_2,F_2}$ muni de la loi $P(P_2)=\dfrac{1}{2}$ et
$P(F_2)=\dfrac{1}{2}$.

La succession des deux lancers peut être alors modélisé par l'univers : $\Omega=\Omega_1\times\Omega_2$
$=\{P_1,F_1\}\times\{P_2,F_2\}=\{(P_1,P_2);(P_1,F_2);(F_1,P_2);(F_1,F_2)\}$ muni de la loi suivante :


\begin{tabular}{|c|c|c|c|c|}
	\hline
	Évènement   & $(P_1,P_2)$                                & $(P_1,F_2)$                                & $(F_1,P_2)$                                & $(F_1,F_2)$                                \\
	\hline
	Probabilité & $\frac{2}{3}\times\frac{1}{2}=\frac{1}{3}$ & $\frac{2}{3}\times\frac{1}{2}=\frac{1}{3}$ & $\frac{1}{3}\times\frac{1}{2}=\frac{1}{6}$ & $\frac{1}{3}\times\frac{1}{2}=\frac{1}{6}$ \\
	\hline
\end{tabular}
}


\section{Variable aléatoire}

\subsection{Variable aléatoire discrète}

\definition{}{
	Soit $\Omega=\{\omega_1;\omega_2;\dots;\omega_k\}$ l'univers associé à une expérience aléatoire. Une variable
	aléatoire $X$ sur l'univers $\Omega$ est une fonction définie sur $\Omega$ et à valeurs dans $\R$. Définir une
	variable aléatoire consiste à associer, à chaque issue $\omega_i$ de l'expérience aléatoire, un réel $x_i$. On note
	alors $\left(X=x_i\right)$ l'évènement formé des issues qui ont pour image $x_i$ par $X$.
}

\definition{}{
	Soit $\Omega$ un univers et $X$ une variable aléatoire réelle sur $\Omega$. Pour tout réel $k$ :
	\begin{itemize}
		\item on note $(X\leq k)$ l'ensemble des issues dont l'image par $X$ est inférieur ou égal à k
		\item on note $(X\geq k)$ l'ensemble des issues dont l'image par $X$ est supérieur ou égal à k
	\end{itemize}
}

\subsection{Loi de probabilité}

\definition{}{
	Soit $X$ une variable aléatoire qui prend les valeurs $\{x_1;x_2;\dots;x_k\}$. Donner la loi de probabilité de $X$,
	c'est donner la valeur $P(X=x_i)$, pour tout $i$ avec $1\leq i\leq k$.

	Les résultats sont généralement présentés sous forme d'un tableau :

	\begin{tabular}{|l|c|c|c|c|}
		\hline
		Valeurs $x_i$ de $X$   & $x_1$      & $x_2$      & \quad\text{ }\dots\quad\text{ } & $x_k$      \\
		\hline
		Probabilité $P(X=x_i)$ & $P(X=x_1)$ & $P(X=x_2)$ & \quad\text{ }\dots\quad\text{ } & $P(X=x_k)$ \\
		\hline
	\end{tabular}
}

\remarque{La somme des probabilités $P(X=x_i)$, pour $i$ allant de $1$ à $k$, est égal à $1$.}

\subsection{Espérance, variance et écart-type}

Soit $X$ une variable aléatoire qui prend en valeurs ${x_1;x_2;\dots;x_k}$ et dont la loi de probabilité est donnée par
le tableau suivant :

\begin{center}
	\begin{tabular}{|l|c|c|c|c|}
		\hline
		Valeurs $x_i$ de $X$   & $x_1$ & $x_2$ & \quad\text{ }\dots\quad\text{ } & $x_k$ \\
		\hline
		Probabilité $P(X=x_i)$ & $p_1$ & $p_2$ & \quad\text{ }\dots\quad\text{ } & $p_k$ \\
		\hline
	\end{tabular}
\end{center}

\definition{}{
	L'espérance de la variable aléatoire $X$ est le réel noté $E(X)$ définie par : $\ds E(X)=\sum_{i=1}^{k}x_ip_i$
}

\remarque{L'espérance d'une variable aléatoire représente la valeur moyenne prise par $X$.}

\definition{}{
La variance de la variable aléatoire $X$ est le réel noté $V(X)$ définie par :
$\ds V(X)=\sum_{i=1}^{k}(x_i-E(X))^2\times p_i$}

	\remarque{La variance d'une variable aléatoire $X$ se calcule aussi avec : $V(X)=\sum_{i=1}^{k} p_i (x_i)^2-E(X)^2$.}

	\definition{}{
		L'écart-type $\sigma(X)$ est défini comme la racine carrée de la variance : $\sigma(X)=\sqrt{V(X)}$.
	}

	\remarque{Par analogie avec les statistiques, de la même façon que $E(X)$ représente une moyenne, $V(X)$ et $\sigma(X)$
		sont des indicateurs de dispersion des valeurs de $X$ autour de $E(X)$. Plus la variance et l'écart-type sont
		grands,	plus les valeurs sont dispersés autour de la moyenne (espérance)}

	\exemple{}{
		On calcule l'espérance de $X$ :
		$\ds E(X)=2\times\frac{1}{6}+1\times\frac{1}{6}-1\times\frac{2}{3}=-\frac{1}{6}\approx-0,17$. \\
		Sur un grand nombre de parties, le gain moyen par partie pour le joueur est $-0,17$. \\Donc le jeu n'est pas favorable au joueur. \\

		On calcule la variance et l'écart-type :
		\begin{itemize}
			\item $\ds V(X)=\left[2-\left(-\frac{1}{6}\right)\right]^2\times \frac{1}{6}+\left[1-\left(-\frac{1}{6}\right)\right]^2\times \frac{1}{6}+\left[-1-\left(-\frac{1}{6}\right)\right]^2\times \frac{2}{3}=\frac{53}{36}\approx1,47$
			\item $\ds \sigma(X)=\sqrt{\frac{53}{36}}\approx1,21$
		\end{itemize}
	}

	\subsection{Somme de variables aléatoires}

	\theoreme{(linéarité de l'espérance)}{
		Pour toutes variables aléatoires $X$ et $Y$ :
		\begin{enumerate}[(i)]
			\item $E(X+Y)=E(X)+E(Y)$
			\item Pour tout réel $a$ : $E(aX)=aE(X)$
		\end{enumerate}
	}

	\theoreme{}{
		Pour toute variable aléatoire :
		\begin{enumerate}[(i)]
			\item $V(X)=E(X^2)-(E(X))^2$ (formule de Koenig-Huygens)
			\item Pour tous réel $a$ : $V(aX)=a^2V(X)$
		\end{enumerate}
	}

	\definition{}{
		On dit que deux variables aléatoires $X$ et $Y$ sont indépendantes lorsque pour tous réels $k$ et $l$, les
		événements $(X=k)$ et $(Y=l)$ sont indépendants.
	}

	\theoreme{}{
		Pour toutes variables aléatoires $X$ et $Y$, si $X$ et $Y$ sont indépendantes, alors $V(X+Y)=V(X)+V(Y)$.
	}

	\section{La loi binomiale}

	\definitions{}{
		\vspace{-10pt}
		\begin{itemize}
			\item On appelle épreuve de Bernoulli tout expérience aléatoire modélisée par un univers à deux issues. On note
			      cet univers $\Omega=\{S;E\}$.
			\item Si de plus, l'univers est muni d'une loi de probabilité qui à l'issue $S$ associe la probabilité $p$, on
			      parle alors d'une épreuves de Bernoulli de paramètre $p$.
		\end{itemize}
	}

	\exemple{}{
		On lance un dé à six faces. On note $S$ l'issue « obtenir la face 6 » et $E$ l'issue « ne pas obtenir la face 6 ».

		On associe à $S$ la probabilité $\dfrac{1}{6}$ et à $E$ l'issue $\dfrac{5}{6}$. On a ainsi défini une épreuve de
		Bernoulli de paramètre $\dfrac{1}{6}$
	}

	\remarques{
		\begin{itemize}
			\item Les notations $S$ et $E$ font référence respectivements aux mots « succès » et « échec ».
			\item Dans une épreuve de Bernoulli de paramètre $p$, la probabilité de l'issue $E$ est égal à $1-p$.
		\end{itemize}
	}

	\definition{}{
		Soit $n$ un entier naturel non nul et $p$ un réel compris entre $0$ et $1$. On appelle « schéma de Bernoulli de
		paramètres $n$ et $p$ » la succession de $n$ épreuves de Bernoulli de paramètre $p$ identiques et indépendantes.
	}

	\remarque{L'univers associé à un schéma de Bernoulli de paramètre $n$ et $p$ est $\Omega^n=\{S;E\}^n$.}

	\exemple{}{
		On lance trois fois un dé à six faces bien équilibré. On modélise chacun des lancers de la manière décrite dans
		l'exemple précédent.

		On a alors défini un schéma de Bernoulli de paramètres $n=3$ et $p=\dfrac{1}{6}$.

		L'univers associé est le produit cartésien $\Omega^3=\{S;E\}^3$. Représenter la situation par un arbre peut aider à
		énumérer toutes les issues $\Omega^3=\{(S,S,S);(S,S,E);(S,E,S);(S,E,E);(E,S,S);$
		$(E,S,E);(E,E,S);(E,E,E)\}$.
	}

	\definition{}{
	Soit $n$ un entier naturel non nul et $p$ un réel compris entre $0$ et $1$. On considère un schéma de Bernoulli de
	paramètres $(n,p)$ et on note $\Omega={S;E}$. On définit une variable aléatoire $X$ sur $\Omega^n$ en associant à
	chaque issue le nombre d'apparition de $S$. La loi de probabilité de $X$ est appelée « loi binomiale de paramètres
$n$ et $p$ ».
}

\exemple{}{
	D'après l'exemple précédent, on définit une variable aléatoire $X$ sur $\Omega^3$ en associant à chaque issue le
	nombre d'apparitions de $S$ (ici, le nombre d'apparitions de « pile »). La loi de probabilité de $X$ suivante est la
	loi binomiale de paramètres $n=3$ et $p=\dfrac{1}{2}$.

	\begin{tabular}{|l|c|c|c|c|}
		\hline
		Valeurs $x_i$ de $X$   & $0$           & $1$           & $2$           & $3$           \\
		\hline
		Probabilité $P(X=x_i)$ & $\frac{1}{8}$ & $\frac{3}{8}$ & $\frac{3}{8}$ & $\frac{1}{8}$ \\
		\hline
	\end{tabular}
}



\end{document}