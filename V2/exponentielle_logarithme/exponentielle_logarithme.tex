\documentclass[10pt]{article}

\newcommand{\DocumentTitle}{Fonction exponentielle et logarithme népérien}
\newcommand{\DocumentTheme}{Analyse}
\newcommand{\DocumentType}{Cours}

% Preamble Section
\usepackage[a4paper, margin=1in]{geometry} % Sets the paper size to A4 and all margins to 1 inch
\usepackage[utf8]{inputenc}   % Allows for input of international characters
\usepackage[T1]{fontenc}      % Utilisation de l'encodage T1
\usepackage{lmodern}          % Support des polices avec le module 'french'
\usepackage[french]{babel}    % Ajout du support pour le français
\usepackage[fleqn]{amsmath}   % Imports the AMS math package for advanced math formatting ([fleqn] -> align alignat to the left)
\usepackage{amsfonts}         % Imports AMS fonts for math fonts
\usepackage{amssymb}          % Imports AMS symbols for math symbols
\usepackage{amsthm}           % Imports AMS theorems
\usepackage{fancyhdr}         % Import to custom the page footer
\usepackage{mdframed}         % Imports styles
\usepackage{enumerate}        % Imports styles for enumerate
\usepackage{multicol}
\usepackage{xcolor}           % Couleurs
\usepackage{tikz}             % Package général pour les graphiques
\usepackage{pgfplots}         % Compléments pour les graphiques
\usepackage{tkz-tab}          % Tableaux de variations
\usepackage{lastpage}         % Pour avoir le total de page dans le footer
\usepackage{makecell}         % Retour à la ligne dans une case d'un tabular
\usepackage{stmaryrd}         % Intervalles entières : \llbracket et \rrbracket

\pgfplotsset{compat=newest}   % Active les dernières fonctionnalités de pgfplots

% Enlève l'indentation des bloc de paragraphes et d'équation
\setlength{\parindent}{0pt}
\setlength{\mathindent}{0pt}

% ========== Title Section ==========

\title{\DocumentTitle} % Title of the document
\author{\DocumentTheme\space-\space\DocumentType} % Author's name
\date{} % Date

\pagestyle{fancy}
\fancyhf{} % Clear all header and footer fields

% ========== Footer Section ==========

\newcommand{\customfooter}{
    \fancyfoot[L]{\DocumentTheme\space-\space\DocumentType}
    \fancyfoot[C]{\DocumentTitle}
    \fancyfoot[R]{\thepage/\pageref{LastPage}}
    % Remove the line below the header and above the footer
    \renewcommand{\headrulewidth}{0pt}
    \renewcommand{\footrulewidth}{0pt}
}

% Apply the custom footer to the fancy style
\customfooter

% Apply the custom footer to the plain style (used on the first page)
\fancypagestyle{plain}{\customfooter}

% ========== Sections ==========

% Sections : I. / II. / III. ...
\renewcommand{\thesection}{\Roman{section}}
% Subsection : I.1 / I.2 / I.3 ...
\renewcommand{\thesubsection}{\thesection. \arabic{subsection}}
% Subsubsection : I.1.a / I.1.b / I.1.c ...
\renewcommand{\thesubsubsection}{\thesubsection. \alph{subsubsection}}

% ========== Colors (from Geogebra) ==========

\definecolor{green}{HTML}{006400}
\definecolor{red}{HTML}{CC0000}
\definecolor{blue}{HTML}{0000FF}
\definecolor{orange}{HTML}{FF5500}
\definecolor{purple}{HTML}{9933FF}
\definecolor{gray}{HTML}{666666}
\definecolor{brown}{HTML}{993300}

\newcommand{\green}[1]{\color{green}{#1}\color{black}}
\newcommand{\red}[1]{\color{red}{#1}\color{black}}
\newcommand{\blue}[1]{\color{blue}{#1}\color{black}}
\newcommand{\orange}[1]{\color{orange}{#1}\color{black}}
\newcommand{\purple}[1]{\color{purple}{#1}\color{black}}
\newcommand{\gray}[1]{\color{gray}{#1}\color{black}}
\newcommand{\brown}[1]{\color{brown}{#1}\color{black}}


% ========== Définition(s) ==========

% Style de la boîte de définition(s)
\mdfdefinestyle{DefinitionStyle}{
    leftmargin=0cm,
    rightmargin=0cm,
    linecolor=black,
    linewidth=2pt,
    topline=false,
    bottomline=false,
    rightline=false
}

% Définition de la commande '\definition'
\newcommand{\definition}[2]{%
    \begin{mdframed}[style=DefinitionStyle]
        \ifstrempty{#1}{% Teste si le premier argument est vide
            \textbf{Définition :}\ % Si vide, n'affiche pas de titre de théorème
        }{%
            \textbf{Définition #1 :}\ % Si non vide, affiche le titre avec le premier argument
        }\\#2
    \end{mdframed}
}

% Définition de la commande '\definitions'
\newcommand{\definitions}[2]{%
    \begin{mdframed}[style=DefinitionStyle]
        \ifstrempty{#1}{% Teste si le premier argument est vide
            \textbf{Définitions :}\ % Si vide, n'affiche pas de titre de théorème
        }{%
            \textbf{Définitions #1 :}\ % Si non vide, affiche le titre avec le premier argument
        }\\#2
    \end{mdframed}
}

% ========== Propriété(s) et Théorème(s) ==========

% Style de la boîte de propriété(s) / théorème(s)
\mdfdefinestyle{ProprieteStyle}{
    leftmargin=0cm,
    rightmargin=0cm,
    linecolor=black,
    linewidth=1pt
}

% Création d'un compteur pour les propriétés
\newcounter{propriete}

% Définition de la commande '\propriete'
\newcommand{\propriete}[2]{%
    \refstepcounter{propriete}% Incrémente le compteur de théorème
    \begin{mdframed}[style=ProprieteStyle]
        \ifstrempty{#1}{% Teste si le premier argument est vide
            \textbf{\thepropriete. Propriété :}\ % Si vide, n'affiche pas de titre de théorème
        }{%
            \textbf{\thepropriete. Propriété #1 :}\ % Si non vide, affiche le titre avec le premier argument
        }\\#2
    \end{mdframed}
}

% Définition de la commande 'proprietes'
\newcommand{\proprietes}[2]{%
    \refstepcounter{propriete}% Incrémente le compteur de théorème
    \begin{mdframed}[style=ProprieteStyle]
        \ifstrempty{#1}{% Teste si le premier argument est vide
            \textbf{\thepropriete. Propriétés :}\ % Si vide, n'affiche pas de titre de théorème
        }{%
            \textbf{\thepropriete. Propriétés #1 :}\ % Si non vide, affiche le titre avec le premier argument
        }\\#2
    \end{mdframed}
}



% Création d'un compteur pour les théorèmes
\newcounter{theoreme}

% Définition de la commande '\theoreme'
\newcommand{\theoreme}[2]{%
    \refstepcounter{theoreme}% Incrémente le compteur de théorème
    \begin{mdframed}[style=ProprieteStyle]
        \ifstrempty{#1}{% Teste si le premier argument est vide
            \textbf{\thetheoreme. Théorème :}\ % Si vide, n'affiche pas de titre de théorème
        }{%
            \textbf{\thetheoreme. Théorème #1 :}\ % Si non vide, affiche le titre avec le premier argument
        }\\#2
    \end{mdframed}
}

% Définition de la commande '\theoremes'
\newcommand{\theoremes}[2]{%
    \refstepcounter{theoreme}% Incrémente le compteur de théorème
    \begin{mdframed}[style=ProprieteStyle]
        \ifstrempty{#1}{% Teste si le premier argument est vide
            \textbf{\thetheoreme. Théorèmes :}\ % Si vide, n'affiche pas de titre de théorème
        }{%
            \textbf{\thetheoreme. Théorèmes #1 :}\ % Si non vide, affiche le titre avec le premier argument
        }\\#2
    \end{mdframed}
}

% ========== Exemple(s) ==========

% Définition de la commande 'exemple'
\newcommand{\exemple}[2]{
    \textbf{Exemple :} #1
    \begin{quote}
        #2
    \end{quote}
}

% Définition de la commande 'exemples'
\newcommand{\exemples}[2]{
    \textbf{Exemples :} #1
    \begin{quote}
        #2
    \end{quote}
}

% ========== Remarques ==========

% Définition de la commande 'remarque'
\newcommand{\remarque}[1]{
    \textbf{Remarque :} #1
}

% Définition de la commande 'remarques'
\newcommand{\remarques}[1]{
    \textbf{Remarques :} #1
}

% ========== Démonstration ==========

\newcommand{\demonstration}[2]{
    \textbf{Démonstration :} #1
    \begin{quote}
        #2
    \end{quote}
}

% ========== Notations ==========

% Ensembles
\newcommand{\C}{\mathbb{C}}
\newcommand{\R}{\mathbb{R}}
\newcommand{\Q}{\mathbb{Q}}
\newcommand{\D}{\mathbb{D}}
\newcommand{\Z}{\mathbb{Z}}
\newcommand{\N}{\mathbb{N}}

% Limites
\let\oldlim\lim
\renewcommand{\lim}[1]{\mathop{\displaystyle\oldlim}\limits_{#1}}

% Opérateurs
\newcommand{\x}{\times}
\newcommand{\equival}{\Leftrightarrow}

% Vecteurs
\renewcommand{\Vec}[1]{\overrightarrow{#1}}
\newcommand*{\norme}[1]{\left\lVert\vv{#1}\right\rVert}
\newcommand{\ps}[2]{\ensuremath{\vv{#1}.\vv{#2}}}

\newcommand{\vcoord}[2]{\begin{pmatrix} #1 \\ #2 \end{pmatrix}}
\newcommand{\Vcoord}[3]{\begin{pmatrix} #1 \\ #2 \\ #3 \end{pmatrix}}

% ========== Tableaux ==========
\newcommand{\boldhline}{\hline\noalign{\vskip 0pt}\hline}

\begin{document}

\maketitle

\section{Généralités sur la fonction exponentielle}

\definition{}{
	Il existe une unique fonction $f$ dérivable sur $\R$ telle que $f'=f$ et $f(0)=1$. Cette fonction est
	appelée fonction exponentielle et se note $\exp$. Ainsi, pour tout réel $x$, on a $\exp'(x)=\exp(x)$ et $\exp(0)=1$.
}

\subsection{Propriétés algébriques}

\propriete{(lemme)}{
	Pour tout réel $x$, on a $\exp(x)\not=0$.
}

\proprietes{}{
	Pour tous réels $x$ et $y$, on a :
	\begin{enumerate}[(i)]
		\item $\exp(x+y)=\exp(x)\times\exp(y)$ \quad (appelé relation fonctionnelle)
		\item $\exp(-x)=\dfrac{1}{\exp(x)}$
		\item $\exp(x-y)=\dfrac{\exp(x)}{\exp(y)}$
		\item $\exp(nx)=\exp(x)^n$
	\end{enumerate}
}

\subsection{La notation de l'exponentielle}

\definition{}{
	L'image de $1$ par la fonction $\exp$ est le nombre noté $e$, appelé constante d'Euler. Ainsi, $\exp(1)=e$.
}

\remarques{
	\begin{itemize}
		\item La fonction $\exp$ possède les mêmes propriété algébriques que les fonctions puissances. On notera
		      donc $\exp(x)=e^x$.
		\item $e \approx 2,71828182845904\dots$
	\end{itemize}
}

\proprietes{}{
	Pour tous réels $x$ et $y$, on a :
	\vspace{-10pt}
	\begin{multicols}{2}
		\begin{enumerate}[(i)]
			\item $e^{x+y}=e^x\times e^y$
			\item $e^{-x}=\dfrac{1}{e^x}$
			\item $e^{x-y}=\dfrac{e^x}{e^y}$
			\item $(e^x)^n=e^{nx}$
		\end{enumerate}
	\end{multicols}
	\vspace{5pt}
}

\newpage

\section{Étude et applications de la fonction exponentielle}

\subsection{Signe et variations de la fonction exponentielle}

\propriete{}{
	La fonction $\exp$ est strictement positive sur $\R$. Autrement dit : pour tout nombre réel $x$, $e^x>0$.
}

\propriete{}{
	La fonction $\exp$ est strictement croissante sur $\R$.
}

\subsection{Fonctions définies avec l'exponentielle}

\propriete{}{
Soit $a\in\R$. Soit la fonction $f(x)=e^{ax}$.
\begin{enumerate}[(i)]
	\item Pour $a>0$, la fonction $f$ est strictement croissante sur $\R$.
	\item Pour $a<0$, la fonction $f$ est strictement décroissante sur $\R$.
\end{enumerate}
}

\propriete{}{
Soient $a$ et $b$ deux réels. La fonction $f$ définie sur $\R$ par $f(x)=e^{ax+b}$ est dérivable sur $\R$ et pour
tout réel $x$, $f'(x)=a\times e^{ax+b}$.
}

\subsection{Équations et inéquations}

\propriete{}{
	Pour tous réels $a$ et $b$, on a :
	\begin{enumerate}[(i)]
		\item $e^a=e^b \Leftrightarrow a=b$
		\item $e^a<e^b \Leftrightarrow a<b$
		\item $e^a\leq e^b \Leftrightarrow a\leq b$
	\end{enumerate}
}


\section{Généralités sur la fonction logarithme népérien}

\propriete{(lemme)}{
	Pour tout réel $a>0$, il existe un unique réel $b$ tel que $a=e^b$.
}

\subsection{Définition et notation}

\definition{}{
	On appelle logarithme népérien d'un réel $a>0$, le nombre réel $b$ tel que $e^b=a$. On le note $\ln(a)=b$.
}

\remarques{
	\begin{itemize}
		\item $\ln(0)$ n'existe pas. En effet, $e^x\not=0$ pour tout $x\in\mathbb{R}$
		\item Pour tout entier $n\geq 2$, $\ln(n)$ n'est pas rationel.
	\end{itemize}
}

\exemples{}{
	\begin{itemize}
		\item $\ln(1) = 0$ (car $e^0=1$)
		\item $\ln(e) = 1$ (car $e^1=e$)
	\end{itemize}
}

\newpage

\subsection{Propriétés algébriques}

\propriete{}{
	Pour tout réel $x>0$, $y>0$ et pour tout entier relatif $n$ :
	\begin{enumerate}[(i)]
		\item $e^{\ln(x)}=x$
		\item $\ln(e^x)=x$
		\item $\ln(xy) = \ln(x) + \ln(y)$
		\item $\ln(\frac{x}{y}) = \ln(x) - \ln(y)$
		\item $\ln(x^n)=n\ln(x)$
	\end{enumerate}
}

\section{Propriétés graphiques}

\subsection{Symétrie des courbes représentatives}

\begin{minipage}{0.5\textwidth}
	Dans un repère orthonormé, on note $d$ la droite d'équation $x=y$.\\
	La symétrie axiale par rapport à la droite $d$ a pour effet d'échanger les abscisses et les ordonnées, c'est à dire
	qu'elle transforme tout point de coordonnées $(x;y)$ en un point de coordonnées $(x;y)$. \\

	\theoreme{}{
		Les courbes représentatives de la fonction $\exp$ et $\ln$ sont symétriques l'une de l'autre par rapport à la
		droite $d$.
	}

\end{minipage}
\hfill
\begin{minipage}{0.4\textwidth}
	\begin{tikzpicture}
		\begin{axis}[
				axis lines = middle,
				axis equal,
				samples=200,
				xlabel = $x$,
				ylabel = {$y$},
				xmin=-3.5, xmax=5.5,
				ymin=-2, ymax=4,
				xtick = {-3, -2,...,5},
				ytick = {-3, -2,...,5},
				grid = both,
				minor tick num = 1,
				major grid style = {lightgray},
				minor grid style = {lightgray!25}
			]
			\addplot[red, thick] {exp(x)};
			\addplot[blue, thick] {ln(x)};
			\addplot[green, thick] {x};
		\end{axis}
		\node at (1,2.6) {$\red{\exp(x)}$};
		\node at (4.6,4.8) {$\green{y=x}$};
		\node at (3.5,1) {$\blue{\ln(x)}$};
	\end{tikzpicture}
\end{minipage}

\subsection{Dérivation de la fonction logarithme}

\propriete{}{
	Si pour tout réel $x>0$, $f(x)=\ln(x)$ alors $f$ est dérivable, et pour tout $x>0$ : $f'(x)=\dfrac{1}{x}$
}

\proprietes{(corrolaires)}{
	\vspace{-10pt}
	\begin{itemize}
		\item La fonction $\ln(x)$ est strictement croissante sur $]0;+\infty[$.
		\item Pour tous réels $a$ et $b$ strictements positifs, $\ln(a)<\ln(b)\Leftrightarrow a<b$.
	\end{itemize}
}

\section{Compléments sur le logarithme décimal}

Soit $n$ un entier et en posant $x=10^n$, on a $\ln(x)=n\times\ln(10)$ donc $n=\dfrac{\ln(x)}{\ln(10)}$.

Le nombre $n$ est appelé le logarithme décimal de $x$ noté $\log_10(x)$ ou plus simplement $\log(x)$. On peut
généraliser cette définition pour tout réel strictement positif :

\definition{}{
	Pour tout réel $a>0$, on définit le logarithme décimal de a par $\log(a)=\dfrac{\ln(a)}{\ln(10)}$.
}

\newpage

\proprietes{}{
	Pour tous réels $x>0$, $y>0$ et pour tout entier relatif $n$, on a :
	\vspace{-10pt}
	\begin{multicols}{2}
		\begin{enumerate}[(i)]
			\item $\log(10^n)=n$
			\item $\log(x \times y)=\log(x)+\log(y)$
			\item $\log\left(\dfrac{x}{y}\right)=\log(x)-\log(y)$
			\item $\log(x^n)=n\times\log(n)$
		\end{enumerate}
	\end{multicols}
	\vspace{5pt}
	}

\remarque{Tout nombre réel $x$ peut s'écrire sous la forme $x=a\times10^n$ avec $a\in[1;10[$ (c'est ce qu'on appelle
l'écriture scientifique de $x$). Dans ce cas, l'entier $n$ est égal à la partie entière de $\log(x)$.}

\exemple{}{
	Soit $x=123,4=1,234\times10^2$. La partie entière de $\log(x)$ est égal à $2$, c'est à dire que $2\leq\log(x)<3$.
}


\end{document}