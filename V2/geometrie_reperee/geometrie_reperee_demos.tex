\documentclass[10pt]{article}

\newcommand{\DocumentTitle}{Géométrie repérée}
\newcommand{\DocumentTheme}{Géométrie}
\newcommand{\DocumentType}{Démonstrations}

% Preamble Section
\usepackage[a4paper, margin=1in]{geometry} % Sets the paper size to A4 and all margins to 1 inch
\usepackage[utf8]{inputenc}   % Allows for input of international characters
\usepackage[T1]{fontenc}      % Utilisation de l'encodage T1
\usepackage{lmodern}          % Support des polices avec le module 'french'
\usepackage[french]{babel}    % Ajout du support pour le français
\usepackage[fleqn]{amsmath}   % Imports the AMS math package for advanced math formatting ([fleqn] -> align alignat to the left)
\usepackage{amsfonts}         % Imports AMS fonts for math fonts
\usepackage{amssymb}          % Imports AMS symbols for math symbols
\usepackage{amsthm}           % Imports AMS theorems
\usepackage{fancyhdr}         % Import to custom the page footer
\usepackage{mdframed}         % Imports styles
\usepackage{enumerate}        % Imports styles for enumerate
\usepackage{multicol}
\usepackage{xcolor}           % Couleurs
\usepackage{tikz}             % Package général pour les graphiques
\usepackage{pgfplots}         % Compléments pour les graphiques
\usepackage{tkz-tab}          % Tableaux de variations
\usepackage{lastpage}         % Pour avoir le total de page dans le footer
\usepackage{makecell}         % Retour à la ligne dans une case d'un tabular
\usepackage{stmaryrd}         % Intervalles entières : \llbracket et \rrbracket

\pgfplotsset{compat=newest}   % Active les dernières fonctionnalités de pgfplots

% Enlève l'indentation des bloc de paragraphes et d'équation
\setlength{\parindent}{0pt}
\setlength{\mathindent}{0pt}

% ========== Title Section ==========

\title{\DocumentTitle} % Title of the document
\author{\DocumentTheme\space-\space\DocumentType} % Author's name
\date{} % Date

\pagestyle{fancy}
\fancyhf{} % Clear all header and footer fields

% ========== Footer Section ==========

\newcommand{\customfooter}{
    \fancyfoot[L]{\DocumentTheme\space-\space\DocumentType}
    \fancyfoot[C]{\DocumentTitle}
    \fancyfoot[R]{\thepage/\pageref{LastPage}}
    % Remove the line below the header and above the footer
    \renewcommand{\headrulewidth}{0pt}
    \renewcommand{\footrulewidth}{0pt}
}

% Apply the custom footer to the fancy style
\customfooter

% Apply the custom footer to the plain style (used on the first page)
\fancypagestyle{plain}{\customfooter}

% ========== Sections ==========

% Sections : I. / II. / III. ...
\renewcommand{\thesection}{\Roman{section}}
% Subsection : I.1 / I.2 / I.3 ...
\renewcommand{\thesubsection}{\thesection. \arabic{subsection}}
% Subsubsection : I.1.a / I.1.b / I.1.c ...
\renewcommand{\thesubsubsection}{\thesubsection. \alph{subsubsection}}

% ========== Colors (from Geogebra) ==========

\definecolor{green}{HTML}{006400}
\definecolor{red}{HTML}{CC0000}
\definecolor{blue}{HTML}{0000FF}
\definecolor{orange}{HTML}{FF5500}
\definecolor{purple}{HTML}{9933FF}
\definecolor{gray}{HTML}{666666}
\definecolor{brown}{HTML}{993300}

\newcommand{\green}[1]{\color{green}{#1}\color{black}}
\newcommand{\red}[1]{\color{red}{#1}\color{black}}
\newcommand{\blue}[1]{\color{blue}{#1}\color{black}}
\newcommand{\orange}[1]{\color{orange}{#1}\color{black}}
\newcommand{\purple}[1]{\color{purple}{#1}\color{black}}
\newcommand{\gray}[1]{\color{gray}{#1}\color{black}}
\newcommand{\brown}[1]{\color{brown}{#1}\color{black}}


% ========== Définition(s) ==========

% Style de la boîte de définition(s)
\mdfdefinestyle{DefinitionStyle}{
    leftmargin=0cm,
    rightmargin=0cm,
    linecolor=black,
    linewidth=2pt,
    topline=false,
    bottomline=false,
    rightline=false
}

% Définition de la commande '\definition'
\newcommand{\definition}[2]{%
    \begin{mdframed}[style=DefinitionStyle]
        \ifstrempty{#1}{% Teste si le premier argument est vide
            \textbf{Définition :}\ % Si vide, n'affiche pas de titre de théorème
        }{%
            \textbf{Définition #1 :}\ % Si non vide, affiche le titre avec le premier argument
        }\\#2
    \end{mdframed}
}

% Définition de la commande '\definitions'
\newcommand{\definitions}[2]{%
    \begin{mdframed}[style=DefinitionStyle]
        \ifstrempty{#1}{% Teste si le premier argument est vide
            \textbf{Définitions :}\ % Si vide, n'affiche pas de titre de théorème
        }{%
            \textbf{Définitions #1 :}\ % Si non vide, affiche le titre avec le premier argument
        }\\#2
    \end{mdframed}
}

% ========== Propriété(s) et Théorème(s) ==========

% Style de la boîte de propriété(s) / théorème(s)
\mdfdefinestyle{ProprieteStyle}{
    leftmargin=0cm,
    rightmargin=0cm,
    linecolor=black,
    linewidth=1pt
}

% Création d'un compteur pour les propriétés
\newcounter{propriete}

% Définition de la commande '\propriete'
\newcommand{\propriete}[2]{%
    \refstepcounter{propriete}% Incrémente le compteur de théorème
    \begin{mdframed}[style=ProprieteStyle]
        \ifstrempty{#1}{% Teste si le premier argument est vide
            \textbf{\thepropriete. Propriété :}\ % Si vide, n'affiche pas de titre de théorème
        }{%
            \textbf{\thepropriete. Propriété #1 :}\ % Si non vide, affiche le titre avec le premier argument
        }\\#2
    \end{mdframed}
}

% Définition de la commande 'proprietes'
\newcommand{\proprietes}[2]{%
    \refstepcounter{propriete}% Incrémente le compteur de théorème
    \begin{mdframed}[style=ProprieteStyle]
        \ifstrempty{#1}{% Teste si le premier argument est vide
            \textbf{\thepropriete. Propriétés :}\ % Si vide, n'affiche pas de titre de théorème
        }{%
            \textbf{\thepropriete. Propriétés #1 :}\ % Si non vide, affiche le titre avec le premier argument
        }\\#2
    \end{mdframed}
}



% Création d'un compteur pour les théorèmes
\newcounter{theoreme}

% Définition de la commande '\theoreme'
\newcommand{\theoreme}[2]{%
    \refstepcounter{theoreme}% Incrémente le compteur de théorème
    \begin{mdframed}[style=ProprieteStyle]
        \ifstrempty{#1}{% Teste si le premier argument est vide
            \textbf{\thetheoreme. Théorème :}\ % Si vide, n'affiche pas de titre de théorème
        }{%
            \textbf{\thetheoreme. Théorème #1 :}\ % Si non vide, affiche le titre avec le premier argument
        }\\#2
    \end{mdframed}
}

% Définition de la commande '\theoremes'
\newcommand{\theoremes}[2]{%
    \refstepcounter{theoreme}% Incrémente le compteur de théorème
    \begin{mdframed}[style=ProprieteStyle]
        \ifstrempty{#1}{% Teste si le premier argument est vide
            \textbf{\thetheoreme. Théorèmes :}\ % Si vide, n'affiche pas de titre de théorème
        }{%
            \textbf{\thetheoreme. Théorèmes #1 :}\ % Si non vide, affiche le titre avec le premier argument
        }\\#2
    \end{mdframed}
}

% ========== Exemple(s) ==========

% Définition de la commande 'exemple'
\newcommand{\exemple}[2]{
    \textbf{Exemple :} #1
    \begin{quote}
        #2
    \end{quote}
}

% Définition de la commande 'exemples'
\newcommand{\exemples}[2]{
    \textbf{Exemples :} #1
    \begin{quote}
        #2
    \end{quote}
}

% ========== Remarques ==========

% Définition de la commande 'remarque'
\newcommand{\remarque}[1]{
    \textbf{Remarque :} #1
}

% Définition de la commande 'remarques'
\newcommand{\remarques}[1]{
    \textbf{Remarques :} #1
}

% ========== Démonstration ==========

\newcommand{\demonstration}[2]{
    \textbf{Démonstration :} #1
    \begin{quote}
        #2
    \end{quote}
}

% ========== Notations ==========

% Ensembles
\newcommand{\C}{\mathbb{C}}
\newcommand{\R}{\mathbb{R}}
\newcommand{\Q}{\mathbb{Q}}
\newcommand{\D}{\mathbb{D}}
\newcommand{\Z}{\mathbb{Z}}
\newcommand{\N}{\mathbb{N}}

% Limites
\let\oldlim\lim
\renewcommand{\lim}[1]{\mathop{\displaystyle\oldlim}\limits_{#1}}

% Opérateurs
\newcommand{\x}{\times}
\newcommand{\equival}{\Leftrightarrow}

% Vecteurs
\renewcommand{\Vec}[1]{\overrightarrow{#1}}
\newcommand*{\norme}[1]{\left\lVert\vv{#1}\right\rVert}
\newcommand{\ps}[2]{\ensuremath{\vv{#1}.\vv{#2}}}

\newcommand{\vcoord}[2]{\begin{pmatrix} #1 \\ #2 \end{pmatrix}}
\newcommand{\Vcoord}[3]{\begin{pmatrix} #1 \\ #2 \\ #3 \end{pmatrix}}

% ========== Tableaux ==========
\newcommand{\boldhline}{\hline\noalign{\vskip 0pt}\hline}

% ----- DEBUT DU DOCUMENT -----
\begin{document}

% Style et numérotation
\maketitle

\demonstration{(Propriété 1)}{
  Soit $(d)$ la droite passant par un point $A(x_A;y_A)$ et dont on connait le vecteur normal $\vec{n}\vcoord{a}{b}$.
  \vspace{-10pt}
  \begin{align*}
    M(x;y)\in (d) & \equival \Vec{AM}\vcoord{x-x_A}{y-y_A} \text{ et } \vec{n}\vcoord{a}{b} \text{ sont orthogonaux} \\
                  & \equival \Vec{AM}\pdt\vec{n} = 0                                                                 \\
                  & \equival a(x-x_A)+b(y-y_B)=0                                                                     \\
                  & \equival ax+bx\underbrace{-ax_A-by_A}_{=c}=0
  \end{align*}
}

\demonstration{(Propriété 2)}{
  \vspace{-10pt}
  \begin{align*}
    \text{Soit } M(x;y) \in C & \equival\Omega M=R                     &  &                                                     \\
                              & \equival\Omega M^2=R^2                 &  & \textit{ avec } \Omega M=\sqrt{(x-x_0)^2+(y-y_0)^2} \\
                              & \equival\Omega (x-x_0)^2+(y-y_0)^2=R^2 &  &
  \end{align*}
}

\demonstration{(Propriété 3)}{
  \vspace{-10pt}
  \begin{align*}
    M(x;y)\in C & \equival\text{le triangle $MAB$ est rectangle en $M$} &  &                                                                                       \\
                & \equival\Vec{AM}\pdt\Vec{BM}=0                        &  & \text{ avec } \Vec{AM}\vcoord{x-x_A}{y-y_A} \text{ et } \Vec{BM}\vcoord{x-x_B}{y-y_B} \\
                & \equival (x-x_A)(x-x_B)+(y-y_A)(y-y_B)=0              &  &
  \end{align*}

}


\demonstration{(Propriété 4)}{
  \begin{itemize}
    \item On va montrer que pour tout point $A$ appartenant à $P$ et distinct de son sommet, il existe un point $B$ distinct de $A$ appartenant à $P$ et ayant la même ordonnée que $A$. \\
          Soient deux points $A(x_A;y_A)$ et $B(x_B;y_B)$ appartenant à $P$.\\

          On a donc $y_A=ax{_A}^2+bx_A+c$ et $y_B=ax{_B}^2+bx_B+c$.
          \begin{equation*}
            \begin{split}
              y_A=y_B&\equival ax{_A}^2+bx_A+c = ax{_B}^2+bx_B+c \\
              &\equival a(x{_A}^2-x{_B}^2) + b(x_A-x_B)=0 \\
              &\equival (x{_A}-x{_B})\left[ a(x_A-x_B) +b \right]=0 \\
              &\equival (x{_A}-x{_B})=0 \text{ ou }  a(x_A-x_B) +b=0 \\
              &\equival (x{_A}=x{_B}) \text{ ou }  x_A+x_B=\frac{-b}{a}
            \end{split}
          \end{equation*}
          Donc, si un point $A(x_A;y_A)$ distinct du sommet de la parabole appartient à $P$, le point $ B\left(-\dfrac{b}{a}-x_A;y_A\right)$ appartient à $P$, est distinct de $A$ et a la même ordonnée que $A$. \\
          La courbe admet donc bien deux points distincts d'ordonnée $y_A$.

\newpage

    \item On va déterminer les coordonnées du milieur $I$ de $[AB]$. \\
          Les points $A$ et $B$ ont la même ordonnée donc $ y_I=\dfrac{y_A+y_B}{2}=\dfrac{2y_A}{2}=y_A$. \\
          De plus, d'après ce qui précède, $ x_A+x_B=-\dfrac{b}{a}$, donc $\dfrac{x_A+x_B}{2}=-\dfrac{b}{2a}$, et donc $ x_I=\dfrac{x_A+x_B}{2}=-\dfrac{b}{2a}$. \\
          On en déduit donc que $ I\left(-\dfrac{b}{2a};y_A\right)$.

    \item Soit $\Delta$ la droite d'équation $ x=-\dfrac{b}{2a}$. Les points $A$ et $B$ ont la même ordonnée, donc le vecteur $\Vec{AB}$ a pour coordonnées $\vcoord{x_B-x_A}{0}$.
          Un vecteur directeur de $\Delta$ est $\vec{j}\vcoord{0}{1}$. Or $\Vec{AB}\pdt\vec{j}=(x_B-x_A)\times0+0\times1=0$.
          Les vecteurs sont donc orthogonaux, donc la droite $\Delta$ est orthogonale au segment $[AB]$.
          De plus, le point $I$, milieur de $[AB]$, appartient à $\Delta$. Donc $\Delta$ est la médiatrice du segment $[AB]$.
          $B$ est donc le symétrique de $A$ par rapport à $\Delta$.

    \item Si $A$ est le point de la courbe d'abscisse $-\dfrac{b}{2a}$, alors $A$ appartient à $\Delta$, c'est le sommet de la parabole.
          Donc $A$ invariant par symétrie par rapport à $\Delta$. On a alors $x_A=x_B$. \\
          Tout point de la parabole admet ainsi un symétrique par rapport à $\Delta$, ce qui signifie que $\Delta$ est axe de symétrie de la parabole.
  \end{itemize}
}



\end{document}
% ----- FIN DU DOCUMENT -----