\documentclass[10pt]{article}

\newcommand{\DocumentTitle}{Logique}
\newcommand{\DocumentTheme}{Séminaire}
\newcommand{\DocumentType}{Cours}

% Preamble Section
\usepackage[a4paper, margin=1in]{geometry} % Sets the paper size to A4 and all margins to 1 inch
\usepackage[utf8]{inputenc}   % Allows for input of international characters
\usepackage[T1]{fontenc}      % Utilisation de l'encodage T1
\usepackage{lmodern}          % Support des polices avec le module 'french'
\usepackage[french]{babel}    % Ajout du support pour le français
\usepackage[fleqn]{amsmath}   % Imports the AMS math package for advanced math formatting ([fleqn] -> align alignat to the left)
\usepackage{amsfonts}         % Imports AMS fonts for math fonts
\usepackage{amssymb}          % Imports AMS symbols for math symbols
\usepackage{amsthm}           % Imports AMS theorems
\usepackage{yhmath}           % Used for "\wideparen{}" 
\usepackage{fancyhdr}         % Import to custom the page footer
\usepackage{mdframed}         % Imports styles
\usepackage{enumerate}        % Imports styles for enumerate
\usepackage{multicol}
\usepackage{xcolor}           % Couleurs
\usepackage{tikz}             % Package général pour les graphiques
\usetikzlibrary{calc}
\usepackage{pgfplots}         % Compléments pour les graphiques
\usepackage{tkz-tab}          % Tableaux de variations
\usepackage{lastpage}         % Pour avoir le total de page dans le footer
\usepackage{makecell}         % Retour à la ligne dans une case d'un tabular
\usepackage{stmaryrd}         % Intervalles entières : \llbracket et \rrbracket
\usepackage{cancel}           % Permet de barrer des termes
\usepackage{forest}           % Permet de créer des arbres pondérés
\usepackage{tabularx}         % Permet de générer des tableaux (ex: cours fonction dérivées et application)
\usepackage{multirow}         % Permet de faire plusieurs lignes dans des cases de talbeaux


\pgfplotsset{compat=newest}   % Active les dernières fonctionnalités de pgfplots

% Enlève l'indentation des bloc de paragraphes et d'équation
\setlength{\parindent}{0pt}
\setlength{\mathindent}{0pt}

% ========== Title Section ==========

\title{\DocumentTitle} % Title of the document
\author{\DocumentTheme\space-\space\DocumentType} % Author's name
\date{} % Date

\pagestyle{fancy}
\fancyhf{} % Clear all header and footer fields

% ========== Footer Section ==========

\newcommand{\customfooter}{
    \fancyfoot[L]{\DocumentTheme\space-\space\DocumentType}
    \fancyfoot[C]{\DocumentTitle}
    \fancyfoot[R]{\thepage/\pageref{LastPage}}
    % Remove the line below the header and above the footer
    \renewcommand{\headrulewidth}{0pt}
    \renewcommand{\footrulewidth}{0pt}
}

% Apply the custom footer to the fancy style
\customfooter

% Apply the custom footer to the plain style (used on the first page)
\fancypagestyle{plain}{\customfooter}

% ========== Sections ==========

% Sections : I. / II. / III. ...
\renewcommand{\thesection}{\Roman{section}}
% Subsection : I.1 / I.2 / I.3 ...
\renewcommand{\thesubsection}{\thesection. \arabic{subsection}}
% Subsubsection : I.1.a / I.1.b / I.1.c ...
\renewcommand{\thesubsubsection}{\thesubsection. \alph{subsubsection}}

% ========== Colors (from Geogebra) ==========

\definecolor{green}{HTML}{006400}
\definecolor{red}{HTML}{CC0000}
\definecolor{blue}{HTML}{0000FF}
\definecolor{orange}{HTML}{FF5500}
\definecolor{purple}{HTML}{9933FF}
\definecolor{gray}{HTML}{666666}
\definecolor{brown}{HTML}{993300}
\definecolor{black}{HTML}{000000}
\definecolor{white}{HTML}{FFFFFF}

\newcommand{\green}[1]{\color{green}{#1}\color{black}}
\newcommand{\red}[1]{\color{red}{#1}\color{black}}
\newcommand{\blue}[1]{\color{blue}{#1}\color{black}}
\newcommand{\orange}[1]{\color{orange}{#1}\color{black}}
\newcommand{\purple}[1]{\color{purple}{#1}\color{black}}
\newcommand{\gray}[1]{\color{gray}{#1}\color{black}}
\newcommand{\brown}[1]{\color{brown}{#1}\color{black}}


% ========== Définition(s) ==========

% Style de la boîte de définition(s)
\mdfdefinestyle{DefinitionStyle}{
    leftmargin=0cm,
    rightmargin=0cm,
    linecolor=black,
    linewidth=2pt,
    topline=false,
    bottomline=false,
    rightline=false
}

% Définition de la commande '\definition'
\newcommand{\definition}[2]{%
    \begin{mdframed}[style=DefinitionStyle]
        \ifstrempty{#1}{% Teste si le premier argument est vide
            \textbf{Définition :}\ % Si vide, n'affiche pas de titre de théorème
        }{%
            \textbf{Définition #1 :}\ % Si non vide, affiche le titre avec le premier argument
        }\\#2
    \end{mdframed}
}

% Définition de la commande '\definitions'
\newcommand{\definitions}[2]{%
    \begin{mdframed}[style=DefinitionStyle]
        \ifstrempty{#1}{% Teste si le premier argument est vide
            \textbf{Définitions :}\ % Si vide, n'affiche pas de titre de théorème
        }{%
            \textbf{Définitions #1 :}\ % Si non vide, affiche le titre avec le premier argument
        }\\#2
    \end{mdframed}
}

% ========== Propriété(s) et Théorème(s) ==========

% Style de la boîte de propriété(s) / théorème(s)
\mdfdefinestyle{ProprieteStyle}{
    leftmargin=0cm,
    rightmargin=0cm,
    linecolor=black,
    linewidth=1pt
}

% Création d'un compteur pour les propriétés
\newcounter{propriete}

% Définition de la commande '\propriete'
\newcommand{\propriete}[2]{%
    \refstepcounter{propriete}% Incrémente le compteur de théorème
    \begin{mdframed}[style=ProprieteStyle]
        \ifstrempty{#1}{% Teste si le premier argument est vide
            \textbf{\thepropriete. Propriété :}\ % Si vide, n'affiche pas de titre de théorème
        }{%
            \textbf{\thepropriete. Propriété #1 :}\ % Si non vide, affiche le titre avec le premier argument
        }\\#2
    \end{mdframed}
}

% Définition de la commande 'proprietes'
\newcommand{\proprietes}[2]{%
    \refstepcounter{propriete}% Incrémente le compteur de théorème
    \begin{mdframed}[style=ProprieteStyle]
        \ifstrempty{#1}{% Teste si le premier argument est vide
            \textbf{\thepropriete. Propriétés :}\ % Si vide, n'affiche pas de titre de théorème
        }{%
            \textbf{\thepropriete. Propriétés #1 :}\ % Si non vide, affiche le titre avec le premier argument
        }\\#2
    \end{mdframed}
}

% Création d'un compteur pour les théorèmes
\newcounter{theoreme}

% Définition de la commande '\theoreme'
\newcommand{\theoreme}[2]{%
    \refstepcounter{theoreme}% Incrémente le compteur de théorème
    \begin{mdframed}[style=ProprieteStyle]
        \ifstrempty{#1}{% Teste si le premier argument est vide
            \textbf{\thetheoreme. Théorème :}\ % Si vide, n'affiche pas de titre de théorème
        }{%
            \textbf{\thetheoreme. Théorème #1 :}\ % Si non vide, affiche le titre avec le premier argument
        }\\#2
    \end{mdframed}
}

% Définition de la commande '\theoremes'
\newcommand{\theoremes}[2]{%
    \refstepcounter{theoreme}% Incrémente le compteur de théorème
    \begin{mdframed}[style=ProprieteStyle]
        \ifstrempty{#1}{% Teste si le premier argument est vide
            \textbf{\thetheoreme. Théorèmes :}\ % Si vide, n'affiche pas de titre de théorème
        }{%
            \textbf{\thetheoreme. Théorèmes #1 :}\ % Si non vide, affiche le titre avec le premier argument
        }\\#2
    \end{mdframed}
}

\newcommand{\corrolaire}[2]{%
    \begin{mdframed}[style=ProprieteStyle]
        \ifstrempty{#1}{% Teste si le premier argument est vide
            \textbf{Corrolaire :}\ % Si vide, n'affiche pas de titre de théorème
        }{%
            \textbf{Corrolaire #1 :}\ % Si non vide, affiche le titre avec le premier argument
        }\\#2
    \end{mdframed}
}

% ========== Exemple(s) ==========

% Définition de la commande 'exemple'
\newcommand{\exemple}[2]{
    \textbf{Exemple :} #1
    \begin{quote}
        #2
    \end{quote}
}

% Définition de la commande 'exemples'
\newcommand{\exemples}[2]{
    \textbf{Exemples :} #1
    \begin{quote}
        #2
    \end{quote}
}

% ========== Remarques ==========

% Définition de la commande 'remarque'
\newcommand{\remarque}[1]{
    \textbf{Remarque :} #1
}

% Définition de la commande 'remarques'
\newcommand{\remarques}[1]{
    \textbf{Remarques :}
    \begin{quote}
        #1
    \end{quote}
}

% ========== Démonstration ==========

\newcommand{\demonstration}[2]{
    \ifstrempty{#1}{% Teste si le premier argument est vide
        \textbf{Démonstration :}\ % Si vide, n'affiche pas de titre de théorème
    }{%
        \textbf{Démonstration #1 :}\ % Si non vide, affiche le titre avec le premier argument
    }
    \begin{quote}
        #2
    \end{quote}
}
% ========== Autres blocs ==========

\newcommand{\newbloc}[2]{
    \textbf{#1}
    \begin{quote}
        #2
    \end{quote}
}

% ========== Notations ==========

% Style
\newcommand{\ds}{\displaystyle}
\newcommand{\llb}{\llbracket}
\newcommand{\rrb}{\rrbracket}

% Ensembles
\newcommand{\C}{\mathbb{C}}
\newcommand{\R}{\mathbb{R}}
\newcommand{\Q}{\mathbb{Q}}
\newcommand{\D}{\mathbb{D}}
\newcommand{\Z}{\mathbb{Z}}
\newcommand{\N}{\mathbb{N}}

% Limites
\let\oldlim\lim
\renewcommand{\lim}[1]{\mathop{\displaystyle\oldlim}\limits_{#1}}

% Opérateurs
\newcommand{\x}{\times}
\newcommand{\equival}{\Leftrightarrow}
\newcommand{\involve}{\Rightarrow}

% Ensembles
\renewcommand{\bar}[1]{\overline{#1}}
\renewcommand{\part}[1]{\mathcal{P}({#1})}  % Ensemle des parties de E
\newcommand{\rel}{\mathcal{R}} % Relation entre deux ensembles

% Vecteurs
\renewcommand{\Vec}[1]{\overrightarrow{#1}}
\newcommand*{\norme}[1]{\|#1\|}
\newcommand{\pdt}{\mathbin{\vcenter{\hbox{\scalebox{0.6}{\textbullet}}}}}
\newcommand{\vcoord}[2]{\begin{pmatrix} #1 \\ #2 \end{pmatrix}}
\newcommand{\Vcoord}[3]{\begin{pmatrix} #1 \\ #2 \\ #3 \end{pmatrix}}

% Fonctions
\newcommand{\aire}[1]{\text{aire}\left( #1 \right)}
\newcommand{\Int}{\ds\int}


% ========== Systèmes ==========
\newcommand{\sys}[2]{\begin{cases}#1\\#2\end{cases}}
\newcommand{\Sys}[3]{\begin{cases}#1\\#2\\#3\end{cases}}

% ========== Tableaux ==========
\newcommand{\boldhline}{\hline\noalign{\vskip 0pt}\hline}

% ========== Graphiques ==========

% Triange de Pascal
\newcommand{\binomial}[2]{\pgfmathparse{int((#1)!/((#2)!*((#1)-(#2))!))}\pgfmathresult}
\newcommand{\Pascal}[1]{
    \begin{tikzpicture}[scale=1, every node/.style={scale=1.2}]
        \foreach \n in {0,...,#1} {
                \foreach \k in {0,...,\n} {
                        \node at (\k-\n/2,-\n) {$\binomial{\n}{\k}$};
                        \draw (\k-\n/2,-\n) +(-0.5,-0.5) rectangle ++(0.5,0.5);
                    }
            }
    \end{tikzpicture}
}

\newcommand{\Pascalbinome}[1]{
    \begin{tikzpicture}[scale=1, every node/.style={scale=0.9}]
        \foreach \n in {0,...,#1} {
                \foreach \k in {0,...,\n} {
                        \pgfmathtruncatemacro{\symk}{min(\k,\n-\k)}
                        \node at (\k-\n/2,-\n) {$\ds\binom{\n}{\symk}$};
                        \draw (\k-\n/2,-\n) +(-0.5,-0.5) rectangle ++(0.5,0.5);
                    }
            }
    \end{tikzpicture}
}

\begin{document}

\maketitle

\section{Notions de logique}

\subsection{Assertion, prédicat et négation}

\definition{(assertion)}{
    Une assertion ou proposition est une phrase qui est ou bien vraie ou bien fausse. On dira que cette assertion a deux
    valeurs de vérité.
}

\remarque{On appelle cette logique « logique du tiers exclu » puisqu'on exclu les phrases avec deux valeurs de vérité.}

\exemples{}{
    \begin{itemize}
        \item « $2$ est un nombre impaire » $\rightarrow$ FAUX.
        \item « $3$ est un nombre premier » $\rightarrow$ VRAI.
        \item « $n$ est un entier pair » $\rightarrow$ PROBLEME : la valeur de vérité dépend de $n$.
    \end{itemize}
}

\definition{(prédicat)}{
    Une assertion prenant en compte un paramètre $n$ est appelé un prédicat. On le note $P(n)$.
}

\begin{minipage}{0.5\textwidth}
    \definition{(négation)}{
        Si $P$ est une assertion, on note $\neg P$ (ou non $P$) la négation de $P$.
    }
\end{minipage}
\hfill
\begin{minipage}{0.4\textwidth}
    \[
        \begin{array}{|c|c|}
            \hline
            P    & \neg P \\
            \hline
            Vrai & Faux   \\
            \hline
            Faux & Vrai   \\
            \hline
        \end{array}
    \]
\end{minipage}

\vspace{10pt}

\remarque{Au lieu d'écrire « $P$ est vraie », on écrit simplement « $P$ ».}

\subsection{Connecteurs logiques}

À partir de deux propostions, on peut en former une troisième. \\

\begin{minipage}{0.7\textwidth}
    \definitions{(conjonction et disjonction)}{
        Soit $P$ et $Q$ deux propositions. Alors :
        \begin{enumerate}[(i)]
            \item la proposition $P \land Q$ est vraie ssi $P$ est vraie ET $Q$ est vraie.
            \item la proposition $P \lor Q$ est vraie ssi l'une des deux proposition (ou les deux) est vraie.
        \end{enumerate}
    }
\end{minipage}
\hfill
\begin{minipage}{0.25\textwidth}
    \[
        \begin{array}{|c|c|c|c|}
            \hline
            P & Q & P \land Q & P \lor Q \\
            \hline
            V & V & V         & V        \\
            \hline
            V & F & F         & V        \\
            \hline
            F & V & F         & V        \\
            \hline
            F & F & F         & F        \\
            \hline
        \end{array}
    \]
\end{minipage}

\vspace{10pt}

\begin{minipage}{0.7\textwidth}
    \definitions{(implication et équivalence)}{
        Soient $P$ et $Q$ deux propositions.
        \begin{enumerate}[(i)]
            \item L'implication « $P\involve Q$ » (on lit « $P$ implication $Q$ ») est fausse lorsque $P$ est vraie et $Q$
                  est fausse.
            \item L'équivalence « $P\equival Q$ » (on lit « $P$ équivalence $Q$ ») est vraie lorsque $P$ et $Q$ ont la même
                  valeur de vérité.
        \end{enumerate}
    }
\end{minipage}
\hfill
\begin{minipage}{0.25\textwidth}
    \[
        \begin{array}{|c|c|c|c|}
            \hline
            P & Q & P\involve Q & P\equival Q \\
            \hline
            V & V & V           & V           \\
            \hline
            V & F & F           & F           \\
            \hline
            F & V & V           & F           \\
            \hline
            F & F & V           & V           \\
            \hline
        \end{array}
    \]
\end{minipage}

\newpage

\exemples{}{
    \begin{itemize}
        \item « 2 est pair $\involve$ 3 est pair » $\rightarrow$ Faux
        \item « 3 est pair $\involve$ 2 est pair » $\rightarrow$ Vrai
    \end{itemize}
}

\definitions{(implique, équivaut)}{
    Soient $P$ et $Q$ deux propositions.
    \begin{enumerate}[(i)]
        \item Lorsque $P\involve Q$ est vraie, on dit « $P$ implique $Q$ » ou « si $P$ alors $Q$ ».
        \item Lorsque $P\equival Q$ est vraie, on dit « $P$ équivaut à $Q$ » ou « $P$ si et seulement si $Q$ ».
    \end{enumerate}
}

\theoremes{}{
    Soient $P$ et $Q$ deux propositions.
    \begin{enumerate}[(i)]
        \item $\neg(\neg P) \equival P$
        \item $\neg(P\land Q) \equival \neg P \lor\neg Q$ (1$^\text{ère}$ loi de Morgan)
        \item $\neg(P\lor Q) \equival \neg P \land\neg Q$ (2$^\text{eme}$ loi de Morgan)
        \item $P\involve Q \equival \neg P \lor Q$
        \item $\neg(P \involve Q) \equival P\land\neg Q$
        \item $P\equival Q \equival (P\involve Q)\land(Q \involve P)$
        \item $P\equival Q \equival \neg P \equival \neg Q$
    \end{enumerate}
}


\definition{(réciproque et contraposée)}{
    Soient $P$ et $Q$ deux propositions.
    \begin{enumerate}[(i)]
        \item La réciproque de l'implication $P\involve Q$ est l'implication $Q\involve P$.
        \item La contraposée de l'implication $P\involve Q$ est l'implication $\neg Q\involve\neg P$.
    \end{enumerate}
}

\exemple{Pythagore}{
    \begin{itemize}
        \item Théorème : si $ABC$ est un triangle rectangle en $A$ alors $AB^2+AC^2=BC^2$.
        \item Réciproque :  si $AB^2+AC^2=BC^2$ alors $ABC$ est un triangle rectangle en $A$.
        \item Contraposée : si $AB^2+AC^2\not=BC^2$ alors $ABC$ n'est un triangle rectangle en $A$.
    \end{itemize}
}

\theoreme{}{
    \vspace{-10pt}
    \begin{itemize}
        \item Une implication et sa contraposée sont équivalentes.
        \item Deux propositions sont équivalentes si les implications dans les deux sens sont vraies.
    \end{itemize}
}

\subsection{Quantificateurs}

\definition{(universel et existenciel)}{
    \vspace{-10pt}
    \begin{enumerate}[(i)]
        \item Le quantificateur universel « $\forall$ » se lit « pour tout ».
        \item Le quantificateur existenciel « $\exists$ » se lit « il existe ».
    \end{enumerate}
}

\remarques{
    \begin{itemize}
        \item « $\exists!$ » signifie « il existe un unique »
        \item La négation de « $\forall$ » est « $\exists$ » et vice versa.
        \item On peut intervertir les quantificateur de même nature.
    \end{itemize}
}

\end{document}