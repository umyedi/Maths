\documentclass[10pt]{article}

\newcommand{\DocumentTitle}{Ensembles et applications}
\newcommand{\DocumentTheme}{Séminaire}
\newcommand{\DocumentType}{Cours}

% Preamble Section
\usepackage[a4paper, margin=1in]{geometry} % Sets the paper size to A4 and all margins to 1 inch
\usepackage[utf8]{inputenc}   % Allows for input of international characters
\usepackage[T1]{fontenc}      % Utilisation de l'encodage T1
\usepackage{lmodern}          % Support des polices avec le module 'french'
\usepackage[french]{babel}    % Ajout du support pour le français
\usepackage[fleqn]{amsmath}   % Imports the AMS math package for advanced math formatting ([fleqn] -> align alignat to the left)
\usepackage{amsfonts}         % Imports AMS fonts for math fonts
\usepackage{amssymb}          % Imports AMS symbols for math symbols
\usepackage{amsthm}           % Imports AMS theorems
\usepackage{fancyhdr}         % Import to custom the page footer
\usepackage{mdframed}         % Imports styles
\usepackage{enumerate}        % Imports styles for enumerate
\usepackage{multicol}
\usepackage{xcolor}           % Couleurs
\usepackage{tikz}             % Package général pour les graphiques
\usepackage{pgfplots}         % Compléments pour les graphiques
\usepackage{tkz-tab}          % Tableaux de variations
\usepackage{lastpage}         % Pour avoir le total de page dans le footer
\usepackage{makecell}         % Retour à la ligne dans une case d'un tabular
\usepackage{stmaryrd}         % Intervalles entières : \llbracket et \rrbracket

\pgfplotsset{compat=newest}   % Active les dernières fonctionnalités de pgfplots

% Enlève l'indentation des bloc de paragraphes et d'équation
\setlength{\parindent}{0pt}
\setlength{\mathindent}{0pt}

% ========== Title Section ==========

\title{\DocumentTitle} % Title of the document
\author{\DocumentTheme\space-\space\DocumentType} % Author's name
\date{} % Date

\pagestyle{fancy}
\fancyhf{} % Clear all header and footer fields

% ========== Footer Section ==========

\newcommand{\customfooter}{
    \fancyfoot[L]{\DocumentTheme\space-\space\DocumentType}
    \fancyfoot[C]{\DocumentTitle}
    \fancyfoot[R]{\thepage/\pageref{LastPage}}
    % Remove the line below the header and above the footer
    \renewcommand{\headrulewidth}{0pt}
    \renewcommand{\footrulewidth}{0pt}
}

% Apply the custom footer to the fancy style
\customfooter

% Apply the custom footer to the plain style (used on the first page)
\fancypagestyle{plain}{\customfooter}

% ========== Sections ==========

% Sections : I. / II. / III. ...
\renewcommand{\thesection}{\Roman{section}}
% Subsection : I.1 / I.2 / I.3 ...
\renewcommand{\thesubsection}{\thesection. \arabic{subsection}}
% Subsubsection : I.1.a / I.1.b / I.1.c ...
\renewcommand{\thesubsubsection}{\thesubsection. \alph{subsubsection}}

% ========== Colors (from Geogebra) ==========

\definecolor{green}{HTML}{006400}
\definecolor{red}{HTML}{CC0000}
\definecolor{blue}{HTML}{0000FF}
\definecolor{orange}{HTML}{FF5500}
\definecolor{purple}{HTML}{9933FF}
\definecolor{gray}{HTML}{666666}
\definecolor{brown}{HTML}{993300}

\newcommand{\green}[1]{\color{green}{#1}\color{black}}
\newcommand{\red}[1]{\color{red}{#1}\color{black}}
\newcommand{\blue}[1]{\color{blue}{#1}\color{black}}
\newcommand{\orange}[1]{\color{orange}{#1}\color{black}}
\newcommand{\purple}[1]{\color{purple}{#1}\color{black}}
\newcommand{\gray}[1]{\color{gray}{#1}\color{black}}
\newcommand{\brown}[1]{\color{brown}{#1}\color{black}}


% ========== Définition(s) ==========

% Style de la boîte de définition(s)
\mdfdefinestyle{DefinitionStyle}{
    leftmargin=0cm,
    rightmargin=0cm,
    linecolor=black,
    linewidth=2pt,
    topline=false,
    bottomline=false,
    rightline=false
}

% Définition de la commande '\definition'
\newcommand{\definition}[2]{%
    \begin{mdframed}[style=DefinitionStyle]
        \ifstrempty{#1}{% Teste si le premier argument est vide
            \textbf{Définition :}\ % Si vide, n'affiche pas de titre de théorème
        }{%
            \textbf{Définition #1 :}\ % Si non vide, affiche le titre avec le premier argument
        }\\#2
    \end{mdframed}
}

% Définition de la commande '\definitions'
\newcommand{\definitions}[2]{%
    \begin{mdframed}[style=DefinitionStyle]
        \ifstrempty{#1}{% Teste si le premier argument est vide
            \textbf{Définitions :}\ % Si vide, n'affiche pas de titre de théorème
        }{%
            \textbf{Définitions #1 :}\ % Si non vide, affiche le titre avec le premier argument
        }\\#2
    \end{mdframed}
}

% ========== Propriété(s) et Théorème(s) ==========

% Style de la boîte de propriété(s) / théorème(s)
\mdfdefinestyle{ProprieteStyle}{
    leftmargin=0cm,
    rightmargin=0cm,
    linecolor=black,
    linewidth=1pt
}

% Création d'un compteur pour les propriétés
\newcounter{propriete}

% Définition de la commande '\propriete'
\newcommand{\propriete}[2]{%
    \refstepcounter{propriete}% Incrémente le compteur de théorème
    \begin{mdframed}[style=ProprieteStyle]
        \ifstrempty{#1}{% Teste si le premier argument est vide
            \textbf{\thepropriete. Propriété :}\ % Si vide, n'affiche pas de titre de théorème
        }{%
            \textbf{\thepropriete. Propriété #1 :}\ % Si non vide, affiche le titre avec le premier argument
        }\\#2
    \end{mdframed}
}

% Définition de la commande 'proprietes'
\newcommand{\proprietes}[2]{%
    \refstepcounter{propriete}% Incrémente le compteur de théorème
    \begin{mdframed}[style=ProprieteStyle]
        \ifstrempty{#1}{% Teste si le premier argument est vide
            \textbf{\thepropriete. Propriétés :}\ % Si vide, n'affiche pas de titre de théorème
        }{%
            \textbf{\thepropriete. Propriétés #1 :}\ % Si non vide, affiche le titre avec le premier argument
        }\\#2
    \end{mdframed}
}



% Création d'un compteur pour les théorèmes
\newcounter{theoreme}

% Définition de la commande '\theoreme'
\newcommand{\theoreme}[2]{%
    \refstepcounter{theoreme}% Incrémente le compteur de théorème
    \begin{mdframed}[style=ProprieteStyle]
        \ifstrempty{#1}{% Teste si le premier argument est vide
            \textbf{\thetheoreme. Théorème :}\ % Si vide, n'affiche pas de titre de théorème
        }{%
            \textbf{\thetheoreme. Théorème #1 :}\ % Si non vide, affiche le titre avec le premier argument
        }\\#2
    \end{mdframed}
}

% Définition de la commande '\theoremes'
\newcommand{\theoremes}[2]{%
    \refstepcounter{theoreme}% Incrémente le compteur de théorème
    \begin{mdframed}[style=ProprieteStyle]
        \ifstrempty{#1}{% Teste si le premier argument est vide
            \textbf{\thetheoreme. Théorèmes :}\ % Si vide, n'affiche pas de titre de théorème
        }{%
            \textbf{\thetheoreme. Théorèmes #1 :}\ % Si non vide, affiche le titre avec le premier argument
        }\\#2
    \end{mdframed}
}

% ========== Exemple(s) ==========

% Définition de la commande 'exemple'
\newcommand{\exemple}[2]{
    \textbf{Exemple :} #1
    \begin{quote}
        #2
    \end{quote}
}

% Définition de la commande 'exemples'
\newcommand{\exemples}[2]{
    \textbf{Exemples :} #1
    \begin{quote}
        #2
    \end{quote}
}

% ========== Remarques ==========

% Définition de la commande 'remarque'
\newcommand{\remarque}[1]{
    \textbf{Remarque :} #1
}

% Définition de la commande 'remarques'
\newcommand{\remarques}[1]{
    \textbf{Remarques :} #1
}

% ========== Démonstration ==========

\newcommand{\demonstration}[2]{
    \textbf{Démonstration :} #1
    \begin{quote}
        #2
    \end{quote}
}

% ========== Notations ==========

% Ensembles
\newcommand{\C}{\mathbb{C}}
\newcommand{\R}{\mathbb{R}}
\newcommand{\Q}{\mathbb{Q}}
\newcommand{\D}{\mathbb{D}}
\newcommand{\Z}{\mathbb{Z}}
\newcommand{\N}{\mathbb{N}}

% Limites
\let\oldlim\lim
\renewcommand{\lim}[1]{\mathop{\displaystyle\oldlim}\limits_{#1}}

% Opérateurs
\newcommand{\x}{\times}
\newcommand{\equival}{\Leftrightarrow}

% Vecteurs
\renewcommand{\Vec}[1]{\overrightarrow{#1}}
\newcommand*{\norme}[1]{\left\lVert\vv{#1}\right\rVert}
\newcommand{\ps}[2]{\ensuremath{\vv{#1}.\vv{#2}}}

\newcommand{\vcoord}[2]{\begin{pmatrix} #1 \\ #2 \end{pmatrix}}
\newcommand{\Vcoord}[3]{\begin{pmatrix} #1 \\ #2 \\ #3 \end{pmatrix}}

% ========== Tableaux ==========
\newcommand{\boldhline}{\hline\noalign{\vskip 0pt}\hline}

\begin{document}

\maketitle

\section{Généralités sur les ensembles}

\begin{itemize}
    \item Un ensemble est une collection d'éléments ou d'objets. Les élément (ou objets) peuvent être des nombres, des points, des
          vecteurs... En général, un élément sera noté par une lettre minuscule et un ensemble par une lettre majuscule.
    \item Soient $E$ un ensemble et $x$ un objet de $E$. 0n dit alors que $x$ est un élément de $E$ ou
          $x$ appartient à $E$ et on écrit $x \in E$. Si $x$ n'est pas un élément de l'ensemble $E$, on écrit $x \not\in E$.
    \item Un ensemble peut être écrit en extension (list exclusive de tous les éléments) ou en compréhension
          (les éléments sont défini par une propriété).
\end{itemize}

\definition{(ensemble vide)}{
    L'ensemble vide, noté $\emptyset$, est défini comme étant l'ensemble vérifiant $x\not\in E$ pour tout objet $x$.
}


\section{Inclusion, égalité et parties d'un ensemble}

\definitions{}{
    \vspace{-10pt}
    \begin{itemize}
        \item Soient $E$ et $F$ deux ensembles. On dit que $E$ est inclus dans $F$, noté $E \subset F$, si
              $\forall x\in E, (x\in E \implies x\in F)$.
        \item On dit que $E$ n'est pas inclus dans $F$, noté $E\not\subset F$, s'il existe au moins un élément de $E$ qui
              n'appartient pas à $F$.
    \end{itemize}
}

\proprietes{}{
    Soient E, F et G trois ensembles, alors :
    \begin{enumerate}[(i)]
        \item $E\subset E$
        \item $(E\subset F \land F\subset G) \implies E\subset G$ (transitivité)
    \end{enumerate}
}

\definition{}{
    Soient $E$ et $F$ deux ensembles. On dit que $E$ et $F$ sont égaux et on note
    $E = F$ , lorsqu'ils sont constitués de mêmes éléments. Sinon, on dit qu'ils sont dis-
    tincts, on note $E \not= F$.
}

\propriete{}{
    Deux parties $A$ et $B$ d'un ensemble de $E$ sont égales si et seulement si $A\subset B\land B\subset A$.
}

\definition{}{
    L'ensemble des parties de $E$ est l'ensemble dont les éléments sont les parties de $E$. On le note souvent
    $\mathcal{P}(E)$.
}

\remarques{
    \begin{itemize}
        \item $\mathcal{P}(E)=\{A, A\subset E\}$\
        \item $A\in\mathcal{P}(E) \equival A \subset E$
    \end{itemize}
}

\newpage

\section{Opérations sur les parties d'un ensemble}

\definitions{(réunion, intersection, différence et complémentaire)}{
    Soient $A$ et $B$ deux parties d'un ensemble $E$.
    \begin{enumerate}[(i)]
        \item La réunion de $A$ et de $B$ est l'ensemble des éléments de $E$ qui appartiennent à $A$ OU à $B$. On note
              $A\cap B$.
        \item L'intersection de $A$ et de $B$ est l'ensemble des éléments de $E$ qui appartiennent à $A$ ET à $B$. On note
              $A\cup B$.
        \item La différence de $A$ et de $B$ est l'ensemble des éléments de $E$ qui appartiennent à $A$ et qui
              n'appartiennent pas à $B$. On note $A\setminus B$ ou $A-B$.
        \item La partie complémentaire de $A$ dans $E$ est $E\setminus A$. On note $\bar{A}$ ou parfois $\complement_E^A$.
    \end{enumerate}
}


\section{Règles de calcul}

\subsection{Intersection}

\propriete{(intersection)}{
    Soient $A$, $B$ et $C$ trois parties de $E$. Alors :
    \begin{enumerate}[(i)]
        \item $A\cap B = B\cap A$ (commutativité)
        \item $A\cap (B \cap C) = (A\cap B) \cap C$ (associativité)
        \item $A\cap\emptyset = A$,  $A\cap A=A$ et $A\cap E=E$
        \item $A\cap B=A\equival A\subset B$
    \end{enumerate}
}

\subsection{Union}

\propriete{(union)}{
    Soient $A$ et $B$ deux parties de $E$. Alors :
    \begin{enumerate}[(i)]
        \item $A\cup B = B\cup A$ (commutativité)
        \item $A\cup (B \cup C) = (A\cup B) \cup C$ (associativité)
        \item $A\cup\emptyset = A$ et $A\cup A=A$
        \item $A\cup B\equival A\subset B$
    \end{enumerate}
}

\subsection{Intersection et union}

\propriete{}{
    Soient $A$, $B$ et $C$ trois parties de $E$. Alors :
    \begin{enumerate}[(i)]
        \item $\bar{\bar{A}}=A$
        \item $\bar{(A\cap B)} \equival \bar{A}\cup\bar{B}$
        \item $\bar{(A\cup B)} \equival \bar{A}\cap\bar{B}$
        \item $A\cap(B\cup C) = (A\cap B)\cup(A\cap C)$
        \item $A\cup(B\cap C) = (A\cup B)\cap(A\cup C)$
    \end{enumerate}
}

\section{Produit cartésien}

\definition{}{
    Soient $E$ et $F$ deux ensembles. Le produit cartésien de $E$ par $F$, noté $E\times F$, est l'ensemble des couples
    $(x,y)$ où $x\in E$ et $y\in F$.
}

\newpage

\section{Applications}

\subsection{Définitions}

\definition{}{
    Soient $E$ et $F$ deux ensembles.
    
    \begin{itemize}
       \item Une application (ou fonction) $f:E\rightarrow F$ est la donnée pour chaque élément $x\in E$ d'un unique
       élément de $F$ noté $f(x)$.
       \item Deux applications $f,g:E\rightarrow F$ sont égales si et seulement si : $\forall x\in E, f(x)=g(x)$.
       On note alors $f=g$.
       \item Le graphe de $f:E\rightarrow F$ est l'ensemble $\Gamma=\{(x, f(x)), x\in E\}\subset E\times F$
    \end{itemize}
}

\subsection{Image direct, image réciproque}

\definition{}{
    Soient $A\subset E$, $B \subset F$ et $f:E\rightarrow F$.
    \begin{itemize}
       \item L'image directe de $A$ par $f$ est l'ensemble $$f(A)=\{f(x), x\in A\}$$
       \item L'image réciproque de $B$ par $f$ est l'ensemble $$f^{-1}(B)=\{x\in E, f(x)\in B\}$$
    \end{itemize}
}

\subsection{Antécédent}

\definition{}{
    Soit $y$ un élément fixé dans $F$. Tout élément $x\in E$ tel que $f(x)=y$ est un antécédent de $y$. L'ensemble des
    antécédent de $y$ est $f^{-1}(\{y\})$.
}

\subsection{La loi rond}

\definition{}{
    Soient $E$, $F$, $G$ trois ensembles et $f:E\rightarrow F$, $g:F\rightarrow G$ deux applications. La composée de $f$
    et $g$ notée $g\circ f$ est l'application de $E$ dans $G$ défini par $(g\circ f)(x)=g(f(x))$ où $x\in E$
}

\propriete{}{
    Soient $E$, $F$, $G$, $H$ quatre ensembles et $f:E\rightarrow F$, $g:F\rightarrow G$, $h:G\rightarrow H$ trois
    applications. Alors $(h\circ g)\circ f = h\circ (g\circ f)$.
}

\remarque{En général, $f\circ g \not= g\circ f$.}


\section{Ensembles finis}

\subsection{Injection, surjection et bijection}

\definition{(injection, surjection et bijection)}{
    Soient $E$ et $F$ deux ensembles et $f:E\rightarrow F$ une application.
    \begin{itemize}
       \item L'application $f$ est dite injective si : 
       $$\forall(x,x')\in E^2 : f(x)=f(x') \Rightarrow x=x'$$
       \item L'application $f$ est dite surjective si :
       $$\forall y\in F, \exists x\in E : y=f(x)$$

       \item L'application $f$ est dite bijective si elle est à la fois injective et surjective.
    \end{itemize}
}

\newpage

\remarques{
    \begin{itemize}
       \item L'injection assure l'unicité tandis que la surjection assure l'existence.
       \item $f:E\rightarrow F$ injective $\equival \forall(x,x')\in E^2: x\not=x' \Rightarrow f(x)\not=f(x')$
       \item $f:E\rightarrow F$ sujective $\equival f(E)=F$
       \item $f:E\rightarrow F$ bijective $\equival \forall y\in F, \exists! x\in E: y=f(x)$
    \end{itemize}
}

\definition{(ensemble fini, cardinal)}{
    Soit $E$ un ensemble et $n\in\N^*$.
    \begin{itemize}
       \item L'ensemble $E$ est dit fini s'il est vide ou s'il existe une bijection de $\{1,2,3,\dots,n\}$ sur $E$.
       \item L'entier $n$ est unique et s'appelle le cardinal de $E$. On le note $Card(E)$.
    \end{itemize}
}

\remarque{Il y existe différentes notations du cardinal : $Card(E)=n(E)=\#E=\left|E\right|=\bar{E}=\bar{\bar{E}}$}

\propriete{}{
    Soient $E$ et $F$ deux ensembles finis et $f$ une application de $E$ dans $F$.
    \begin{enumerate}[(i)]
       \item Si $f$ est injective alors $Card(E) \leq Card(F)$.
       \item Si $f$ est surjective alors $Card(E) \geq Card(F)$.
       \item Si $f$ est bijective alors $Card(E) = Card(F)$.
    \end{enumerate}
}


\propriete{}{
    Soient $E$, $F$ deux ensembles finis et $f: E \rightarrow F$ une application. Si $Card(E)=Card(F)$ alors les
    assertions suivantes sont équivalentes :
    \begin{enumerate}[(i)]
       \item $f$ est injective,
       \item $f$ est surjective,
       \item $f$ est bijective.
    \end{enumerate}
}

\subsection{Nombre d'applications}

\definition{}{
    Soient $E$ et $F$ deux ensembles. L'ensemble des application de $E$ dans $F$ est noté $F^E$.
}


\propriete{}{
    Soient $E$ et $F$ deux ensembles finis. Alors le nombre d'applications différentes de $E$ dans $F$ est
    $Card(E)^{Card(F)}$.
}

\propriete{}{
    Soient $E$ et $F$ deux ensembles finis avec $n=Card(E)$ et $p=Card(F)$. Alors le nombre d'injections de $E$ dans $F$
    est $p\times (p-1) \times\dots\times (p - (n-1))$. 
}

\propriete{}{
    Le nombre de bijections d'un ensemble $E$ de cardinal $n$ dans lui-même est $n!$.
}


\subsection{Parties d'un ensemble fini}

\definition{}{
    Soit $E$ un ensemble. Une partition de $E$ est une famille de parties $(A_i)_{i\in I}$ telle que :
    \begin{itemize}
       \item $\ds\bigcup_{i\in I} A_i=E$
       \item $\forall(i,j)\in I^2 : i\not=j\Rightarrow A_i\cap A_j=\emptyset$
    \end{itemize}
}

\propriete{}{
    Soit $E$ un ensemble fini.
    \begin{enumerate}[(i)]
       \item Si $A$ est une partie de $E$ alors $A$ est fini et $Card(A)\leq Card(E)$
       \item Si $A\subset E$ et $Card(A)=Card(E)$, alors $E=A$.
    \end{enumerate}
}

\propriete{}{
    Soit $E$ un ensemble fini.
    \begin{enumerate}[(i)]
       \item Soient $A\subset E$ et $B\subset E$ tels que $A\cap B=\emptyset$. Alors $$Card(A\cup B)=Card(A) + Card(B)$$.
       \item Soient $A\subset E$ et $B\subset E$. Alors $$Card(A\cup B)=Card(A) + Card(B)-Card(A\cap B)$$.
       \item Soit $A\subset E$. Alors $$Card(\bar{A}) = Card(E) - Card(A)$$.
    \end{enumerate}
}

\propriete{}{
    Soit $E$ un ensemble fini de cardinal $n$. Alors $Card(\part{E})=2^n$.
}

\section{Binôme de Newton}

\definition{}{
    Soit $n\in\N^*$. On note $\ds n! = 1\times2\times\dots\times n = \prod_{k=1}^{n} k$ appelé factorielle de $n$.
    \\ Par convention, $0! = 1$.
}

\definition{}{
    Soient $n$ et $p$ deux entiers tels que $0\leq p\leq n$. On défini le symbole $\ds\binom{n}{k}$ par
    $$\ds\binom{n}{k}=\frac{n!}{k!(n-k)!}$$
}

\remarque{La notation du binôme peut aussi s'écrire $\ds C_n^k$.}

\propriete{}{
    \vspace{-10pt}
    \begin{enumerate}[(i)]
       \item $\ds \binom{n}{0} = \binom{n}{n} = 1$
       \item $\ds \binom{n}{k} = \binom{n}{n-k}$ (relation de Pascal)
       \item $\ds \binom{n}{k} = \binom{n-1}{k} + \binom{n-1}{k-1}$
    \end{enumerate}
}

\theoreme{(formule du binôme de Newton)}{
    Soient $a$ et $b$ deux nombres réels. Pour tout entier naturel $n$, on a :
    $$\ds (a+b)^n=\sum_{k=0}^{n} \binom{n}{k}a^k\times b^{n-k} = \sum_{k=0}^{n} \binom{n}{k}a^{n-k}\times b^k$$
}


\end{document}