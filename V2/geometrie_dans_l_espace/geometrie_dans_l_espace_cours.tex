\documentclass[10pt]{article}

\newcommand{\DocumentTitle}{Géométrie dans l'espace}
\newcommand{\DocumentTheme}{Géométrie}
\newcommand{\DocumentType}{Cours}

% Preamble Section
\usepackage[a4paper, margin=1in]{geometry} % Sets the paper size to A4 and all margins to 1 inch
\usepackage[utf8]{inputenc}   % Allows for input of international characters
\usepackage[T1]{fontenc}      % Utilisation de l'encodage T1
\usepackage{lmodern}          % Support des polices avec le module 'french'
\usepackage[french]{babel}    % Ajout du support pour le français
\usepackage[fleqn]{amsmath}   % Imports the AMS math package for advanced math formatting ([fleqn] -> align alignat to the left)
\usepackage{amsfonts}         % Imports AMS fonts for math fonts
\usepackage{amssymb}          % Imports AMS symbols for math symbols
\usepackage{amsthm}           % Imports AMS theorems
\usepackage{fancyhdr}         % Import to custom the page footer
\usepackage{mdframed}         % Imports styles
\usepackage{enumerate}        % Imports styles for enumerate
\usepackage{multicol}
\usepackage{xcolor}           % Couleurs
\usepackage{tikz}             % Package général pour les graphiques
\usepackage{pgfplots}         % Compléments pour les graphiques
\usepackage{tkz-tab}          % Tableaux de variations
\usepackage{lastpage}         % Pour avoir le total de page dans le footer
\usepackage{makecell}         % Retour à la ligne dans une case d'un tabular
\usepackage{stmaryrd}         % Intervalles entières : \llbracket et \rrbracket

\pgfplotsset{compat=newest}   % Active les dernières fonctionnalités de pgfplots

% Enlève l'indentation des bloc de paragraphes et d'équation
\setlength{\parindent}{0pt}
\setlength{\mathindent}{0pt}

% ========== Title Section ==========

\title{\DocumentTitle} % Title of the document
\author{\DocumentTheme\space-\space\DocumentType} % Author's name
\date{} % Date

\pagestyle{fancy}
\fancyhf{} % Clear all header and footer fields

% ========== Footer Section ==========

\newcommand{\customfooter}{
    \fancyfoot[L]{\DocumentTheme\space-\space\DocumentType}
    \fancyfoot[C]{\DocumentTitle}
    \fancyfoot[R]{\thepage/\pageref{LastPage}}
    % Remove the line below the header and above the footer
    \renewcommand{\headrulewidth}{0pt}
    \renewcommand{\footrulewidth}{0pt}
}

% Apply the custom footer to the fancy style
\customfooter

% Apply the custom footer to the plain style (used on the first page)
\fancypagestyle{plain}{\customfooter}

% ========== Sections ==========

% Sections : I. / II. / III. ...
\renewcommand{\thesection}{\Roman{section}}
% Subsection : I.1 / I.2 / I.3 ...
\renewcommand{\thesubsection}{\thesection. \arabic{subsection}}
% Subsubsection : I.1.a / I.1.b / I.1.c ...
\renewcommand{\thesubsubsection}{\thesubsection. \alph{subsubsection}}

% ========== Colors (from Geogebra) ==========

\definecolor{green}{HTML}{006400}
\definecolor{red}{HTML}{CC0000}
\definecolor{blue}{HTML}{0000FF}
\definecolor{orange}{HTML}{FF5500}
\definecolor{purple}{HTML}{9933FF}
\definecolor{gray}{HTML}{666666}
\definecolor{brown}{HTML}{993300}

\newcommand{\green}[1]{\color{green}{#1}\color{black}}
\newcommand{\red}[1]{\color{red}{#1}\color{black}}
\newcommand{\blue}[1]{\color{blue}{#1}\color{black}}
\newcommand{\orange}[1]{\color{orange}{#1}\color{black}}
\newcommand{\purple}[1]{\color{purple}{#1}\color{black}}
\newcommand{\gray}[1]{\color{gray}{#1}\color{black}}
\newcommand{\brown}[1]{\color{brown}{#1}\color{black}}


% ========== Définition(s) ==========

% Style de la boîte de définition(s)
\mdfdefinestyle{DefinitionStyle}{
    leftmargin=0cm,
    rightmargin=0cm,
    linecolor=black,
    linewidth=2pt,
    topline=false,
    bottomline=false,
    rightline=false
}

% Définition de la commande '\definition'
\newcommand{\definition}[2]{%
    \begin{mdframed}[style=DefinitionStyle]
        \ifstrempty{#1}{% Teste si le premier argument est vide
            \textbf{Définition :}\ % Si vide, n'affiche pas de titre de théorème
        }{%
            \textbf{Définition #1 :}\ % Si non vide, affiche le titre avec le premier argument
        }\\#2
    \end{mdframed}
}

% Définition de la commande '\definitions'
\newcommand{\definitions}[2]{%
    \begin{mdframed}[style=DefinitionStyle]
        \ifstrempty{#1}{% Teste si le premier argument est vide
            \textbf{Définitions :}\ % Si vide, n'affiche pas de titre de théorème
        }{%
            \textbf{Définitions #1 :}\ % Si non vide, affiche le titre avec le premier argument
        }\\#2
    \end{mdframed}
}

% ========== Propriété(s) et Théorème(s) ==========

% Style de la boîte de propriété(s) / théorème(s)
\mdfdefinestyle{ProprieteStyle}{
    leftmargin=0cm,
    rightmargin=0cm,
    linecolor=black,
    linewidth=1pt
}

% Création d'un compteur pour les propriétés
\newcounter{propriete}

% Définition de la commande '\propriete'
\newcommand{\propriete}[2]{%
    \refstepcounter{propriete}% Incrémente le compteur de théorème
    \begin{mdframed}[style=ProprieteStyle]
        \ifstrempty{#1}{% Teste si le premier argument est vide
            \textbf{\thepropriete. Propriété :}\ % Si vide, n'affiche pas de titre de théorème
        }{%
            \textbf{\thepropriete. Propriété #1 :}\ % Si non vide, affiche le titre avec le premier argument
        }\\#2
    \end{mdframed}
}

% Définition de la commande 'proprietes'
\newcommand{\proprietes}[2]{%
    \refstepcounter{propriete}% Incrémente le compteur de théorème
    \begin{mdframed}[style=ProprieteStyle]
        \ifstrempty{#1}{% Teste si le premier argument est vide
            \textbf{\thepropriete. Propriétés :}\ % Si vide, n'affiche pas de titre de théorème
        }{%
            \textbf{\thepropriete. Propriétés #1 :}\ % Si non vide, affiche le titre avec le premier argument
        }\\#2
    \end{mdframed}
}



% Création d'un compteur pour les théorèmes
\newcounter{theoreme}

% Définition de la commande '\theoreme'
\newcommand{\theoreme}[2]{%
    \refstepcounter{theoreme}% Incrémente le compteur de théorème
    \begin{mdframed}[style=ProprieteStyle]
        \ifstrempty{#1}{% Teste si le premier argument est vide
            \textbf{\thetheoreme. Théorème :}\ % Si vide, n'affiche pas de titre de théorème
        }{%
            \textbf{\thetheoreme. Théorème #1 :}\ % Si non vide, affiche le titre avec le premier argument
        }\\#2
    \end{mdframed}
}

% Définition de la commande '\theoremes'
\newcommand{\theoremes}[2]{%
    \refstepcounter{theoreme}% Incrémente le compteur de théorème
    \begin{mdframed}[style=ProprieteStyle]
        \ifstrempty{#1}{% Teste si le premier argument est vide
            \textbf{\thetheoreme. Théorèmes :}\ % Si vide, n'affiche pas de titre de théorème
        }{%
            \textbf{\thetheoreme. Théorèmes #1 :}\ % Si non vide, affiche le titre avec le premier argument
        }\\#2
    \end{mdframed}
}

% ========== Exemple(s) ==========

% Définition de la commande 'exemple'
\newcommand{\exemple}[2]{
    \textbf{Exemple :} #1
    \begin{quote}
        #2
    \end{quote}
}

% Définition de la commande 'exemples'
\newcommand{\exemples}[2]{
    \textbf{Exemples :} #1
    \begin{quote}
        #2
    \end{quote}
}

% ========== Remarques ==========

% Définition de la commande 'remarque'
\newcommand{\remarque}[1]{
    \textbf{Remarque :} #1
}

% Définition de la commande 'remarques'
\newcommand{\remarques}[1]{
    \textbf{Remarques :} #1
}

% ========== Démonstration ==========

\newcommand{\demonstration}[2]{
    \textbf{Démonstration :} #1
    \begin{quote}
        #2
    \end{quote}
}

% ========== Notations ==========

% Ensembles
\newcommand{\C}{\mathbb{C}}
\newcommand{\R}{\mathbb{R}}
\newcommand{\Q}{\mathbb{Q}}
\newcommand{\D}{\mathbb{D}}
\newcommand{\Z}{\mathbb{Z}}
\newcommand{\N}{\mathbb{N}}

% Limites
\let\oldlim\lim
\renewcommand{\lim}[1]{\mathop{\displaystyle\oldlim}\limits_{#1}}

% Opérateurs
\newcommand{\x}{\times}
\newcommand{\equival}{\Leftrightarrow}

% Vecteurs
\renewcommand{\Vec}[1]{\overrightarrow{#1}}
\newcommand*{\norme}[1]{\left\lVert\vv{#1}\right\rVert}
\newcommand{\ps}[2]{\ensuremath{\vv{#1}.\vv{#2}}}

\newcommand{\vcoord}[2]{\begin{pmatrix} #1 \\ #2 \end{pmatrix}}
\newcommand{\Vcoord}[3]{\begin{pmatrix} #1 \\ #2 \\ #3 \end{pmatrix}}

% ========== Tableaux ==========
\newcommand{\boldhline}{\hline\noalign{\vskip 0pt}\hline}

\begin{document}

\maketitle

\section{Droites, plans et vecteurs de l'espace}

\subsection{Droites et plans}

\textbf{Axiomes :}
\begin{itemize}
    \item Par deux points distincts de l'espace passe une unique droite.
    \item Par trois points non alignés de l'espace passe un unique plan.
    \item L'intersection de deux plans distincts est soit une droite, soit l'ensemble vide.
    \item Si une droite a au moins deux points d'intersections avec un plan, alors elle est incluse dans ce plan.
\end{itemize}

\definitions{}{
    \vspace{-10pt}
    \begin{itemize}
        \item Des points de l'espace sont dits coplanaires s'il existe un plan les contenant tous.
        \item Des droites de l'espace sont dites coplanaires s'il existe un plan les contenant toutes.
        \item Deux droites de l'espace sont dites sécantes si elles ont un unique point d'intersection.
        \item Deux droites de l'espace sont dites parallèles si elles sont coplanaires et non-sécantes.
    \end{itemize}
}

\begin{minipage}[t]{0.3\textwidth}
    \vspace{0pt}
    \begin{tikzpicture}[scale=2]
        % Définition des sommets du cube
        \coordinate (A) at (0,0,1);
        \coordinate (B) at (1,0,1);
        \coordinate (C) at (1,0,0);
        \coordinate (D) at (0,0,0);
        \coordinate (E) at (0,1,0);
        \coordinate (F) at (0,1,1);
        \coordinate (G) at (1,1,1);
        \coordinate (H) at (1,1,0);

        % Dessin des arêtes du cube
        \draw[thick] (A) -- (B) -- (G) -- (F) -- (A);
        \draw[thick] (B) -- (C) -- (H) -- (G);
        \draw[thick] (F) -- (E) -- (H) ;
        \draw[dashed, thick] (D) -- (A);
        \draw[dashed, thick] (D) -- (C);
        \draw[dashed, thick] (D) -- (E);

        % Placement des noms des points
        \node at (A) [left] {$A$};
        \node at (B) [right] {$B$};
        \node at (C) [right] {$C$};
        \node at (D) [left] {$D$};
        \node at (E) [left] {$E$};
        \node at (F) [left] {$F$};
        \node at (G) [right] {$G$};
        \node at (H) [right] {$H$};
    \end{tikzpicture}
\end{minipage}%
\begin{minipage}[t]{0.7\textwidth}
    \vspace{5pt}
    \exemple{}{Dans le cube $ABCDEFGH$ ci-contre, les droites $(AB)$ et $(ED)$ ne sont ni sécantes ni parallèles : elles
        sont non coplanaires.}
\end{minipage}

\subsection{Décomposition d'un vecteur et repérage dans un plan}

\theoremes{/ Définitions}{
    Dans un plan, soient $A$, $B$ et $C$ trois points non alignés.
    \begin{enumerate}[(i)]
        \item Pour tout vecteur $\vec{u}$ du plan, il existe un unique couple de réels $(x,y)$ tel que
              $\vec{u}=x\Vec{AB}+y\Vec{AC}$.
              \begin{itemize}
                  \item On dit que $\vec{u}$ est écrit sous la forme d'une combinaison linéaire de
                        $\Vec{AB}$ et $\Vec{AC}$.
                  \item Les réels $x$ et $y$ sont appelés les coordonnées de $\vec{u}$ dans la base
                        $(\Vec{AB},\Vec{AC})$.
              \end{itemize}
        \item Pour tout point $M$ du plan, il existe un unique couple de réels $(x,y)$ tel que
              $\Vec{AM}=x\Vec{AB}+y\Vec{AC}$.
              \begin{itemize}
                  \item Les réels $x$ et $y$ sont appelés les coordonnées de $M$ dans le repère
                        $(A, \Vec{AB}, \Vec{AC})$.
              \end{itemize}
    \end{enumerate}
}

\subsection{Vecteurs de l'espace}

Soient $A$, $B$, $C$ et $D$ quatre points de l'espace. On dit que les vecteurs $\Vec{AB}$ et $\Vec{CD}$ sont égaux si
les points $A$, $B$, $C$ et $D$ sont coplanaires et si $ABDC$ est un parallélogramme. On note $\Vec{AB}=\Vec{CD}$.
\newline

Un vecteur de l'espace est caractérisé par sa direction, son sens et sa norme.\\
On définit (de la même façon que dans un plan) les sommes de vecteurs et la multiplication d'un vecteur par un réel. Les
mêmes règles de calcul restent valables dans l'espace.

La notion de colinéarité des vecteurs se généralise aussi à l'espace :
\begin{itemize}
    \item Deux vecteurs de l'espace $\vec{u}$ et $\vec{v}$ sont dits colinéaires s'il existe un réel $k$ tel que
          $\vec{u}=k\vec{v}$.
    \item Si $A$ et $B$ sont deux points de l'espace et si les vecteurs $\Vec{AB}$ et $\vec{u}$ sont colinéaires, on dit
          alors que $\vec{u}$ dirige la droite $(AB)$, ou encore que $\vec{u}$ est un vecteur directeur de la droite
          $(AB)$.
\end{itemize}

\theoreme{du parallélisme de deux droites}{
    Deux droites de l'espace sont parallèles si et seulement si elles sont dirigées par des vecteurs colinéaires.
}

\section{Repérage dans l'espace}

\subsection{Bases et repères dans l'espace}

\theoremes{/ Définitions}{
    Soient $A$, $B$, $C$ et $D$ quatre points non coplanaires de l'espace :
    \begin{enumerate}[(i)]
        \item Pour tout vecteur de l'espace $\vec{u}$, il existe trois réels $x$, $y$ et $z$ tels que
              $\vec{u}=x\Vec{AB}+y\Vec{AC}+z\Vec{AD}$.
              \begin{itemize}
                  \item Le triplet $(x,y,z)$ est unique et $x$, $y$, $z$ sont appelés coordonnées du vecteur $\vec{u}$
                        dans la base $(\Vec{AB},\Vec{AC},\Vec{AD})$.
                  \item On note $\vec{u}\Vcoord{x}{y}{z}$.
              \end{itemize}
        \item Pour tout point de l'espace $M$, il existe des réels $(x,y,z)$ tels que
              $\Vec{AM}=x\Vec{AB}+y\Vec{AC}+z\Vec{AD}$.
              \begin{itemize}
                  \item Le triplet $(x,y,z)$ est unique et $x$, $y$, $z$ sont appelés coordonnées du point $M$ dans le
                        repère $(A,\Vec{AB},\Vec{AC},\Vec{AD})$.
                  \item On note $M\Vcoord{x}{y}{z}$.
              \end{itemize}
    \end{enumerate}
}

\theoremes{}{
    Dans une base de l'espace :
    \begin{enumerate}[(i)]
        \item Pour tous vecteurs $\vec{u}$ et $\vec{v}$, si $\vec{u}\Vcoord{x}{y}{z}$ et $\vec{v}\Vcoord{x'}{y'}{z'}$
              alors $\vec{u}+\vec{v}\Vcoord{x+x'}{y+y'}{z+z'}$.
        \item Pour tout réel $k$ et pour tout vecteur $\vec{u}$, si $\vec{u}\Vcoord{x}{y}{z}$ alors
              $k\vec{u}\Vcoord{kx}{ky}{kz}$.
    \end{enumerate}
}

\theoreme{}{
    Dans un repère de l'espace : pour tous points $A$ et $B$, si $A(x_A,y_A,z_A)$ et $B(x_B,y_B,z_B)$ alors
    $\Vec{AB}\Vcoord{x_B-x_A}{y_B-y_A}{z_B-z_A}$.
}

\newpage

\subsection{Représentation paramétrique d'une droite}

\theoremes{/ Définitions}{
    Soient $A$ un point et $\vec{u}$ un vecteur de l'espace.
    On note $(d)$ la droite dirigée par $\vec{u}$ et passant par $A$.\\
    \begin{enumerate}[(i)]
        \item Dans un repère de l'espace, pour tout point $M(x,y,z)$ :
              $M\in(d)\equival\exists k\in\R, \Vec{AM}=k\vec{u}$.
        \item Si de plus on note $A(x_A,y_A,z_A)$ et $\vec{u}\Vcoord{x_{\vec{u}}}{y_{\vec{u}}}{z_{\vec{u}}}$ alors
              $M\in(d)\equival\exists k\in\R, \Sys{x=x_A+kx_{\vec{u}}}{y=y_A+ky_{\vec{u}}}{z=z_A+kz_{\vec{u}}}$
    \end{enumerate}
    Ce système d'équations est appelé une représentation paramétrique de la droite $(d)$. On appelle $k$ le paramètre.
}

\remarque{Il existe une infinité de représentations paramétriques d'une même droite.\\}

\exemple{}{
    Soient $A(0;4;0)$ et $B(1;4;2)$. On calcule $\Vec{AB}\Vcoord{1-0}{4-4}{2-0}\equival\Vec{AB}\Vcoord{1}{0}{2}$.

    Une représentation paramétrique de la droite $(AB):\Sys{x=0+1k}{y=4+0k}{z=0+2k}\equival\Sys{x=k}{y=4}{z=2k}$

}

\section{Position relative d'une droite et d'un plan}

\definitions{}{
    \vspace{-10pt}
    \begin{itemize}
        \item Une droite $(d)$ est parallèle à un plan $(P)$ s'il existe une droite $(d')$ incluse dans $(P)$ et
              parallèle à $(d)$.
        \item Une droite $(d)$ est strictement parallèle à $(P)$ si elle est parallèle à $(P)$ et non incluse dans $(P)$.
        \item Une droite est sécante à un plan $(P)$ si elle a un unique point d'intersection avec $(P)$.
        \item Un vecteur $\vec{u}$ est directeur d'un plan $(P)$ s'il existe une droite $(d)$ incluse dans $(P)$ et
              dirigée par $\vec{u}$.
    \end{itemize}
}

\theoreme{}{
    Pour tous points $A$, $B$, $C$ et $D$ de l'espace : $A$, $B$, $C$ et $D$ sont coplanaires $\equival \exists k\in\R,
        \exists k'\in\R : \Vec{AD}=k\Vec{AB}+k'\Vec{AC}$.
}

\definitions{}{
    Trois vecteurs $\vec{u}$, $\vec{v}$, et $\vec{w}$ sont dits coplanaires s'il existe deux réels $k$ et $k'$ tels que
    $\vec{w}=k\vec{u}+k'\vec{v}$
}

\remarque{Si $\vec{w}=k\vec{u}+k'\vec{v}$, on dit aussi que $\vec{w}$ est une combinaison linéaire de $\vec{u}$ et
    $\vec{v}$}.

\theoreme{}{
    Soit $(d)$ une droite et soit $(P)$ un plan. On suppose que $\vec{u}$ est un vecteur directeur de $(d)$ et que
    $(\vec{v},\vec{w})$
    est un couple de vecteurs directeurs non colinéaires de $(P)$ alors on a :\\
    $(d)$ est parallèle à $(P) \equival\vec{u}$, $\vec{v}$ et $\vec{w}$ sont coplanaires.
}

\newpage

\section{Produit scalaire dans l'espace}

\definition{}{
    Soient $\vec{u}$ et $\vec{v}$ deux vecteurs de l'espace.
    On pose $\vec{u}\pdt\vec{v}=\dfrac{1}{2}\left(\norme{\vec{u}}^2+\norme{\vec{v}}^2-\norme{\vec{v}-\vec{u}}^2\right)$
}

\theoremes{}{
    Pour tous vecteurs $\vec{u}$ et $\vec{v}$ de l'espace :
    \begin{enumerate}[(i)]
        \item $\vec{u}$ et $\vec{v}$ sont orthogonaux $\equival$ $\vec{u}\pdt\vec{v}=0$
        \item $\vec{u}\pdt\vec{v}=\norme{\vec{u}}\times\norme{\vec{v}}\times\cos{(\widehat{\vec{u},\vec{v}})}$
        \item Si $\vec{u}$ et $\vec{v}$ sont colinéaires
              \begin{enumerate}[(a)]
                  \item et de même sens, alors $\vec{u}\pdt\vec{v}=\norme{\vec{u}}\times\norme{\vec{v}}$
                  \item et de sens opposés, alors $\vec{u}\pdt\vec{v}=-\norme{\vec{u}}\times\norme{\vec{v}}$
              \end{enumerate}
        \item $\vec{u}\pdt\vec{v}=\vec{v}\pdt\vec{u}$
        \item $\forall k\in\R: (k\vec{u})\pdt\vec{v} = \vec{u}\pdt(k\vec{v})=k(\vec{u}\pdt\vec{v})$
    \end{enumerate}
}

\theoreme{}{
    Pour tous points $A$, $B$ et $C$ de l'espace : si $H$ est le projeté orthogonal de $B$ sur $(AC)$ alors
    $\Vec{AB}\pdt\Vec{AC}=\Vec{AH}\pdt\Vec{AC}$.
}

\theoreme{}{
    Pour tous vecteurs $\vec{u}$, $\vec{v}$ et $\vec{w}$ de l'espace : $\vec{u}\pdt(\vec{v}+\vec{w})=\vec{u}\pdt\vec{v}
        +\vec{u}\pdt\vec{w}$.
}

\definition{}{
    On appelle repère orthonormé de l'espace, la donnée d'un point $O$ et de trois vecteurs
    $\vec{i}$, $\vec{j}$ et $\vec{k}$ avec :
    \begin{itemize}
        \item $\vec{i}$, $\vec{j}$ et $\vec{k}$ orthogonaux deux à deux,
        \item $\norme{\vec{i}}=\norme{\vec{j}}=\norme{\vec{k}}=1$
    \end{itemize}
}

\remarque{On dit que les vecteurs $\vec{i}$, $\vec{j}$ et $\vec{k}$ sont des vecteurs unitaires.}

\theoreme{}{
    Dans un repère orthonormé de l'espace $(O,\vec{i},\vec{j},\vec{k})$ : si $\vec{u}\Vcoord{x}{y}{z}$ et
    $\vec{u}\Vcoord{x'}{y'}{z'}$ alors $\vec{u}\pdt\vec{v}=xx'+yy'+zz'$
}

\corrolaire{}{
    Dans un repère orthonormé de l'espace : si $\vec{u}\Vcoord{x}{y}{z}$ alors $\norme{\vec{u}}=\sqrt{x^2+y^2+z^2}$.
}

\remarque{Si $A(x_A,y_A,z_A)$ et $B(x_B,y_B,z_B)$ alors on déduit $AB=\sqrt{(x_B-x_A)^2+(y_B-y_A)^2+(z_B-z_A)^2}$.}

\newpage

\section{Orthogonalité et équations cartésiennes de plan}

\subsection{Droite perpendiculaire à un plan et vecteur normal à un plan}

\definitions{}{
    \vspace{-10pt}
    \begin{itemize}
        \item Deux droites sont orthogonales si leurs vecteurs directeurs sont orthogonaux.
        \item Deux droites sont perpendiculaires si elles sont orthogonales et sécantes.
        \item Une droite est perpendiculaire (ou orthogonale) à un plan si elle est orthogonale à deux droites non
              parallèles de ce plan.
        \item Un vecteur est normal à un plan s'il est orthogonal à deux vecteurs directeur non colinéaires de ce plan.
    \end{itemize}
}

\theoreme{}{
    \vspace{-10pt}
    \begin{itemize}
        \item Si un vecteur est normal à un plan, alors il est orthogonal à tous les vecteurs directeurs de ce plan.
        \item Si une droite est perpendiculaire à un plan, alors elle est orthogonale à toutes les droites de ce plan.
    \end{itemize}
}

\theoreme{}{
    \vspace{-10pt}
    \begin{itemize}
        \item Deux droites perpendiculaires à un même plan sont parallèles.
        \item Deux vecteurs normaux à un même plan sont colinéaires.
    \end{itemize}
}

\subsection{Équations cartésiennes d'un plan}

\theoreme{}{
    Soit $(P)$ un plan et $\vec{n}$ un vecteur normal à $(P)$.
    \begin{enumerate}[(i)]
        \item Si $\vec{u}$ est un vecteur directeur d'une droite $(d)$ alors on a l'équivalence suivante :
              $$
                  (d) \text{ est parallèle à } (P) \equival \vec{u}\pdt\vec{v}=0
              $$
        \item Si $A$ est un point appartenant à $(P)$ et $M$ un point de l'espace, alors on a l'équivalence suivante :
              $$
                  M\in(P)\equival\Vec{AM}\pdt\vec{n}=0
              $$
    \end{enumerate}
}

\definitions{}{
    Dans un repère orthonormé : si $\vec{n}\Vcoord{a}{b}{c}$ alors $\Vec{AM}\pdt\vec{n}=0$ peut s'écrire sous la forme
    $ax+by+cz+d=0$ avec $d$ réel. Cette équation s'appelle une équation cartésienne du plan $(P)$.
}

\theoreme{}{
    Soient $a$, $b$, $c$ et $d$ quatre réels avec $(a,b,c)\not=(0,0,0)$. L'ensemble des points $M(x,y,z)$ vérifiant
    $ax+by+cz+d=0$ forme un unique plan de vecteur normal $\vec{n}\Vcoord{a}{b}{c}$.
}

\newpage

\section{Positions relatives de plans et de droites, théorèmes d'incidence}

\theoreme{}{
    Deux droites de l'espace sont soit:
    \begin{itemize}
        \item coplanaires, et dans ce cas, elle sont soit non parallèles, soit sécantes.
        \item non coplanaires, et dans ce cas, elles sont ni parallèles ni sécantes.
    \end{itemize}
}

\theoreme{}{
    Une droite et un plan de l'espace sont soit :
    \begin{itemize}
        \item parallèles et dans ce cas, soit ils n'ont aucun point d'intersection, soit la droite est incluse dans le
              plan.
        \item non parallèles et dans ce cas, ils sont sécants et leur intersection est un unique point.
    \end{itemize}
}

\definitions{}{
    \vspace{-10pt}
    \begin{itemize}
        \item Deux plans sont dits strictement coplanaires si leur intersection est l'ensemble vide.
        \item Deux plans sont dits parallèles s'il sont confondus ou d'intersection vide.
        \item Deux plans sont dits sécants s'il ne sont pas parallèles.
        \item Deux plans sont dits perpendiculaires s'il existe une droites incluse dans l'un et perpendiculaire dans
              l'autre.
    \end{itemize}
}

\theoremes{}{
    Soient $(d)$ et $(d')$ deux droites. Soient $(P)$ et $(P')$ deux plans.
    \begin{enumerate}[(i)]
        \item $d$ est parallèle à $(P) \equival $ Un vecteur directeur de $(d)$ est orthogonal à un vecteur normal à
              $(P)$.
        \item $(d)$ est perpendiculaire à $(P) \equival$ Un vecteur directeur de $(d)$ est colinéaire à un vecteur.
              normal à $(P)$.
        \item $(P)$ et $(P')$ sont parallèles $\equival$ Un vecteur normal à $(P)$ est colinéaire à un vecteur normal à
              $(P')$.
        \item $(P)$ et $(P')$ sont perpendiculaires $\equival$ Un vecteur normal à $(P)$ est orthogonal à un vecteur
              normal à $(P')$.
    \end{enumerate}
}

\theoreme{d'incidence}{
    Soient $(P)$, $(P')$ et $(P'')$ trois plans. Si $(P)$ et $(P')$ sont parallèles et $(P'')$ est sécant à $(P)$ et
    $(P')$, alors les droites d'intersection de $(P)$ et $(P')$ avec $(P'')$ sont parallèles.
}

\theoreme{du toit}{
    Soient $(P)$ et $(P')$ deux plans sécants. On note $\Delta$ la droite d'intersection de $(P)$ et $(P')$. Soient
    $(d)$ une droite incluse dans $(P)$ et $(d')$ une droite incluse dans $(P')$.
    Si $(d)$ et $(d')$ sont parallèles, alors $\Delta$ est parallèle à $(d)$ et $(d')$.
}

\section{Formule liant le produit scalaire et la norme}

\theoreme{(identités remarquables)}{
    Pour tous vecteurs de l'espace $\vec{u}$ et $\vec{v}$ :
    \begin{enumerate}[(i)]
        \item $\norme{\vec{u}+\vec{v}}^2=\norme{\vec{u}}^2+2\vec{u}\pdt\vec{v}+\norme{\vec{v}}^2$
        \item $\norme{\vec{u}-\vec{v}}^2=\norme{\vec{u}}^2-2\vec{u}\pdt\vec{v}+\norme{\vec{v}}^2$
        \item $(\vec{u}+\vec{v})\pdt(\vec{u}-\vec{v})=\norme{\vec{u}}^2-\norme{\vec{v}}^2$
    \end{enumerate}
}

\theoreme{(formules de polarisation)}{
    Pour tous vecteurs de l'espace $\vec{u}$ et $\vec{v}$ :
    \begin{enumerate}[(i)]
        \item $\vec{u}\pdt\vec{v}=\dfrac{1}{2}\Bigl(\norme{\vec{u}+\vec{v}}^2-\norme{\vec{u}}^2-\norme{\vec{v}}^2\Bigr)$
        \item $\vec{u}\pdt\vec{v}=\dfrac{1}{4}\Bigl(\norme{\vec{u}+\vec{v}}^2-\norme{\vec{u}-\vec{v}}^2\Bigr)$
    \end{enumerate}
}


\end{document}