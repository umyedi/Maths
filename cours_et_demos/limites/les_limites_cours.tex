\documentclass[11pt,a4paper]{article}
\usepackage[T1]{fontenc}
\usepackage[utf8]{inputenc}
\usepackage{amsfonts}
\usepackage{amssymb}
\usepackage{mdframed}
\usepackage{tikz}
\usepackage{tkz-tab}
\usepackage{pgfplots}
\usepackage{xcolor}
\usepackage{fancyhdr}
\usepackage{lastpage}
\usepackage[fleqn]{amsmath}
\setlength{\mathindent}{0pt}

% Spécifications du document
\newcommand{\doctitre}{Les limites} % Ex: Le second degré
\newcommand{\docniveau}{T$^{\text{le}}$ Spécialité mathématiques} % Ex: ^{\text{re}}$ Spécialité mathématiques
\newcommand{\doctheme}{Analyse} %Ex: Algèbre
\newcommand{\doctype}{Cours} % Ex: Démonstrations
\newcommand{\docshorttype}{Cours} % Démo

% Couleurs pour les graphiques
\definecolor{dark_green}{HTML}{008000}

% Paramètres du document
\RequirePackage{geometry}
\geometry{tmargin=1cm,bmargin=1.9cm,lmargin=1.9cm,rmargin=1.9cm}
\setlength{\parindent}{0pt}
\title{\doctitre}
\author{\docniveau \\ \doctheme\text{ - }\doctype}
\date{}
\fancypagestyle{custom}{
  \fancyhf{}
  \renewcommand{\headrulewidth}{0pt}
  \lfoot{\doctheme\text{ - }\docshorttype}
  \cfoot{\doctitre} % Change \titre to \doctitre
  \rfoot{\thepage/\pageref{LastPage}}
}

% Styles pour les mdframed
\mdfdefinestyle{definitionStyle}{
    leftline=true,
    rightline=false,
    topline=false,
    bottomline=false,
    linewidth=2pt,
    linecolor=black,
    innertopmargin=0pt,
    innerbottommargin=0pt,
    innerrightmargin=0pt,
    innerleftmargin=5pt,
}

\mdfdefinestyle{proprieteStyle}{
    linewidth=1pt,
    linecolor=black,
    innertopmargin=5pt,
    innerbottommargin=5pt,
    innerrightmargin=5pt,
    innerleftmargin=5pt,
}
% ----- DEBUT DU DOCUMENT -----
\begin{document}

% Style et numérotation
\maketitle
\pagestyle{custom}
\thispagestyle{custom}

\section{Limites finies d'une fonction en $+\infty$}



\begin{mdframed}[style=definitionStyle]
    \textbf{Définition :} ~\\
    Soit $f$ une fonction et $l$ un réel.
    Dire que \guillemotleft $f(x)$ tend vers $l$ quand $x$ tend vers $+\infty$ \guillemotright\text{ }signifie \\
    $\forall \varepsilon > 0\text{, } \exists A\in\mathbb{R}\text{, } \forall x>A\text{ : } l-\varepsilon < f(x) < l+\varepsilon$. \\
    On note $\displaystyle \lim_{x \to +\infty} f(x) = l$.
\end{mdframed}

\begin{mdframed}[style=proprieteStyle]
    \textbf{Théorème$^{1}$ :} ~\\
    Pour toutes fonctions $f$ et $g$ et pour tous réel $l$ et $l'$ : \\
    Si $\displaystyle\lim_{x \to +\infty} f(x) = l$ et $\displaystyle\lim_{x \to +\infty} g(x) = l'$ et $l<l'$ alors
    il existe un réel $A$ tel que pour tout $x>A$, $f(x)<g(x)$.
\end{mdframed}

\textbf{Remarque :} Une conséquence de ce théorème est que la limite d'une fonction est unique si elle existe. En effet,
si on applique ce théorème à une fonction $f$ avec elle-même, on obtient $f(x)<f(x)$ ce qui n'a pas de sens.

\begin{mdframed}[style=proprieteStyle]
    \textbf{Théorème$^{2}$ de comparaison des limites :} ~\\
    Soient $f$ et $g$ deux fonctions et $l$ et $l'$ deux réel. \\
    Si $\displaystyle\lim_{x \to +\infty} f(x) = l$ et $\displaystyle\lim_{x \to +\infty} g(x) = l'$ et il existe $A$ réel
    tel que pour tout $x>A$ : $f(x)\leq g(x)$ alors $l\leq l'$.
\end{mdframed}

\textbf{Remarque :} Attention, même si $f(x)<g(x)$, leur limites peuvent quand même être égales
(ex : $g(x)=\frac{1}{x}$ et $f(x)=\frac{-1}{x}$ tendent toutes les deux vers $0$.)

\begin{mdframed}[style=proprieteStyle]
    \textbf{Théorème$^3$ des gendarmes \emph{(admis)} :} ~\\
    Soient $f$, $g$ et $h$ trois fonctions et $l$ un réel. \\
    Si $\displaystyle\lim_{x \to +\infty} f(x) = l$ et $\displaystyle\lim_{x \to +\infty} h(x) = l$ et s'il existe $A$
    réel tel que pour tout $x>A$, $f(x)\leq g(x)\leq h(x)$ alors $\displaystyle\lim_{x \to +\infty} g(x) = l$.
\end{mdframed}

% \begin{mdframed}[style=proprieteStyle]
%     \textbf{Théorème$^4$ des limites de références :}
%     \begin{enumerate}[(i)]
%         \item $\displaystyle\lim_{x \to +\infty} \frac{1}{x} = 0$
%         \item $\displaystyle\lim_{x \to +\infty} \frac{1}{\sqrt{x}} = 0$
%         \item $\displaystyle \forall n\in\mathbb{N}^*, \lim_{x \to +\infty} \frac{1}{x^n} = 0$
%     \end{enumerate}
% \end{mdframed}


\end{document}
% ----- FIN DU DOCUMENT -----