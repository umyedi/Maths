\documentclass[10pt]{article}

\newcommand{\DocumentTitle}{Continuité}
\newcommand{\DocumentTheme}{Analyse}
\newcommand{\DocumentType}{Cours}

% Preamble Section
\usepackage[a4paper, margin=1in]{geometry} % Sets the paper size to A4 and all margins to 1 inch
\usepackage[utf8]{inputenc}   % Allows for input of international characters
\usepackage[T1]{fontenc}      % Utilisation de l'encodage T1
\usepackage{lmodern}          % Support des polices avec le module 'french'
\usepackage[french]{babel}    % Ajout du support pour le français
\usepackage[fleqn]{amsmath}   % Imports the AMS math package for advanced math formatting ([fleqn] -> align alignat to the left)
\usepackage{amsfonts}         % Imports AMS fonts for math fonts
\usepackage{amssymb}          % Imports AMS symbols for math symbols
\usepackage{amsthm}           % Imports AMS theorems
\usepackage{yhmath}           % Used for "\wideparen{}" 
\usepackage{fancyhdr}         % Import to custom the page footer
\usepackage{mdframed}         % Imports styles
\usepackage{enumerate}        % Imports styles for enumerate
\usepackage{multicol}
\usepackage{xcolor}           % Couleurs
\usepackage{tikz}             % Package général pour les graphiques
\usetikzlibrary{calc}
\usepackage{pgfplots}         % Compléments pour les graphiques
\usepackage{tkz-tab}          % Tableaux de variations
\usepackage{lastpage}         % Pour avoir le total de page dans le footer
\usepackage{makecell}         % Retour à la ligne dans une case d'un tabular
\usepackage{stmaryrd}         % Intervalles entières : \llbracket et \rrbracket
\usepackage{cancel}           % Permet de barrer des termes
\usepackage{forest}           % Permet de créer des arbres pondérés
\usepackage{tabularx}         % Permet de générer des tableaux (ex: cours fonction dérivées et application)
\usepackage{multirow}         % Permet de faire plusieurs lignes dans des cases de talbeaux


\pgfplotsset{compat=newest}   % Active les dernières fonctionnalités de pgfplots

% Enlève l'indentation des bloc de paragraphes et d'équation
\setlength{\parindent}{0pt}
\setlength{\mathindent}{0pt}

% ========== Title Section ==========

\title{\DocumentTitle} % Title of the document
\author{\DocumentTheme\space-\space\DocumentType} % Author's name
\date{} % Date

\pagestyle{fancy}
\fancyhf{} % Clear all header and footer fields

% ========== Footer Section ==========

\newcommand{\customfooter}{
    \fancyfoot[L]{\DocumentTheme\space-\space\DocumentType}
    \fancyfoot[C]{\DocumentTitle}
    \fancyfoot[R]{\thepage/\pageref{LastPage}}
    % Remove the line below the header and above the footer
    \renewcommand{\headrulewidth}{0pt}
    \renewcommand{\footrulewidth}{0pt}
}

% Apply the custom footer to the fancy style
\customfooter

% Apply the custom footer to the plain style (used on the first page)
\fancypagestyle{plain}{\customfooter}

% ========== Sections ==========

% Sections : I. / II. / III. ...
\renewcommand{\thesection}{\Roman{section}}
% Subsection : I.1 / I.2 / I.3 ...
\renewcommand{\thesubsection}{\thesection. \arabic{subsection}}
% Subsubsection : I.1.a / I.1.b / I.1.c ...
\renewcommand{\thesubsubsection}{\thesubsection. \alph{subsubsection}}

% ========== Colors (from Geogebra) ==========

\definecolor{green}{HTML}{006400}
\definecolor{red}{HTML}{CC0000}
\definecolor{blue}{HTML}{0000FF}
\definecolor{orange}{HTML}{FF5500}
\definecolor{purple}{HTML}{9933FF}
\definecolor{gray}{HTML}{666666}
\definecolor{brown}{HTML}{993300}
\definecolor{black}{HTML}{000000}
\definecolor{white}{HTML}{FFFFFF}

\newcommand{\green}[1]{\color{green}{#1}\color{black}}
\newcommand{\red}[1]{\color{red}{#1}\color{black}}
\newcommand{\blue}[1]{\color{blue}{#1}\color{black}}
\newcommand{\orange}[1]{\color{orange}{#1}\color{black}}
\newcommand{\purple}[1]{\color{purple}{#1}\color{black}}
\newcommand{\gray}[1]{\color{gray}{#1}\color{black}}
\newcommand{\brown}[1]{\color{brown}{#1}\color{black}}


% ========== Définition(s) ==========

% Style de la boîte de définition(s)
\mdfdefinestyle{DefinitionStyle}{
    leftmargin=0cm,
    rightmargin=0cm,
    linecolor=black,
    linewidth=2pt,
    topline=false,
    bottomline=false,
    rightline=false
}

% Définition de la commande '\definition'
\newcommand{\definition}[2]{%
    \begin{mdframed}[style=DefinitionStyle]
        \ifstrempty{#1}{% Teste si le premier argument est vide
            \textbf{Définition :}\ % Si vide, n'affiche pas de titre de théorème
        }{%
            \textbf{Définition #1 :}\ % Si non vide, affiche le titre avec le premier argument
        }\\#2
    \end{mdframed}
}

% Définition de la commande '\definitions'
\newcommand{\definitions}[2]{%
    \begin{mdframed}[style=DefinitionStyle]
        \ifstrempty{#1}{% Teste si le premier argument est vide
            \textbf{Définitions :}\ % Si vide, n'affiche pas de titre de théorème
        }{%
            \textbf{Définitions #1 :}\ % Si non vide, affiche le titre avec le premier argument
        }\\#2
    \end{mdframed}
}

% ========== Propriété(s) et Théorème(s) ==========

% Style de la boîte de propriété(s) / théorème(s)
\mdfdefinestyle{ProprieteStyle}{
    leftmargin=0cm,
    rightmargin=0cm,
    linecolor=black,
    linewidth=1pt
}

% Création d'un compteur pour les propriétés
\newcounter{propriete}

% Définition de la commande '\propriete'
\newcommand{\propriete}[2]{%
    \refstepcounter{propriete}% Incrémente le compteur de théorème
    \begin{mdframed}[style=ProprieteStyle]
        \ifstrempty{#1}{% Teste si le premier argument est vide
            \textbf{\thepropriete. Propriété :}\ % Si vide, n'affiche pas de titre de théorème
        }{%
            \textbf{\thepropriete. Propriété #1 :}\ % Si non vide, affiche le titre avec le premier argument
        }\\#2
    \end{mdframed}
}

% Définition de la commande 'proprietes'
\newcommand{\proprietes}[2]{%
    \refstepcounter{propriete}% Incrémente le compteur de théorème
    \begin{mdframed}[style=ProprieteStyle]
        \ifstrempty{#1}{% Teste si le premier argument est vide
            \textbf{\thepropriete. Propriétés :}\ % Si vide, n'affiche pas de titre de théorème
        }{%
            \textbf{\thepropriete. Propriétés #1 :}\ % Si non vide, affiche le titre avec le premier argument
        }\\#2
    \end{mdframed}
}

% Création d'un compteur pour les théorèmes
\newcounter{theoreme}

% Définition de la commande '\theoreme'
\newcommand{\theoreme}[2]{%
    \refstepcounter{theoreme}% Incrémente le compteur de théorème
    \begin{mdframed}[style=ProprieteStyle]
        \ifstrempty{#1}{% Teste si le premier argument est vide
            \textbf{\thetheoreme. Théorème :}\ % Si vide, n'affiche pas de titre de théorème
        }{%
            \textbf{\thetheoreme. Théorème #1 :}\ % Si non vide, affiche le titre avec le premier argument
        }\\#2
    \end{mdframed}
}

% Définition de la commande '\theoremes'
\newcommand{\theoremes}[2]{%
    \refstepcounter{theoreme}% Incrémente le compteur de théorème
    \begin{mdframed}[style=ProprieteStyle]
        \ifstrempty{#1}{% Teste si le premier argument est vide
            \textbf{\thetheoreme. Théorèmes :}\ % Si vide, n'affiche pas de titre de théorème
        }{%
            \textbf{\thetheoreme. Théorèmes #1 :}\ % Si non vide, affiche le titre avec le premier argument
        }\\#2
    \end{mdframed}
}

\newcommand{\corrolaire}[2]{%
    \begin{mdframed}[style=ProprieteStyle]
        \ifstrempty{#1}{% Teste si le premier argument est vide
            \textbf{Corrolaire :}\ % Si vide, n'affiche pas de titre de théorème
        }{%
            \textbf{Corrolaire #1 :}\ % Si non vide, affiche le titre avec le premier argument
        }\\#2
    \end{mdframed}
}

% ========== Exemple(s) ==========

% Définition de la commande 'exemple'
\newcommand{\exemple}[2]{
    \textbf{Exemple :} #1
    \begin{quote}
        #2
    \end{quote}
}

% Définition de la commande 'exemples'
\newcommand{\exemples}[2]{
    \textbf{Exemples :} #1
    \begin{quote}
        #2
    \end{quote}
}

% ========== Remarques ==========

% Définition de la commande 'remarque'
\newcommand{\remarque}[1]{
    \textbf{Remarque :} #1
}

% Définition de la commande 'remarques'
\newcommand{\remarques}[1]{
    \textbf{Remarques :}
    \begin{quote}
        #1
    \end{quote}
}

% ========== Démonstration ==========

\newcommand{\demonstration}[2]{
    \ifstrempty{#1}{% Teste si le premier argument est vide
        \textbf{Démonstration :}\ % Si vide, n'affiche pas de titre de théorème
    }{%
        \textbf{Démonstration #1 :}\ % Si non vide, affiche le titre avec le premier argument
    }
    \begin{quote}
        #2
    \end{quote}
}
% ========== Autres blocs ==========

\newcommand{\newbloc}[2]{
    \textbf{#1}
    \begin{quote}
        #2
    \end{quote}
}

% ========== Notations ==========

% Style
\newcommand{\ds}{\displaystyle}
\newcommand{\llb}{\llbracket}
\newcommand{\rrb}{\rrbracket}

% Ensembles
\newcommand{\C}{\mathbb{C}}
\newcommand{\R}{\mathbb{R}}
\newcommand{\Q}{\mathbb{Q}}
\newcommand{\D}{\mathbb{D}}
\newcommand{\Z}{\mathbb{Z}}
\newcommand{\N}{\mathbb{N}}

% Limites
\let\oldlim\lim
\renewcommand{\lim}[1]{\mathop{\displaystyle\oldlim}\limits_{#1}}

% Opérateurs
\newcommand{\x}{\times}
\newcommand{\equival}{\Leftrightarrow}
\newcommand{\involve}{\Rightarrow}

% Ensembles
\renewcommand{\bar}[1]{\overline{#1}}
\renewcommand{\part}[1]{\mathcal{P}({#1})}  % Ensemle des parties de E
\newcommand{\rel}{\mathcal{R}} % Relation entre deux ensembles

% Vecteurs
\renewcommand{\Vec}[1]{\overrightarrow{#1}}
\newcommand*{\norme}[1]{\|#1\|}
\newcommand{\pdt}{\mathbin{\vcenter{\hbox{\scalebox{0.6}{\textbullet}}}}}
\newcommand{\vcoord}[2]{\begin{pmatrix} #1 \\ #2 \end{pmatrix}}
\newcommand{\Vcoord}[3]{\begin{pmatrix} #1 \\ #2 \\ #3 \end{pmatrix}}

% Fonctions
\newcommand{\aire}[1]{\text{aire}\left( #1 \right)}
\newcommand{\Int}{\ds\int}


% ========== Systèmes ==========
\newcommand{\sys}[2]{\begin{cases}#1\\#2\end{cases}}
\newcommand{\Sys}[3]{\begin{cases}#1\\#2\\#3\end{cases}}

% ========== Tableaux ==========
\newcommand{\boldhline}{\hline\noalign{\vskip 0pt}\hline}

% ========== Graphiques ==========

% Triange de Pascal
\newcommand{\binomial}[2]{\pgfmathparse{int((#1)!/((#2)!*((#1)-(#2))!))}\pgfmathresult}
\newcommand{\Pascal}[1]{
    \begin{tikzpicture}[scale=1, every node/.style={scale=1.2}]
        \foreach \n in {0,...,#1} {
                \foreach \k in {0,...,\n} {
                        \node at (\k-\n/2,-\n) {$\binomial{\n}{\k}$};
                        \draw (\k-\n/2,-\n) +(-0.5,-0.5) rectangle ++(0.5,0.5);
                    }
            }
    \end{tikzpicture}
}

\newcommand{\Pascalbinome}[1]{
    \begin{tikzpicture}[scale=1, every node/.style={scale=0.9}]
        \foreach \n in {0,...,#1} {
                \foreach \k in {0,...,\n} {
                        \pgfmathtruncatemacro{\symk}{min(\k,\n-\k)}
                        \node at (\k-\n/2,-\n) {$\ds\binom{\n}{\symk}$};
                        \draw (\k-\n/2,-\n) +(-0.5,-0.5) rectangle ++(0.5,0.5);
                    }
            }
    \end{tikzpicture}
}

\begin{document}

\maketitle

\section{Fonctions continues}

\subsection{Fonction continue en un réel}

\definitions{}{
	Soient $a$ un réel, $I$ un intervalle contenant $a$ et $f$ une fonction définie sur $I$.
	On dit que $f$ est continue en $a$ si les limites à droite et à gauche, quand $x$ tend vers $a$, de $f(x)$ existent
	et sont toutes égales à $f(a)$, autrement dit, si $\lim{x\to a^+}f(x)=\lim{x\to a^-}f(x)=f(a)$.\\
	Dans le cas contraire, on dit que $f$ admet une discontinuité en $a$.
}

\remarque{Pour indiquer que les limites à droite et à gauche, quand $x$ tend vers $a$, de $f(x)$ existent et sont égales
	à $f(a)$, on écrira simplement $\lim{x\to a}f(x)=f(a)$.}

\exemple{}{
	\begin{enumerate}[(1)]
		\item La fonction valeur absolue est définie pour tout réel $x$ par $\|x\|=
			      \begin{cases}
				      x  & \text{si } x\geq0 \\
				      -x & \text{si } x<0
			      \end{cases}$

		      Quand $x$ tend vers $0$ :
		      \begin{itemize}
			      \item la limite à droite est $\lim{x\to0^+}\|x\|=\lim{x\to0}x=0$,
			      \item la limite à gauche est $\lim{x\to0^-}\|x\|=\lim{x\to0}-x=0$.
			      \item De plus, l'image de $0$ est $\|0\|=0$.\\
		      \end{itemize}
		      La fonction valeur absolue est donc continue en $0$.
		\item La partie entière d'un réel $x$ est par définition l'unique entier relatif $n$ tel que $n\leq x<n+1$. On
		      la note $E(x)$. Pour tout entier relatif $a$, la fonction partie entière admet une discontinuité en $a$. \\
		      Démontrons par exemple que la fonction partie entière n'est pas continue en $a=2$.
		      \begin{itemize}
			      \item Si $1\leq x<2$ alors $E(x)=1$ dont la limite à gauche, quand $x$ tend vers $2$, de la fonction partie
			            entière est $\lim{x\to2^-}=1$.
			      \item Si $2\leq x<3$ alors $E(x)=2$ dont la limite à droite, quand $x$ tend vers $2$, de la fonction partie
			            entière est $\lim{x\to2^+}=2$.
		      \end{itemize}
		      La limite à gauche et la limité à droite sont différentes : cette fonction n'est donc pas continue en $2$. La
		      démonstration serait identique pour n'importe quel entier relatif.
	\end{enumerate}
}

\subsection{Fonction continue sur un intervalle}

\definitions{}{
	Soient $I$ un intervalle et $f$ une fonction définie sur $I$. On dit que $f$ est continue sur $I$ si pour tout réel
	$a$ appartenant à $I$, $f$ est continue en $a$.
}

\theoreme{(continuité des fonctions usuelles)}{
	\vspace{-10pt}
	\begin{enumerate}
		\item Les fonctions affines, polynômes, racine carrée, valeur absolue, cosinus, sinus, exponentielles  et
		      logarithmes, sont continues sur chaque intervalle où elles sont définies.
		\item Toute fonction construite à partir des précédentes par somme, produit, quotient ou composition, est
		      continue sur chaque intervalle où elle est définie.
	\end{enumerate}
}

\newpage

\exemple{}{
	\begin{itemize}
		\item La fonction $f$ définie pour tout réel $x$ par $f(x)=e^{3x-5}$, est continue sur $\R$ car elle est obtenue
		      par la composition d'une fonction affine et de la fonction exponentielle.
		\item La fonction $f$ définie pour tout réel $x$ par $g(x)=\dfrac{\sqrt{x}}{x^2-1}$ est continue sur $]0;1[$ et
		      sur $]1;+\infty[$ car elle est obtenue par quotient de la fonction racine carrée et d'une fonction polynôme.
	\end{itemize}
}

\theoreme{(continuité des fonctions dérivables)}{
	Soient $I$ un intervalle et $f$ une fonction définie sur $I$. Si $f$ est dérivable sur $I$ alors $f$ est continue
	sur $I$.
}

\section{Théorème des valeurs intermédiaires}

\theoreme{des valeurs intermédiaires (TVI)}{
	Soit $f$ une fonction définie sur un intervalle $I$. Soient $a$ et $b$ deux réels dans $I$. Pour tout $k$ réel, si
	$f$ est continue sur $I$ et si $f(a)<k<f(b)$ alors il existe au moins un réel $c$ dans l'intervalle $[a;b]$ tel que
	$f(x)=k$.
}

\definitions{}{
	Soient $I$ et $J$ deux intervalles. Soit $f:I\mapsto J$. On dit que $f$ est une bijection de $I$ dans $J$ si tout
	réel de $J$ admet un unique antécédent dans $I$.
}

\exemple{}{
	\begin{itemize}
		\item Une fonction affine non constante est une bijection de $\R$ dans $\R$.
		\item La fonction exponentielle est une bijection de $\R$ dans $]0;+\infty[$.
	\end{itemize}
}

\theoreme{de la bijection (corrolaire)}{
	Soit $f$ une fonction définie sur un intervalle $I$. Soient $a$ et $b$ deux réels dans $I$. Il existe un réel $k$
	tel que si $f$ est continue et strictement croissante et $f(a)<k<f(b)$ alors il existe un unique réel $c$ dans
	l'intervalle $[a;b]$ tel que $f(c)=k$.
}

\remarque{Le théorème de la bijection est également valable pour une fonction strictement décroissante.}

\exemple{démontrons l'existence d'une solution pour l'équation $\cos(x)=x$}{

Comme $-1\leq\cos(x)\leq1$, les solutions éventuelles sont à chercher dans l'intervalle $[-1;1]$.

Si $\dfrac{-\pi}{2}<x<0$, alors $\cos(x)>0$ et donc $\cos(x)\not=x$ ce qui veut dire qu'on peut réduire l'intervalle
de recherche à $[0;1]$.

Étudions la fonction $f(x)=\cos(x)-x$ sur $[0;1]$ : $f'(x)=-\sin(x)-1<0$ car $\forall x\in[0;1]: \sin(x)>0$ donc $f$
est strictement décroissante sur $[0;1]$. De plus, $f$ est continue car c'est une somme de fonctions de référence.

Aussi, $f(0)=1>0$ et $f(1)=\cos(1)-1<0$ car $\cos(1)<0$

Conclusion : d'après le théorème de la bijection, l'équation $f(x)=0$ admet donc une unique solution dans
l'intervalle $]0;1[$. Donc $\cos(x)=x$ admet une unique solution réelle.
}

\section{Limite d'une suite par récurrence}

\theoreme{}{
	Soit $f$ une fonction définie sur un intervalle $I$. Soit $a\in I$. Si $f$ est continue en $a$, alors pour toute
	suite $u_n$ convergeant vers $a$, la suite $(f(u_n))$ converge vers $a$.
}

\theoreme{du point fixe}{
	Soif $f$ une fonction définie sur $I$. Soit $u_n$ la suite définie par récurrence telle que pour tout $n\in\N :
		u_n\in I$ et $u_{n+1}=f(u_n)$. Si $u_n$ converge vers un réel $l$, et $f$ est continue en $l$ alors $f(l)=l$.
}

\exemple{
	Soit $(u_n):
		\begin{cases}
			u_n=0 \\
			\forall n\in\N : u_{n+1}=\sqrt{\dfrac{u_n}{2}+1}
		\end{cases}$
}{
	On peut prouver que $u_n$ est croissante et majorée par $2$ donc $u_n$ converge vers un réel $l$.

	La fonction $f(x)=\sqrt(\dfrac{x}{2}+1)$ étant continue sur l'intervalle où elle est définie, d'après le théorème du
	point fixe :
	\begin{align*}
		f(l)=l & \Leftrightarrow l=\sqrt{\dfrac{l}{2}+1} \\
		       & \Leftrightarrow l^2=\dfrac{l}{2}+1 \qquad \text{avec } l>0 \\
		       & \Leftrightarrow 2l^2-l-2=0  \qquad \text{avec } l>0\\
               & \text{ On calcule} \quad \Delta=17>0 \quad \text{donc} \quad x_1=\dfrac{1-\sqrt{17}}{4}<0 \quad;\quad  x_2=\dfrac{1+\sqrt{17}}{4}>0 \\
               & \Leftrightarrow l=x_2=\dfrac{1+\sqrt{17}}{4}
	\end{align*}

    Conclusion $\ds\lim{n\to+\infty}u_n=\frac{1+\sqrt{17}}{4}$.
}






\end{document}