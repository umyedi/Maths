\documentclass[10pt]{article}

\newcommand{\DocumentTitle}{Titre du document}
\newcommand{\DocumentTheme}{Thème}
\newcommand{\DocumentType}{Cours}

% Preamble Section
\usepackage[a4paper, margin=1in]{geometry} % Sets the paper size to A4 and all margins to 1 inch
\usepackage[utf8]{inputenc}   % Allows for input of international characters
\usepackage[T1]{fontenc}      % Utilisation de l'encodage T1
\usepackage{lmodern}          % Support des polices avec le module 'french'
\usepackage[french]{babel}    % Ajout du support pour le français
\usepackage[fleqn]{amsmath}   % Imports the AMS math package for advanced math formatting ([fleqn] -> align alignat to the left)
\usepackage{amsfonts}         % Imports AMS fonts for math fonts
\usepackage{amssymb}          % Imports AMS symbols for math symbols
\usepackage{amsthm}           % Imports AMS theorems
\usepackage{fancyhdr}         % Import to custom the page footer
\usepackage{mdframed}         % Imports styles
\usepackage{enumerate}        % Imports styles for enumerate
\usepackage{multicol}
\usepackage{xcolor}           % Couleurs
\usepackage{tikz}             % Package général pour les graphiques
\usepackage{pgfplots}         % Compléments pour les graphiques
\usepackage{tkz-tab}          % Tableaux de variations
\usepackage{lastpage}         % Pour avoir le total de page dans le footer
\usepackage{makecell}         % Retour à la ligne dans une case d'un tabular
\usepackage{stmaryrd}         % Intervalles entières : \llbracket et \rrbracket

\pgfplotsset{compat=newest}   % Active les dernières fonctionnalités de pgfplots

% Enlève l'indentation des bloc de paragraphes et d'équation
\setlength{\parindent}{0pt}
\setlength{\mathindent}{0pt}

% ========== Title Section ==========

\title{\DocumentTitle} % Title of the document
\author{\DocumentTheme\space-\space\DocumentType} % Author's name
\date{} % Date

\pagestyle{fancy}
\fancyhf{} % Clear all header and footer fields

% ========== Footer Section ==========

\newcommand{\customfooter}{
    \fancyfoot[L]{\DocumentTheme\space-\space\DocumentType}
    \fancyfoot[C]{\DocumentTitle}
    \fancyfoot[R]{\thepage/\pageref{LastPage}}
    % Remove the line below the header and above the footer
    \renewcommand{\headrulewidth}{0pt}
    \renewcommand{\footrulewidth}{0pt}
}

% Apply the custom footer to the fancy style
\customfooter

% Apply the custom footer to the plain style (used on the first page)
\fancypagestyle{plain}{\customfooter}

% ========== Sections ==========

% Sections : I. / II. / III. ...
\renewcommand{\thesection}{\Roman{section}}
% Subsection : I.1 / I.2 / I.3 ...
\renewcommand{\thesubsection}{\thesection. \arabic{subsection}}
% Subsubsection : I.1.a / I.1.b / I.1.c ...
\renewcommand{\thesubsubsection}{\thesubsection. \alph{subsubsection}}

% ========== Colors (from Geogebra) ==========

\definecolor{green}{HTML}{006400}
\definecolor{red}{HTML}{CC0000}
\definecolor{blue}{HTML}{0000FF}
\definecolor{orange}{HTML}{FF5500}
\definecolor{purple}{HTML}{9933FF}
\definecolor{gray}{HTML}{666666}
\definecolor{brown}{HTML}{993300}

\newcommand{\green}[1]{\color{green}{#1}\color{black}}
\newcommand{\red}[1]{\color{red}{#1}\color{black}}
\newcommand{\blue}[1]{\color{blue}{#1}\color{black}}
\newcommand{\orange}[1]{\color{orange}{#1}\color{black}}
\newcommand{\purple}[1]{\color{purple}{#1}\color{black}}
\newcommand{\gray}[1]{\color{gray}{#1}\color{black}}
\newcommand{\brown}[1]{\color{brown}{#1}\color{black}}


% ========== Définition(s) ==========

% Style de la boîte de définition(s)
\mdfdefinestyle{DefinitionStyle}{
    leftmargin=0cm,
    rightmargin=0cm,
    linecolor=black,
    linewidth=2pt,
    topline=false,
    bottomline=false,
    rightline=false
}

% Définition de la commande '\definition'
\newcommand{\definition}[2]{%
    \begin{mdframed}[style=DefinitionStyle]
        \ifstrempty{#1}{% Teste si le premier argument est vide
            \textbf{Définition :}\ % Si vide, n'affiche pas de titre de théorème
        }{%
            \textbf{Définition #1 :}\ % Si non vide, affiche le titre avec le premier argument
        }\\#2
    \end{mdframed}
}

% Définition de la commande '\definitions'
\newcommand{\definitions}[2]{%
    \begin{mdframed}[style=DefinitionStyle]
        \ifstrempty{#1}{% Teste si le premier argument est vide
            \textbf{Définitions :}\ % Si vide, n'affiche pas de titre de théorème
        }{%
            \textbf{Définitions #1 :}\ % Si non vide, affiche le titre avec le premier argument
        }\\#2
    \end{mdframed}
}

% ========== Propriété(s) et Théorème(s) ==========

% Style de la boîte de propriété(s) / théorème(s)
\mdfdefinestyle{ProprieteStyle}{
    leftmargin=0cm,
    rightmargin=0cm,
    linecolor=black,
    linewidth=1pt
}

% Création d'un compteur pour les propriétés
\newcounter{propriete}

% Définition de la commande '\propriete'
\newcommand{\propriete}[2]{%
    \refstepcounter{propriete}% Incrémente le compteur de théorème
    \begin{mdframed}[style=ProprieteStyle]
        \ifstrempty{#1}{% Teste si le premier argument est vide
            \textbf{\thepropriete. Propriété :}\ % Si vide, n'affiche pas de titre de théorème
        }{%
            \textbf{\thepropriete. Propriété #1 :}\ % Si non vide, affiche le titre avec le premier argument
        }\\#2
    \end{mdframed}
}

% Définition de la commande 'proprietes'
\newcommand{\proprietes}[2]{%
    \refstepcounter{propriete}% Incrémente le compteur de théorème
    \begin{mdframed}[style=ProprieteStyle]
        \ifstrempty{#1}{% Teste si le premier argument est vide
            \textbf{\thepropriete. Propriétés :}\ % Si vide, n'affiche pas de titre de théorème
        }{%
            \textbf{\thepropriete. Propriétés #1 :}\ % Si non vide, affiche le titre avec le premier argument
        }\\#2
    \end{mdframed}
}



% Création d'un compteur pour les théorèmes
\newcounter{theoreme}

% Définition de la commande '\theoreme'
\newcommand{\theoreme}[2]{%
    \refstepcounter{theoreme}% Incrémente le compteur de théorème
    \begin{mdframed}[style=ProprieteStyle]
        \ifstrempty{#1}{% Teste si le premier argument est vide
            \textbf{\thetheoreme. Théorème :}\ % Si vide, n'affiche pas de titre de théorème
        }{%
            \textbf{\thetheoreme. Théorème #1 :}\ % Si non vide, affiche le titre avec le premier argument
        }\\#2
    \end{mdframed}
}

% Définition de la commande '\theoremes'
\newcommand{\theoremes}[2]{%
    \refstepcounter{theoreme}% Incrémente le compteur de théorème
    \begin{mdframed}[style=ProprieteStyle]
        \ifstrempty{#1}{% Teste si le premier argument est vide
            \textbf{\thetheoreme. Théorèmes :}\ % Si vide, n'affiche pas de titre de théorème
        }{%
            \textbf{\thetheoreme. Théorèmes #1 :}\ % Si non vide, affiche le titre avec le premier argument
        }\\#2
    \end{mdframed}
}

% ========== Exemple(s) ==========

% Définition de la commande 'exemple'
\newcommand{\exemple}[2]{
    \textbf{Exemple :} #1
    \begin{quote}
        #2
    \end{quote}
}

% Définition de la commande 'exemples'
\newcommand{\exemples}[2]{
    \textbf{Exemples :} #1
    \begin{quote}
        #2
    \end{quote}
}

% ========== Remarques ==========

% Définition de la commande 'remarque'
\newcommand{\remarque}[1]{
    \textbf{Remarque :} #1
}

% Définition de la commande 'remarques'
\newcommand{\remarques}[1]{
    \textbf{Remarques :} #1
}

% ========== Démonstration ==========

\newcommand{\demonstration}[2]{
    \textbf{Démonstration :} #1
    \begin{quote}
        #2
    \end{quote}
}

% ========== Notations ==========

% Ensembles
\newcommand{\C}{\mathbb{C}}
\newcommand{\R}{\mathbb{R}}
\newcommand{\Q}{\mathbb{Q}}
\newcommand{\D}{\mathbb{D}}
\newcommand{\Z}{\mathbb{Z}}
\newcommand{\N}{\mathbb{N}}

% Limites
\let\oldlim\lim
\renewcommand{\lim}[1]{\mathop{\displaystyle\oldlim}\limits_{#1}}

% Opérateurs
\newcommand{\x}{\times}
\newcommand{\equival}{\Leftrightarrow}

% Vecteurs
\renewcommand{\Vec}[1]{\overrightarrow{#1}}
\newcommand*{\norme}[1]{\left\lVert\vv{#1}\right\rVert}
\newcommand{\ps}[2]{\ensuremath{\vv{#1}.\vv{#2}}}

\newcommand{\vcoord}[2]{\begin{pmatrix} #1 \\ #2 \end{pmatrix}}
\newcommand{\Vcoord}[3]{\begin{pmatrix} #1 \\ #2 \\ #3 \end{pmatrix}}

% ========== Tableaux ==========
\newcommand{\boldhline}{\hline\noalign{\vskip 0pt}\hline}

\begin{document}

\maketitle

\section{I. Équation cartésienne d'une droite et vecteur normal}

\definition{}{
  Soit $d$ une droite de vecteur directeur $\vec{u}$. \\
  Un vecteur normal à la droite $(d)$ est un vecteur non nul orthogonal au vecteur $\vec{u}$.
}

\textbf{Schéma :}

\begin{tikzpicture}
  % Points
  \coordinate (A) at (2,0.5);
  \coordinate (M) at (6,6/4);

  % Droite (d)
  \draw[thick] (0,0) -- (8,2) node[pos=1, above left] {$d$};

  % Points marked with cross
  \draw[green] (A) ++(-0.1,-0.1) -- ++(0.2,0.2);
  \draw[green] (A) ++(-0.1,0.1) -- ++(0.2,-0.2);
  \node[green, below left] at (A) {$A$};

  \draw[green] (M) ++(-0.1,-0.1) -- ++(0.2,0.2);
  \draw[green] (M) ++(-0.1,0.1) -- ++(0.2,-0.2);
  \node[green, above left] at (M) {$M$};

  % Vecteur u et n
  \draw[thick, red, ->] (4,0.5) -- (5,0.75) node[midway, below] {$\vec{u}$};
  \draw[thick, blue, ->] (3.75,1.25) -- (3.5,2.25) node[midway, left] {$\vec{n}$};

  % Texte
  \node at (9,1) {$\blue{\vec{n}}$ est un vecteur normal};
  \node at (8.20,0.5) {à la droite $(d)$};
\end{tikzpicture}

\propriete{}{
  Soient $a$, $b$ et $c$ trois réels tels que $(a,b)\not=(0,0)$. \\
  Dans un repère orthonormé, le vecteur $\vec{n}\vcoord{a}{b}$ est normal à la droite $(d)$ si et seulement si la droite
  admet une équation cartésienne de la forme $ax+by+c=0$ avec $c$ un réel à déterminer.
}


\exemple{}{
  On cherche à déterminer une équation cartésienne de la droite $(d)$ passant par le point $A(5;-1)$ et de vecteur normal $\displaystyle \vec{n} \vcoord{2}{-3}$.
  \begin{equation*}
    \begin{split}
      M(x;y)\in (d) & \equival \Vec{AM} \vcoord{x-5}{y+1} \text{ et } \vec{n} \vcoord{2}{-3} \text{ sont orthogonaux}\\
      & \equival \Vec{AM}\pdt\vec{n} = 0 \\
      & \equival 2(x-5)-3(y+1)=0 \\
      & \equival 2x-10-3y-3=0 \\
      & \equival 2x-3y-13=0
    \end{split}
  \end{equation*}
  Donc une équation cartésienne de $(d)$ est $2x-3y-13=0$.
}

\section{II. Équation cartésienne d'un cercle}

\definition{}{
  On appelle cercle de centre $\Omega$ et de rayon $r>0$ l'ensemble des points $M$ du plan qui vérifie $\Omega M=r$.
}


\propriete{de l'équation d'un cercle connaissant son centre et son rayon}{
  Le plan est muni d'un repère orthonormé $(O,I,J)$. Soit $C$ le cercle de centre $\Omega(x_0;y_0)$ et de rayon $R$.
  Une équation du cercle $C$ est $(x-x_0)^2+(y-y_0)^2=R^2$.
}

\newpage

\exemple{}{
  \begin{minipage}{0.3\textwidth}
    \begin{tikzpicture}[scale=0.8]
      \draw[help lines, gray!30, step=1] (-2,-1) grid (6,7);
      \draw[thick, color=red] (2,3) circle [radius=3];
      \draw[thick,->] (-2,0) -- (6,0) node[above left] {$x$};
      \draw[thick,->] (0,-1) -- (0,7) node[below left] {$y$};
      \draw[thick] (1,0.1) -- (1,-0.1) node[below] {$I$};
      \draw[thick] (0.1,1) -- (-0.1,1) node[left] {$J$};
      \draw[green,thick] (2,3) -- (5,3) node[midway, above] {$3$};
      \draw[thin] (-0.05,-0.05) -- (0.05,0.05) node[below left] {$O$};
      \draw[thin] (-0.05,0.05) -- (0.05,-0.05);
      \coordinate (A) at (2,3);
      \draw[ultra thin, line width=0.4pt] (A) ++(-0.05,-0.05) -- ++(0.1,0.1) (A) ++(-0.05,0.05) -- ++(0.1,-0.1) node[anchor=east] {\small$A$};
    \end{tikzpicture}
  \end{minipage}
  \hfill
  \begin{minipage}{0.6\textwidth}
    On cherche à déterminer l'équation du cercle de centre $A(2;3)$ et de rayon $3$.
    \begin{equation*}
      \begin{split}
        M(x;y)\in C&\equival AM = 3 \\
        & \equival AM^2=9 \text{ avec } AM=\sqrt{(x-2)^2+(y-3)^2} \\
        & \equival (x-2)^2+(y-3)^2=9 \\
        & \equival x^2-4x+4+y^2-6x+9=9 \\
        & \equival x^2+y^2-4x-6x+4=0 \\
      \end{split}
    \end{equation*}
    Une équation cartésienne du cercle de centre $A(2;3)$ et de rayon $3$ est $(x-2)^2+(y-3)^2=9$ ou $x^2+y^2-4x-6x+4=0$. ~\\ ~\\
  \end{minipage}
}

\begin{minipage}{0.8\textwidth}
  \propriete{de l'équation d'un cercle connaissant son diamètre}{
      Soit $C$ le cercle de diamètre $[AB]$. \\
      Un point $M(x;y)$ appartient au cercle $C$ si et seulement si $\Vec{MA}\pdt\Vec{MB}=0$.
      Une équation cartésienne du cercle $C$ est donc $(x-x_A)(x-x_B)+(y-y_A)(y-y_B)=0$.
  }
\end{minipage}
\hfill
\begin{minipage}{0.17\textwidth}
  \begin{tikzpicture}[scale=0.5,>=stealth]

    % Définir les coordonnées des points A et B
    \coordinate (A) at (0,0);
    \coordinate (B) at (4,0);

    % Calculer le milieu du segment AB et le rayon du cercle
    \coordinate (M) at ($(A)!0.5!(B)$);
    \pgfmathsetmacro{\radius}{0.485*sqrt(17)}

    % Dessiner un arc de cercle pour représenter le lieu des points possibles pour le point C
    \draw[black, thick] (M) circle (\radius);

    % Choisir un angle pour positionner C sur l'arc de cercle
    \pgfmathsetmacro{\angle}{120}
    \coordinate (C) at ($(M)+(\angle:\radius)$);

    % Dessinez les vecteurs AB et AC
    \draw[red, thick] (A) -- (B) node[midway,anchor=north east] {};
    \draw[green, thick] (A) -- (C) node[midway,anchor=south east] {};
    \draw[green, thick] (C) -- (B) node[midway,anchor=south west] {};

    % % Ajouter une marque d'angle droit en M
    \draw[green, thick] (0.92, 1.575) -- (1.06, 1.5) -- (1.15, 1.65);

    % Marquer les points A, B et M avec des croix plus petites et plus épaisses
    \draw[ultra thin, line width=0.4pt] (A) ++(-0.05,-0.05) -- ++(0.1,0.1) (A) ++(-0.05,0.05) -- ++(0.1,-0.1) node[anchor=east] {\small$A$};
    \draw[ultra thin, line width=0.4pt] (B) ++(-0.05,-0.05) -- ++(0.1,0.1) (B) ++(-0.05,0.05) -- ++(0.1,-0.1) node[anchor=south west] {\small$B$};
    \draw[ultra thin, line width=0.4pt] (C) ++(-0.05,-0.05) -- ++(0.1,0.1) (C) ++(-0.05,0.05) -- ++(0.1,-0.1) node[anchor=south east] {\small$M$};
  \end{tikzpicture}
\end{minipage}

\section{III. Équation cartésienne d'une parabole}

\definition{}{
  Soit $a$, $b$ et $c$ trois réels tels que $a\not=0$. \\
  Soit $f$ une fonction polynôme du second degré définie par $f(x)=ax^2+bx+c$. \\
  La courbe représentative de la fonction $f$ qui a pour équation $y=ax^2+bx+c$ est une parabole.
}


\propriete{}{
  Cette courbe représentative admet pour axe de symétrie de la droite d'équation $\displaystyle x=\frac{-b}{2a}$ et pour sommet le point $\displaystyle S\left(\frac{-b}{2a};f\left(\frac{-b}{2a}\right)\right)$.
}

\exemple{}{
  On cherche à déterminer le sommet et l'axe de symétrie de la parabole d'équation $y=-x^2+2x-5$. \\
  \begin{minipage}{0.5\textwidth}
    Axe de symétrie : \\
    On a $a=-1$, $b=2$ et $c=-5$. \\
    On calcule $\displaystyle x=\frac{-b}{2a}=\frac{-2}{2(-1)}=1$. \\
    Donc l'axe de symétrie de la parabole est $x=1$. \\
  \end{minipage}
  \hfill
  \begin{minipage}{0.5\textwidth}
    Sommet : \\
    Donc le sommet a pour abscisse $1$. \\
    Son ordonnée est $y=-1^2+2\times1-5=-4$. \\
    Donc $S\left(1;-4\right)$
    \vspace*{20pt}
  \end{minipage}
}



\end{document}
% ----- FIN DU DOCUMENT -----