\documentclass[10pt]{article}

\newcommand{\DocumentTitle}{Nombres complexes}
\newcommand{\DocumentTheme}{Algèbre}
\newcommand{\DocumentType}{Cours}

% Preamble Section
\usepackage[a4paper, margin=1in]{geometry} % Sets the paper size to A4 and all margins to 1 inch
\usepackage[utf8]{inputenc}   % Allows for input of international characters
\usepackage[T1]{fontenc}      % Utilisation de l'encodage T1
\usepackage{lmodern}          % Support des polices avec le module 'french'
\usepackage[french]{babel}    % Ajout du support pour le français
\usepackage[fleqn]{amsmath}   % Imports the AMS math package for advanced math formatting ([fleqn] -> align alignat to the left)
\usepackage{amsfonts}         % Imports AMS fonts for math fonts
\usepackage{amssymb}          % Imports AMS symbols for math symbols
\usepackage{amsthm}           % Imports AMS theorems
\usepackage{yhmath}           % Used for "\wideparen{}" 
\usepackage{fancyhdr}         % Import to custom the page footer
\usepackage{mdframed}         % Imports styles
\usepackage{enumerate}        % Imports styles for enumerate
\usepackage{multicol}
\usepackage{xcolor}           % Couleurs
\usepackage{tikz}             % Package général pour les graphiques
\usetikzlibrary{calc}
\usepackage{pgfplots}         % Compléments pour les graphiques
\usepackage{tkz-tab}          % Tableaux de variations
\usepackage{lastpage}         % Pour avoir le total de page dans le footer
\usepackage{makecell}         % Retour à la ligne dans une case d'un tabular
\usepackage{stmaryrd}         % Intervalles entières : \llbracket et \rrbracket
\usepackage{cancel}           % Permet de barrer des termes
\usepackage{forest}           % Permet de créer des arbres pondérés
\usepackage{tabularx}         % Permet de générer des tableaux (ex: cours fonction dérivées et application)
\usepackage{multirow}         % Permet de faire plusieurs lignes dans des cases de talbeaux


\pgfplotsset{compat=newest}   % Active les dernières fonctionnalités de pgfplots

% Enlève l'indentation des bloc de paragraphes et d'équation
\setlength{\parindent}{0pt}
\setlength{\mathindent}{0pt}

% ========== Title Section ==========

\title{\DocumentTitle} % Title of the document
\author{\DocumentTheme\space-\space\DocumentType} % Author's name
\date{} % Date

\pagestyle{fancy}
\fancyhf{} % Clear all header and footer fields

% ========== Footer Section ==========

\newcommand{\customfooter}{
    \fancyfoot[L]{\DocumentTheme\space-\space\DocumentType}
    \fancyfoot[C]{\DocumentTitle}
    \fancyfoot[R]{\thepage/\pageref{LastPage}}
    % Remove the line below the header and above the footer
    \renewcommand{\headrulewidth}{0pt}
    \renewcommand{\footrulewidth}{0pt}
}

% Apply the custom footer to the fancy style
\customfooter

% Apply the custom footer to the plain style (used on the first page)
\fancypagestyle{plain}{\customfooter}

% ========== Sections ==========

% Sections : I. / II. / III. ...
\renewcommand{\thesection}{\Roman{section}}
% Subsection : I.1 / I.2 / I.3 ...
\renewcommand{\thesubsection}{\thesection. \arabic{subsection}}
% Subsubsection : I.1.a / I.1.b / I.1.c ...
\renewcommand{\thesubsubsection}{\thesubsection. \alph{subsubsection}}

% ========== Colors (from Geogebra) ==========

\definecolor{green}{HTML}{006400}
\definecolor{red}{HTML}{CC0000}
\definecolor{blue}{HTML}{0000FF}
\definecolor{orange}{HTML}{FF5500}
\definecolor{purple}{HTML}{9933FF}
\definecolor{gray}{HTML}{666666}
\definecolor{brown}{HTML}{993300}
\definecolor{black}{HTML}{000000}
\definecolor{white}{HTML}{FFFFFF}

\newcommand{\green}[1]{\color{green}{#1}\color{black}}
\newcommand{\red}[1]{\color{red}{#1}\color{black}}
\newcommand{\blue}[1]{\color{blue}{#1}\color{black}}
\newcommand{\orange}[1]{\color{orange}{#1}\color{black}}
\newcommand{\purple}[1]{\color{purple}{#1}\color{black}}
\newcommand{\gray}[1]{\color{gray}{#1}\color{black}}
\newcommand{\brown}[1]{\color{brown}{#1}\color{black}}


% ========== Définition(s) ==========

% Style de la boîte de définition(s)
\mdfdefinestyle{DefinitionStyle}{
    leftmargin=0cm,
    rightmargin=0cm,
    linecolor=black,
    linewidth=2pt,
    topline=false,
    bottomline=false,
    rightline=false
}

% Définition de la commande '\definition'
\newcommand{\definition}[2]{%
    \begin{mdframed}[style=DefinitionStyle]
        \ifstrempty{#1}{% Teste si le premier argument est vide
            \textbf{Définition :}\ % Si vide, n'affiche pas de titre de théorème
        }{%
            \textbf{Définition #1 :}\ % Si non vide, affiche le titre avec le premier argument
        }\\#2
    \end{mdframed}
}

% Définition de la commande '\definitions'
\newcommand{\definitions}[2]{%
    \begin{mdframed}[style=DefinitionStyle]
        \ifstrempty{#1}{% Teste si le premier argument est vide
            \textbf{Définitions :}\ % Si vide, n'affiche pas de titre de théorème
        }{%
            \textbf{Définitions #1 :}\ % Si non vide, affiche le titre avec le premier argument
        }\\#2
    \end{mdframed}
}

% ========== Propriété(s) et Théorème(s) ==========

% Style de la boîte de propriété(s) / théorème(s)
\mdfdefinestyle{ProprieteStyle}{
    leftmargin=0cm,
    rightmargin=0cm,
    linecolor=black,
    linewidth=1pt
}

% Création d'un compteur pour les propriétés
\newcounter{propriete}

% Définition de la commande '\propriete'
\newcommand{\propriete}[2]{%
    \refstepcounter{propriete}% Incrémente le compteur de théorème
    \begin{mdframed}[style=ProprieteStyle]
        \ifstrempty{#1}{% Teste si le premier argument est vide
            \textbf{\thepropriete. Propriété :}\ % Si vide, n'affiche pas de titre de théorème
        }{%
            \textbf{\thepropriete. Propriété #1 :}\ % Si non vide, affiche le titre avec le premier argument
        }\\#2
    \end{mdframed}
}

% Définition de la commande 'proprietes'
\newcommand{\proprietes}[2]{%
    \refstepcounter{propriete}% Incrémente le compteur de théorème
    \begin{mdframed}[style=ProprieteStyle]
        \ifstrempty{#1}{% Teste si le premier argument est vide
            \textbf{\thepropriete. Propriétés :}\ % Si vide, n'affiche pas de titre de théorème
        }{%
            \textbf{\thepropriete. Propriétés #1 :}\ % Si non vide, affiche le titre avec le premier argument
        }\\#2
    \end{mdframed}
}

% Création d'un compteur pour les théorèmes
\newcounter{theoreme}

% Définition de la commande '\theoreme'
\newcommand{\theoreme}[2]{%
    \refstepcounter{theoreme}% Incrémente le compteur de théorème
    \begin{mdframed}[style=ProprieteStyle]
        \ifstrempty{#1}{% Teste si le premier argument est vide
            \textbf{\thetheoreme. Théorème :}\ % Si vide, n'affiche pas de titre de théorème
        }{%
            \textbf{\thetheoreme. Théorème #1 :}\ % Si non vide, affiche le titre avec le premier argument
        }\\#2
    \end{mdframed}
}

% Définition de la commande '\theoremes'
\newcommand{\theoremes}[2]{%
    \refstepcounter{theoreme}% Incrémente le compteur de théorème
    \begin{mdframed}[style=ProprieteStyle]
        \ifstrempty{#1}{% Teste si le premier argument est vide
            \textbf{\thetheoreme. Théorèmes :}\ % Si vide, n'affiche pas de titre de théorème
        }{%
            \textbf{\thetheoreme. Théorèmes #1 :}\ % Si non vide, affiche le titre avec le premier argument
        }\\#2
    \end{mdframed}
}

\newcommand{\corrolaire}[2]{%
    \begin{mdframed}[style=ProprieteStyle]
        \ifstrempty{#1}{% Teste si le premier argument est vide
            \textbf{Corrolaire :}\ % Si vide, n'affiche pas de titre de théorème
        }{%
            \textbf{Corrolaire #1 :}\ % Si non vide, affiche le titre avec le premier argument
        }\\#2
    \end{mdframed}
}

% ========== Exemple(s) ==========

% Définition de la commande 'exemple'
\newcommand{\exemple}[2]{
    \textbf{Exemple :} #1
    \begin{quote}
        #2
    \end{quote}
}

% Définition de la commande 'exemples'
\newcommand{\exemples}[2]{
    \textbf{Exemples :} #1
    \begin{quote}
        #2
    \end{quote}
}

% ========== Remarques ==========

% Définition de la commande 'remarque'
\newcommand{\remarque}[1]{
    \textbf{Remarque :} #1
}

% Définition de la commande 'remarques'
\newcommand{\remarques}[1]{
    \textbf{Remarques :}
    \begin{quote}
        #1
    \end{quote}
}

% ========== Démonstration ==========

\newcommand{\demonstration}[2]{
    \ifstrempty{#1}{% Teste si le premier argument est vide
        \textbf{Démonstration :}\ % Si vide, n'affiche pas de titre de théorème
    }{%
        \textbf{Démonstration #1 :}\ % Si non vide, affiche le titre avec le premier argument
    }
    \begin{quote}
        #2
    \end{quote}
}
% ========== Autres blocs ==========

\newcommand{\newbloc}[2]{
    \textbf{#1}
    \begin{quote}
        #2
    \end{quote}
}

% ========== Notations ==========

% Style
\newcommand{\ds}{\displaystyle}
\newcommand{\llb}{\llbracket}
\newcommand{\rrb}{\rrbracket}

% Ensembles
\newcommand{\C}{\mathbb{C}}
\newcommand{\R}{\mathbb{R}}
\newcommand{\Q}{\mathbb{Q}}
\newcommand{\D}{\mathbb{D}}
\newcommand{\Z}{\mathbb{Z}}
\newcommand{\N}{\mathbb{N}}

% Limites
\let\oldlim\lim
\renewcommand{\lim}[1]{\mathop{\displaystyle\oldlim}\limits_{#1}}

% Opérateurs
\newcommand{\x}{\times}
\newcommand{\equival}{\Leftrightarrow}
\newcommand{\involve}{\Rightarrow}

% Ensembles
\renewcommand{\bar}[1]{\overline{#1}}
\renewcommand{\part}[1]{\mathcal{P}({#1})}  % Ensemle des parties de E
\newcommand{\rel}{\mathcal{R}} % Relation entre deux ensembles

% Vecteurs
\renewcommand{\Vec}[1]{\overrightarrow{#1}}
\newcommand*{\norme}[1]{\|#1\|}
\newcommand{\pdt}{\mathbin{\vcenter{\hbox{\scalebox{0.6}{\textbullet}}}}}
\newcommand{\vcoord}[2]{\begin{pmatrix} #1 \\ #2 \end{pmatrix}}
\newcommand{\Vcoord}[3]{\begin{pmatrix} #1 \\ #2 \\ #3 \end{pmatrix}}

% Fonctions
\newcommand{\aire}[1]{\text{aire}\left( #1 \right)}
\newcommand{\Int}{\ds\int}


% ========== Systèmes ==========
\newcommand{\sys}[2]{\begin{cases}#1\\#2\end{cases}}
\newcommand{\Sys}[3]{\begin{cases}#1\\#2\\#3\end{cases}}

% ========== Tableaux ==========
\newcommand{\boldhline}{\hline\noalign{\vskip 0pt}\hline}

% ========== Graphiques ==========

% Triange de Pascal
\newcommand{\binomial}[2]{\pgfmathparse{int((#1)!/((#2)!*((#1)-(#2))!))}\pgfmathresult}
\newcommand{\Pascal}[1]{
    \begin{tikzpicture}[scale=1, every node/.style={scale=1.2}]
        \foreach \n in {0,...,#1} {
                \foreach \k in {0,...,\n} {
                        \node at (\k-\n/2,-\n) {$\binomial{\n}{\k}$};
                        \draw (\k-\n/2,-\n) +(-0.5,-0.5) rectangle ++(0.5,0.5);
                    }
            }
    \end{tikzpicture}
}

\newcommand{\Pascalbinome}[1]{
    \begin{tikzpicture}[scale=1, every node/.style={scale=0.9}]
        \foreach \n in {0,...,#1} {
                \foreach \k in {0,...,\n} {
                        \pgfmathtruncatemacro{\symk}{min(\k,\n-\k)}
                        \node at (\k-\n/2,-\n) {$\ds\binom{\n}{\symk}$};
                        \draw (\k-\n/2,-\n) +(-0.5,-0.5) rectangle ++(0.5,0.5);
                    }
            }
    \end{tikzpicture}
}

\begin{document}

\maketitle

\section{Ensemble des nombres complexes}

\subsection{Notion de nombre complexe}

\propriete{}{
    Il existe un ensemble des nombres complexes (noté $\C$) qui possède les propriétés suiante :
    \begin{enumerate}[(i)]
       \item $\R\subset\C$
       \item L'addition et la multiplication des réels se prolongent aux nombres complexes et les règles de calcul sont
             les mêmes.
       \item Il existe un nombre complexe $i$ tel que $i^2=-1$.
       \item Tout nombre complexe $z$ s'écrit $z=a+ib$ avec $a$ et $b$ des réels.
    \end{enumerate}
}

\definitions{(nombre complexe)}{
    L'écriture $z=a+ib$ d'un nombre complexe avec $(a,b)\in\R^2$ est appelé forme algébrique (ou cartésienne) de $z$
    avec :
    \begin{itemize}
       \item $a$ la partie réelle de $z$ noté $a=\Re(z)$.
       \item $b$ la partie imaginaire de $z$ noté $a=\Im(z)$.
    \end{itemize}
    }

\remarques{
    Soit $z$ un nombre complexe.
    \begin{itemize}
       \item On a $z=\Re(z) + i\times\Im(z)$
       \item Si $\Re(z)=0$, on dit que $z$ est imaginaire pur. On note $i\R$ l'ensemble des imaginaires purs.
       \item Si $\Im(z)=0$, $z$ est réel.
    \end{itemize}
}

\theoreme{}{
    Soient $z=a+ib$ et $z'=a'+ib'$ deux nombres complexes avec $a$, $b$, $a'$ et $b'$ des réels. Alors : 
    $z=z' \equival \sys{a=a'}{b=b'}$.
}

\corrolaire{}{
    Tout nombre complexe $z$ s'écrit de manière unique $z=a+ib$ avec $(a,b)\in\R^2$.
}

\proprietes{(conséquence)}{
    Soient $z=a+ib$ et $z'=a'+ib'$ deux nombres complexes avec $a$, $b$, $a'$ et $b'$ des réels. Alors :
    \begin{enumerate}[(i)]
       \item $z\not= z' \equival a\not=a' \lor b\not=b'$
       \item $z=0 \equival a=0 \land b=0$
       \item $z\not=0 \equival a\not=0 \lor b\not=0$
    \end{enumerate}
}

\newpage

\subsection{Opérations sur les nombres complexes}

\proprietes{(addition et multiplication)}{
    Soient $z=a+ib$ et $z'=a'+ib'$ deux nombres complexes avec $a$, $b$, $a'$ et $b'$ des réels. Alors :
    \begin{enumerate}[(i)]
       \item $z+z' = (a+a') + i(b+b')$
       \item $z\times z' = (aa'-bb')+ i(ab'-a'b)$
    \end{enumerate}
}

\definition{(opposé)}{
    Pour tout nombre complexe $z=a+ib$ avec $(a,b)\in\R^2$, il existe un unique nombre complexe $z'$ tel que $z+z'=0$.
    On appelle $z'$ l'opposé de $z$ et on le note $-z=(-a)+i(-b)$. 
}

\definition{(soustraction)}{
    Soient $z=a+ib$ et $z'=a'+ib'$ deux nombres complexes avec $a$, $b$, $a'$ et $b'$ des réels. Alors $z-z'$ est défini
    par $z+(-z')$ et on a $z-z'=(a-a')+i(b-b')$.
}

\definition{(inverse)}{
    Pour tout nombre complexe $z=a+ib$ avec $a$, $b$ des réels et $z\not=0$, il existe un unique nombre complexe $z'$
    tel que $z\times z'=1$. On appelle $z'$ inverse de $z$ et on le note
    $\dfrac{1}{z}=\dfrac{a}{a^2+b^2}+i\times\dfrac{b}{a^2+b^2}$.   
}

\definition{(quotient)}{
    Soient $z$ et $z'$ deux nombres complexes tels que $z'\not=0$. Alors $\dfrac{z}{z'}$ est défini par
    $\dfrac{z}{z'}=z\times \dfrac{1}{z'}$.
}

\section{Conjugué d'un nombre complexe}

\definition{}{
    Soit $z$ un nombre complexe tel que $z=a+ib$ avec $(a,b)\in\R^2$. Alors le conjugué de $z$, noté $\bar{z}$, est le
    nombre complexe défini par $\bar{z}=a-ib$.
}

\proprietes{}{
    Pour tout nombre complexe $z$, on a : \\

    \begin{minipage}{0.22\textwidth}
        \begin{enumerate}[(i)]
           \item $z+\bar{z} = 2\Re(z)$
           \item $z-\bar{z} = 2i\Im(z)$
        \end{enumerate}
    \end{minipage}
    \hfill
    \begin{minipage}{0.25\textwidth}
        \begin{enumerate}[(i)]
            \setcounter{enumi}{2}
            \item $z\in\R \equival z=\bar{z}$
            \item $z\in i\R \equival z=-\bar{z}$
        \end{enumerate}
    \end{minipage}
    \hfill
    \begin{minipage}{0.3\textwidth}
        \begin{enumerate}[(i)]
            \setcounter{enumi}{4}
            \item $\bar{\bar{z}}=z$
            \item $z\bar{z}=\Re(z)^2+\Im(z)^2$
        \end{enumerate}
    \end{minipage}
}

\proprietes{(opérations et conjugué)}{
    Pour tous nombres complexes $z$ et $z'$, on a :

    \begin{minipage}{0.5\textwidth}
        \begin{enumerate}[(i)]
        \item $\bar{-z}=-\bar{z}$
        \item $\bar{z+z'}=\bar{z}+\bar{z'}$
        \item $\bar{z\times z'}=\bar{z}\times \bar{z'}$
        \item $\forall n\in\N, \bar{(z^n)}=(\bar{z})^n$
        \end{enumerate}
    \end{minipage}
    \hfill
    \begin{minipage}{0.5\textwidth}
        \setcounter{enumi}{4}
        \begin{enumerate}[(i)]
           \item Si $z\not=0$, $\bar{\left(\dfrac{1}{z}\right)}=\dfrac{1}{\bar{z}}$
           \item Si $z\not=0$, $\bar{\left(\dfrac{z'}{z}\right)}=\dfrac{\bar{z'}}{\bar{z}}$

        \end{enumerate}
    \end{minipage}
}

\newpage

\section{Équations du second degré à coefficients réels}

\propriete{}{
    On considère l'équation $z^2=a$ avec $a$ un réel.
    \begin{enumerate}[(i)]
       \item Si $a>0$, alors l'équation admet deux solutions réelles : $\sqrt a$ et $-\sqrt a$.
       \item Si $a<0$, alors l'équation admet deux solutions réelles : $i\sqrt {-a}$ et $-i\sqrt {-a}$.
    \end{enumerate}
}

\proprietes{(solutions)}{
    On considère l'équation $az^2+bz+c=0$ avec $a$, $b$, $c$ trois réels et $a\not=0$. Soit $\Delta=b^2-4ac$ le
    discriminant de cette équation.
    \begin{enumerate}[(i)]
       \item Si $\Delta>0$, alors l'équation admet deux solutions réelles distinctes : 
       $$z_1=\dfrac{-b-\sqrt \Delta}{2a} \text{ et } z_2=\dfrac{-b+\sqrt \Delta}{2a}$$
       \item Si $\Delta=0$, alors l'équation admet une unique solution réelle :
       $$z_0=\dfrac{-b}{2a}$$.
       \item Si $\Delta<0$, alors l'équation admet deux solutions complexes conjuguées :
       $$z_1=\dfrac{-b-i\sqrt {-\Delta}}{2a} \text{ et } z_2=\dfrac{-b+i\sqrt {-\Delta}}{2a}$$
    \end{enumerate}
}

\proprietes{(factorisation)}{
    On considère l'équation $az^2+bz+c=0$ avec $a$, $b$, $c$ trois réels et $a\not=0$. Soit $\Delta=b^2-4ac$ le
    discriminant de cette équation.
    \begin{enumerate}[(i)]
       \item Si $\Delta\not=0$, alors on a $az^2+bz+c=a(z-z_1)(z-z_2)$.
       \item Si $\Delta=0$, alors on a $az^2+bz+c=a(z-z_0)^2$.
    \end{enumerate}
}

\section{Représentation dans le plan complexe}

Dans toute la suite du chapitre, on munit le plan d'un repère $(O,\Vec{u},\Vec{v})$ orthonormé direct.

\subsection{Définitions}

\begin{minipage}{0.65\textwidth}
    \definitions{}{
        À tout nombre complexe $z = a + ib$, avec $a$ et $b$ réels, on peut associer :
    
        \begin{itemize}
            \item L'unique point $M(a ; b)$ appelé point image de $z$.
            \item L'unique vecteur $\Vec{w}=\vcoord{a}{b}$ appelé vecteur image de $z$.
        \end{itemize}
    
        Réciproquement :
    
        \begin{itemize}
            \item À tout point $M(a ; b)$ avec $a$ et $b$ deux réels, on peut associer l'unique nombre complexe $z = a + ib$
            appelé affixe du point $M$.
            \item À tout vecteur vecteur $\Vec{w}=\vcoord{a}{b}$ avec $a$ et $b$ deux réels, on peut associer l'unique nombre
            complexe $z = a + ib$ appelé affixe du vecteur $\Vec{w}$.
        \end{itemize}
    }
\end{minipage}
\hfill
\begin{minipage}{0.3\textwidth}

    \begin{tikzpicture}
        % Axes
        \draw[->] (-1,0) -- (4,0) node[above] {Axes de réels};
        \draw[->] (0,0) -- (1,0) node[below left] {$\Vec{u}$};
        \draw[->] (0,-1) -- (0,4) node[right] {Axe des imaginaires purs};
        \draw[->] (0,0) -- (0,1) node[below left] {$\Vec{v}$};
        
        % Point M(a,b)
        \coordinate (M) at (2,3);
        \fill (M) circle (2pt) node[above right] {$M(a+ib)$};
        
        % Vector from O(0,0) to M(a,b)
        \draw[->, thick] (0,0) -- (M) node[midway, above, sloped] {$\Vec{w}(a+ib)$};
    
        % Origin point O(0,0)
        \fill (0,0) circle (2pt) node[below left] {$O$};
        
        % Dotted lines to show coordinates of M
        \draw[dashed] (M) -- (2,0) node[below] {$a$};
        \draw[dashed] (M) -- (0,3) node[left] {$b$};
    \end{tikzpicture}
\end{minipage}

\newpage

\remarques{
    \begin{itemize}
        \item Les nombres réels sont les affixes des points de l'axe des abscisses aussi appelé : axe des réels.
        \item Les nombres imaginaires purs sont les affixes des points de l'axe des ordonnées aussi appelé : axe des imaginaires purs.
        \item Lorsqu'un point ou un vecteur est repéré par son affixe, le plan est appelé le plan complexe.
        \item L'affixe de $M$ est souvent noté $z_M$ et la donnée d'un point $M$ d'affixe $z_M$ est souvent notée $M(z_M)$.
        \item L'affixe de $\Vec{w}$ est souvent noté $z_{\Vec{w}}$, et la donnée d'un vecteur $w$ d'affixe $z_{\Vec{w}}$ est souvent notée 
        $ \Vec{w}(z_{\Vec{w}})$ .
        
    \end{itemize}
}

\subsection{Propriétés}

Des propriétés connues de géométrie sur les vecteurs et points donnent les propriétés suivantes.

\proprietes{}{
    Soient $A(z_A)$ et $B(z_B)$ deux points du plan complexe. Soient $ \Vec{w_1}(z_{\Vec{w_1}})$ et
    $\Vec{w_2}(z_{\Vec{w_2}})$  deux vecteurs du plan complexe. Soit $\lambda\in\R$. \\

    \begin{minipage}{0.4\textwidth}
        \begin{enumerate}[(i)]
            \item $A=B \equival z_A=z_B$ . 
            \item $\Vec{w_1}=\Vec{w_2} \equival z_{\Vec{w_1}}=z_{\Vec{w_2}}$.
            \item $\Vec{AB}$ a pour affixe $z_B-z_A$.
        \end{enumerate}
    \end{minipage}
    \hfill
    \begin{minipage}{0.6\textwidth}
        \begin{enumerate}[(i)]
            \setcounter{enumi}{3}
            \item Le milieu du segment $[AB]$ a pour affixe $\dfrac{z_A+z_B}{2}$.
            \item Le vecteur $\Vec{w_1}+\Vec{w_2}$ a pour affixe $z_1+z_2$.
            \item $\lambda \Vec{w_1}$ a pour affixe $\lambda z_1$.
        \end{enumerate}
    \end{minipage} 
}

\subsection{Conjugué et opposé}

\propriete{}{
    \vspace{-10pt}
    \begin{enumerate}[(i)]
       \item Les points $M$ d'affixe $z$ et $M'$ d'affixe $\bar{z}$ sont symétriques par rapport à l'axe des réels.
       \item Les points $M$ d'affixe $z$ et $M''$ d'affixe $-z$ sont symétriques par rapport à l'origine du repère.
    \end{enumerate}
}

\section{Module et argument d'un nombre complexe}

\subsection{Module}

\definition{}{
    Soit $M$ un point d'affixe $z$. Le module de $z$, noté $|z|$ est le réel positif défini par $|z| = OM$.
    Si $z = a + ib$ avec $a$ et $b$ deux réels. Alors $|z| = \sqrt{a^2 + b^2}$.
}

\remarque{Si $z = z'$, alors $|z| = |z'|$. Mais la réciproque est fausse. Contre-exemple avec $z=1+i$ et $z'=1-i$.
$|z| = |z'|=\sqrt{2}$ et $z \not= z'$.}

\proprietes{}{
    Soit $z$ un nombre complexe. \\
    \begin{minipage}{0.3\textwidth}
        \begin{enumerate}[(i)]
           \item $|z|^2=z \bar{z}$
            \item $|\bar{z}|=|z|$
        \end{enumerate}
    \end{minipage}
    \hfill
    \begin{minipage}{0.7\textwidth}
        \begin{enumerate}[(i)]
            \setcounter{enumi}{2}
           \item $|-z|=|z|$
            \item $|z|=0 \equival z=0$
        \end{enumerate}
    \end{minipage}
}

\remarque{corrolaire de (i) : $|z|= \sqrt{z \overline{z}}$ (utile en pratique).}

\propriete{}{
    Soit $A(z_A)$ et $B(z_B)$. On a $AB = |z_B - z_A| = |z_A - z_B|$
}

\newpage

\proprietes{}{
    \vspace{-10pt}
    \begin{minipage}{0.4\textwidth}
        \begin{enumerate}[(i)]
            \item Produit : $|zz'|=|z||z'|$
            \item Puissance : $|z^n|=|z|^n$  
        \end{enumerate}
    \end{minipage}
    \hfill
    \begin{minipage}{0.6\textwidth}
        \begin{enumerate}[(i)]
            \setcounter{enumi}{2}
            \item Inverse : $\left| \dfrac{1}{z} \right|=\dfrac{1}{|z|}$
            \item Quotient : $\left| \dfrac{z}{z'} \right|=\dfrac{|z|}{|z'|}$ 
        \end{enumerate}
    \end{minipage}
    \vspace{10pt}
}

\propriete{(inégalité triangulaire)}{
    Soient $z$ et $z'$ deux nombres complexes, on a $|z+z'| \leq |z|+|z'|$
}

\subsection{Argument}

\definition{}{
    Soit un point $M$ d'affixe non nulle $z$.
    On appelle argument de $z$, noté $\arg(z)$ une mesure, en radians, de l'angle $(\Vec{u};OM)$.
}

\remarques{
    \begin{itemize}
        \item Un nombre complexe non nul possède une infinité d'arguments de la forme $\arg(z) [2\pi]$.
        
        \item 0 n'a pas d'argument car dans ce cas l'angle $(\Vec{u};OM)$ n'est pas défini. 
    \end{itemize}
}

\propriete{}{
    Soit $z$ un nombre complexe non nul. \\

    \begin{minipage}{0.4\textwidth}
        \begin{enumerate}[(i)]
            \item $z\in\R \equival \arg(z)=0 \mod{\pi}$
            \item $z\in i\R \equival \arg(z)= \dfrac{\pi}{2} \mod{\pi}$
        \end{enumerate}
    \end{minipage}
    \hfill
    \begin{minipage}{0.6\textwidth}
        \begin{enumerate}[(i)]
            \setcounter{enumi}{2}
            \item $\arg({\bar{z}})=-\arg(z) \mod{2 \pi}$
            \item $\arg({-z})=\arg(z)+ \pi \mod{2 \pi}$
        \end{enumerate}
    \end{minipage}
}

\section{Forme trigonométrique d'un nombre complexe}

\subsection{Définition}


\propriete{}{
    Soit $z=a+ib$ un nombre complexe non nul. On pose : $\theta = \arg(z)$. \\
    On a alors : $a=|z| \cos(\theta)$ et $b=|z| \sin(\theta)$.
}

\definition{}{
    On appelle forme trigonométrique d'un nombre complexe non nul l'écriture :
    $z=|z|(\cos(\theta)+i \sin(\theta))$ avec $\theta = \arg(z)$.
}

\propriete{}{
    Deux nombres complexes non nuls sont égaux si, et seulement si, ils ont le même module et le même argument (modulo $2\pi$).
}

\subsection{Relations trigonométrique et propriétés des arguments}

\propriete{(formule d'addition)}{
    Pour tous réel $a$ et $b$, \\

    \begin{minipage}{0.5\textwidth}
        \begin{enumerate}[(i)]
            \item $\cos(a-b)=\cos(a)\cos(b) + \sin(a)\sin(b)$ 
            \item $\cos(a+b)=\cos(a)\cos(b) - \sin(a)\sin(b)$ 
        \end{enumerate}
    \end{minipage}
    \hfill
    \begin{minipage}{0.5\textwidth}
        \begin{enumerate}[(i)]
            \setcounter{enumi}{2}
            \item $\sin(a-b)=\sin(a)\cos(b) - \cos(a)\sin(b)$ 
            \item $\sin(a+b)=\sin(a)\cos(b) + \cos(a)\sin(b)$ 
        \end{enumerate}
    \end{minipage}
}

\propriete{(formule de duplication)}{
    Pour tous réel $a$,
    \begin{enumerate}[(i)]
       \item $\cos(2a) = \cos^2(a) - \sin^2(a) = 1 - 2\sin^2(a) = 2\cos^2(a) - 1$
       \item $\sin(2a)=2\sin(a)\cos(a)$
    \end{enumerate}
}

\propriete{(argument et opérations)}{
    Soient $z$ et $z'$ deux nombres complexes non nuls et $n$ un entier naturel non nul.
    \begin{enumerate}[(i)]
       \item $\arg(zz') = \arg(z)+\arg(z') \mod{2\pi}$
       \item $\arg(z^n) = n\arg(z) \mod{2\pi}$
       \item $\arg\left(\dfrac{1}{z}\right) = -\arg(z) \mod{2\pi}$
       \item $\arg\left(\dfrac{z}{z'}\right) = \arg(z) - \arg(z') \mod{2\pi}$
    \end{enumerate}
}

\remarques{
    A l'aide des arguments, on peut gérer différentes situations em géométrie, par exemple, avec $A$, $B$, $C$ et $D$
    quatre points d'affixes respectives $a$, $b$, $c$ et $d$ :
    \begin{itemize}
       \item Situation d'alignement : \\
        $A$, $B$ et $C$ alignés $\equival \arg (b-a) = \arg (c-a) \mod{\pi}$
       \item Situation de parallèlisme : \\
        $(AB) \sslash (CD) \equival \arg (b-a) = \arg (d-c) \mod{\pi}$
       \item Situation de perpendicularité : \\
        $(AB) \perp (CD) \equival \arg (b-a) = \arg (d-c) + \dfrac{\pi}{2} \mod{\pi}$
    \end{itemize}
}

\section{Forme exponentielle}

\subsection{Définition}

\theoreme{(fonction exponentielle complexe)}{
    Soit $f$ la fonction défini sur $\R$ par $f(\theta)=\cos\theta+i\sin\theta$.
    \begin{itemize}
       \item Pour tous réels $\theta$ et $\theta'$ on a $f(\theta+\theta')=f(\theta)\times f(\theta')$.
       De plus, $f(0)=1$.
       \item Par analogie avec la fonction exponentielle dans $\R$, on pose $f(\theta)=e^{i\theta}$, soit
       $e^{i\theta}=\cos\theta+i\sin\theta$.
       \item On a $|e^{i\theta}|=1$
    \end{itemize}
}

\remarque{On peut écrire $e^{i\pi}-1=0$. Cette relation possède la particularité de relier les grandes branches des
mathématiques : l'analyse (avec la fonction $e$), l'algèbre (avec le nombre $i$) et la géométrie (avec le nombre $\pi$)}

\definition{}{
    Tout nombre complexe $z$ non nul de module $r$ et d'argument $\theta$ s'écrit sous sa forme exponentielle
    $z=re^{i\theta}$.
}

\proprietes{}{
    Pour tous nombres réels $\theta$ et $\theta'$, on a :
    \begin{enumerate}[(i)]
       \item $e^{i\theta}e^{i\theta'}=e^{i(\theta+\theta')}$
       \item $\dfrac{1}{e^{i\theta}} = e^{-i\theta}$
       \item $\dfrac{e^{i\theta}}{e^{i\theta'}} = e^{i(\theta-\theta')}$
       \item $\bar{e^{i\theta}} = e^{-i\theta}$
    \end{enumerate}
}

\propriete{(formule de Moivre)}{
    $\forall\theta\in\R,\forall n\in\Z : \left(e^{i\theta}\right)^n = e^{in\theta}$ c'est à dire
    $(\cos\theta+i\sin\theta)^n = \cos(n\theta) + i\sin(n\theta)$
}

\propriete{(formule d'Euler)}{
    $\forall\theta\in\R : \sys{\cos\theta=\dfrac{e^{i\theta}+e^{-i\theta}}{2}}{\sin\theta=\dfrac{e^{i\theta}-e^{-i\theta}}{2}}$
}

\end{document}