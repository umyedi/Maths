\documentclass[10pt]{article} % Sets the document class to 'article' with a font size of 12pt

\newcommand{\DocumentTitle}{Les limites}
\newcommand{\DocumentTheme}{Analyse}
\newcommand{\DocumentType}{Cours}

% Preamble Section
\usepackage[a4paper, margin=1in]{geometry} % Sets the paper size to A4 and all margins to 1 inch
\usepackage[utf8]{inputenc}   % Allows for input of international characters
\usepackage[T1]{fontenc}      % Utilisation de l'encodage T1
\usepackage{lmodern}          % Support des polices avec le module 'french'
\usepackage[french]{babel}    % Ajout du support pour le français
\usepackage[fleqn]{amsmath}   % Imports the AMS math package for advanced math formatting ([fleqn] -> align alignat to the left)
\usepackage{amsfonts}         % Imports AMS fonts for math fonts
\usepackage{amssymb}          % Imports AMS symbols for math symbols
\usepackage{amsthm}           % Imports AMS theorems
\usepackage{yhmath}           % Used for "\wideparen{}" 
\usepackage{fancyhdr}         % Import to custom the page footer
\usepackage{mdframed}         % Imports styles
\usepackage{enumerate}        % Imports styles for enumerate
\usepackage{multicol}
\usepackage{xcolor}           % Couleurs
\usepackage{tikz}             % Package général pour les graphiques
\usetikzlibrary{calc}
\usepackage{pgfplots}         % Compléments pour les graphiques
\usepackage{tkz-tab}          % Tableaux de variations
\usepackage{lastpage}         % Pour avoir le total de page dans le footer
\usepackage{makecell}         % Retour à la ligne dans une case d'un tabular
\usepackage{stmaryrd}         % Intervalles entières : \llbracket et \rrbracket
\usepackage{cancel}           % Permet de barrer des termes
\usepackage{forest}           % Permet de créer des arbres pondérés
\usepackage{tabularx}         % Permet de générer des tableaux (ex: cours fonction dérivées et application)
\usepackage{multirow}         % Permet de faire plusieurs lignes dans des cases de talbeaux


\pgfplotsset{compat=newest}   % Active les dernières fonctionnalités de pgfplots

% Enlève l'indentation des bloc de paragraphes et d'équation
\setlength{\parindent}{0pt}
\setlength{\mathindent}{0pt}

% ========== Title Section ==========

\title{\DocumentTitle} % Title of the document
\author{\DocumentTheme\space-\space\DocumentType} % Author's name
\date{} % Date

\pagestyle{fancy}
\fancyhf{} % Clear all header and footer fields

% ========== Footer Section ==========

\newcommand{\customfooter}{
    \fancyfoot[L]{\DocumentTheme\space-\space\DocumentType}
    \fancyfoot[C]{\DocumentTitle}
    \fancyfoot[R]{\thepage/\pageref{LastPage}}
    % Remove the line below the header and above the footer
    \renewcommand{\headrulewidth}{0pt}
    \renewcommand{\footrulewidth}{0pt}
}

% Apply the custom footer to the fancy style
\customfooter

% Apply the custom footer to the plain style (used on the first page)
\fancypagestyle{plain}{\customfooter}

% ========== Sections ==========

% Sections : I. / II. / III. ...
\renewcommand{\thesection}{\Roman{section}}
% Subsection : I.1 / I.2 / I.3 ...
\renewcommand{\thesubsection}{\thesection. \arabic{subsection}}
% Subsubsection : I.1.a / I.1.b / I.1.c ...
\renewcommand{\thesubsubsection}{\thesubsection. \alph{subsubsection}}

% ========== Colors (from Geogebra) ==========

\definecolor{green}{HTML}{006400}
\definecolor{red}{HTML}{CC0000}
\definecolor{blue}{HTML}{0000FF}
\definecolor{orange}{HTML}{FF5500}
\definecolor{purple}{HTML}{9933FF}
\definecolor{gray}{HTML}{666666}
\definecolor{brown}{HTML}{993300}
\definecolor{black}{HTML}{000000}
\definecolor{white}{HTML}{FFFFFF}

\newcommand{\green}[1]{\color{green}{#1}\color{black}}
\newcommand{\red}[1]{\color{red}{#1}\color{black}}
\newcommand{\blue}[1]{\color{blue}{#1}\color{black}}
\newcommand{\orange}[1]{\color{orange}{#1}\color{black}}
\newcommand{\purple}[1]{\color{purple}{#1}\color{black}}
\newcommand{\gray}[1]{\color{gray}{#1}\color{black}}
\newcommand{\brown}[1]{\color{brown}{#1}\color{black}}


% ========== Définition(s) ==========

% Style de la boîte de définition(s)
\mdfdefinestyle{DefinitionStyle}{
    leftmargin=0cm,
    rightmargin=0cm,
    linecolor=black,
    linewidth=2pt,
    topline=false,
    bottomline=false,
    rightline=false
}

% Définition de la commande '\definition'
\newcommand{\definition}[2]{%
    \begin{mdframed}[style=DefinitionStyle]
        \ifstrempty{#1}{% Teste si le premier argument est vide
            \textbf{Définition :}\ % Si vide, n'affiche pas de titre de théorème
        }{%
            \textbf{Définition #1 :}\ % Si non vide, affiche le titre avec le premier argument
        }\\#2
    \end{mdframed}
}

% Définition de la commande '\definitions'
\newcommand{\definitions}[2]{%
    \begin{mdframed}[style=DefinitionStyle]
        \ifstrempty{#1}{% Teste si le premier argument est vide
            \textbf{Définitions :}\ % Si vide, n'affiche pas de titre de théorème
        }{%
            \textbf{Définitions #1 :}\ % Si non vide, affiche le titre avec le premier argument
        }\\#2
    \end{mdframed}
}

% ========== Propriété(s) et Théorème(s) ==========

% Style de la boîte de propriété(s) / théorème(s)
\mdfdefinestyle{ProprieteStyle}{
    leftmargin=0cm,
    rightmargin=0cm,
    linecolor=black,
    linewidth=1pt
}

% Création d'un compteur pour les propriétés
\newcounter{propriete}

% Définition de la commande '\propriete'
\newcommand{\propriete}[2]{%
    \refstepcounter{propriete}% Incrémente le compteur de théorème
    \begin{mdframed}[style=ProprieteStyle]
        \ifstrempty{#1}{% Teste si le premier argument est vide
            \textbf{\thepropriete. Propriété :}\ % Si vide, n'affiche pas de titre de théorème
        }{%
            \textbf{\thepropriete. Propriété #1 :}\ % Si non vide, affiche le titre avec le premier argument
        }\\#2
    \end{mdframed}
}

% Définition de la commande 'proprietes'
\newcommand{\proprietes}[2]{%
    \refstepcounter{propriete}% Incrémente le compteur de théorème
    \begin{mdframed}[style=ProprieteStyle]
        \ifstrempty{#1}{% Teste si le premier argument est vide
            \textbf{\thepropriete. Propriétés :}\ % Si vide, n'affiche pas de titre de théorème
        }{%
            \textbf{\thepropriete. Propriétés #1 :}\ % Si non vide, affiche le titre avec le premier argument
        }\\#2
    \end{mdframed}
}

% Création d'un compteur pour les théorèmes
\newcounter{theoreme}

% Définition de la commande '\theoreme'
\newcommand{\theoreme}[2]{%
    \refstepcounter{theoreme}% Incrémente le compteur de théorème
    \begin{mdframed}[style=ProprieteStyle]
        \ifstrempty{#1}{% Teste si le premier argument est vide
            \textbf{\thetheoreme. Théorème :}\ % Si vide, n'affiche pas de titre de théorème
        }{%
            \textbf{\thetheoreme. Théorème #1 :}\ % Si non vide, affiche le titre avec le premier argument
        }\\#2
    \end{mdframed}
}

% Définition de la commande '\theoremes'
\newcommand{\theoremes}[2]{%
    \refstepcounter{theoreme}% Incrémente le compteur de théorème
    \begin{mdframed}[style=ProprieteStyle]
        \ifstrempty{#1}{% Teste si le premier argument est vide
            \textbf{\thetheoreme. Théorèmes :}\ % Si vide, n'affiche pas de titre de théorème
        }{%
            \textbf{\thetheoreme. Théorèmes #1 :}\ % Si non vide, affiche le titre avec le premier argument
        }\\#2
    \end{mdframed}
}

\newcommand{\corrolaire}[2]{%
    \begin{mdframed}[style=ProprieteStyle]
        \ifstrempty{#1}{% Teste si le premier argument est vide
            \textbf{Corrolaire :}\ % Si vide, n'affiche pas de titre de théorème
        }{%
            \textbf{Corrolaire #1 :}\ % Si non vide, affiche le titre avec le premier argument
        }\\#2
    \end{mdframed}
}

% ========== Exemple(s) ==========

% Définition de la commande 'exemple'
\newcommand{\exemple}[2]{
    \textbf{Exemple :} #1
    \begin{quote}
        #2
    \end{quote}
}

% Définition de la commande 'exemples'
\newcommand{\exemples}[2]{
    \textbf{Exemples :} #1
    \begin{quote}
        #2
    \end{quote}
}

% ========== Remarques ==========

% Définition de la commande 'remarque'
\newcommand{\remarque}[1]{
    \textbf{Remarque :} #1
}

% Définition de la commande 'remarques'
\newcommand{\remarques}[1]{
    \textbf{Remarques :}
    \begin{quote}
        #1
    \end{quote}
}

% ========== Démonstration ==========

\newcommand{\demonstration}[2]{
    \ifstrempty{#1}{% Teste si le premier argument est vide
        \textbf{Démonstration :}\ % Si vide, n'affiche pas de titre de théorème
    }{%
        \textbf{Démonstration #1 :}\ % Si non vide, affiche le titre avec le premier argument
    }
    \begin{quote}
        #2
    \end{quote}
}
% ========== Autres blocs ==========

\newcommand{\newbloc}[2]{
    \textbf{#1}
    \begin{quote}
        #2
    \end{quote}
}

% ========== Notations ==========

% Style
\newcommand{\ds}{\displaystyle}
\newcommand{\llb}{\llbracket}
\newcommand{\rrb}{\rrbracket}

% Ensembles
\newcommand{\C}{\mathbb{C}}
\newcommand{\R}{\mathbb{R}}
\newcommand{\Q}{\mathbb{Q}}
\newcommand{\D}{\mathbb{D}}
\newcommand{\Z}{\mathbb{Z}}
\newcommand{\N}{\mathbb{N}}

% Limites
\let\oldlim\lim
\renewcommand{\lim}[1]{\mathop{\displaystyle\oldlim}\limits_{#1}}

% Opérateurs
\newcommand{\x}{\times}
\newcommand{\equival}{\Leftrightarrow}
\newcommand{\involve}{\Rightarrow}

% Ensembles
\renewcommand{\bar}[1]{\overline{#1}}
\renewcommand{\part}[1]{\mathcal{P}({#1})}  % Ensemle des parties de E
\newcommand{\rel}{\mathcal{R}} % Relation entre deux ensembles

% Vecteurs
\renewcommand{\Vec}[1]{\overrightarrow{#1}}
\newcommand*{\norme}[1]{\|#1\|}
\newcommand{\pdt}{\mathbin{\vcenter{\hbox{\scalebox{0.6}{\textbullet}}}}}
\newcommand{\vcoord}[2]{\begin{pmatrix} #1 \\ #2 \end{pmatrix}}
\newcommand{\Vcoord}[3]{\begin{pmatrix} #1 \\ #2 \\ #3 \end{pmatrix}}

% Fonctions
\newcommand{\aire}[1]{\text{aire}\left( #1 \right)}
\newcommand{\Int}{\ds\int}


% ========== Systèmes ==========
\newcommand{\sys}[2]{\begin{cases}#1\\#2\end{cases}}
\newcommand{\Sys}[3]{\begin{cases}#1\\#2\\#3\end{cases}}

% ========== Tableaux ==========
\newcommand{\boldhline}{\hline\noalign{\vskip 0pt}\hline}

% ========== Graphiques ==========

% Triange de Pascal
\newcommand{\binomial}[2]{\pgfmathparse{int((#1)!/((#2)!*((#1)-(#2))!))}\pgfmathresult}
\newcommand{\Pascal}[1]{
    \begin{tikzpicture}[scale=1, every node/.style={scale=1.2}]
        \foreach \n in {0,...,#1} {
                \foreach \k in {0,...,\n} {
                        \node at (\k-\n/2,-\n) {$\binomial{\n}{\k}$};
                        \draw (\k-\n/2,-\n) +(-0.5,-0.5) rectangle ++(0.5,0.5);
                    }
            }
    \end{tikzpicture}
}

\newcommand{\Pascalbinome}[1]{
    \begin{tikzpicture}[scale=1, every node/.style={scale=0.9}]
        \foreach \n in {0,...,#1} {
                \foreach \k in {0,...,\n} {
                        \pgfmathtruncatemacro{\symk}{min(\k,\n-\k)}
                        \node at (\k-\n/2,-\n) {$\ds\binom{\n}{\symk}$};
                        \draw (\k-\n/2,-\n) +(-0.5,-0.5) rectangle ++(0.5,0.5);
                    }
            }
    \end{tikzpicture}
}

\begin{document}

\maketitle

\section{Limite finie d'une fonction en $+\infty$}

\definition{}{
    Soit $f$ une fonction. Soit $l$ un réel.\\
    Dire que \guillemotleft $f(x)$ tend vers $l$ quand $x$ tend vers $+\infty$ \guillemotright\text{ }signifie
    $\forall \varepsilon > 0\text{, } \exists A\in\mathbb{R}\text{, } \forall x>A\text{ : } l-\varepsilon < f(x) < l+\varepsilon$.
    On note $\lim{x\to+\infty} f(x) = l$.
}

\theoreme{}{
    Pour toutes fonctions $f$ et $g$ et pour tout réels $l$ et $l'$ : \\
    Si $\lim{x\to+\infty} f(x) = l$ et $\lim{x\to+\infty} g(x) = l'$ et $l<l'$ alors
    $\exists A\in\R, \forall x>A : f(x)<g(x)$.}

\remarque{Une conséquence de ce théorème est que la limite d'une fonction est unique si elle existe. En effet,
    si on applique ce théorème à une fonction $f$ avec elle-même, on obtient $f(x)<f(x)$ ce qui n'a pas de sens.
}

\theoreme{de comparaison des limites}{
    Soient $f$ et $g$ deux fonctions. Soit $l$ et $l'$ deux réels. \\
    Si $\lim{x\to+\infty} f(x) = l$ et $\lim{x\to+\infty} g(x) = l'$ et $\exists A\in\R, \forall x>A$ : $f(x)\leq g(x)$ alors $l\leq l'$.
}

\remarque{Attention, même si $f(x)<g(x)$, leur limites peuvent quand même être égales
    (ex : $g(x)=\frac{1}{x}$ et $f(x)=\frac{-1}{x}$ tendent toutes les deux vers $0$.)
}

\theoreme{des gendarmes}{
    Soient $f$, $g$ et $h$ trois fonctions. Soit $l$ un réel. \\
    Si $\lim{x\to+\infty} f(x) = l$ et $\lim{x\to+\infty} h(x) = l$ et $\exists A\in\R, \forall x>A$, $f(x)\leq g(x)\leq h(x)$ alors $\lim{x\to+\infty} g(x) = l$.
}

\theoreme{des limites de référence}{
    \vspace{-10pt}
    \begin{enumerate}[(i)]
        \item $\lim{x\to+\infty} \dfrac{1}{x} = 0$
        \item $\lim{x\to+\infty} \dfrac{1}{\sqrt{x}} = 0$
        \item $\forall n\in\N^*, \lim{x\to+\infty} \dfrac{1}{x^n} = 0$
    \end{enumerate}
}

\newpage

\theoreme{de linéarité}{
    Soient $f$ et $g$ deux fonctions. Soit $l$ et $l'$ deux réels.
    Si $\lim{x\to+\infty}f(x)=l$ et $\lim{x\to+\infty} g(x)=l'$ alors :
    \begin{enumerate}[(i)]
        \item $\lim{x\to+\infty}f(x)+g(x)=l+l'$
        \item $\forall k\in\R \lim{x\to+\infty}f(x)\x k = k\x l$
        \item $\lim{x\to+\infty} f(x)\x g(x) = l \x l'$
        \item Si $l'\not=0$ et $\exists A\in\R, \forall x>A : g(x)\not= 0$ alors $\lim{x\to+\infty} \dfrac{f(x)}{g(x)} = \dfrac{l}{l'}$
    \end{enumerate}
}



\section{Limites de suites}

\subsection{Limites finies : suites convergentes}

\definition{}{
    Soit $u$ une suite. Soit $l$ un réel.\\
    On dit que la suite $u$ tend vers $l$ quand $n$ tend vers $+\infty$ lorsque la proposition $(P)$ suivante est vérifiée :
    $(P)$: \frquote{$\forall\varepsilon\in\R^+_*, \exists N\in\N, \forall n\geq N : l-\varepsilon<u_n<l+\varepsilon$}\\
    On dit que la suite $u$ converge vers $l$ et on note : $\lim{n\to+\infty}u_n=l$
}

\theoreme{}{
    Soient $u$ et $v$ deux suites. Soient $l$ et $l'$ deux réels.\\
    Si $\lim{n\to+\infty}u_n=l$ et $\lim{n\to+\infty}v_n=l'$ et $l<l'$ alors $\exists N\in\N, \forall n\geq N : u_n < v_n$
}

\remarque{Ce théorème a pour corollaire que la limite d'une suite est unique si elle existe.}

\theoreme{de comparaison des limites}{
    Soient $u$ et $v$ deux suites. Soient $l$ et $l'$ deux réels.\\
    Si $\exists N\in\N, \forall n\geq N : u_n\leq v_n$ et $\lim{n\to+\infty}u_n=l$ et $\lim{n\to+\infty}v_n=l'$ alors $l\leq l'$.
}

\theoreme{des gendarmes}{
    Soient $u$, $v$ et $w$ trois suites. Soient $l$ un réel.\\
    Si $\exists N\in\N, \forall n\geq N : u_n\leq v_n \leq w_n$ et $\lim{n\to+\infty}u_n=l$ et $\lim{n\to+\infty}w_n=l$
    alors $\lim{n\to+\infty}v_n=l$
}

\remarque{Comme pour les fonctions, les deux théorèmes précédents permettent de comparer les limites de fonctions simples,
    appelées \frquote{limites de référence}, avec des limites de fonctions plus élaborées.}

\theoreme{des limites de référence}{
    \vspace{-10pt}
    \begin{enumerate}[(i)]
        \item $\lim{n\to+\infty}\dfrac{1}{\sqrt{n}}=0$
        \item $\forall p\in\N^* : \lim{n\to+\infty}\dfrac{1}{n^p}=0$
    \end{enumerate}
}

\theoreme{de linéarité}{
    Soient $u$ et $v$ deux suites. Soient $l$ et $l'$ deux réels.\\
    Si $\lim{n\to+\infty}u_n=l$ et $\lim{n\to+\infty}v_n=l'$ alors:
    \begin{enumerate}[(i)]
        \item $\lim{n\to+\infty}u_n+ v_n=l+l'$
        \item $\forall k\in\R : \lim{n\to+\infty}k\x u_n = k\x l$
    \end{enumerate}
}

\theoreme{produit et quotient}{
    Soient $u$ et $v$ deux suites. Soient $l$ et $l'$ deux réels.\\
    Si $\lim{n\to+\infty}u_n=l$ et $\lim{n\to+\infty}v_n=l'$ alors:
    \begin{enumerate}[(i)]
        \item $\lim{n\to+\infty}u_n\x v_n=l\x l'$
        \item Si $l'\not=0$ et $\exists N\in\N, \forall n\geq N :v_n\not=0$ alors $\lim{n\to+\infty}\dfrac{u_n}{v_n} = \dfrac{l}{l'}$
    \end{enumerate}
}

\theoreme{}{
    Pour toute suite $u$ :
    \begin{enumerate}[(i)]
        \item Si $u$ est croissante et majorée alors $u$ converge.
        \item Si $u$ est décroissante et minorée alors $u$ converge.
    \end{enumerate}
}

\subsection{Limites infinies}

\definition{}{
    On dit que la suite $u$ tend vers $+\infty$ quand $n$ tend vers $+\infty$ lorsque la propostion $(P)$ suivante est vérifiée :
    $(P)$ : \frquote{$\forall A\in\R, \exists N\in\N,\forall n\geq N : u_n>A$}.\\
    On dit que la suite $u$ diverge vers $+\infty$ et on note $\lim{n\to+\infty}u_n=+\infty$.
}

\theoreme{de comparaison}{
    Soient $u$ et $v$ deux suites.\\
    Si $\exists N\in\N, \forall n\geq N : u_n\geq v_n$ et $\lim{n\to+\infty}v_n=+\infty$ alors $\lim{n\to+\infty}u_n=+\infty$.
}

\theoreme{des suites croissantes}{
    Soit $u$ une suite. Si $u$ est croissante et non majorée, alors $\lim{n\to+\infty}u_n=+\infty$.
}

\remarque{Une suite non majorée ne diverge pas forcément vers $+\infty$ (ex: $\forall n\in\N:u_n=(-1)^n$}).

\theoreme{des limites de référence}{
    \vspace{-10pt}
    \begin{enumerate}[(i)]
        \item $\lim{n\to+\infty}\sqrt{n}=+\infty$
        \item $\forall p\in\N^* : \lim{n\to+\infty}n^p=+\infty$
    \end{enumerate}
}

\theoreme{somme et produit}{
    Soient $u$ et $v$ deux suites. Si $\lim{n\to+\infty}u_n=+\infty$ et $\lim{n\to+\infty}v_n=+\infty$ alors:
    \begin{enumerate}[(i)]
        \item $\lim{n\to+\infty}u_n+v_n=+\infty$
        \item $\forall k\in\R^*_+ : \lim{n\to+\infty}k\x u_n = +\infty$
        \item $\lim{n\to+\infty}u_n\x v_n = +\infty$
    \end{enumerate}
}

\remarque{Si $\lim{n\to+\infty}u_n=+\infty$ et $\lim{n\to+\infty}v_n=+\infty$, on ne peut rien déduire de la limite de $\dfrac{u_n}{u_v}$}.

\newpage

\theoreme{des limites de référence}{
    Soit $u$ une suite à termes strictement positifs.
    \begin{enumerate}[(i)]
        \item Si $\lim{n\to+\infty}u_n=+\infty$ alors $\lim{n\to+\infty}\dfrac{1}{u_n}=0$
        \item Si $\lim{n\to+\infty}u_n=0$ alors $\lim{n\to+\infty}\dfrac{1}{u_n}=+\infty$
    \end{enumerate}
}

\theoreme{}{
    Soit $u$ une suite. Soient $r$ et $q$ deux réels.
    \begin{enumerate}[(i)]
        \item Si $u$ est arithmétique de raison $r$ :
              \begin{enumerate}[\scriptsize1.]
                  \item Si $r>0$ alors $\lim{n\to+\infty}u_n=+\infty$
                  \item Si $r<0$ alors $\lim{n\to+\infty}u_n=-\infty$
              \end{enumerate}
        \item Si $u$ est géométrique de raison $q$ et $u_0>0$:
              \begin{enumerate}[\scriptsize1.]
                  \item Si $q>1$ alors $\lim{n\to+\infty}u_n=+\infty$
                  \item Si $-1<q<1$ alors $\lim{n\to+\infty}u_n=0$
                  \item Si $q\leq-1$ alors $u$ diverge sans limite
              \end{enumerate}
    \end{enumerate}
}

\subsection{Complément sur les suites divergeant vers $-\infty$}

\definition{}{
    On dit que la suite $u$ tend vers $-\infty$ quand $n$ tend vers $+\infty$ lorsque la propostion $(P)$ suivante est vérifiée :
    $(P)$ : \frquote{$\forall A\in\R, \exists N\in\N,\forall n\geq N : u_n<A$}.\\
    On dit que la suite $u$ diverge vers $-\infty$ et on note $\lim{n\to+\infty}u_n=-\infty$.
}

\theoreme{}{
    Soit $u$ une suite. $\lim{n\to+\infty}u_n=-\infty \equival \lim{n\to+\infty}-(u_n)=+\infty$
}

\theoreme{de comparaison}{
    Soient $u$ et $v$ deux suites.\\
    Si $\exists N\in\N, \forall n\geq N : u_n\leq v_n$ et $\lim{n\to+\infty}v_n=-\infty$ alors $\lim{n\to+\infty}u_n=-\infty$.

}

\theoreme{}{
    Soit $u$ une suite. Si $u$ est décroissante et non minorée alors $\lim{n\to+\infty}u_n=-\infty$.
}

\theoreme{somme et produit}{
    Soient $u$ et $v$ deux suites.
    \begin{enumerate}[(i)]
        \item Si $\lim{n\to+\infty}u_n=-\infty$ et $\lim{n\to+\infty}v_n=-\infty$ alors $\lim{n\to+\infty}u_n+v_n=-\infty$
        \item $\forall k\in\R^*_+$ : Si $\lim{n\to+\infty}u_n=-\infty$ alors $\lim{n\to+\infty}k\x u_n=-\infty$
        \item Si $\lim{n\to+\infty}u_n=-\infty$ et $\lim{n\to+\infty}v_n=-\infty$ alors $\lim{n\to+\infty}u_n\x v_n=+\infty$
        \item Si $\lim{n\to+\infty}u_n=+\infty$ et $\lim{n\to+\infty}v_n=-\infty$ alors $\lim{n\to+\infty}u_n\x v_n=-\infty$
    \end{enumerate}
}

\newpage

\theoreme{de la limite de l'inverse}{
    Soit $u$ une suite à termes strictement négatifs.
    \begin{enumerate}[(i)]
        \item Si $\lim{n\to+\infty}u_n=-\infty$ alors $\lim{n\to+\infty}\dfrac{1}{u_n}=0$
        \item Si $\lim{n\to+\infty}u_n=0$ alors $\lim{n\to+\infty}\dfrac{1}{u_n}=-\infty$
    \end{enumerate}
}

\section{Opérations sur les limites}

\renewcommand{\arraystretch}{3}
\newcolumntype{C}{>{\centering\arraybackslash}p{2.6cm}}
\newcolumntype{I}{!{\vrule width 1pt}}

\begin{tabular}{IC|CIC|C|CI}
    \boldhline
    $\lim{n\to+\infty}u_n=$ & $\lim{n\to+\infty}v_n=$     & $\lim{n\to+\infty}u_n+v_n=$ & $\lim{n\to+\infty}u_n\x v_n=$  & $\lim{n\to+\infty}\dfrac{u_n}{v_n}=$ \\
    \boldhline
    $L\not=0$               & $0$                         & $L$                         & $0$                            & $\pm\infty$ (règles des signes)      \\
    \hline
    $0$                     & $0$                         & $0$                         & $0$                            & Forme indéterminée                   \\
    \hline
    $L\not=0$               & $\pm\infty$                 & $\pm\infty$                 & $\pm\infty$ (règle des signes) & $0$                                  \\
    \hline
    $0$                     & $\pm\infty$                 & $\pm\infty$                 & Forme indéterminée             & $0$                                  \\
    \hline
    $+\infty$               & $+\infty$                   & $+\infty$                   & $+\infty$                      & Forme indéterminée                   \\
    \hline
    $+\infty$               & $0$ avec $\forall n, v_n>0$ & $+\infty$                   & Forme indéterminée             & $+\infty$                            \\
    \hline
    $+\infty$               & $-\infty$                   & Forme indéterminée          & $-\infty$                      & Forme indéterminée                   \\
    \hline
    $-\infty$               & $-\infty$                   & $-\infty$                   & $+\infty$                      & Forme indéterminée                   \\
    \hline
    $-\infty$               & $0$ avec $\forall n, v_n>0$ & $-\infty$                   & Forme indéterminée             & $-\infty$                            \\
    \boldhline
\end{tabular}

\vspace{10pt}

\remarques{
\begin{itemize}
    \item Dans le cas où il existe plusieurs exemples donnant des limites différentes, on parle de forme indéterminée.}
    \item On admet que les règles ci-dessus à propos des opérations sur les limites infinies de suite restent
          valables pour les limites infinies de fonctions.
\end{itemize}

\newpage

\section{Compléments sur les limites de fonctions}

\subsection{Limite infinie d'une fonction en $+\infty$}

\definitions{}{
    \begin{itemize}
        \item $\lim{x\to+\infty}f(x)=+\infty$ signifie \frquote{$\forall A\in\R, \exists B\in\R, \forall x\geq B : f(x)>A$}
        \item $\lim{x\to+\infty}f(x)=-\infty$ signifie \frquote{$\forall A\in\R, \exists B\in\R, \forall x\geq B : f(x)<A$}
    \end{itemize}
}


\theoreme{de comparaison}{
    Soient $f$ et $g$ deux fonctions. Si $\exists A\in\R, \forall x>A, f(x)\leq g(x)$ :
    \begin{enumerate}[(i)]
        \item et si $\lim{x\to+\infty}f(x)=+\infty$ alors $\lim{x\to+\infty}g(x)=+\infty$
        \item et si $\lim{x\to+\infty}g(x)=-\infty$ alors $\lim{x\to+\infty}f(x)=-\infty$
    \end{enumerate}
}

\subsection{Limite d'une fonction en $-\infty$}

\definitions{}{
    \begin{itemize}
        \item $\lim{x\to-\infty}f(x)=l$ signifie \frquote{$\forall \varepsilon\in\R^*_+, \exists A\in\R, \forall x<A : l-\varepsilon<f(x)<l+\varepsilon$}
        \item $\lim{x\to-\infty}f(x)=+\infty$ signifie \frquote{$\forall A\in\R, \exists B\in\R, \forall x\leq B : f(x)>A$}
        \item $\lim{x\to-\infty}f(x)=-\infty$ signifie \frquote{$\forall A\in\R, \exists B\in\R, \forall x\leq B : f(x)<A$}
    \end{itemize}
}

\theoreme{des limites de référence}{
    Pour tout entier $p\geq1$ :
    \begin{enumerate}[(i)]
        \item $\lim{x\to-\infty}\dfrac{1}{x^p}=0$
        \item Si $p$ est pair alors $\lim{x\to-\infty}x^p=+\infty$
        \item Si $p$ est impair alors $\lim{x\to-\infty}x^p=-\infty$
    \end{enumerate}
}

\remarque{On admet que tous les théorème de comparaison (ex: théorème des gendarmes) ainsi que toutes les règles à propos
    des opérations restent valables pour les limites quand $x$ tend vers $-\infty$.}

\subsection{Limite à gauche et à droite d'une fonction en un réel}

On considère un réel $a$ et un intervalle $I$ contenant $a$, ainsi qu'une fonction $f$ définie partout sur $I$ sauf en $a$.
Quand $x$ tend vers $a$, la limite de $f(x)$ peut être différente si $x<a$ et si $x>a$.

\definitions{}{
    \\On peut étudier la limite d'une fonction en un réel $a$ :
    \begin{itemize}
        \item par valeurs inférieures à ce réel. On parle de limite à gauche en $a$ et on note $\lim{x\to a^-}$
        \item par valeurs supérieures à ce réel. On parle de limite à droite en $a$ et on note $\lim{x\to a^+}$
    \end{itemize}
}

\newpage

\subsection{Limites de fonctions composées}


\theoreme{de composition des limites}{
    Soient $a$, $b$ et $c$ les variables désignant soit des réels, soit $-\infty$, soit $+\infty$.\\
    Soit $I$ un intervalle dont l'une des bornes est $a$. Soit $u$ une fonction définie sur $I$.
    Soient $f$ et $g$ deux fonctions telles que $\forall x\in I :f(x)=g(u(x))$.\\
    Si $\lim{x\to a}u(x)=b$ et $\lim{x\to b}g(x)=c$ alors $\lim{x\to a}f(x)=c$.
}

\section{Limites des fonctions exponentielle et logarithme népérien}

\subsection{Fonction exponentielle}

\theoreme{}{
    \vspace{-10pt}
    \begin{enumerate}[(i)]
        \item $\lim{x\to+\infty}e^x=+\infty$
        \item $\lim{x\to-\infty}e^x=0$
    \end{enumerate}
}

\theoreme{des croissances comparées}{
    \vspace{-10pt}
    \begin{enumerate}[(i)]
        \item $\lim{x\to+\infty}\dfrac{e^x}{x}=+\infty$
        \item $\lim{x\to-\infty}x\x e^x=0$
    \end{enumerate}
}

\remarques{
    \begin{itemize}
        \item Ce sont des formes indéterminée.
        \item On peut généraliser ce théorème à n'importe quelle puissance de $x$ :\\
              $\forall n\in\N^* : \lim{x\to+\infty}\dfrac{e^x}{x^n}=+\infty$ et $\lim{x\to-\infty}x^n\x e^x=0$
    \end{itemize}
}

\subsection{Fonction logarithme népérien}

\theoreme{}{
    \vspace{-10pt}
    \begin{enumerate}[(i)]
        \item $\lim{x\to+\infty}\ln(x)=+\infty$
        \item $\lim{x\to0}\ln(x)=-\infty$
    \end{enumerate}
}

\theoreme{des croissances comparées}{
    \vspace{-10pt}
    \begin{enumerate}[(i)]
        \item $\lim{x\to+\infty}\dfrac{\ln(x)}{x}=0$
        \item $\lim{x\to0}x\x \ln(x)=0$
    \end{enumerate}
}

\remarques{
    \begin{itemize}
        \item Ce sont des formes indéterminée.
        \item On peut généraliser ce théorème à n'importe quelle puissance de $x$ :\\$\forall n\in\N^* : \lim{x\to+\infty}\dfrac{\ln(x)}{x^n}=0$
                  et $\lim{x\to0}x^n\x \ln(x)=0$
        \item L'ordre des vitesse de croissance des fonctions est le suivant : $\ln(x)$, $\sqrt{x}$, $x$, $x^2$, $x^3$, \dots, $x^n$, $e^x$.
    \end{itemize}
}



\end{document}
