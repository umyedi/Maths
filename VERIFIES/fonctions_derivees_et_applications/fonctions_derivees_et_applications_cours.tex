\documentclass[11pt,a4paper]{article}
\usepackage[utf8]{inputenc}
\usepackage[T1]{fontenc}
\usepackage{amsfonts}
\usepackage{amssymb}
\usepackage{mdframed}
\usepackage{tikz}
\usepackage{tkz-tab}
\usepackage{wrapfig}
\usepackage{pgfplots}
\usepackage{xcolor}
\usepackage{fancyhdr}
\usepackage{lastpage}
\usepackage{tabularx}
\pgfplotsset{compat=1.17}
\usepackage{multirow}
\usepackage[fleqn]{amsmath}
\setlength{\mathindent}{0pt}

% Défini les couleurs pour les graphiques
\definecolor{dark_green}{HTML}{008000}

% Extensions
\RequirePackage{geometry} 
\geometry{tmargin=1cm,bmargin=1.9cm,lmargin=1.9cm,rmargin=1.9cm}

% Paramètres du titre
\def\classe{$1^{\text{re}}$ Spécialité mathématiques}
\def\titre{Fonctions dérivées et applications}
\def\theme{Analyse - Cours}

% Paramètres de numérotation des pages
\fancypagestyle{custom}{
  \fancyhf{}
  \renewcommand{\headrulewidth}{0pt}
  \lfoot{Analyse - Cours}
  \cfoot{Fonctions dérivées et applications}
  \rfoot{\thepage/\pageref{LastPage}}
}

% Définition d'un type de colonne centré
\newcolumntype{Y}{>{\centering\arraybackslash}X}

\usepackage{titlesec} % Pour personnaliser les titres
\usepackage{xcolor} % Pour définir des couleurs

\title{\titre}
\author{\classe \\ \theme}
\date{}

\renewcommand{\familydefault}{\sfdefault}

% Styles pour les mdframed
\mdfdefinestyle{definitionStyle}{
    leftline=true,
    rightline=false,
    topline=false,
    bottomline=false,
    linewidth=2pt,
    linecolor=black,
    innertopmargin=0pt,
    innerbottommargin=0pt,
    innerrightmargin=0pt,
    innerleftmargin=5pt,
}

\mdfdefinestyle{proprieteStyle}{
    linewidth=1pt,
    linecolor=black,
    innertopmargin=5pt,
    innerbottommargin=5pt,
    innerrightmargin=5pt,
    innerleftmargin=5pt,
}

% Supprime l'indentation des paragraphes
\setlength{\parindent}{0pt}

\begin{document}

\maketitle
\pagestyle{custom}
\thispagestyle{custom}

\section*{I. Fonctions dérivées}
\begin{mdframed}[style=definitionStyle]
    \textbf{Définition :} ~\\
    Soit $f$ une fonction définie sur un intervalle $I$. \\
    Si, pour tout réel $a$ de $I$, le nombre dérivé $f'(a)$ existe, on dit que la fonction $f$ est dérivable sur $I$. \\
    On appelle fonction dérivée de $f$ sur $I$ la fonction qui, a tout réel $x\in I$ associe le réel $f'(x)$. \\
    On la note $f'$.
\end{mdframed}

\subsection*{Dérivées des fonctions usuelles}
\renewcommand{\arraystretch}{1.5}
\begin{tabularx}{\linewidth}{|X|Y|Y|X|}
    \hline
    La fonction $f$ est définie par\dots                                       & $f$ est définie sur\dots     & $f$ est dérivable sur\dots   & La fonction dérivée $f'$ est définie par\dots \\
    \hline
    \multirow{2}{*}{}$f(x)=k$ ($k\in\mathbb{R}$)                               & $\mathbb{R}$                 & $\mathbb{R}$                 & $f'(x)=0$                                       \\
    Fonction constante                                                         &                              &                              &                                                 \\
    \hline
    \multirow{3}{*}{}$f(x)=ax+b$                                               &                              &                              &                                                 \\
    ($a$ et $b$ réels)                                                         & $\mathbb{R}$                 & $\mathbb{R}$                 & $f'(x)=a$                                       \\
    Fonction affine                                                            &                              &                              &                                                 \\
    \hline
    \multirow{2}{*}{}$f(x)=x^n$ ($x\in \mathbb{N^*}$)                          & $\mathbb{R}$                 & $\mathbb{R}$                 & $f'(x)=nx^{n-1}$                                \\
    Fonction puissance                                                         &                              &                              &                                                 \\
    \hline
    \multirow{2}{*}{}$\displaystyle{}f(x)=\frac{1}{x}\color{white}\frac{1^\frac{-}{-}}{1}$                         & $]-\infty;0[\cup]0;+\infty[$ & $]-\infty;0[\cup]0;+\infty[$ & $\displaystyle{}f'(x)=\frac{-1}{x^2}$           \\
    Fonction inverse                                                           &                              &                              &                                                 \\
    \hline
    \multirow{2}{*}{}$\displaystyle{}f(x)=\frac{1}{x^n}\color{white}\frac{-}{-}^\frac{-}{-}\color{black}$ & $]-\infty;0[\cup]0;+\infty[$ & $]-\infty;0[\cup]0;+\infty[$ & $\displaystyle{}f'(x)=\frac{-n}{x^{n+1}}$       \\
    Fonction inverse d'une puissance (avec $x\in \mathbb{N^*}$)                                          &                              &                              &                                                 \\
    \hline
    \multirow{2}{*}{}$f(x)=\sqrt{x}$                                           & $[0;+\infty[$                & $]0;+\infty[$                & $\displaystyle{}f'(x)=\frac{1}{2\sqrt{x}}\color{white}\frac{-}{-}^\frac{-}{-}\color{black}$      \\
    Fonction racine carrée                                                     &                              &                              &                                                 \\
    \hline
    \multirow{2}{*}{}$f(x)=\cos{x}$                          & $\mathbb{R}$                 & $\mathbb{R}$                 & $f'(x)=-\sin{x}$                                \\
    Fonction cosinus                                                         &                              &                              &                                                 \\
    \hline
    \multirow{2}{*}{}$f(x)=\sin{x}$                          & $\mathbb{R}$                 & $\mathbb{R}$                 & $f'(x)=\cos{x}$                                \\
    Fonction sinus                                                         &                              &                              &                                                 \\
    \hline
\end{tabularx}

\newpage

\section*{II. Opérations sur les fonctions dérivables}

\begin{mdframed}[style=proprieteStyle]
    \textbf{Propriété :} ~\\
    Soient $u$ et $v$ deux fonctions dérivables sur un intervalle $I$, et $k$ un nombre réel. \\
    Les fonctions suivantes sont dérivables sur $I$ de fonctions dérivées : \\

    \renewcommand{\arraystretch}{2}
    \begin{tabularx}{\linewidth}{|Y|Y|}
        \hline
        \textbf{Fonction}                     & \textbf{Fonction dérivée}                                      \\
        \hline
        Somme $u+v$                           & $(u+v)'=u'+v'$                                                 \\
        \hline
        Produit par un réel $ku$              & $(ku)'=ku'$                                                    \\
        \hline
        Produit $uv$                          & $(uv)'=u'v+uv'$                                                \\
        \hline
        Quotient $\displaystyle\frac{u}{v}\color{white}\frac{-}{\frac{-}{-}}$ (avec $v\not=0$)& $\displaystyle{}\left(\frac{u}{v}\right)'=\frac{u'v-uv'}{v^2}$ \\
        \hline
        Inverse $\displaystyle\frac{1}{u}\color{white}\frac{-}{\frac{-}{-}}$ (avec $u\not=0$)& $\displaystyle\left(\frac{1}{u}\right)'=\frac{-u'}{u^2}$ \\
        \hline
    \end{tabularx}
\end{mdframed}

\textbf{Exemples :}

\begin{minipage}[t]{0.5\textwidth}
    Soit $f(x)=3x\sqrt{x}$ sur $I=]0;+\infty[$. \\

    Posons $u(x)=3x$ et $v(x)=\sqrt{x}$. \\
    On a $u(x)'=3$ et $v(x)'=\frac{1}{2\sqrt{x}}$. \\
    On a $(uv)'=u'v+uv'$.
    \begin{alignat*}{2}
         & \text{Donc } & f'(x) & =u'(x)v(x)+u(x)v'(x)                                                                 \\
         &              &       & = 3\sqrt{x}+3x\times \frac{1}{2\sqrt{x}}                                             \\
         &              &       & = 3\sqrt{x}+\frac{3x}{2\sqrt{x}} \color{dark_green} \times \frac{\sqrt{x}}{\sqrt{x}} \\
         &              &       & = 3\sqrt{x}+ \frac{3x\sqrt{x}}{2x}                                                   \\
         &              &       & = 3\sqrt{x}+1,5\sqrt{x}=4,5\sqrt{x}
    \end{alignat*}
\end{minipage}
\hfill
\begin{minipage}[t]{0.5\textwidth}
    Soit $g(x)=\frac{2x-1}{x-5}$ sur $I=]-\infty; 5[\cup]5; +\infty[$. \\

    Posons $u(x)=2x-1$ et $v(x)=x-5$. \\
    On a $u(x)'=2$ et $v(x)'=1$. \\
    On a $\displaystyle\left(\frac{u}{v}\right)'=\frac{u'v-uv'}{v^2}$.
    \begin{alignat*}{2}
         & \text{Donc } & g'(x) & =\frac{u'(x)v(x)-u(x)v'(x)}{(v(x))^2} \\
         &              &       & =\frac{2(x-5)-(2x-1)\times1}{(x-5)^2} \\
         &              &       & =\frac{2x-10-2x+1}{(x-5)^2}           \\
         &              &       & =\frac{-9}{(x-5)^2}
    \end{alignat*}
\end{minipage}


\begin{mdframed}[style=proprieteStyle]
    \textbf{Propriété admise :} ~\\
    On considère un intervalle $I$ et $a$ et $b$ deux réels.\\
    Soit $J$ l'intervalle formé des valeurs prises par $ax+b$ lorsque $x$ décrit l'intervalle $I$.\\
    Si la fonction $g$ est dérivable sur $J$, alors la fonction $f$ définie sur $I$ par $f(x)=g(ax+b)$ est dérivable sur $I$ et, pour tout réel $x$ de $I$, on a $f'(x)=a\times g'(ax+b)$.
\end{mdframed}

\textbf{Exemple :} ~\\
Soit $f$ définie sur $[\frac{-2}{5};+\infty[$ par $f(x)=\sqrt{5x+2}$

On a $f:x \longrightarrow 5x+2 \longrightarrow  \sqrt{5x+2}$

La fonction $f$ est de la forme $f(x)=g(ax+b)$ avec $a=5$, $b=2$ et $g:x\mapsto\sqrt{x}$
\begin{alignat*}{2}
     & \text{Donc } & f'(x) & =ag'(ax+b) \text{ avec } g':x\mapsto\frac{1}{2\sqrt{x}} \\
     &              &       & =5\times \frac{1}{\sqrt{5x+2}} =\frac{5}{\sqrt{5x+2}}
\end{alignat*}

Pour tout $x\in[\frac{-2}{5};+\infty[$, $5x+2\in[0;+\infty[$.

            Or $g$ est dérivable sur $]0;+\infty[$.

Donc $f$ est dérivable sur $[\frac{-2}{5};+\infty[$.

\section*{III. Applications de la dérivation}

\subsection*{1. Étude des variations d'une fonction}

\begin{mdframed}[style=proprieteStyle]
    \textbf{Théorème admis :} ~\\
    Soit $f$ une fonction dérivable sur un intervalle $I$, de fonction dérivée $f'$.
    \begin{itemize}
        \item Si $f$ est croissante sur $I$, alors $f'$ est positive sur $I$.
        \item Si $f$ est décroissante sur $I$, alors $f'$ est négative sur $I$.
        \item Si $f$ est constante sur $I$, alors $f'$ est nulle sur $I$.
    \end{itemize}
\end{mdframed}

\begin{mdframed}[style=proprieteStyle]
    \textbf{Théorème réciproque admis :} ~\\
    Soit $f$ une fonction dérivable sur un intervalle $I$, de fonction dérivée $f'$.
    \begin{itemize}
        \item Si $f'$ est strictement positive sur $I$, sauf pour un nombre fini de réel où elle s'annule, alors $f$ est strictement croissante sur $I$.
        \item Si $f'$ est strictement négative sur $I$, sauf pour un nombre fini de réel où elle s'annule, alors $f$ est strictement décroissante sur $I$.
        \item Si $f'$ est nulle sur $I$, alors $f$ est constante sur $I$.
    \end{itemize}
\end{mdframed}

\subsection*{2. Étude des extrema d'une fonction}

\begin{mdframed}[style=definitionStyle]
    \textbf{Définitions :} ~\\
    Soit $f$ une fonction définie sur un intervalle $I$ et $c$ un réel de $I$ et qui n'est pas une borne de $I$.
    \vspace{-4pt}
    \begin{itemize}
        \item Dire que $f(c)$ est un maximum local de $f$ signifie qu'il existe deux réels $a$ et $b$ dans $I$ tels que $c\in]a;b[$ et que pour tout réel $x\in]a;b[$, $f(x) \leq f(c)$.
        \item Dire que $f(c)$ est un minimum local de $f$ signifie qu'il existe deux réels $a$ et $b$ dans $I$ tels que $c\in]a;b[$ et que pour tout réel $x\in]a;b[$, $f(x) \geq f(c)$.
        \item Un extremum local est un minimum ou un maximum local.
    \end{itemize}



\end{mdframed}

\begin{minipage}{0.4\textwidth}
    \begin{tikzpicture}
        \begin{axis}[
                axis lines=middle,
                xmin=-3, xmax=6,
                ymin=-2, ymax=4,
                xtick=\empty,
                ytick=\empty,
                smooth,
                clip=false
            ]
            \addplot+[mark=none, smooth, thick, red, domain=-1:5] {2*sin(0.8*deg(x))+0.8};

            % Calculate the maximum local value of the function
            \pgfmathsetmacro{\xmax}{(3*pi)/4}
            \pgfmathsetmacro{\ymax}{2*sin(0.8*\xmax*180/pi)+0.8}

            % Draw double horizontal arrow
            \draw[dashed, thin, blue] (axis cs:0, 2.8) -- (axis cs:2,2.8);
            \draw[dashed, thin, blue] (axis cs:2, 0) -- (axis cs:2,2.8);
            \draw [<->, dark_green, thick] (axis cs: 1, 2.8) -- (axis cs: 3, 2.8);
            \draw [|-|, dark_green, thick] (axis cs: 1, 0) -- (axis cs: 3, 0);

            \node[label={-90:$\color{dark_green}a$}] at (axis cs:1, 0) {};
            \node[label={-90:$\color{dark_green}b$}] at (axis cs:3, 0.1) {};
            \node[label={-90:$\color{blue}c$}] at (axis cs:2, 0.1) {};
            \node[label={-120:$\color{blue}f(c)$}] at (axis cs:0.2, 3.5) {};

        \end{axis}
    \end{tikzpicture}
\end{minipage}
\hspace{0.05\textwidth}
\begin{minipage}{0.55\textwidth}
    \begin{mdframed}[style=proprieteStyle]
        \textbf{Théorème de la condition nécessaire sur l'existence d'un extremum local \emph{(admis)} :} ~\\
        Soit $f$ une fonction dérivable sur un intervalle ouvert $I$ et $a$ un réel de $I$. \\
        Si $f$ présente un extremum local en $a$ alors $f'(a)=0$.
    \end{mdframed}
    \textbf{Remarque :} La réciproque est fausse. En effet, pour $f:x\mapsto x^3$ on a $f:x\mapsto 3x^2$ donc $f'(0)=0$
    mais $f$ n'admet pas d'extremum local en $O$.
\end{minipage}

\begin{mdframed}[style=proprieteStyle]
    \textbf{Théorème de la condition suffisante sur l'existence d'un extremum local :} ~\\
    Soit $f$ une fonction dérivable sur un intervalle ouvert $I$, de dérivée $f'$ et $a\in I$.
    Si la dérivée $f'$ s'annule en $a$ en changeant de signe en $a$, alors la fonction $f$ admet un extremum local en $a$.
\end{mdframed}

\end{document}