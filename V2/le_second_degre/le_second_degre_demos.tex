\documentclass[10pt]{article} % Sets the document class to 'article' with a font size of 12pt

\newcommand{\DocumentTitle}{Le second degré}
\newcommand{\DocumentTheme}{Algebre}
\newcommand{\DocumentType}{Démonstrations}

% Preamble Section
\usepackage[a4paper, margin=1in]{geometry} % Sets the paper size to A4 and all margins to 1 inch
\usepackage[utf8]{inputenc}   % Allows for input of international characters
\usepackage[T1]{fontenc}      % Utilisation de l'encodage T1
\usepackage{lmodern}          % Support des polices avec le module 'french'
\usepackage[french]{babel}    % Ajout du support pour le français
\usepackage[fleqn]{amsmath}   % Imports the AMS math package for advanced math formatting ([fleqn] -> align alignat to the left)
\usepackage{amsfonts}         % Imports AMS fonts for math fonts
\usepackage{amssymb}          % Imports AMS symbols for math symbols
\usepackage{amsthm}           % Imports AMS theorems
\usepackage{fancyhdr}         % Import to custom the page footer
\usepackage{mdframed}         % Imports styles
\usepackage{enumerate}        % Imports styles for enumerate
\usepackage{multicol}         
\usepackage{xcolor}           % Couleurs
\usepackage{tikz}             % Package général pour les graphiques
\usepackage{pgfplots}         % Compléments pour les graphiques
\usepackage{tkz-tab}          % Tableaux de variations
\usepackage{lastpage}         % Pour avoir le total de page dans le footer

\pgfplotsset{compat=newest}   % Active les dernières fonctionnalités de pgfplots

% Enlève l'indentation des bloc de paragraphes et d'équation
\setlength{\parindent}{0pt}
\setlength{\mathindent}{0pt}

% ========== Title Section ==========

\title{\DocumentTitle} % Title of the document
\author{\DocumentTheme\space-\space\DocumentType} % Author's name
\date{} % Date

\pagestyle{fancy}
\fancyhf{} % Clear all header and footer fields

% ========== Footer Section ==========

\newcommand{\customfooter}{
    \fancyfoot[L]{\DocumentTheme\space-\space\DocumentType}
    \fancyfoot[C]{\DocumentTitle}
    \fancyfoot[R]{\thepage/\pageref{LastPage}}
    % Remove the line below the header and above the footer
    \renewcommand{\headrulewidth}{0pt}
    \renewcommand{\footrulewidth}{0pt}
}

% Apply the custom footer to the fancy style
\customfooter

% Apply the custom footer to the plain style (used on the first page)
\fancypagestyle{plain}{\customfooter}

% ========== Sections ==========

% Sections : I. / II. / III. ...
\renewcommand{\thesection}{\Roman{section}}
% Subsection : I.1 / I.2 / I.3 ...
\renewcommand{\thesubsection}{\thesection. \arabic{subsection}}
% Subsubsection : I.1.a / I.1.b / I.1.c ...
\renewcommand{\thesubsubsection}{\thesubsection. \alph{subsubsection}}

% ========== Colors (from Geogebra) ==========

\definecolor{green}{HTML}{006400}
\definecolor{red}{HTML}{CC0000}
\definecolor{blue}{HTML}{0000FF}
\definecolor{orange}{HTML}{FF5500}
\definecolor{purple}{HTML}{9933FF}
\definecolor{gray}{HTML}{666666}
\definecolor{brown}{HTML}{993300}

\newcommand{\green}[1]{\color{green}{#1}\color{black}}
\newcommand{\red}[1]{\color{red}{#1}\color{black}}
\newcommand{\blue}[1]{\color{blue}{#1}\color{black}}
\newcommand{\orange}[1]{\color{orange}{#1}\color{black}}
\newcommand{\purple}[1]{\color{purple}{#1}\color{black}}
\newcommand{\gray}[1]{\color{gray}{#1}\color{black}}
\newcommand{\brown}[1]{\color{brown}{#1}\color{black}}


% ========== Définition(s) ==========

% Style de la boîte de définition(s)
\mdfdefinestyle{DefinitionStyle}{
    leftmargin=0cm,
    rightmargin=0cm,
    linecolor=black,
    linewidth=2pt,
    topline=false,
    bottomline=false,
    rightline=false
}

% Définition de la commande '\definition'
\newcommand{\definition}[2]{
    \begin{mdframed}[style=DefinitionStyle]
        \textbf{Définition : #1}\\
        #2
    \end{mdframed}
}

% Définition de la commande '\definitions'
\newcommand{\definitions}[2]{
    \begin{mdframed}[style=DefinitionStyle]
        \textbf{Définitions : #1}\\
        #2
    \end{mdframed}
}

% ========== Propriété(s) et Théorème(s) ==========

% Style de la boîte de propriété(s) / théorème(s)
\mdfdefinestyle{ProprieteStyle}{
    leftmargin=0cm,
    rightmargin=0cm,
    linecolor=black,
    linewidth=1pt
}

% Création d'un compteur pour les propriétés
\newcounter{propriete}

% Définition de la commande '\propriete'
\newcommand{\propriete}[2]{
    \refstepcounter{propriete}
    \begin{mdframed}[style=ProprieteStyle] % ou un style différent si vous en avez créé un
        \textbf{Propriété \thepropriete\ :\space#1}\\
        #2
    \end{mdframed}
}

% Définition de la commande 'proprietes'
\newcommand{\proprietes}[2]{
    \refstepcounter{propriete}
    \begin{mdframed}[style=ProprieteStyle] % ou un style différent si vous en avez créé un
        \textbf{Propriétés \thepropriete\ :\space#1}\\
        #2
    \end{mdframed}
}

% Création d'un compteur pour les théorèmes
\newcounter{theoreme}

% Définition de la commande '\theoreme'
\newcommand{\theoreme}[2]{
    \refstepcounter{theoreme}
    \begin{mdframed}[style=ProprieteStyle] % ou un style différent si vous en avez créé un
        \textbf{Théorème \thetheoreme\ :\space#1}\\
        #2
    \end{mdframed}
}

% Définition de la commande '\theoremes'
\newcommand{\theoremes}[2]{
    \refstepcounter{theoreme}
    \begin{mdframed}[style=ProprieteStyle] % ou un style différent si vous en avez créé un
        \textbf{Théorèmes \thetheoreme\ :\space#1}\\
        #2
    \end{mdframed}
}

% ========== Exemple(s) ==========

% Définition de la commande 'exemple'
\newcommand{\exemple}[2]{
    \textbf{Exemple :} #1
    \begin{quote}
        #2
    \end{quote}
}

% Définition de la commande 'exemples'
\newcommand{\exemples}[2]{
    \textbf{Exemples :} #1
    \begin{quote}
        #2
    \end{quote}
}

% ========== Remarques ==========

% Définition de la commande 'remarque'
\newcommand{\remarque}[1]{
    \textbf{Remarque :} #1
}

% Définition de la commande 'remarques'
\newcommand{\remarques}[1]{
    \textbf{Remarques :} #1
}

% ========== Démonstration ==========

\newcommand{\demonstration}[2]{
    \textbf{Démonstration :} #1
    \begin{quote}
        #2
    \end{quote}
}

% ========== Notations ==========

% Ensembles
\newcommand{\C}{\mathbb{C}}
\newcommand{\R}{\mathbb{R}}
\newcommand{\Q}{\mathbb{Q}}
\newcommand{\D}{\mathbb{D}}
\newcommand{\Z}{\mathbb{Z}}
\newcommand{\N}{\mathbb{N}}


\begin{document}

\maketitle % Generates the title

\demonstration{Théorème 1}{
    Soient $a$, $b$ et $c$ des réels tel que $a\not=0$.
    \begin{alignat*}{2}
         & \text{Pour tout $x$ réel, on a } & f(x) & = ax^2+bx+c                                                                                                                                      \\
         &                                  &      & = a\left[x^2+\frac{b}{a}x\right]+c                                                                                                               \\
         &                                  &      & = a\left[x^2+\frac{2b}{2a}x + \green{\left(\frac{b}{2a}\right)^2-\left(\frac{b}{2a}\right)^2}\right]+c \quad\text{IR n°1 : } a^2+2ab+b^2=(a+b)^2 \\
         &                                  &      & = a\left[\left(x+\frac{b}{2a}\right)^2-\frac{b^2}{4a^2}\right]+c                                                                                 \\
         &                                  &      & = a\left(x+\frac{b}{2a}\right)^2-\frac{b^2}{4a}+c                                                                                                \\
         &                                  &      & = a(x-\alpha)+\beta \text{ avec $\alpha=\frac{-b}{2a}$ et $\beta=\frac{-b^2}{4a}+c=f(\alpha)$}
    \end{alignat*}
}

\demonstration{Propriété 1}{
    \begin{enumerate}[(i)]
        \item 1$^{\text{er}}$ cas : $a>0$
              \begin{itemize}
                  \item Soient $x_1$ et $x_2$ deux réels de l'intervalle $]-\infty;\alpha[$ tels que $x_1<x_2<\alpha$.
                                    \begin{align*}
                                        x_1-\alpha            & <  x_2-\alpha < 0                                                                    \\
                                        (x_1-\alpha)^2        & >  (x_2-\alpha)^2 \text{ car }x_1-\alpha \text{ et }x_2-\alpha \text{ sont négatifs} \\
                                        a(x_1-\alpha)^2       & > a(x_2-\alpha)^2 \text{ car } a>0                                                   \\
                                        a(x_1-\alpha)^2+\beta & > a(x_2-\alpha)^2+\beta                                                              \\
                                        f(x_1)                & > f(x_2)
                                    \end{align*}
                                    Donc $f$ est strictement décroissante sur $]-\infty;\alpha[$.
                  \item Soient $x_1$ et $x_2$ deux réels de l'intervalle $[\alpha;\infty[$ tels que $\alpha<x_1<x_2$.
                        \begin{align*}
                            0\leq x_1-\alpha      & <  x_2-\alpha                                                                        \\
                            (x_1-\alpha)^2        & <  (x_2-\alpha)^2 \text{ car }x_1-\alpha \text{ et }x_2-\alpha \text{ sont positifs} \\
                            a(x_1-\alpha)^2       & < a(x_2-\alpha)^2 \text{ car } a>0                                                   \\
                            a(x_1-\alpha)^2+\beta & < a(x_2-\alpha)^2+\beta                                                              \\
                            f(x_1)                & < f(x_2)
                        \end{align*}
                        Donc $f$ est strictement croissante sur $[\alpha;\infty[$.
              \end{itemize}


        \item 2$^{\text{ème}}$ cas : $a<0$
              \begin{itemize}
                  \item Soient $x_1$ et $x_2$ deux réels de l'intervalle $]-\infty;\alpha[$ tels que $x_1<x_2<\alpha$.
                                    \begin{align*}
                                        x_1-\alpha            & <  x_2-\alpha < 0                                                                    \\
                                        (x_1-\alpha)^2        & >  (x_2-\alpha)^2 \text{ car }x_1-\alpha \text{ et }x_2-\alpha \text{ sont positifs} \\
                                        a(x_1-\alpha)^2       & < a(x_2-\alpha)^2 \text{ car } a<0                                                   \\
                                        a(x_1-\alpha)^2+\beta & < a(x_2-\alpha)^2+\beta                                                              \\
                                        f(x_1)                & < f(x_2)
                                    \end{align*}
                                    Donc $f$ est strictement croissante sur $]-\infty;\alpha[$.
                  \item Soient $x_1$ et $x_2$ deux réels de l'intervalle $[\alpha;\infty[$ tels que $\alpha<x_1<x_2$.
                        \begin{align*}
                            0\leq x_1-\alpha      & <  x_2-\alpha                                                                        \\
                            (x_1-\alpha)^2        & <  (x_2-\alpha)^2 \text{ car }x_1-\alpha \text{ et }x_2-\alpha \text{ sont positifs} \\
                            a(x_1-\alpha)^2       & > a(x_2-\alpha)^2 \text{ car } a<0                                                   \\
                            a(x_1-\alpha)^2+\beta & > a(x_2-\alpha)^2+\beta                                                              \\
                            f(x_1)                & > f(x_2)
                        \end{align*}
                        Donc $f$ est strictement décroissante sur $[\alpha;\infty[$.
              \end{itemize}
    \end{enumerate}
}

\demonstration{}{
    Pour toute fonction du second degrée $f(x)=ax^2+bx+c$, on a vu que $f(x)$ peut s'écrire sous la forme $f(x)=a\left(x+\dfrac{b}{2a}\right)+\dfrac{-b}{4a}+c$.
    \begin{alignat*}{2}
         & \text{Donc } & f(x) & = a\left(x+\frac{b}{2a}\right)+\frac{-b}{4a}+\frac{4ac}{4a}                      \\
         &              &      & = a\left[\left(x+\frac{b}{2a}\right)^2+\frac{-b^2}{4a^2}+\frac{4ac}{4a^2}\right] \\
         &              &      & = a\left[\left(x+\frac{b}{2a}\right)^2-\frac{b^2-4ac}{4a^2}\right]               \\
         &              &      & \quad \text{On pose $\Delta=-b^2-4ac$}                                           \\
         &              &      & = a\left[\left(x+\frac{b}{2a}\right)^2-\frac{\Delta}{4a^2}\right]
    \end{alignat*}

    \begin{itemize}
        \item 1$^{\text{er}}$ cas : $\Delta<0$ ~\\
              Si $\Delta<0$ alors $f(x)>0$ pour tout $x\in\mathbb{R}$ (si $a>0$) ou $f(x)<0$ pour tout $x\in\mathbb{R}$ (si $a<0$). \\
              Donc $f(x)=0$ n'admet pas de solution et $f(x)$ n'est pas factorisable.

        \item 2$^{\text{ème}}$ cas : $\Delta=0$ \\
              Si $\Delta=0$ alors $f(x)=a( x+\frac{b}{2a})^2$ ou $f(x)=a(x-\alpha)^2$ avec $\alpha=\frac{-b}{2a}$. \\
              Donc l'équation $f(x)=0$ admet une solution (double) $\alpha$ et $f(x)$ est factorisable ou $f(x)=a(x-\alpha)^2$.

        \item 3$^{\text{ème}}$ cas : $\Delta>0$
              \begin{alignat*}{2}
                   & \text{Si $\Delta>0$ alors } & f(x) & = a\left[\left(x+\frac{b}{2a}\right)^2-\left(\frac{\sqrt{\Delta}}{2a}\right)^2\right]                      \\
                   &                             &      & =a\left(x+\frac{b}{2a}-\frac{\sqrt{\Delta}}{2a}\right)\left(x+\frac{b}{2a}+\frac{\sqrt{\Delta}}{2a}\right) \\
                   &                             &      & =a\left(x-\frac{-b-\sqrt{\Delta}}{2a}\right)\left(x-\frac{-b+\sqrt{\Delta}}{2a}\right)                     \\
              \end{alignat*}
              Donc $f(x)$ est factorisable en $f(x)=a(x-x_1)(x-x_2)$ avec $x_1=\frac{-b-\sqrt{\Delta}}{2a}$ et $x_2=\frac{-b+\sqrt{\Delta}}{2a}$. \\
              L'équation $f(x)=0$ admet deux solutions $x_1$ et $x_2$.
    \end{itemize}
}



\end{document} % Ends the document
