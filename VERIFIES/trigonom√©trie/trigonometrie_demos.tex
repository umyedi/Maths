\documentclass[11pt,a4paper]{article}
\usepackage[utf8]{inputenc}
\usepackage[T1]{fontenc}
\usepackage{amsfonts}
\usepackage{amssymb}
\usepackage{mdframed}
\usepackage{tikz}
\usepackage{tkz-tab}
\usepackage{pgfplots}
\usepackage{xcolor}
\usepackage{fancyhdr}
\usepackage{lastpage}
\usepackage[fleqn]{amsmath}
\usepackage{yhmath}
\usepackage{array}
\setlength{\mathindent}{0pt}

% Spécifications du document
\newcommand{\doctitre}{Trigonométrie} % Ex: Le second degré
\newcommand{\docniveau}{$1^{\text{re}}$ Spécialité mathématiques} % Ex: $1^{\text{re}}$ Spécialité mathématiques
\newcommand{\doctheme}{Analyse} %Ex: Algèbre
\newcommand{\doctype}{Démonstrations} % Ex: Démonstrations
\newcommand{\docshorttype}{Démo} % Démo

% Couleurs pour les graphiques
\definecolor{dark_green}{HTML}{008000}

% Paramètres du document
\RequirePackage{geometry}
\geometry{tmargin=1cm,bmargin=1.9cm,lmargin=1.9cm,rmargin=1.9cm}
\renewcommand{\familydefault}{\sfdefault}
\setlength{\parindent}{0pt}
\title{\doctitre}
\author{\docniveau \\ \doctheme\text{ - }\doctype}
\date{}
\fancypagestyle{custom}{
  \fancyhf{}
  \renewcommand{\headrulewidth}{0pt}
  \lfoot{\doctheme\text{ - }\docshorttype}
  \cfoot{\doctitre} % Change \titre to \doctitre
  \rfoot{\thepage/\pageref{LastPage}}
}

% Styles pour les mdframed
\mdfdefinestyle{definitionStyle}{
    leftline=true,
    rightline=false,
    topline=false,
    bottomline=false,
    linewidth=2pt,
    linecolor=black,
    innertopmargin=0pt,
    innerbottommargin=0pt,
    innerrightmargin=0pt,
    innerleftmargin=5pt,
}

\mdfdefinestyle{proprieteStyle}{
    linewidth=1pt,
    linecolor=black,
    innertopmargin=5pt,
    innerbottommargin=5pt,
    innerrightmargin=5pt,
    innerleftmargin=5pt,
}

\newcolumntype{Y}{>{\centering\arraybackslash}m{1.2cm}}


% ----- DEBUT DU DOCUMENT -----
\begin{document}

% Style et numérotation
\maketitle
\pagestyle{custom}
\thispagestyle{custom}

\section*{I. Lecture sur le cercle trigonométrique}

\subsection*{1. Le cercle trigonométrique}

\subsection*{2. Longueur d'un arc et radian}


\section*{II. Enroulement de la droite des réels sur le cercle trigonométrique}

\section*{III. Cosinus et sinus d'un nombre réel}

\subsection*{1. Définitions}

\subsection*{2. Valeurs remarquables du cosinus et du sinus}

\underline{Démonstration \emph{(valeurs remarquables)} :}

\begin{itemize}
  \item Pour $\frac{\pi}{3}$ : \\ 
  -> Ajouter schéma \\
  Le triangle $IOM$ est isocèle (car $IO=OM$) avec un angle de $60^\circ$. Il est donc équilatéral. \\
  Donc la hauteur $(HM)$ est aussi la médiatrice du segment $[OI]$. \\
  Donc $OH=\frac{1}{2}$, soit $\cos{\left(\frac{\pi}{3}\right)}=\frac{1}{2}$.
  \begin{align*}
    \text{De plus, } &\cos^2{(\frac{\pi}{3})}+\sin^2{(\frac{\pi}{3})}=1\\
    \Leftrightarrow\,\,\,&\sin^2{(\frac{\pi}{3})}=1-\frac{1}{4} \\
    \Leftrightarrow\,\,\,&\sin{(\frac{\pi}{3})}=\sqrt{\frac{3}{4}}
  \end{align*}
  Comme $\sin{(\frac{\pi}{3})}>0$, on a $\sin{(\frac{\pi}{3})}=\sqrt{\frac{3}{4}}=\frac{\sqrt{3}}{\sqrt{4}}=\frac{\sqrt{3}}{2}$
  \item Pour $\frac{\pi}{6}$ : \\
  -> Ajouter schéma
  \begin{align*}
    \text{Par des calculs analogues dans le triangle $OMJ$, on obtient } &OH=\cos{(\frac{\pi}{6})}=\frac{\sqrt{3}}{2} \\
    \text{et } &OH'=\sin{(\frac{\pi}{6})}=\frac{1}{2}.
  \end{align*}
  \item Pour $\frac{\pi}{4}$ : \\
  -> Ajouter schéma \\
  Le triangle $OMH$ est rectangle avec un angle de $45^\circ$ donc il est rectangle isocèle. \\
  Donc $OH=HM$ \\
  D'après le théorème de Pythagore :
  \begin{align*}
    &OH^2+HM^2=OM^2 \\
    \Leftrightarrow \, \, \,&2OH^2=OM^2 \\
    \Leftrightarrow \, \, \,&OH^2=\frac{1}{2} \\
    \Leftrightarrow \, \, \,&OH=\sqrt{\frac{1}{2}}=\frac{\sqrt{1}}{\sqrt{2}} = \frac{1}{\sqrt{2}}\color{dark_green}\times\frac{\sqrt{2}}{\sqrt{2}}\color{black}=\frac{\sqrt{2}}{2}
  \end{align*}
  \begin{align*}
    \text{Donc } &OH=\cos{(\frac{\pi}{4})}=\frac{\sqrt{2}}{2} \\
    \text{et } &HM=\sin{(\frac{\pi}{4})}=\frac{\sqrt{2}}{2}
  \end{align*}
\end{itemize}

\subsection*{3. Lien avec le cosinus et sinus dans un triangle rectangle}

\section*{IV. Fonctions cosinus et sinus}


\end{document}
% ----- FIN DU DOCUMENT -----