\documentclass[10pt]{article}

\newcommand{\DocumentTitle}{Relations}
\newcommand{\DocumentTheme}{Séminaire}
\newcommand{\DocumentType}{Cours}

% Preamble Section
\usepackage[a4paper, margin=1in]{geometry} % Sets the paper size to A4 and all margins to 1 inch
\usepackage[utf8]{inputenc}   % Allows for input of international characters
\usepackage[T1]{fontenc}      % Utilisation de l'encodage T1
\usepackage{lmodern}          % Support des polices avec le module 'french'
\usepackage[french]{babel}    % Ajout du support pour le français
\usepackage[fleqn]{amsmath}   % Imports the AMS math package for advanced math formatting ([fleqn] -> align alignat to the left)
\usepackage{amsfonts}         % Imports AMS fonts for math fonts
\usepackage{amssymb}          % Imports AMS symbols for math symbols
\usepackage{amsthm}           % Imports AMS theorems
\usepackage{yhmath}           % Used for "\wideparen{}" 
\usepackage{fancyhdr}         % Import to custom the page footer
\usepackage{mdframed}         % Imports styles
\usepackage{enumerate}        % Imports styles for enumerate
\usepackage{multicol}
\usepackage{xcolor}           % Couleurs
\usepackage{tikz}             % Package général pour les graphiques
\usetikzlibrary{calc}
\usepackage{pgfplots}         % Compléments pour les graphiques
\usepackage{tkz-tab}          % Tableaux de variations
\usepackage{lastpage}         % Pour avoir le total de page dans le footer
\usepackage{makecell}         % Retour à la ligne dans une case d'un tabular
\usepackage{stmaryrd}         % Intervalles entières : \llbracket et \rrbracket
\usepackage{cancel}           % Permet de barrer des termes
\usepackage{forest}           % Permet de créer des arbres pondérés
\usepackage{tabularx}         % Permet de générer des tableaux (ex: cours fonction dérivées et application)
\usepackage{multirow}         % Permet de faire plusieurs lignes dans des cases de talbeaux


\pgfplotsset{compat=newest}   % Active les dernières fonctionnalités de pgfplots

% Enlève l'indentation des bloc de paragraphes et d'équation
\setlength{\parindent}{0pt}
\setlength{\mathindent}{0pt}

% ========== Title Section ==========

\title{\DocumentTitle} % Title of the document
\author{\DocumentTheme\space-\space\DocumentType} % Author's name
\date{} % Date

\pagestyle{fancy}
\fancyhf{} % Clear all header and footer fields

% ========== Footer Section ==========

\newcommand{\customfooter}{
    \fancyfoot[L]{\DocumentTheme\space-\space\DocumentType}
    \fancyfoot[C]{\DocumentTitle}
    \fancyfoot[R]{\thepage/\pageref{LastPage}}
    % Remove the line below the header and above the footer
    \renewcommand{\headrulewidth}{0pt}
    \renewcommand{\footrulewidth}{0pt}
}

% Apply the custom footer to the fancy style
\customfooter

% Apply the custom footer to the plain style (used on the first page)
\fancypagestyle{plain}{\customfooter}

% ========== Sections ==========

% Sections : I. / II. / III. ...
\renewcommand{\thesection}{\Roman{section}}
% Subsection : I.1 / I.2 / I.3 ...
\renewcommand{\thesubsection}{\thesection. \arabic{subsection}}
% Subsubsection : I.1.a / I.1.b / I.1.c ...
\renewcommand{\thesubsubsection}{\thesubsection. \alph{subsubsection}}

% ========== Colors (from Geogebra) ==========

\definecolor{green}{HTML}{006400}
\definecolor{red}{HTML}{CC0000}
\definecolor{blue}{HTML}{0000FF}
\definecolor{orange}{HTML}{FF5500}
\definecolor{purple}{HTML}{9933FF}
\definecolor{gray}{HTML}{666666}
\definecolor{brown}{HTML}{993300}
\definecolor{black}{HTML}{000000}
\definecolor{white}{HTML}{FFFFFF}

\newcommand{\green}[1]{\color{green}{#1}\color{black}}
\newcommand{\red}[1]{\color{red}{#1}\color{black}}
\newcommand{\blue}[1]{\color{blue}{#1}\color{black}}
\newcommand{\orange}[1]{\color{orange}{#1}\color{black}}
\newcommand{\purple}[1]{\color{purple}{#1}\color{black}}
\newcommand{\gray}[1]{\color{gray}{#1}\color{black}}
\newcommand{\brown}[1]{\color{brown}{#1}\color{black}}


% ========== Définition(s) ==========

% Style de la boîte de définition(s)
\mdfdefinestyle{DefinitionStyle}{
    leftmargin=0cm,
    rightmargin=0cm,
    linecolor=black,
    linewidth=2pt,
    topline=false,
    bottomline=false,
    rightline=false
}

% Définition de la commande '\definition'
\newcommand{\definition}[2]{%
    \begin{mdframed}[style=DefinitionStyle]
        \ifstrempty{#1}{% Teste si le premier argument est vide
            \textbf{Définition :}\ % Si vide, n'affiche pas de titre de théorème
        }{%
            \textbf{Définition #1 :}\ % Si non vide, affiche le titre avec le premier argument
        }\\#2
    \end{mdframed}
}

% Définition de la commande '\definitions'
\newcommand{\definitions}[2]{%
    \begin{mdframed}[style=DefinitionStyle]
        \ifstrempty{#1}{% Teste si le premier argument est vide
            \textbf{Définitions :}\ % Si vide, n'affiche pas de titre de théorème
        }{%
            \textbf{Définitions #1 :}\ % Si non vide, affiche le titre avec le premier argument
        }\\#2
    \end{mdframed}
}

% ========== Propriété(s) et Théorème(s) ==========

% Style de la boîte de propriété(s) / théorème(s)
\mdfdefinestyle{ProprieteStyle}{
    leftmargin=0cm,
    rightmargin=0cm,
    linecolor=black,
    linewidth=1pt
}

% Création d'un compteur pour les propriétés
\newcounter{propriete}

% Définition de la commande '\propriete'
\newcommand{\propriete}[2]{%
    \refstepcounter{propriete}% Incrémente le compteur de théorème
    \begin{mdframed}[style=ProprieteStyle]
        \ifstrempty{#1}{% Teste si le premier argument est vide
            \textbf{\thepropriete. Propriété :}\ % Si vide, n'affiche pas de titre de théorème
        }{%
            \textbf{\thepropriete. Propriété #1 :}\ % Si non vide, affiche le titre avec le premier argument
        }\\#2
    \end{mdframed}
}

% Définition de la commande 'proprietes'
\newcommand{\proprietes}[2]{%
    \refstepcounter{propriete}% Incrémente le compteur de théorème
    \begin{mdframed}[style=ProprieteStyle]
        \ifstrempty{#1}{% Teste si le premier argument est vide
            \textbf{\thepropriete. Propriétés :}\ % Si vide, n'affiche pas de titre de théorème
        }{%
            \textbf{\thepropriete. Propriétés #1 :}\ % Si non vide, affiche le titre avec le premier argument
        }\\#2
    \end{mdframed}
}

% Création d'un compteur pour les théorèmes
\newcounter{theoreme}

% Définition de la commande '\theoreme'
\newcommand{\theoreme}[2]{%
    \refstepcounter{theoreme}% Incrémente le compteur de théorème
    \begin{mdframed}[style=ProprieteStyle]
        \ifstrempty{#1}{% Teste si le premier argument est vide
            \textbf{\thetheoreme. Théorème :}\ % Si vide, n'affiche pas de titre de théorème
        }{%
            \textbf{\thetheoreme. Théorème #1 :}\ % Si non vide, affiche le titre avec le premier argument
        }\\#2
    \end{mdframed}
}

% Définition de la commande '\theoremes'
\newcommand{\theoremes}[2]{%
    \refstepcounter{theoreme}% Incrémente le compteur de théorème
    \begin{mdframed}[style=ProprieteStyle]
        \ifstrempty{#1}{% Teste si le premier argument est vide
            \textbf{\thetheoreme. Théorèmes :}\ % Si vide, n'affiche pas de titre de théorème
        }{%
            \textbf{\thetheoreme. Théorèmes #1 :}\ % Si non vide, affiche le titre avec le premier argument
        }\\#2
    \end{mdframed}
}

\newcommand{\corrolaire}[2]{%
    \begin{mdframed}[style=ProprieteStyle]
        \ifstrempty{#1}{% Teste si le premier argument est vide
            \textbf{Corrolaire :}\ % Si vide, n'affiche pas de titre de théorème
        }{%
            \textbf{Corrolaire #1 :}\ % Si non vide, affiche le titre avec le premier argument
        }\\#2
    \end{mdframed}
}

% ========== Exemple(s) ==========

% Définition de la commande 'exemple'
\newcommand{\exemple}[2]{
    \textbf{Exemple :} #1
    \begin{quote}
        #2
    \end{quote}
}

% Définition de la commande 'exemples'
\newcommand{\exemples}[2]{
    \textbf{Exemples :} #1
    \begin{quote}
        #2
    \end{quote}
}

% ========== Remarques ==========

% Définition de la commande 'remarque'
\newcommand{\remarque}[1]{
    \textbf{Remarque :} #1
}

% Définition de la commande 'remarques'
\newcommand{\remarques}[1]{
    \textbf{Remarques :}
    \begin{quote}
        #1
    \end{quote}
}

% ========== Démonstration ==========

\newcommand{\demonstration}[2]{
    \ifstrempty{#1}{% Teste si le premier argument est vide
        \textbf{Démonstration :}\ % Si vide, n'affiche pas de titre de théorème
    }{%
        \textbf{Démonstration #1 :}\ % Si non vide, affiche le titre avec le premier argument
    }
    \begin{quote}
        #2
    \end{quote}
}
% ========== Autres blocs ==========

\newcommand{\newbloc}[2]{
    \textbf{#1}
    \begin{quote}
        #2
    \end{quote}
}

% ========== Notations ==========

% Style
\newcommand{\ds}{\displaystyle}
\newcommand{\llb}{\llbracket}
\newcommand{\rrb}{\rrbracket}

% Ensembles
\newcommand{\C}{\mathbb{C}}
\newcommand{\R}{\mathbb{R}}
\newcommand{\Q}{\mathbb{Q}}
\newcommand{\D}{\mathbb{D}}
\newcommand{\Z}{\mathbb{Z}}
\newcommand{\N}{\mathbb{N}}

% Limites
\let\oldlim\lim
\renewcommand{\lim}[1]{\mathop{\displaystyle\oldlim}\limits_{#1}}

% Opérateurs
\newcommand{\x}{\times}
\newcommand{\equival}{\Leftrightarrow}
\newcommand{\involve}{\Rightarrow}

% Ensembles
\renewcommand{\bar}[1]{\overline{#1}}
\renewcommand{\part}[1]{\mathcal{P}({#1})}  % Ensemle des parties de E
\newcommand{\rel}{\mathcal{R}} % Relation entre deux ensembles

% Vecteurs
\renewcommand{\Vec}[1]{\overrightarrow{#1}}
\newcommand*{\norme}[1]{\|#1\|}
\newcommand{\pdt}{\mathbin{\vcenter{\hbox{\scalebox{0.6}{\textbullet}}}}}
\newcommand{\vcoord}[2]{\begin{pmatrix} #1 \\ #2 \end{pmatrix}}
\newcommand{\Vcoord}[3]{\begin{pmatrix} #1 \\ #2 \\ #3 \end{pmatrix}}

% Fonctions
\newcommand{\aire}[1]{\text{aire}\left( #1 \right)}
\newcommand{\Int}{\ds\int}


% ========== Systèmes ==========
\newcommand{\sys}[2]{\begin{cases}#1\\#2\end{cases}}
\newcommand{\Sys}[3]{\begin{cases}#1\\#2\\#3\end{cases}}

% ========== Tableaux ==========
\newcommand{\boldhline}{\hline\noalign{\vskip 0pt}\hline}

% ========== Graphiques ==========

% Triange de Pascal
\newcommand{\binomial}[2]{\pgfmathparse{int((#1)!/((#2)!*((#1)-(#2))!))}\pgfmathresult}
\newcommand{\Pascal}[1]{
    \begin{tikzpicture}[scale=1, every node/.style={scale=1.2}]
        \foreach \n in {0,...,#1} {
                \foreach \k in {0,...,\n} {
                        \node at (\k-\n/2,-\n) {$\binomial{\n}{\k}$};
                        \draw (\k-\n/2,-\n) +(-0.5,-0.5) rectangle ++(0.5,0.5);
                    }
            }
    \end{tikzpicture}
}

\newcommand{\Pascalbinome}[1]{
    \begin{tikzpicture}[scale=1, every node/.style={scale=0.9}]
        \foreach \n in {0,...,#1} {
                \foreach \k in {0,...,\n} {
                        \pgfmathtruncatemacro{\symk}{min(\k,\n-\k)}
                        \node at (\k-\n/2,-\n) {$\ds\binom{\n}{\symk}$};
                        \draw (\k-\n/2,-\n) +(-0.5,-0.5) rectangle ++(0.5,0.5);
                    }
            }
    \end{tikzpicture}
}

\begin{document}

\maketitle

\section{Quelques définitions}

\definition{(relation)}{
    Soient $E$ et $F$ des ensembles non vides. Soit $\rel$ une partie de $E\times F$. On considère tous les couples
    $(a,b)$ de $E\times F$ tels que $(a,b)\in\rel$. On note $a\rel b\equival (a,b)\in\rel$ et on appelle $\rel$
    correspondance de $E$ vers $F$.
}

\definition{(relation binaire)}{
    On appelle relation binaire toute correspondance de $E$ sur lui-même.
}

\exemple{}{
    Dans $\part{E}$ : $(A,B)\in\part{E}, A\rel B\equival A\subset B$
}

\section{Relation d'équivalence}

\definition{(réflexivité)}{
    Soit $E$ un ensemble et $\rel$ une relation binaire. La relation $\rel$ est dit réflexive si :
    $$\forall x\in E, x\rel x$$
}

\exemples{}{
    \begin{itemize}
       \item Dans $\part{E}$ : $\forall A\in\part{E}, A\subset A \rightarrow$ Relation réflexive
       \item Soit $(x,y)\in\R^2, x\rel y\equival x<y \rightarrow$ Relation non réflexive
    \end{itemize}
}

\remarque{Soit $\rel$ une relation réflexive sur $E$ ($\rel\subset E\times E$). Alors $\rel$ est réflexive
$\equival \forall x\in E, (x,y)\in\rel$.}

\definition{(symétrie)}{
    La relation $\rel$ est dite symétrique si : $$\forall (x,y)\in E^2, x\rel y \Rightarrow y\rel x$$
}

\exemples{}{ 
    \begin{itemize}
       \item Soit $E$ l'ensemble des droites du plan. Alors $(d, d')\in E^2, d \rel d' \equival d \perp d'$ est une
       relation symétrique.
       \item L'inclusion n'est pas symétrique.
    \end{itemize}
}

\definition{(transitivité)}{
    Soient $(x,y,z)\in E^3$. La relation $\rel$ est transitive si :
    $$\forall(x,y,z)\in E^3, (x \rel y)\land(y \rel z)\Rightarrow x \rel z$$
}

\exemple{}{
    \begin{itemize}
       \item L'inclusion est transitive.
       \item $d \sslash d'$ est transitif.
       \item $d\perp d'$ n'est pas transitif.
    \end{itemize}
}

\definition{(équivalence)}{
    Soit $E$ un ensemble non vide et $\rel$ une relation binaire. On dit que $\rel$ est une relation d'équivalence si :
    \begin{itemize}
       \item $\rel$ est réflexive,
       \item et $\rel$ est transitive.
       \item $\rel$ est symétrique,
    \end{itemize}
}

\definition{(antisymétrique)}{
    Soit $E$ un ensemble non vide et $\rel$ une relation binaire. La relation $\rel$ est dite antisymétrique si :
    $$\forall(x,y)\in E^2, (x \rel y)\land(y \rel x)\Rightarrow x=y$$
}

\exemple{}{
    L'inclusion est une relation antisymétrique (si $A \subset B$ et $B \subset A$, alors $A=B$).
}

\section{Relation d'ordre}

\definition{(relation d'ordre)}{
    Soit $E$ un ensemble non vide et $\rel$ une relation binaire. On dit que $\rel$ est une relation d'ordre si :
    \begin{itemize}
        \item $\rel$ est réflexive,
        \item $\rel$ est transitive,
        \item et $\rel$ est antisymétrique.
    \end{itemize}

}


\definition{(ordre total, ordre partiel)}{
    On dit que la relation d'ordre $\rel$ est une relation d'ordre total si :
    $$\forall(x,y)\in E^2, (x\rel y)\lor(y \rel x)$$
    Sinon, on dit que l'ordre est partiel.
}

\definition{(diagramme de Hasse)}{
    Soit $\rel$ une relation d'ordre sur un ensemble fini $E$. Le diagramme de Hasse de $\rel$ est le graphe dont les
    sommets sont les éléments de $E$ et dont les arcs (ou arêtes) satisfont la propriété suivante. \\
    Il existe un arc de $x$ à $y$ si et seulement si :
    \begin{itemize}
       \item $x\leq y$
       \item et $\exists z\in E, x\leq y\leq z \Rightarrow x=z \lor y=y$.
    \end{itemize}
}

\exemple{}{
    \begin{tikzpicture}
        \node (empty) at (0,0) {$\emptyset$};
        \node [below left of=empty] (one) {$\{1\}$};
        \node [below of=empty] (two) {$\{2\}$};
        \node [below right of=empty] (three) {$\{3\}$};
        \node [below left of=one] (onetwo) {$\{1,2\}$};
        \node [below of=two] (onethree) {$\{1,3\}$};
        \node [below right of=three] (twothree) {$\{2,3\}$};
        \node [below of=onethree] (onetwothree) {$\{1,2,3\}$};
    
        \draw [black, thick, shorten <=-3pt, shorten >=-3pt, -{latex}] (empty) -> (one);
        \draw [black, thick, shorten <=-3pt, shorten >=-3pt, -{latex}] (empty) -> (two);
        \draw [black, thick, shorten <=-3pt, shorten >=-3pt, -{latex}] (empty) -> (three);

        \draw [black, thick, shorten <=-3pt, shorten >=-3pt, -{latex}] (one) -> (onetwo);
        \draw [black, thick, shorten <=-3pt, shorten >=-3pt, -{latex}] (one) -> (onethree);

        \draw [black, thick, shorten <=-3pt, shorten >=-3pt, -{latex}] (two) -> (onetwo);
        \draw [black, thick, shorten <=-3pt, shorten >=-3pt, -{latex}] (two) -> (twothree);

        \draw [black, thick, shorten <=-3pt, shorten >=-3pt, -{latex}] (three) -> (onethree);
        \draw [black, thick, shorten <=-3pt, shorten >=-3pt, -{latex}] (three) -> (twothree);

        \draw [black, thick, shorten <=-3pt, shorten >=-3pt, -{latex}] (onetwo) -> (onetwothree);
        \draw [black, thick, shorten <=-3pt, shorten >=-3pt, -{latex}] (onethree) -> (onetwothree);
        \draw [black, thick, shorten <=-3pt, shorten >=-3pt, -{latex}] (twothree) -> (onetwothree);
    \end{tikzpicture}
}

\end{document}