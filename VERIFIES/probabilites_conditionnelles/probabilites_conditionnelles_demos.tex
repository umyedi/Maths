\documentclass[11pt,a4paper]{article}
\usepackage[utf8]{inputenc}
\usepackage[T1]{fontenc}
\usepackage{amsfonts}
\usepackage{amssymb}
\usepackage{mdframed}
\usepackage{tikz}
\usepackage{tkz-tab}
\usepackage{pgfplots}
\usepackage{xcolor}
\usepackage{fancyhdr}
\usepackage{lastpage}
\usepackage[fleqn]{amsmath}
\setlength{\mathindent}{0pt}

% Spécifications du document
% Spécifications du document
\newcommand{\doctitre}{Probabilités conditionnelles et indépendance} % Ex: Le second degré
\newcommand{\docniveau}{$1^{\text{re}}$ Spécialité mathématiques} % Ex: $1^{\text{re}}$ Spécialité mathématiques
\newcommand{\doctheme}{Probabilités et Statistiques} %Ex: Algèbre
\newcommand{\doctype}{Démonstrations} % Ex: Démonstrations
\newcommand{\docshorttype}{Démo} % Démo

% Couleurs pour les graphiques
\definecolor{dark_green}{HTML}{008000}

% Paramètres du document
\RequirePackage{geometry}
\geometry{tmargin=1cm,bmargin=1.9cm,lmargin=1.9cm,rmargin=1.9cm}
\renewcommand{\familydefault}{\sfdefault}
\setlength{\parindent}{0pt}
\title{\doctitre}
\author{\docniveau \\ \doctheme\text{ - }\doctype}
\date{}
\fancypagestyle{custom}{
  \fancyhf{}
  \renewcommand{\headrulewidth}{0pt}
  \lfoot{Probas et Stats - \docshorttype}
  \cfoot{\doctitre} % Change \titre to \doctitre
  \rfoot{\thepage/\pageref{LastPage}}
}

% Styles pour les mdframed
\mdfdefinestyle{definitionStyle}{
    leftline=true,
    rightline=false,
    topline=false,
    bottomline=false,
    linewidth=2pt,
    linecolor=black,
    innertopmargin=0pt,
    innerbottommargin=0pt,
    innerrightmargin=0pt,
    innerleftmargin=5pt,
}

\mdfdefinestyle{proprieteStyle}{
    linewidth=1pt,
    linecolor=black,
    innertopmargin=5pt,
    innerbottommargin=5pt,
    innerrightmargin=5pt,
    innerleftmargin=5pt,
}
% ----- DEBUT DU DOCUMENT -----
\begin{document}

% Style et numérotation
\maketitle
\pagestyle{custom}
\thispagestyle{custom}

\section*{I. Notion de probabilité conditionnelle}

\subsection*{1. Définition}

\underline{Démonstration :}

On a $(A\cap B)\subset A$.\\
Donc $p(A\cap B) \leq p(A)$. \\
Donc $\frac{p(A\cap B)}{p(A)}\leq 1 \Leftrightarrow p_B(A) \leq 1$ \\
De plus, $p_B(A) \geq 1$ comme quotient de deux positifs.
\begin{alignat*}{2}
   & \text{Calculons } & p_A(B)+p_A(\bar B) & =\frac{p(A\cap B)}{p(A)}+\frac{p(A\cap \bar B)}{p(A)} \\
   &                   &                    & = \frac{p(A\cap B)+p(A\cap \bar B)}{p(A)}
\end{alignat*}
Or $(A\cap B) \cup (A\cap \bar B) = A$. \\
Donc $p(A\cap B)+p(A\cap \bar B)=p(A)$. \\
Donc $p_A(B)+p_A(\bar B)=\frac{p(A)}{p(A)}=1$

\section*{III. Indépendance de deux évènements}

\underline{Démonstration :}

\begin{alignat*}{2}
   & \text{$A$ et $B$ sont indépendants } & \Leftrightarrow & \text{ } p(A\cap B)=p(A)\times p(B)                                       \\
   &                                      & \Leftrightarrow & \text{ } p_A(B)=\frac{p(A\cap B)}{p(A)}=\frac{p(A)\times p(B)}{p(A)}=p(B) \\
   &                                      & \Leftrightarrow & \text{ } p_A(B)=p(B)                                                      \\
\end{alignat*}

\underline{Démonstration :}

\begin{alignat*}{2}
   & \text{Calculons } & p(A\cap \bar B ) & = p(A)-p(A\cap B)                                             \\
   &                   &                  & p(A)-p(A)\times p(B) \text{ car $A$ et $B$ sont indépendants} \\
   &                   &                  & p(A)(1-p(B))                                                  \\
   &                   &                  & p(A)p(\bar B)
\end{alignat*}

Ainsi, les évènements $A$ et $\bar B$ sont indépendants.

\end{document}
% ----- FIN DU DOCUMENT -----