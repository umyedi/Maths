\documentclass[11pt,a4paper]{article}
\usepackage[utf8]{inputenc}
\usepackage[T1]{fontenc}
\usepackage{amsfonts}
\usepackage{amssymb}
\usepackage{mdframed}
\usepackage{tikz}
\usepackage{tkz-tab}
\usepackage{pgfplots}
\usepackage{xcolor}
\usepackage{fancyhdr}
\usepackage{lastpage}
\usepackage[fleqn]{amsmath}
\setlength{\mathindent}{0pt}

% Spécifications du document
\newcommand{\doctitre}{Taux d'évolution} % Ex: Le second degré
\newcommand{\docniveau}{$2^{\text{nd}}$ Générale} % Ex: $1^{\text{re}}$ Spécialité mathématiques
\newcommand{\doctheme}{Probabilités et statistiques} %Ex: Algèbre
\newcommand{\doctype}{Cours} % Ex: Démonstrations
\newcommand{\docshorttype}{Cours} % Démo

% Couleurs pour les graphiques
\definecolor{dark_green}{HTML}{008000}

% Paramètres du document
\RequirePackage{geometry}
\geometry{tmargin=1cm,bmargin=1.9cm,lmargin=1.9cm,rmargin=1.9cm}
\renewcommand{\familydefault}{\sfdefault}
\setlength{\parindent}{0pt}
\title{\doctitre}
\author{\docniveau \\ \doctheme\text{ - }\doctype}
\date{}
\fancypagestyle{custom}{
  \fancyhf{}
  \renewcommand{\headrulewidth}{0pt}
  \lfoot{Probas et Stats\text{ - }\docshorttype}
  \cfoot{\doctitre} % Change \titre to \doctitre
  \rfoot{\thepage/\pageref{LastPage}}
}

% Styles pour les mdframed
\mdfdefinestyle{definitionStyle}{
    leftline=true,
    rightline=false,
    topline=false,
    bottomline=false,
    linewidth=2pt,
    linecolor=black,
    innertopmargin=0pt,
    innerbottommargin=0pt,
    innerrightmargin=0pt,
    innerleftmargin=5pt,
}

\mdfdefinestyle{proprieteStyle}{
    linewidth=1pt,
    linecolor=black,
    innertopmargin=5pt,
    innerbottommargin=5pt,
    innerrightmargin=5pt,
    innerleftmargin=5pt,
}
% ----- DEBUT DU DOCUMENT -----
\begin{document}

% Style et numérotation
\maketitle
\pagestyle{custom}
\thispagestyle{custom}

\section*{I. Taux d'évolution et coefficient multiplicateur}

\begin{mdframed}[style=definitionStyle]
    \textbf{Définitions :}
    \vspace{-4pt}
    \begin{itemize}
        \item La variation absolue d'une évolution est la différence entre la valeur d'arrivée et la valeur de départ : $\Delta V=V_A-V_D$.
        \item Le taux d'évolution ou variation relative est le quotient entre la variation absolue par la valeur de départ : $t=\frac{V_A-V_D}{V_D}$.
        \item On appelle coefficient multiplicateur de l'évolution le nombre $CM=\frac{V_A}{V_D}$.
    \end{itemize}
\end{mdframed}

\begin{mdframed}[style=proprieteStyle]
    \textbf{Propriété :} ~\\
    Pour toute évolution de taux $t$ et de coefficient multiplicateur $CM$, on a $CM=1+t$.
\end{mdframed}

\underline{Démonstration :} ~\\

$\displaystyle\frac{V_A-V_D}{V_D}=\frac{V_A}{V_D}-\frac{V_D}{V_D}=\frac{V_A}{V_D}-1=CM-1$ d'où $CM=1+t$. \\

\textbf{Remarques :} 
\vspace*{-4pt}
\begin{itemize}
    \item Lors d'une augmentation, le taux $t$ est positif et le coefficient multiplicateur est supérieur à $1$.
    \item Lors d'une diminution, le taux $t$ est négatif et le coefficient multiplicateur est inférieur à $1$.
\end{itemize}

\section*{II. Évolution successives}

\subsection*{1. Cas général}

\begin{mdframed}[style=proprieteStyle]
    \textbf{Propriété :} ~\\
    Lorsqu'une quantité subit $n$ évolutions, successives de taux $t_1, t_2,\dots,t_n$, alors le coefficient multiplicateur correspondant à l'évolution global est égal au produit des coefficients multiplicateurs $CM_1,CM_2,\dots,CM_n$ associés respectivements aux évolutions de taux $t_1, t_2,\dots,t_n$.
\end{mdframed}

\subsection*{2. Évolution réciproques}

\begin{mdframed}[style=definitionStyle]
    \textbf{Définition :} ~\\
    Une quantité non nulle $V_D$ subit une évolution de taux $t$ pour devenir égal à $V_A$.
    Le taux réciproque de $t$ est le taux permettant de passer de la valeur $V_A$ à la valeur $V_D$.
\end{mdframed}

\begin{mdframed}[style=proprieteStyle]
    \textbf{Propriété :} ~\\
    Pour que deux évolutions soient réciproques, il faut que leurs coefficients multiplicateurs soient inverses l'un de l'autre.
    Le taux
\end{mdframed}



\end{document}
% ----- FIN DU DOCUMENT -----