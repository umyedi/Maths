\documentclass[11pt,a4paper]{article}
\usepackage[utf8]{inputenc}
\usepackage[T1]{fontenc}
\usepackage{amsfonts}
\usepackage{amssymb}
\usepackage{mdframed}
\usepackage{tikz}
\usepackage{tkz-tab}
\usepackage{pgfplots}
\usepackage{xcolor}
\usepackage{fancyhdr}
\usepackage{lastpage}
\usepackage[fleqn]{amsmath}
\usepackage{tabularx}
\setlength{\mathindent}{0pt}

% Définition d'un type de colonne centré
\newcolumntype{Y}{>{\centering\arraybackslash}X}

% Spécifications du document
\newcommand{\doctitre}{Variables aléatoires réelles} % Ex: Le second degré
\newcommand{\docniveau}{$1^{\text{re}}$ Spécialité mathématiques} % Ex: $1^{\text{re}}$ Spécialité mathématiques
\newcommand{\doctheme}{Probabilités et Statistiques} %Ex: Algèbre
\newcommand{\doctype}{Cours} % Ex: Démonstrations
\newcommand{\docshorttype}{Cours} % Démo

% Couleurs pour les graphiques
\definecolor{dark_green}{HTML}{008000}

% Paramètres du document
\RequirePackage{geometry}
\geometry{tmargin=1cm,bmargin=1.9cm,lmargin=1.9cm,rmargin=1.9cm}
\renewcommand{\familydefault}{\sfdefault}
\setlength{\parindent}{0pt}
\title{\doctitre}
\author{\docniveau \\ \doctheme\text{ - }\doctype}
\date{}
\fancypagestyle{custom}{
  \fancyhf{}
  \renewcommand{\headrulewidth}{0pt}
  \lfoot{\text{Probas et Stats - }\docshorttype}
  \cfoot{\doctitre} % Change \titre to \doctitre
  \rfoot{\thepage/\pageref{LastPage}}
}

% Styles pour les mdframed
\mdfdefinestyle{definitionStyle}{
    leftline=true,
    rightline=false,
    topline=false,
    bottomline=false,
    linewidth=2pt,
    linecolor=black,
    innertopmargin=0pt,
    innerbottommargin=0pt,
    innerrightmargin=0pt,
    innerleftmargin=5pt,
}

\mdfdefinestyle{proprieteStyle}{
    linewidth=1pt,
    linecolor=black,
    innertopmargin=5pt,
    innerbottommargin=5pt,
    innerrightmargin=5pt,
    innerleftmargin=5pt,
}
% ----- DEBUT DU DOCUMENT -----
\begin{document}

% Style et numérotation
\maketitle
\pagestyle{custom}
\thispagestyle{custom}

\section*{I. Variable aléatoire et loi de probabilité}

\subsection*{1. Définition d'une variable aléatoire discrète}

\begin{mdframed}[style=definitionStyle]
    \textbf{Définition :} ~\\
    Soit $\Omega=\{\omega_1;\omega_2;\dots;\omega_k\}$ l'univers associé à une expérience aléatoire. \\
    Une variable aléatoire $X$ sur l'univers $\Omega$ est une fonction définie sur $\Omega$ et à valeurs dans $\mathbb{R}$. \\
    Définir une variable aléatoire consiste à associer, à chaque issue $\omega_i$ de l'expérience aléatoire, un réel $x_i$. \\
    On note alors $\left(X=x_i\right)$ l'évènement formé des issues qui ont pour image $x_i$ par $X$.
\end{mdframed}

\textbf{Exemple :} ~\\
On lance un dé équilibré à 6 faces. \\
On gagne 2€ si le « 2 » sort, 1€ si le « 1 » sort et on perd 1€ dans tous les autres cas. \\
On appelle $X$ la variable aléatoire qui à chaque issue associe le gain obtenu. \\

On a $\Omega=\{1;2;3;4;5;6\}$ et $\Omega(X)=\{2;1;-1\}$. \\
On précise les évènements de $X$ :
\vspace{-4pt}
\begin{itemize}
    \item $(X=2)=\{2\}$
    \item $(X=1)=\{1\}$
    \item $(X=-1)=\{3;4;5;6\}$
\end{itemize}

\subsection*{Loi de probabilité d'une variable aléatoire discrète}

\begin{mdframed}[style=definitionStyle]
    \textbf{Définition :} ~\\
    Soit $X$ une variable aléatoire qui prend les valeurs $\{x_1;x_2;\dots;x_k\}$. \\
    Donner la loi de probabilité de $X$, c'est donner la valeur $p(X=x_i)$, pour tout $i$ avec $1\leq i\leq k$.
    \vspace*{4pt}

    Les résultats sont généralement présentés sous forme d'un tableau : \\
    \vspace*{-8pt}

    \renewcommand{\arraystretch}{1.6}
    \begin{tabular}{|l|c|c|c|c|}
        \hline
        Valeurs $x_i$ de $X$ & $x_1$      & $x_2$      & \quad\text{ }\dots\quad\text{ } & $x_k$      \\
        \hline
        Probabilité $p(X=x_i)$     & $p(X=x_1)$ & $p(X=x_2)$ & \quad\text{ }\dots\quad\text{ } & $p(X=x_k)$ \\
        \hline
    \end{tabular}
\end{mdframed}

\textbf{Remarque :} La somme des probabilités $p(X=x_i)$, pour $i$ allant de $1$ à $k$, est égal à $1$. \\

\textbf{Exemple \emph{(suite)} :} ~\\

On donne la loi de probabilité de $X$ : $\displaystyle{\color{white}\frac{1}{1}}$\\
\renewcommand{\arraystretch}{2}
\begin{tabular}{|l|c|c|c|c|}
    \hline
    Valeurs $x_i$ de $X$      & $2$                        & $1$                        & $-1$                                   \\
    \hline
    Probabilité  $p(X=x_i)\displaystyle{\color{white}\frac{1}{\frac{1}{1}}}$ & $\displaystyle\frac{1}{6}$ & $\displaystyle\frac{1}{6}$ & $\displaystyle\frac{4}{6}=\frac{2}{3}$ \\
    \hline
\end{tabular}

\newpage

\section*{II. Espérance, variance et écart-type d'une variable aléatoire}

Soit $X$ une variable aléatoire qui prend en valeurs ${x_1;x_2;\dots;x_k}$ et dont la loi de probabilité est donnée par le tableau suivant :
\vspace{-12pt}

\begin{center}
    \renewcommand{\arraystretch}{1.6}
    \begin{tabular}{|l|c|c|c|c|}
        \hline
        Valeurs $x_i$ de $X$ & $x_1$      & $x_2$      & \quad\text{ }\dots\quad\text{ } & $x_k$      \\
        \hline
        Probabilité $p(X=x_i)$      & $p_1$ & $p_2$ & \quad\text{ }\dots\quad\text{ } & $p_k$ \\
        \hline
    \end{tabular}
\end{center}
\vspace*{8pt}

\begin{mdframed}[style=definitionStyle]
    \textbf{Définition de l'espérance :} ~\\
    L'espérance de la variable aléatoire $X$ est le réel noté $E(X)$ définie par :
    $\displaystyle E(X)=\sum_{i=1}^{k}x_ip_i$          
\end{mdframed}

\textbf{Remarques :} L'espérance d'une variable aléatoire représente la valeur moyenne prise par $X$. \\

Lorsque les valeurs prises par $X$ représentent les gains ou les pertes à un jeu, alors $E(X)$ représente le gain moyen par partie :
\vspace*{-4pt}
\begin{itemize}
    \item Si $E(X)>0$ alors le jeu est favorable au joueur.
    \item Si $E(X)<0$ alors le jeu est défavorable au joueur.
    \item Si $E(X)=0$ alors le jeu est équitable.
\end{itemize}


\begin{mdframed}[style=definitionStyle]
    \textbf{Définition de la variance :} ~\\
    La variance de la variable aléatoire $X$ est le réel noté $V(X)$ définie par :
    $\displaystyle V(X)=\sum_{i=1}^{k}\left[x_i-E(X)\right]^2\times p_i$
\end{mdframed}

\textbf{Remarque :} La variance d'une variable aléatoire $X$ se calcule aussi avec la formule $\displaystyle V(X)=\sum_{i=1}^{k} p_i x_i^2-E(X)^2$

\begin{mdframed}[style=definitionStyle]
    \textbf{Définition de l'écart-type :} ~\\
    L'écart-type $\sigma(X)$ est défini comme la racine carrée de la variance : $\sigma(X)=\sqrt{V(X)}$.
\end{mdframed}

\textbf{Remarque :} Par analogie avec les statistiques, de la même façon que $E(X)$ représente une moyenne, $V(X)$ et $\sigma(X)$ sont des indicateurs de dispersion des valeurs de $X$ autour de $E(X)$. \\
Plus la variance et l'écart-type sont grands, plus les valeurs sont dispersés autour de la moyenne (espérance). \\

\textbf{Exemple \emph{(suite)}:} ~\\
On calcule l'espérance de $X$ :
$\displaystyle E(X)=2\times\frac{1}{6}+1\times\frac{1}{6}-1\times\frac{2}{3}=-\frac{1}{6}\simeq-0,17$. \\
Sur un grand nombre de parties, le gain moyen par partie pour le joueur est $-0,17$. \\Donc le jeu n'est pas favorable au joueur. \\

On calcule la variance et l'écart-type :
\begin{itemize}
    \item $\displaystyle V(X)=\left[2-\left(-\frac{1}{6}\right)\right]^2\times \frac{1}{6}+\left[1-\left(-\frac{1}{6}\right)\right]^2\times \frac{1}{6}+\left[-1-\left(-\frac{1}{6}\right)\right]^2\times \frac{2}{3}=\frac{53}{36}\simeq1,47$
    \item $\displaystyle \sigma(X)=\sqrt{\frac{53}{36}}\simeq1,21$
\end{itemize}
\end{document}
% ----- FIN DU DOCUMENT -----