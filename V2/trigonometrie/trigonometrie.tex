\documentclass[10pt]{article}

\newcommand{\DocumentTitle}{Trigonométrie}
\newcommand{\DocumentTheme}{Analyse}
\newcommand{\DocumentType}{Cours}
% Preamble Section
\usepackage[a4paper, margin=1in]{geometry} % Sets the paper size to A4 and all margins to 1 inch
\usepackage[utf8]{inputenc}   % Allows for input of international characters
\usepackage[T1]{fontenc}      % Utilisation de l'encodage T1
\usepackage{lmodern}          % Support des polices avec le module 'french'
\usepackage[french]{babel}    % Ajout du support pour le français
\usepackage[fleqn]{amsmath}   % Imports the AMS math package for advanced math formatting ([fleqn] -> align alignat to the left)
\usepackage{amsfonts}         % Imports AMS fonts for math fonts
\usepackage{amssymb}          % Imports AMS symbols for math symbols
\usepackage{amsthm}           % Imports AMS theorems
\usepackage{yhmath}           % Used for "\wideparen{}" 
\usepackage{fancyhdr}         % Import to custom the page footer
\usepackage{mdframed}         % Imports styles
\usepackage{enumerate}        % Imports styles for enumerate
\usepackage{multicol}
\usepackage{xcolor}           % Couleurs
\usepackage{tikz}             % Package général pour les graphiques
\usetikzlibrary{calc}
\usepackage{pgfplots}         % Compléments pour les graphiques
\usepackage{tkz-tab}          % Tableaux de variations
\usepackage{lastpage}         % Pour avoir le total de page dans le footer
\usepackage{makecell}         % Retour à la ligne dans une case d'un tabular
\usepackage{stmaryrd}         % Intervalles entières : \llbracket et \rrbracket
\usepackage{cancel}           % Permet de barrer des termes
\usepackage{forest}           % Permet de créer des arbres pondérés
\usepackage{tabularx}         % Permet de générer des tableaux (ex: cours fonction dérivées et application)
\usepackage{multirow}         % Permet de faire plusieurs lignes dans des cases de talbeaux


\pgfplotsset{compat=newest}   % Active les dernières fonctionnalités de pgfplots

% Enlève l'indentation des bloc de paragraphes et d'équation
\setlength{\parindent}{0pt}
\setlength{\mathindent}{0pt}

% ========== Title Section ==========

\title{\DocumentTitle} % Title of the document
\author{\DocumentTheme\space-\space\DocumentType} % Author's name
\date{} % Date

\pagestyle{fancy}
\fancyhf{} % Clear all header and footer fields

% ========== Footer Section ==========

\newcommand{\customfooter}{
    \fancyfoot[L]{\DocumentTheme\space-\space\DocumentType}
    \fancyfoot[C]{\DocumentTitle}
    \fancyfoot[R]{\thepage/\pageref{LastPage}}
    % Remove the line below the header and above the footer
    \renewcommand{\headrulewidth}{0pt}
    \renewcommand{\footrulewidth}{0pt}
}

% Apply the custom footer to the fancy style
\customfooter

% Apply the custom footer to the plain style (used on the first page)
\fancypagestyle{plain}{\customfooter}

% ========== Sections ==========

% Sections : I. / II. / III. ...
\renewcommand{\thesection}{\Roman{section}}
% Subsection : I.1 / I.2 / I.3 ...
\renewcommand{\thesubsection}{\thesection. \arabic{subsection}}
% Subsubsection : I.1.a / I.1.b / I.1.c ...
\renewcommand{\thesubsubsection}{\thesubsection. \alph{subsubsection}}

% ========== Colors (from Geogebra) ==========

\definecolor{green}{HTML}{006400}
\definecolor{red}{HTML}{CC0000}
\definecolor{blue}{HTML}{0000FF}
\definecolor{orange}{HTML}{FF5500}
\definecolor{purple}{HTML}{9933FF}
\definecolor{gray}{HTML}{666666}
\definecolor{brown}{HTML}{993300}
\definecolor{black}{HTML}{000000}
\definecolor{white}{HTML}{FFFFFF}

\newcommand{\green}[1]{\color{green}{#1}\color{black}}
\newcommand{\red}[1]{\color{red}{#1}\color{black}}
\newcommand{\blue}[1]{\color{blue}{#1}\color{black}}
\newcommand{\orange}[1]{\color{orange}{#1}\color{black}}
\newcommand{\purple}[1]{\color{purple}{#1}\color{black}}
\newcommand{\gray}[1]{\color{gray}{#1}\color{black}}
\newcommand{\brown}[1]{\color{brown}{#1}\color{black}}


% ========== Définition(s) ==========

% Style de la boîte de définition(s)
\mdfdefinestyle{DefinitionStyle}{
    leftmargin=0cm,
    rightmargin=0cm,
    linecolor=black,
    linewidth=2pt,
    topline=false,
    bottomline=false,
    rightline=false
}

% Définition de la commande '\definition'
\newcommand{\definition}[2]{%
    \begin{mdframed}[style=DefinitionStyle]
        \ifstrempty{#1}{% Teste si le premier argument est vide
            \textbf{Définition :}\ % Si vide, n'affiche pas de titre de théorème
        }{%
            \textbf{Définition #1 :}\ % Si non vide, affiche le titre avec le premier argument
        }\\#2
    \end{mdframed}
}

% Définition de la commande '\definitions'
\newcommand{\definitions}[2]{%
    \begin{mdframed}[style=DefinitionStyle]
        \ifstrempty{#1}{% Teste si le premier argument est vide
            \textbf{Définitions :}\ % Si vide, n'affiche pas de titre de théorème
        }{%
            \textbf{Définitions #1 :}\ % Si non vide, affiche le titre avec le premier argument
        }\\#2
    \end{mdframed}
}

% ========== Propriété(s) et Théorème(s) ==========

% Style de la boîte de propriété(s) / théorème(s)
\mdfdefinestyle{ProprieteStyle}{
    leftmargin=0cm,
    rightmargin=0cm,
    linecolor=black,
    linewidth=1pt
}

% Création d'un compteur pour les propriétés
\newcounter{propriete}

% Définition de la commande '\propriete'
\newcommand{\propriete}[2]{%
    \refstepcounter{propriete}% Incrémente le compteur de théorème
    \begin{mdframed}[style=ProprieteStyle]
        \ifstrempty{#1}{% Teste si le premier argument est vide
            \textbf{\thepropriete. Propriété :}\ % Si vide, n'affiche pas de titre de théorème
        }{%
            \textbf{\thepropriete. Propriété #1 :}\ % Si non vide, affiche le titre avec le premier argument
        }\\#2
    \end{mdframed}
}

% Définition de la commande 'proprietes'
\newcommand{\proprietes}[2]{%
    \refstepcounter{propriete}% Incrémente le compteur de théorème
    \begin{mdframed}[style=ProprieteStyle]
        \ifstrempty{#1}{% Teste si le premier argument est vide
            \textbf{\thepropriete. Propriétés :}\ % Si vide, n'affiche pas de titre de théorème
        }{%
            \textbf{\thepropriete. Propriétés #1 :}\ % Si non vide, affiche le titre avec le premier argument
        }\\#2
    \end{mdframed}
}

% Création d'un compteur pour les théorèmes
\newcounter{theoreme}

% Définition de la commande '\theoreme'
\newcommand{\theoreme}[2]{%
    \refstepcounter{theoreme}% Incrémente le compteur de théorème
    \begin{mdframed}[style=ProprieteStyle]
        \ifstrempty{#1}{% Teste si le premier argument est vide
            \textbf{\thetheoreme. Théorème :}\ % Si vide, n'affiche pas de titre de théorème
        }{%
            \textbf{\thetheoreme. Théorème #1 :}\ % Si non vide, affiche le titre avec le premier argument
        }\\#2
    \end{mdframed}
}

% Définition de la commande '\theoremes'
\newcommand{\theoremes}[2]{%
    \refstepcounter{theoreme}% Incrémente le compteur de théorème
    \begin{mdframed}[style=ProprieteStyle]
        \ifstrempty{#1}{% Teste si le premier argument est vide
            \textbf{\thetheoreme. Théorèmes :}\ % Si vide, n'affiche pas de titre de théorème
        }{%
            \textbf{\thetheoreme. Théorèmes #1 :}\ % Si non vide, affiche le titre avec le premier argument
        }\\#2
    \end{mdframed}
}

\newcommand{\corrolaire}[2]{%
    \begin{mdframed}[style=ProprieteStyle]
        \ifstrempty{#1}{% Teste si le premier argument est vide
            \textbf{Corrolaire :}\ % Si vide, n'affiche pas de titre de théorème
        }{%
            \textbf{Corrolaire #1 :}\ % Si non vide, affiche le titre avec le premier argument
        }\\#2
    \end{mdframed}
}

% ========== Exemple(s) ==========

% Définition de la commande 'exemple'
\newcommand{\exemple}[2]{
    \textbf{Exemple :} #1
    \begin{quote}
        #2
    \end{quote}
}

% Définition de la commande 'exemples'
\newcommand{\exemples}[2]{
    \textbf{Exemples :} #1
    \begin{quote}
        #2
    \end{quote}
}

% ========== Remarques ==========

% Définition de la commande 'remarque'
\newcommand{\remarque}[1]{
    \textbf{Remarque :} #1
}

% Définition de la commande 'remarques'
\newcommand{\remarques}[1]{
    \textbf{Remarques :}
    \begin{quote}
        #1
    \end{quote}
}

% ========== Démonstration ==========

\newcommand{\demonstration}[2]{
    \ifstrempty{#1}{% Teste si le premier argument est vide
        \textbf{Démonstration :}\ % Si vide, n'affiche pas de titre de théorème
    }{%
        \textbf{Démonstration #1 :}\ % Si non vide, affiche le titre avec le premier argument
    }
    \begin{quote}
        #2
    \end{quote}
}
% ========== Autres blocs ==========

\newcommand{\newbloc}[2]{
    \textbf{#1}
    \begin{quote}
        #2
    \end{quote}
}

% ========== Notations ==========

% Style
\newcommand{\ds}{\displaystyle}
\newcommand{\llb}{\llbracket}
\newcommand{\rrb}{\rrbracket}

% Ensembles
\newcommand{\C}{\mathbb{C}}
\newcommand{\R}{\mathbb{R}}
\newcommand{\Q}{\mathbb{Q}}
\newcommand{\D}{\mathbb{D}}
\newcommand{\Z}{\mathbb{Z}}
\newcommand{\N}{\mathbb{N}}

% Limites
\let\oldlim\lim
\renewcommand{\lim}[1]{\mathop{\displaystyle\oldlim}\limits_{#1}}

% Opérateurs
\newcommand{\x}{\times}
\newcommand{\equival}{\Leftrightarrow}
\newcommand{\involve}{\Rightarrow}

% Ensembles
\renewcommand{\bar}[1]{\overline{#1}}
\renewcommand{\part}[1]{\mathcal{P}({#1})}  % Ensemle des parties de E
\newcommand{\rel}{\mathcal{R}} % Relation entre deux ensembles

% Vecteurs
\renewcommand{\Vec}[1]{\overrightarrow{#1}}
\newcommand*{\norme}[1]{\|#1\|}
\newcommand{\pdt}{\mathbin{\vcenter{\hbox{\scalebox{0.6}{\textbullet}}}}}
\newcommand{\vcoord}[2]{\begin{pmatrix} #1 \\ #2 \end{pmatrix}}
\newcommand{\Vcoord}[3]{\begin{pmatrix} #1 \\ #2 \\ #3 \end{pmatrix}}

% Fonctions
\newcommand{\aire}[1]{\text{aire}\left( #1 \right)}
\newcommand{\Int}{\ds\int}


% ========== Systèmes ==========
\newcommand{\sys}[2]{\begin{cases}#1\\#2\end{cases}}
\newcommand{\Sys}[3]{\begin{cases}#1\\#2\\#3\end{cases}}

% ========== Tableaux ==========
\newcommand{\boldhline}{\hline\noalign{\vskip 0pt}\hline}

% ========== Graphiques ==========

% Triange de Pascal
\newcommand{\binomial}[2]{\pgfmathparse{int((#1)!/((#2)!*((#1)-(#2))!))}\pgfmathresult}
\newcommand{\Pascal}[1]{
    \begin{tikzpicture}[scale=1, every node/.style={scale=1.2}]
        \foreach \n in {0,...,#1} {
                \foreach \k in {0,...,\n} {
                        \node at (\k-\n/2,-\n) {$\binomial{\n}{\k}$};
                        \draw (\k-\n/2,-\n) +(-0.5,-0.5) rectangle ++(0.5,0.5);
                    }
            }
    \end{tikzpicture}
}

\newcommand{\Pascalbinome}[1]{
    \begin{tikzpicture}[scale=1, every node/.style={scale=0.9}]
        \foreach \n in {0,...,#1} {
                \foreach \k in {0,...,\n} {
                        \pgfmathtruncatemacro{\symk}{min(\k,\n-\k)}
                        \node at (\k-\n/2,-\n) {$\ds\binom{\n}{\symk}$};
                        \draw (\k-\n/2,-\n) +(-0.5,-0.5) rectangle ++(0.5,0.5);
                    }
            }
    \end{tikzpicture}
}

\newcolumntype{Y}{>{\centering\arraybackslash}m{1.2cm}}

\begin{document}

\maketitle

\section{Lecture sur le cercle trigonométrique}


\begin{minipage}{0.4\textwidth}
    \begin{tikzpicture}[scale=2]
        % Axes
        \draw[thick] (-1.3,0) -- (1.4,0);
        \draw[thick] (0,-1.3) -- (0,1.4);
        % Cercle trigonométrique
        \draw [thick] (0,0) circle (1);
        % Repère (O,I,J)
        \draw [red, thick] (0,0) node[below left] {$O$} -- (1,0) node[below right] {$I$} node[midway,below] {$1$};
        \draw [thick] (0,0) -- (0,1) node[above left] {$\red J$};
        % Angle
        \draw[->, thick] (1.1,0.3) arc (12:60:1);
        \node at (1.05,0.75) {+};
    \end{tikzpicture}
\end{minipage}
\hfill
\begin{minipage}{0.6\textwidth}
    \subsection{Le cercle trigonométrique}
    \definition{}{
        Dans un repère orhonormé $(O, I, J)$, le cercle trigonométrique de centre $O$ est le cercle qui a pour rayon $1$ et qui est muni d'un sens direct, le sens trigonométrique.
    }
    \subsection{Longueur d'un arc et radian}
    \propriete{}{
        Sur un cercle trigonométrique, la longueur de l'arc de cercle $\wideparen{IM}$ (exprimé dans l'unité de longueur du repère), est proportionnelle à la mesure de l'angle $\widehat{OIM}$ exprimé en degrés.
    }
\end{minipage}

\begin{minipage}{0.6\textwidth}
    En effet, le périmètre du cercle est $P=2\pi R=2\pi$. \\
    On a donc
    \renewcommand{\arraystretch}{1.5}
    \begin{tabular}{|c|c|}
        \hline
        $2\pi$           & $360^\circ$     \\ \hline
        $\wideparen{IM}$ & $\widehat{IOM}$ \\ \hline
    \end{tabular}
    car $\wideparen{IM}=\dfrac{2\pi}{360}\x\widehat{IOM}=\dfrac{\pi}{180}\x\widehat{IOM}$ \\
\end{minipage}
\hfill
\begin{minipage}{0.3\textwidth}
    \begin{tikzpicture}[scale=1]
        % Axes
        \draw (-1.3,0) -- (1.4,0);
        \draw (0,-1.3) -- (0,1.4);
        % Cercle trigonométrique
        \draw  (0,0) circle (1);
        % Angle marker
        \draw[green] (0.2,0) arc (0:135:0.2) node[above right] {\tiny{$\quad\widehat{IOM}$}};
        \fill[green, opacity=0.4] (0,0) -- (0.2,0) arc (0:135:0.2) -- cycle;
        % Repère (O,I,J)
        \draw (0,0) node[below left] {\scriptsize{$O$}} -- (1,0) node[below right] {\scriptsize{$I$}};
        \draw  (0,0) -- (0,1) node[above left] {\scriptsize{$J$}};
        % Point M
        \node[above left] at (-0.707,0.707) {\scriptsize{$M$}};
        \draw[mark size=1.5pt,mark=+] plot coordinates {(-0.707,0.707)};
        \draw (0,0) -- (-0.707,0.707);
    \end{tikzpicture}
\end{minipage}


\begin{minipage}{0.2\textwidth}
    \begin{tikzpicture}[scale=1]
        % Axes
        \draw (-1.3,0) -- (1.4,0);
        \draw (0,-1.3) -- (0,1.4);
        % Cercle trigonométrique
        \draw  (0,0) circle (1);
        % Angle marker
        \draw[green] (0.2,0) arc (0:55:0.2) node[midway,above right] {\tiny{$\widehat{IOU}$}};
        \draw[red] (1,0) arc (0:57:1) node[midway, right] {\scriptsize{$1$}};
        \fill[green, opacity=0.4] (0,0) -- (0.2,0) arc (0:55:0.2) -- cycle;
        % Repère (O,I,J)
        \draw[red] (0,0) node[below left] {\scriptsize{$O$}} -- (1,0) node[below right] {\scriptsize{$I$}} node[midway,  below] {\scriptsize{$1$}};
        \draw  (0,0) -- (0,1) node[above left] {\scriptsize{$J$}};
        % Point M
        \node[above right] at (0.54,0.84) {\scriptsize{$U$}};
        \draw[mark size=1.5pt,mark=+] plot coordinates {(0.54,0.84)};
        \draw (0,0) -- (0.54,0.84);
    \end{tikzpicture}
\end{minipage}
\hfill
\begin{minipage}{0.8\textwidth}
    \definition{}{
        Soit $U$ le point du cercle trigonométrique tel que l'arc $\wideparen{IU}$ ait pour longueur $1$ (exprimé dans l'unité de la longueur du repère). \\
        On définit un radian, noté $1\,\mathrm{rad}$, comme étant la mesure de l'angle $\widehat{IOU}$.
    }
\end{minipage}

\vspace{10pt}

\exemple{}{
    \renewcommand{\arraystretch}{1.5}
    \begin{tabular}{|l|c|c|c|c|c|c|c|c|c|}
        \hline
        Mesure de l'angle $\widehat{IOM}$ en degrés  & $360$  & $180$ & $90$            & $270$            & $30$            & $45$            & $60$            & $1$               & $\frac{180}{\pi}$ \\ \hline
        Longueur de l'arc $\wideparen{IM}$           & $2\pi$ & $\pi$ & $\frac{\pi}{2}$ & $\frac{3\pi}{2}$ & $\frac{\pi}{6}$ & $\frac{\pi}{4}$ & $\frac{\pi}{3}$ & $\frac{\pi}{180}$ & $1$               \\ \hline
        Mesure de l'angle $\widehat{IOM}$ en radians & $2\pi$ & $\pi$ & $\frac{\pi}{2}$ & $\frac{3\pi}{2}$ & $\frac{\pi}{6}$ & $\frac{\pi}{4}$ & $\frac{\pi}{3}$ & $\frac{\pi}{180}$ & $1$               \\ \hline
    \end{tabular}
}

\newpage

\section{Enroulement de la droite des réels sur le cercle trigonométrique}

Sur le cercle trigonométrique, on choisit un point comme origine et on enroule la droite des réels sur le cercle.

\begin{center}
    \begin{tikzpicture}[scale=2]
        % Axes
        \draw[thick] (-1.3,0) -- (1.4,0);
        \draw[thick] (0,-1.3) -- (0,1.4);
        % Cercle trigonométrique
        \draw [thick] (0,0) circle (1);
        % Repère (O,I,J)
        \draw [thick] (0,0) node[below left] {$O$} -- (1,0) node[below right] {$I$} node[midway,below] {$1$};
        \draw [thick] (0,0) -- (0,1) node[above left] {$J$};
        % Point M
        \node[below left] at (0.5,0.865) {\scriptsize{$M$}};
        \draw[mark size=1.5pt,mark=+] plot coordinates {(0.5,0.865)};
        % Curved arrow
        \draw[thick, red, dashed, ->] (1,1) to[out=170,in=50] (0.5,0.865);
        % Droite
        \draw[red, thick] (1,1.4) -- (1,-1.2);
        \node[right, red] at (1,1) {$k=1$};
        \draw[red, mark size=1.5pt, mark=x] plot coordinates {(1,1)};

    \end{tikzpicture}
\end{center}

\propriete{}{
    \vspace{-10pt}
    \begin{itemize}
        \item En enroulant la droite des réels sur le cercle trigonométrique, on associe à tout réel $x$ un unique point $M$ sur le cercle. \\
              On dit alors que $M$ est l'image de $x$ sur le cercle $C$.
        \item Réciproquement, à tout point $M$ du cercle trigonométrique correspondent une infinité de valeurs qui peuvent être considérés comme les abscisses des points de la droite. \\
              Si $x$ est l'un d'entre eux, les autres abscisses sont $x+2\pi$, $x+4\pi$, $x-2\pi$, $x-4\pi$\dots
    \end{itemize}
}

\textbf{Schéma d'un cercle trigonométrique avec des valeurs remarquables :}
\begin{center}
    \begin{tikzpicture}[scale=5]
        % Axes
        \draw[thick] (-1,0) -- (1,0) node[right] {};
        \draw[thick] (0,-1) -- (0,1) node[above] {};
        % Cercle trigonométrique
        \draw [thick] (0,0) circle (1);
        % Repère (O,I,J)
        \draw [thick] (0,0) node[below left] {\LARGE{$O$}} -- (1,0) node[below right] {\LARGE{$I$}};
        \draw [thick] (0,0) -- (0,1) node[above left] {\LARGE{$J$}};

        % Points 0
        \node[above right] at (1,0) {\LARGE{$0$}};
        \draw[mark size=0.4pt,mark=*] plot coordinates {(1,0)};
        % Points pi/2
        \node[above right] at (0,1) {\LARGE{$\frac{\pi}{2}$}};
        \draw[mark size=0.4pt,mark=*] plot coordinates {(0,1)};
        % Points pi
        \node[left] at (-1,0) {\LARGE{$\pi$}};
        \draw[mark size=0.4pt,mark=*] plot coordinates {(-1,0)};
        % Points -pi/2
        \node[below] at (0,-1) {\LARGE{$\frac{3\pi}{2}$}};
        \draw[mark size=0.4pt,mark=*] plot coordinates {(0,-1)};

        % Points pi/3
        \foreach \angle/\label/\pos in {60/{$\frac{\pi}{3}$}/right, 120/{$\frac{2\pi}{3}$}/left, 240/{$\frac{4\pi}{3}$}/left, 300/{$\frac{5\pi}{3}$}/right} {
        % Coordonnées polaires (r,θ) pour placer les points
        \coordinate (P\angle) at (\angle:1);
        % Marqueurs de points
        \draw[mark size=0.4pt,mark=*] plot coordinates {(P\angle)};
        % Étiquettes de points
        \path (P\angle) -- ++(\angle:0.05) node[\pos, green] {\LARGE\label};
        % Lignes en pointillés
        \draw[dashed, blue] (P\angle) -- (P\angle |- 0,0);
        \draw[dashed, green] (P\angle) -- (0,0 -| P\angle);
        }
        % Lignes reliant les points
        \draw[dashed, green] (P60) -- (P120);
        \draw[dashed, green] (P240) -- (P300);

        % Points pi/4
        \foreach \angle/\label/\pos in {45/{$\frac{\pi}{4}$}/right, 135/{$\frac{3\pi}{4}$}/left, 225/{$\frac{5\pi}{4}$}/left, 315/{$\frac{7\pi}{4}$}/right} {
        % Coordonnées polaires (r,θ) pour placer les points
        \coordinate (P\angle) at (\angle:1);
        % Marqueurs de points
        \draw[mark size=0.4pt,mark=*] plot coordinates {(P\angle)};
        % Étiquettes de points
        \path (P\angle) -- ++(\angle:0.05) node[\pos, red] {\LARGE\label};
        % Lignes en pointillés
        \draw[dashed, red] (P\angle) -- (P\angle |- 0,0);
        \draw[dashed, red] (P\angle) -- (0,0 -| P\angle);
        }
        % Lignes reliant les points
        \draw[dashed, red] (P45) -- (P135);
        \draw[dashed, red] (P225) -- (P315);

        % Points pi/6
        \foreach \angle/\label/\pos in {30/{$\frac{\pi}{6}$}/right, 150/{$\frac{5\pi}{6}$}/left, 210/{$\frac{7\pi}{6}$}/left, 330/{$\frac{11\pi}{6}$}/right} {
        % Coordonnées polaires (r,θ) pour placer les points
        \coordinate (P\angle) at (\angle:1);
        % Marqueurs de points
        \draw[mark size=0.4pt,mark=*] plot coordinates {(P\angle)};
        % Étiquettes de points
        \path (P\angle) -- ++(\angle:0.05) node[\pos, blue] {\LARGE\label};
        % Lignes en pointillés
        \draw[dashed, blue] (P\angle) -- (P\angle |- 0,0);
        \draw[dashed, blue] (P\angle) -- (0,0 -| P\angle);
        }
        % Lignes reliant les points
        \draw[dashed, blue] (P30) -- (P150);
        \draw[dashed, blue] (P210) -- (P330);

        % Points pi/3
        \foreach \angle/\label/\pos in {60/{$\frac{\pi}{3}$}/right, 120/{$\frac{2\pi}{3}$}/left, 240/{$\frac{4\pi}{3}$}/left, 300/{$\frac{5\pi}{3}$}/right} {
        % Coordonnées polaires (r,θ) pour placer les points
        \coordinate (P\angle) at (\angle:1);
        % Marqueurs de points
        \draw[mark size=0.4pt,mark=*] plot coordinates {(P\angle)};
        % Étiquettes de points
        \path (P\angle) -- ++(\angle:0.05) node[\pos, green] {\LARGE\label};
        }

        % Points pi/4
        \foreach \angle/\label/\pos in {45/{$\frac{\pi}{4}$}/right, 135/{$\frac{3\pi}{4}$}/left, 225/{$\frac{5\pi}{4}$}/left, 315/{$\frac{7\pi}{4}$}/right} {
        % Coordonnées polaires (r,θ) pour placer les points
        \coordinate (P\angle) at (\angle:1);
        % Marqueurs de points
        \draw[mark size=0.4pt,mark=*] plot coordinates {(P\angle)};
        % Étiquettes de points
        \path (P\angle) -- ++(\angle:0.05) node[\pos, red] {\LARGE\label};
        }

        % Points pi/6
        \foreach \angle/\label/\pos in {30/{$\frac{\pi}{6}$}/right, 150/{$\frac{5\pi}{6}$}/left, 210/{$\frac{7\pi}{6}$}/left, 330/{$\frac{11\pi}{6}$}/right} {
        % Coordonnées polaires (r,θ) pour placer les points
        \coordinate (P\angle) at (\angle:1);
        % Marqueurs de points
        \draw[mark size=0.4pt,mark=*] plot coordinates {(P\angle)};
        % Étiquettes de points
        \path (P\angle) -- ++(\angle:0.05) node[\pos, blue] {\LARGE\label};
        }
    \end{tikzpicture}
\end{center}

\newpage

\section{Cosinus et sinus d'un nombre réel}

\subsection{Définitions}

\definition{}{
    $C$ est le cercle trigonométrique de centre $O$ et $(O,I,J)$ un repère orhonormé direct. $x$ est un nombre réel et $M$ est le points image du réel $x$ sur le cercle trigonométrique.
    \begin{itemize}
        \item Le cosinus de $x$, noté $\cos{x}$, est l'abscisse de $M$ dans le repère $(O,I,J)$.
        \item Le sinus de $x$, noté $\sin{x}$, est l'ordonnée de $M$ dans le repère $(O,I,J)$.
    \end{itemize}
}


\begin{minipage}{0.6\textwidth}
    \propriete{}{
        Pour tout réel $x$ et tout entier relatif $k$, on a :
        \begin{itemize}
            \item $\cos^2{x}+\sin^2{x}=1$
            \item $\cos(x+2k\pi)=\cos{x}$
            \item $-1\leq\cos{x}\leq1$
            \item $-1\leq\sin{x}\leq1$
        \end{itemize}
    }
\end{minipage}
\hfill
\begin{minipage}{0.3\textwidth}
    \begin{tikzpicture}[scale=2]
        % Cercle trigonométrique
        \draw [thick] (0,0) circle (1);
        % Repère (O,I,J)
        \draw [thick] (0,0) node[below left] {$O$} -- (1,0) node[below right] {$I$};
        \draw [thick] (0,0) -- (0,1) node[above left] {$J$};
        \node at (-0.8,0.8) {$C$};
        % Point M
        \node[right] at (0.707,0.707) {\scriptsize{$M(x)$}};
        \node[blue,mark size=0.4pt,mark=*, below] at (0.707, 0) {$\cos{x}$};
        \node[blue, left] at (0, 0.707) {$\sin{x}$};
        \draw[blue, thick] (0,0) -- (0.707,0.707) node[midway, above left] {};
        \draw[blue, thick, dashed] (0.707,0) -- (0.707,0.707) node[midway, above left] {};
        \draw[blue, thick, dashed] (0,0.707) -- (0.707,0.707) node[midway, above left] {};
        \draw[mark size=0.5pt,mark=*] plot coordinates {(0.707,0.707)};
        % Axes
        \draw[thick] (-1.3,0) -- (1.4,0);
        \draw[thick] (0,-1.3) -- (0,1.4);
        \draw[blue, mark size=0.5pt,mark=*] plot coordinates {(0.707,0)};
        \draw[blue, mark size=0.5pt,mark=*] plot coordinates {(0,0.707)};
    \end{tikzpicture}
\end{minipage}

\subsection{Valeurs remarquables du cosinus et du sinus}

\renewcommand{\arraystretch}{2.5}
\begin{tabular}{|Y|Y|Y|Y|Y|Y|}
    \hline
    \large{$x$}       & \Large{$0$} & \Large{$\frac{\pi}{2}$} & \Large{$\frac{\pi}{3}$}      & \Large{$\frac{\pi}{4}$}      & \Large{$\frac{\pi}{6}$}      \\
    \hline
    \large{$\cos{x}$} & \Large{$1$} & \Large{$0$}             & \Large{$\frac{1}{2}$}        & \Large{$\frac{\sqrt{2}}{2}$} & \Large{$\frac{\sqrt{3}}{2}$} \\
    \hline
    \large{$\sin{x}$} & \Large{$0$} & \Large{$1$}             & \Large{$\frac{\sqrt{3}}{2}$} & \Large{$\frac{\sqrt{2}}{2}$} & \Large{$\frac{1}{2}$}        \\
    \hline
\end{tabular}

\subsection{Lien avec le cosinus et sinus dans un triangle rectangle}

\begin{minipage}{0.6\textwidth}
    On considère le cercle trigonométrique et la tangente $D$ au cercle.

    Soit $x\in]0;\frac{\pi}{2}[$ et $M$ son point image.

                Soit $H$ le projeté orhogonal de $M$ sur $(OI)$.

                Soit $H'$ le projeté orhogonal de $M$ sur $(OJ)$.


                On a alors :
                \begin{itemize}
                    \item $\cos(\widehat{IOM})=\frac{\text{côtés adjacents}}{\text{hypoténuse}}=\frac{OH}{OM}=\frac{OH}{1}=OH=\cos{x}$
                    \item $\sin(\widehat{IOM})=\frac{\text{côtés opposé}}{\text{hypoténuse}}=\frac{HM}{OM}=\frac{HM}{1}=OH'=\sin{x}$
                \end{itemize}

                La notion de cosinus et de sinus d'un réel de l'intervalle $]0;\frac{\pi}{2}[$ \\
    coïncident avec les notions de cosinus et sinus d'angle aigu vues \\ aux  collège.
\end{minipage} 
\hfill
\begin{minipage}{0.4\textwidth}
    \begin{tikzpicture}[scale=2.8]
        % Cercle trigonométrique
        \draw [thick] (0,0) circle (1);
        % Repère (O,I,J)
        \draw [thick] (0,0) node[below left] {$O$} -- (1,0) node[below right] {$I$};
        \draw [thick] (0,0) -- (0,1) node[above left] {$J$};
        % Point M
        \node[above right] at (0.707,0.707) {$M$};
        \node[blue,mark size=0.4pt,mark=*, below] at (0.707, 0) {$H$};
        \node[blue, left] at (0, 0.707) {$H'$};
        \draw[blue, thick] (0,0) -- (0.707,0.707) node[midway, above left] {};
        \draw[blue, thick, dashed] (0.707,0) -- (0.707,0.707) node[midway, above left] {};
        \draw[blue, thick, dashed] (0,0.707) -- (0.707,0.707) node[midway, above left] {};
        \draw[mark size=0.5pt,mark=*] plot coordinates {(0.707,0.707)};
        % Axes
        \draw[thick] (-1.3,0) -- (1.4,0);
        \draw[thick] (0,-1.3) -- (0,1.4);
        \draw[blue, mark size=0.5pt,mark=*] plot coordinates {(0.707,0)};
        \draw[blue, mark size=0.5pt,mark=*] plot coordinates {(0,0.707)};
        % Droite
        \draw[red, thick] (1,1.4) -- (1,-1.2);
        \draw[red, thick, dashed] (0.707,0.707) -- (1,0.707) node[right] {$x$};
        \draw[red, mark size=0.5pt,mark=*] plot coordinates {(1,0.707)};

        % Angle marker
        \node[green] at (0.4,0.15) {$\widehat{IOM}$};
        \draw[green] (0.2,0) arc (0:45:0.2) {};
        \draw[red] (1,0) arc (0:45:1) node[midway, right] {};
        \fill[green, opacity=0.4] (0,0) -- (0.2,0) arc (0:45:0.2) -- cycle;

    \end{tikzpicture}
\end{minipage}

\newpage

\section{Fonctions cosinus et sinus}

\definition{}{
    \vspace{-10pt}
    \begin{itemize}
        \item On appelle fonction cosinus la fonction notée $\cos$ définie sur $\mathbb{R}$ par $\cos:x\mapsto\cos x$.
        \item On appelle fonction sinus la fonction notée $\sin$ définie sur $\mathbb{R}$ par $\sin:x\mapsto\sin x$.
    \end{itemize}
}

\propriete{}{
    Pour tout réel $x$, $\cos(-x)=\cos{x}$ et $\sin(-x)=-\sin{x}$. \\
    Ainsi, la fonction cosinus est paire et la fonction sinus est impaire.
}


\textbf{Conséquences graphique :}
\begin{itemize}
    \item La courbe représentative de la fonction $\cos$ est symétrique par rapport à l'axe des ordonnées.
    \item La courbe représentative de la fonction $\sin$ est symétrique par rapport à l'origine du repère.
\end{itemize}

\propriete{}{
    Pour tout réel $x$, $\cos{x}=\cos(x+2\pi)$ et $\sin{x}=\sin(x+2\pi)$.\\
    On dit que les fonctions cosinus et sinus sont des fonction périodiques de période $2\pi$.
}


\textbf{Conséquence graphique :} ~\\
Les courbes représentatives de la fonction cosinus et sinus se reproduisent identiques à elles-mêmes sur un intervalle de longueur $2\pi$.

\propriete{(variations des fonctions cosinus et sinus)}{
    Les deux propriétés précédentes nous permettent de réduire l'intervalle d'étude des deux fonctions $\cos$ et $\sin$ à l'intervalle $[0;\pi]$. \\
    \begin{tikzpicture}
        \tkzTabInit[espcl=4]{$x$ / 1, Variations de $\cos{x}$ / 2}{$0$, $\pi$}
        \tkzTabVar{+/ $1$, -/$-1$ }
    \end{tikzpicture}
    \begin{tikzpicture}[scale=0.5]
        \begin{axis}[
                axis lines=middle,
                xtick={-3.1416, -1.5708, 1.5708, 3.1416, 4.7124, 6.2832},
                xticklabels={$-\pi$, $-\frac{\pi}{2}$, $\frac{\pi}{2}$, $\pi$, $\frac{3\pi}{2}$},
                ytick={-1,0,1},
                ymin=-1.2, ymax=1.4,
                xmin=-3.5, xmax=5.5,
                grid=major,
                grid style={dashed,black!30},
                width=\textwidth,
                height=0.4\textwidth,
                scale only axis,
                axis line style={->, >=stealth},
                set layers
            ]
            \addplot[red,smooth,samples=200,domain=-3.5:6.5] {cos(deg(x))};
            \node at (axis cs:0.7,1.2) {$\cos{x}$};
            \node at (axis cs:5.4,-0.2) {$x$};
        \end{axis}
    \end{tikzpicture}

    \text{ }

    \begin{tikzpicture}
        \tkzTabInit[espcl=2]{$x$ / 1, Variations de $\sin{x}$ /2}{$0$, $ $, $\pi$}
        \tkzTabVar{-/ $0$, +/$1$, -/$-1$ }
    \end{tikzpicture}
    \begin{tikzpicture}[scale=0.5]
        \begin{axis}[
                axis lines=middle,
                xtick={-3.1416, -1.5708, 1.5708, 3.1416, 4.7124},
                xticklabels={$-\pi$, $-\frac{\pi}{2}$, $\frac{\pi}{2}$, $\pi$, $\frac{3\pi}{2}$},
                ytick={-1,0,1},
                ymin=-1.2, ymax=1.4,
                xmin=-3.5, xmax=5.5,
                grid=major,
                grid style={dashed,black!30},
                width=\textwidth,
                height=0.4\textwidth,
                scale only axis,
                axis line style={->, >=stealth},
                set layers
            ]
            \addplot[red,smooth,samples=200,domain=-3.5:6.5] {sin(deg(x))};
            \node at (axis cs:0.7,1.2) {$\sin{x}$};
            \node at (axis cs:5.4,-0.2) {$x$};
        \end{axis}
    \end{tikzpicture}
}

\section{La fontion tangente}

\definition{}{
    On appelle fonction tangente la fonction notée $\tan$ définie sur $\mathbb{R}$ par $\tan:x\mapsto\tan x=\dfrac{\sin x}{\cos x}$ avec $\cos{x}\not=0$.
    On note $D$ son ensemble de définition tel que $D=\R-\left\{ k\in\R;\dfrac{\pi}{2}+k\right\}$.
}




\end{document}