\documentclass[10pt]{article}

\newcommand{\DocumentTitle}{Nombres complexes}
\newcommand{\DocumentTheme}{Algèbre}
\newcommand{\DocumentType}{Cours}

% Preamble Section
\usepackage[a4paper, margin=1in]{geometry} % Sets the paper size to A4 and all margins to 1 inch
\usepackage[utf8]{inputenc}   % Allows for input of international characters
\usepackage[T1]{fontenc}      % Utilisation de l'encodage T1
\usepackage{lmodern}          % Support des polices avec le module 'french'
\usepackage[french]{babel}    % Ajout du support pour le français
\usepackage[fleqn]{amsmath}   % Imports the AMS math package for advanced math formatting ([fleqn] -> align alignat to the left)
\usepackage{amsfonts}         % Imports AMS fonts for math fonts
\usepackage{amssymb}          % Imports AMS symbols for math symbols
\usepackage{amsthm}           % Imports AMS theorems
\usepackage{fancyhdr}         % Import to custom the page footer
\usepackage{mdframed}         % Imports styles
\usepackage{enumerate}        % Imports styles for enumerate
\usepackage{multicol}
\usepackage{xcolor}           % Couleurs
\usepackage{tikz}             % Package général pour les graphiques
\usepackage{pgfplots}         % Compléments pour les graphiques
\usepackage{tkz-tab}          % Tableaux de variations
\usepackage{lastpage}         % Pour avoir le total de page dans le footer
\usepackage{makecell}         % Retour à la ligne dans une case d'un tabular
\usepackage{stmaryrd}         % Intervalles entières : \llbracket et \rrbracket

\pgfplotsset{compat=newest}   % Active les dernières fonctionnalités de pgfplots

% Enlève l'indentation des bloc de paragraphes et d'équation
\setlength{\parindent}{0pt}
\setlength{\mathindent}{0pt}

% ========== Title Section ==========

\title{\DocumentTitle} % Title of the document
\author{\DocumentTheme\space-\space\DocumentType} % Author's name
\date{} % Date

\pagestyle{fancy}
\fancyhf{} % Clear all header and footer fields

% ========== Footer Section ==========

\newcommand{\customfooter}{
    \fancyfoot[L]{\DocumentTheme\space-\space\DocumentType}
    \fancyfoot[C]{\DocumentTitle}
    \fancyfoot[R]{\thepage/\pageref{LastPage}}
    % Remove the line below the header and above the footer
    \renewcommand{\headrulewidth}{0pt}
    \renewcommand{\footrulewidth}{0pt}
}

% Apply the custom footer to the fancy style
\customfooter

% Apply the custom footer to the plain style (used on the first page)
\fancypagestyle{plain}{\customfooter}

% ========== Sections ==========

% Sections : I. / II. / III. ...
\renewcommand{\thesection}{\Roman{section}}
% Subsection : I.1 / I.2 / I.3 ...
\renewcommand{\thesubsection}{\thesection. \arabic{subsection}}
% Subsubsection : I.1.a / I.1.b / I.1.c ...
\renewcommand{\thesubsubsection}{\thesubsection. \alph{subsubsection}}

% ========== Colors (from Geogebra) ==========

\definecolor{green}{HTML}{006400}
\definecolor{red}{HTML}{CC0000}
\definecolor{blue}{HTML}{0000FF}
\definecolor{orange}{HTML}{FF5500}
\definecolor{purple}{HTML}{9933FF}
\definecolor{gray}{HTML}{666666}
\definecolor{brown}{HTML}{993300}

\newcommand{\green}[1]{\color{green}{#1}\color{black}}
\newcommand{\red}[1]{\color{red}{#1}\color{black}}
\newcommand{\blue}[1]{\color{blue}{#1}\color{black}}
\newcommand{\orange}[1]{\color{orange}{#1}\color{black}}
\newcommand{\purple}[1]{\color{purple}{#1}\color{black}}
\newcommand{\gray}[1]{\color{gray}{#1}\color{black}}
\newcommand{\brown}[1]{\color{brown}{#1}\color{black}}


% ========== Définition(s) ==========

% Style de la boîte de définition(s)
\mdfdefinestyle{DefinitionStyle}{
    leftmargin=0cm,
    rightmargin=0cm,
    linecolor=black,
    linewidth=2pt,
    topline=false,
    bottomline=false,
    rightline=false
}

% Définition de la commande '\definition'
\newcommand{\definition}[2]{%
    \begin{mdframed}[style=DefinitionStyle]
        \ifstrempty{#1}{% Teste si le premier argument est vide
            \textbf{Définition :}\ % Si vide, n'affiche pas de titre de théorème
        }{%
            \textbf{Définition #1 :}\ % Si non vide, affiche le titre avec le premier argument
        }\\#2
    \end{mdframed}
}

% Définition de la commande '\definitions'
\newcommand{\definitions}[2]{%
    \begin{mdframed}[style=DefinitionStyle]
        \ifstrempty{#1}{% Teste si le premier argument est vide
            \textbf{Définitions :}\ % Si vide, n'affiche pas de titre de théorème
        }{%
            \textbf{Définitions #1 :}\ % Si non vide, affiche le titre avec le premier argument
        }\\#2
    \end{mdframed}
}

% ========== Propriété(s) et Théorème(s) ==========

% Style de la boîte de propriété(s) / théorème(s)
\mdfdefinestyle{ProprieteStyle}{
    leftmargin=0cm,
    rightmargin=0cm,
    linecolor=black,
    linewidth=1pt
}

% Création d'un compteur pour les propriétés
\newcounter{propriete}

% Définition de la commande '\propriete'
\newcommand{\propriete}[2]{%
    \refstepcounter{propriete}% Incrémente le compteur de théorème
    \begin{mdframed}[style=ProprieteStyle]
        \ifstrempty{#1}{% Teste si le premier argument est vide
            \textbf{\thepropriete. Propriété :}\ % Si vide, n'affiche pas de titre de théorème
        }{%
            \textbf{\thepropriete. Propriété #1 :}\ % Si non vide, affiche le titre avec le premier argument
        }\\#2
    \end{mdframed}
}

% Définition de la commande 'proprietes'
\newcommand{\proprietes}[2]{%
    \refstepcounter{propriete}% Incrémente le compteur de théorème
    \begin{mdframed}[style=ProprieteStyle]
        \ifstrempty{#1}{% Teste si le premier argument est vide
            \textbf{\thepropriete. Propriétés :}\ % Si vide, n'affiche pas de titre de théorème
        }{%
            \textbf{\thepropriete. Propriétés #1 :}\ % Si non vide, affiche le titre avec le premier argument
        }\\#2
    \end{mdframed}
}



% Création d'un compteur pour les théorèmes
\newcounter{theoreme}

% Définition de la commande '\theoreme'
\newcommand{\theoreme}[2]{%
    \refstepcounter{theoreme}% Incrémente le compteur de théorème
    \begin{mdframed}[style=ProprieteStyle]
        \ifstrempty{#1}{% Teste si le premier argument est vide
            \textbf{\thetheoreme. Théorème :}\ % Si vide, n'affiche pas de titre de théorème
        }{%
            \textbf{\thetheoreme. Théorème #1 :}\ % Si non vide, affiche le titre avec le premier argument
        }\\#2
    \end{mdframed}
}

% Définition de la commande '\theoremes'
\newcommand{\theoremes}[2]{%
    \refstepcounter{theoreme}% Incrémente le compteur de théorème
    \begin{mdframed}[style=ProprieteStyle]
        \ifstrempty{#1}{% Teste si le premier argument est vide
            \textbf{\thetheoreme. Théorèmes :}\ % Si vide, n'affiche pas de titre de théorème
        }{%
            \textbf{\thetheoreme. Théorèmes #1 :}\ % Si non vide, affiche le titre avec le premier argument
        }\\#2
    \end{mdframed}
}

% ========== Exemple(s) ==========

% Définition de la commande 'exemple'
\newcommand{\exemple}[2]{
    \textbf{Exemple :} #1
    \begin{quote}
        #2
    \end{quote}
}

% Définition de la commande 'exemples'
\newcommand{\exemples}[2]{
    \textbf{Exemples :} #1
    \begin{quote}
        #2
    \end{quote}
}

% ========== Remarques ==========

% Définition de la commande 'remarque'
\newcommand{\remarque}[1]{
    \textbf{Remarque :} #1
}

% Définition de la commande 'remarques'
\newcommand{\remarques}[1]{
    \textbf{Remarques :} #1
}

% ========== Démonstration ==========

\newcommand{\demonstration}[2]{
    \textbf{Démonstration :} #1
    \begin{quote}
        #2
    \end{quote}
}

% ========== Notations ==========

% Ensembles
\newcommand{\C}{\mathbb{C}}
\newcommand{\R}{\mathbb{R}}
\newcommand{\Q}{\mathbb{Q}}
\newcommand{\D}{\mathbb{D}}
\newcommand{\Z}{\mathbb{Z}}
\newcommand{\N}{\mathbb{N}}

% Limites
\let\oldlim\lim
\renewcommand{\lim}[1]{\mathop{\displaystyle\oldlim}\limits_{#1}}

% Opérateurs
\newcommand{\x}{\times}
\newcommand{\equival}{\Leftrightarrow}

% Vecteurs
\renewcommand{\Vec}[1]{\overrightarrow{#1}}
\newcommand*{\norme}[1]{\left\lVert\vv{#1}\right\rVert}
\newcommand{\ps}[2]{\ensuremath{\vv{#1}.\vv{#2}}}

\newcommand{\vcoord}[2]{\begin{pmatrix} #1 \\ #2 \end{pmatrix}}
\newcommand{\Vcoord}[3]{\begin{pmatrix} #1 \\ #2 \\ #3 \end{pmatrix}}

% ========== Tableaux ==========
\newcommand{\boldhline}{\hline\noalign{\vskip 0pt}\hline}

\begin{document}

\maketitle

% Content of the document


\section{Ensemble des nombres complexes}

\subsection{Notion de nombre complexe}

\propriete{}{
    Il existe un ensemble des nombres complexes (noté $\C$) qui possède les propriétés suiante :
    \begin{enumerate}[(i)]
       \item $\R\subset\C$
       \item L'addition et la multiplication des réels se prolongent aux nombres complexes et les règles de calcul sont
             les mêmes.
       \item Il existe un nombre complexe $i$ tel que $i^2=-1$.
       \item Tout nombre complexe $z$ s'écrit $z=a+ib$ avec $a$ et $b$ des réels.
    \end{enumerate}
}

\definitions{(nombre complexe)}{
    L'écriture $z=a+ib$ d'un nombre complexe avec $(a,b)\in\R^2$ est appelé forme algébrique (ou cartésienne) de $z$
    avec :
    \begin{itemize}
       \item $a$ la partie réelle de $z$ noté $a=\Re(z)$.
       \item $b$ la partie imaginaire de $z$ noté $a=\Im(z)$.
    \end{itemize}
    }

\remarques{
    Soit $z$ un nombre complexe.
    \begin{itemize}
       \item On a $z=\Re(z) + i\times\Im(z)$
       \item Si $\Re(z)=0$, on dit que $z$ est imaginaire pur. On note $i\R$ l'ensemble des imaginaires purs.
       \item Si $\Im(z)=0$, $z$ est réel.
    \end{itemize}
}

\theoreme{}{
    Soient $z=a+ib$ et $z'=a'+ib'$ deux nombres complexes avec $a$, $b$, $a'$ et $b'$ des réels. Alors : 
    $z=z' \equival \sys{a=a'}{b=b'}$.
}

\corrolaire{}{
    Tout nombre complexe $z$ s'écrit de manière unique $z=a+ib$ avec $(a,b)\in\R^2$.
}

\proprietes{(conséquence)}{
    Soient $z=a+ib$ et $z'=a'+ib'$ deux nombres complexes avec $a$, $b$, $a'$ et $b'$ des réels. Alors :
    \begin{enumerate}[(i)]
       \item $z\not= z' \equival a\not=a' \lor b\not=b'$
       \item $z=0 \equival a=0 \land b=0$
       \item $z\not=0 \equival a\not=0 \lor b\not=0$
    \end{enumerate}
}

\newpage

\subsection{Opérations sur les nombres complexes}

\proprietes{(addition et multiplication)}{
    Soient $z=a+ib$ et $z'=a'+ib'$ deux nombres complexes avec $a$, $b$, $a'$ et $b'$ des réels. Alors :
    \begin{enumerate}[(i)]
       \item $z+z' = (a+a') + i(b+b')$
       \item $z\times z' = (aa'-bb')+ i(ab'-a'b)$
    \end{enumerate}
}

\definition{(opposé)}{
    Pour tout nombre complexe $z=a+ib$ avec $(a,b)\in\R^2$, il existe un unique nombre complexe $z'$ tel que $z+z'=0$.
    On appelle $z'$ l'opposé de $z$ et on le note $-z=(-a)+i(-b)$. 
}

\definition{(soustraction)}{
    Soient $z=a+ib$ et $z'=a'+ib'$ deux nombres complexes avec $a$, $b$, $a'$ et $b'$ des réels. Alors $z-z'$ est défini
    par $z+(-z')$ et on a $z-z'=(a-a')+i(b-b')$.
}

\definition{(inverse)}{
    Pour tout nombre complexe $z=a+ib$ avec $a$, $b$ des réels et $z\not=0$, il existe un unique nombre complexe $z'$
    tel que $z\times z'=1$. On appelle $z'$ inverse de $z$ et on le note
    $\dfrac{1}{z}=\dfrac{a}{a^2+b^2}+i\times\dfrac{b}{a^2+b^2}$.   
}

\definition{(quotient)}{
    Soient $z$ et $z'$ deux nombres complexes tels que $z'\not=0$. Alors $\dfrac{z}{z'}$ est défini par
    $\dfrac{z}{z'}=z\times \dfrac{1}{z'}$.
}

\section{Conjugué d'un nombre complexe}

\definition{}{
    Soit $z$ un nombre complexe tel que $z=a+ib$ avec $(a,b)\in\R^2$. Alors le conjugué de $z$, noté $\bar{z}$, est le
    nombre complexe défini par $\bar{z}=a-ib$.
}

\proprietes{}{
    Pour tout nombre complexe $z$, on a : \\

    \begin{minipage}{0.22\textwidth}
        \begin{enumerate}[(i)]
           \item $z+\bar{z} = 2\Re(z)$
           \item $z-\bar{z} = 2i\Im(z)$
        \end{enumerate}
    \end{minipage}
    \hfill
    \begin{minipage}{0.25\textwidth}
        \begin{enumerate}[(i)]
            \setcounter{enumi}{2}
            \item $z\in\R \equival z=\bar{z}$
            \item $z\in i\R \equival z=-\bar{z}$
        \end{enumerate}
    \end{minipage}
    \hfill
    \begin{minipage}{0.3\textwidth}
        \begin{enumerate}[(i)]
            \setcounter{enumi}{4}
            \item $\bar{\bar{z}}=z$
            \item $z\bar{z}=\Re(z)^2+\Im(z)^2$
        \end{enumerate}
    \end{minipage}
}

\proprietes{(opérations et conjugué)}{
    Pour tous nombres complexes $z$ et $z'$, on a :

    \begin{minipage}{0.5\textwidth}
        \begin{enumerate}[(i)]
        \item $\bar{-z}=-\bar{z}$
        \item $\bar{z+z'}=\bar{z}+\bar{z'}$
        \item $\bar{z\times z'}=\bar{z}\times \bar{z'}$
        \item $\forall n\in\N, \bar{(z^n)}=(\bar{z})^n$
        \end{enumerate}
    \end{minipage}
    \hfill
    \begin{minipage}{0.5\textwidth}
        \setcounter{enumi}{4}
        \begin{enumerate}[(i)]
           \item Si $z\not=0$, $\bar{\left(\dfrac{1}{z}\right)}=\dfrac{1}{\bar{z}}$
           \item Si $z\not=0$, $\bar{\left(\dfrac{z'}{z}\right)}=\dfrac{\bar{z'}}{\bar{z}}$

        \end{enumerate}
    \end{minipage}
}


\end{document}