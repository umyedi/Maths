\documentclass[10pt]{article}

\newcommand{\DocumentTitle}{Les suites}
\newcommand{\DocumentTheme}{Algèbre}
\newcommand{\DocumentType}{Cours}

% Preamble Section
\usepackage[a4paper, margin=1in]{geometry} % Sets the paper size to A4 and all margins to 1 inch
\usepackage[utf8]{inputenc}   % Allows for input of international characters
\usepackage[T1]{fontenc}      % Utilisation de l'encodage T1
\usepackage{lmodern}          % Support des polices avec le module 'french'
\usepackage[french]{babel}    % Ajout du support pour le français
\usepackage[fleqn]{amsmath}   % Imports the AMS math package for advanced math formatting ([fleqn] -> align alignat to the left)
\usepackage{amsfonts}         % Imports AMS fonts for math fonts
\usepackage{amssymb}          % Imports AMS symbols for math symbols
\usepackage{amsthm}           % Imports AMS theorems
\usepackage{yhmath}           % Used for "\wideparen{}" 
\usepackage{fancyhdr}         % Import to custom the page footer
\usepackage{mdframed}         % Imports styles
\usepackage{enumerate}        % Imports styles for enumerate
\usepackage{multicol}
\usepackage{xcolor}           % Couleurs
\usepackage{tikz}             % Package général pour les graphiques
\usetikzlibrary{calc}
\usepackage{pgfplots}         % Compléments pour les graphiques
\usepackage{tkz-tab}          % Tableaux de variations
\usepackage{lastpage}         % Pour avoir le total de page dans le footer
\usepackage{makecell}         % Retour à la ligne dans une case d'un tabular
\usepackage{stmaryrd}         % Intervalles entières : \llbracket et \rrbracket
\usepackage{cancel}           % Permet de barrer des termes
\usepackage{forest}           % Permet de créer des arbres pondérés
\usepackage{tabularx}         % Permet de générer des tableaux (ex: cours fonction dérivées et application)
\usepackage{multirow}         % Permet de faire plusieurs lignes dans des cases de talbeaux


\pgfplotsset{compat=newest}   % Active les dernières fonctionnalités de pgfplots

% Enlève l'indentation des bloc de paragraphes et d'équation
\setlength{\parindent}{0pt}
\setlength{\mathindent}{0pt}

% ========== Title Section ==========

\title{\DocumentTitle} % Title of the document
\author{\DocumentTheme\space-\space\DocumentType} % Author's name
\date{} % Date

\pagestyle{fancy}
\fancyhf{} % Clear all header and footer fields

% ========== Footer Section ==========

\newcommand{\customfooter}{
    \fancyfoot[L]{\DocumentTheme\space-\space\DocumentType}
    \fancyfoot[C]{\DocumentTitle}
    \fancyfoot[R]{\thepage/\pageref{LastPage}}
    % Remove the line below the header and above the footer
    \renewcommand{\headrulewidth}{0pt}
    \renewcommand{\footrulewidth}{0pt}
}

% Apply the custom footer to the fancy style
\customfooter

% Apply the custom footer to the plain style (used on the first page)
\fancypagestyle{plain}{\customfooter}

% ========== Sections ==========

% Sections : I. / II. / III. ...
\renewcommand{\thesection}{\Roman{section}}
% Subsection : I.1 / I.2 / I.3 ...
\renewcommand{\thesubsection}{\thesection. \arabic{subsection}}
% Subsubsection : I.1.a / I.1.b / I.1.c ...
\renewcommand{\thesubsubsection}{\thesubsection. \alph{subsubsection}}

% ========== Colors (from Geogebra) ==========

\definecolor{green}{HTML}{006400}
\definecolor{red}{HTML}{CC0000}
\definecolor{blue}{HTML}{0000FF}
\definecolor{orange}{HTML}{FF5500}
\definecolor{purple}{HTML}{9933FF}
\definecolor{gray}{HTML}{666666}
\definecolor{brown}{HTML}{993300}
\definecolor{black}{HTML}{000000}
\definecolor{white}{HTML}{FFFFFF}

\newcommand{\green}[1]{\color{green}{#1}\color{black}}
\newcommand{\red}[1]{\color{red}{#1}\color{black}}
\newcommand{\blue}[1]{\color{blue}{#1}\color{black}}
\newcommand{\orange}[1]{\color{orange}{#1}\color{black}}
\newcommand{\purple}[1]{\color{purple}{#1}\color{black}}
\newcommand{\gray}[1]{\color{gray}{#1}\color{black}}
\newcommand{\brown}[1]{\color{brown}{#1}\color{black}}


% ========== Définition(s) ==========

% Style de la boîte de définition(s)
\mdfdefinestyle{DefinitionStyle}{
    leftmargin=0cm,
    rightmargin=0cm,
    linecolor=black,
    linewidth=2pt,
    topline=false,
    bottomline=false,
    rightline=false
}

% Définition de la commande '\definition'
\newcommand{\definition}[2]{%
    \begin{mdframed}[style=DefinitionStyle]
        \ifstrempty{#1}{% Teste si le premier argument est vide
            \textbf{Définition :}\ % Si vide, n'affiche pas de titre de théorème
        }{%
            \textbf{Définition #1 :}\ % Si non vide, affiche le titre avec le premier argument
        }\\#2
    \end{mdframed}
}

% Définition de la commande '\definitions'
\newcommand{\definitions}[2]{%
    \begin{mdframed}[style=DefinitionStyle]
        \ifstrempty{#1}{% Teste si le premier argument est vide
            \textbf{Définitions :}\ % Si vide, n'affiche pas de titre de théorème
        }{%
            \textbf{Définitions #1 :}\ % Si non vide, affiche le titre avec le premier argument
        }\\#2
    \end{mdframed}
}

% ========== Propriété(s) et Théorème(s) ==========

% Style de la boîte de propriété(s) / théorème(s)
\mdfdefinestyle{ProprieteStyle}{
    leftmargin=0cm,
    rightmargin=0cm,
    linecolor=black,
    linewidth=1pt
}

% Création d'un compteur pour les propriétés
\newcounter{propriete}

% Définition de la commande '\propriete'
\newcommand{\propriete}[2]{%
    \refstepcounter{propriete}% Incrémente le compteur de théorème
    \begin{mdframed}[style=ProprieteStyle]
        \ifstrempty{#1}{% Teste si le premier argument est vide
            \textbf{\thepropriete. Propriété :}\ % Si vide, n'affiche pas de titre de théorème
        }{%
            \textbf{\thepropriete. Propriété #1 :}\ % Si non vide, affiche le titre avec le premier argument
        }\\#2
    \end{mdframed}
}

% Définition de la commande 'proprietes'
\newcommand{\proprietes}[2]{%
    \refstepcounter{propriete}% Incrémente le compteur de théorème
    \begin{mdframed}[style=ProprieteStyle]
        \ifstrempty{#1}{% Teste si le premier argument est vide
            \textbf{\thepropriete. Propriétés :}\ % Si vide, n'affiche pas de titre de théorème
        }{%
            \textbf{\thepropriete. Propriétés #1 :}\ % Si non vide, affiche le titre avec le premier argument
        }\\#2
    \end{mdframed}
}

% Création d'un compteur pour les théorèmes
\newcounter{theoreme}

% Définition de la commande '\theoreme'
\newcommand{\theoreme}[2]{%
    \refstepcounter{theoreme}% Incrémente le compteur de théorème
    \begin{mdframed}[style=ProprieteStyle]
        \ifstrempty{#1}{% Teste si le premier argument est vide
            \textbf{\thetheoreme. Théorème :}\ % Si vide, n'affiche pas de titre de théorème
        }{%
            \textbf{\thetheoreme. Théorème #1 :}\ % Si non vide, affiche le titre avec le premier argument
        }\\#2
    \end{mdframed}
}

% Définition de la commande '\theoremes'
\newcommand{\theoremes}[2]{%
    \refstepcounter{theoreme}% Incrémente le compteur de théorème
    \begin{mdframed}[style=ProprieteStyle]
        \ifstrempty{#1}{% Teste si le premier argument est vide
            \textbf{\thetheoreme. Théorèmes :}\ % Si vide, n'affiche pas de titre de théorème
        }{%
            \textbf{\thetheoreme. Théorèmes #1 :}\ % Si non vide, affiche le titre avec le premier argument
        }\\#2
    \end{mdframed}
}

\newcommand{\corrolaire}[2]{%
    \begin{mdframed}[style=ProprieteStyle]
        \ifstrempty{#1}{% Teste si le premier argument est vide
            \textbf{Corrolaire :}\ % Si vide, n'affiche pas de titre de théorème
        }{%
            \textbf{Corrolaire #1 :}\ % Si non vide, affiche le titre avec le premier argument
        }\\#2
    \end{mdframed}
}

% ========== Exemple(s) ==========

% Définition de la commande 'exemple'
\newcommand{\exemple}[2]{
    \textbf{Exemple :} #1
    \begin{quote}
        #2
    \end{quote}
}

% Définition de la commande 'exemples'
\newcommand{\exemples}[2]{
    \textbf{Exemples :} #1
    \begin{quote}
        #2
    \end{quote}
}

% ========== Remarques ==========

% Définition de la commande 'remarque'
\newcommand{\remarque}[1]{
    \textbf{Remarque :} #1
}

% Définition de la commande 'remarques'
\newcommand{\remarques}[1]{
    \textbf{Remarques :}
    \begin{quote}
        #1
    \end{quote}
}

% ========== Démonstration ==========

\newcommand{\demonstration}[2]{
    \ifstrempty{#1}{% Teste si le premier argument est vide
        \textbf{Démonstration :}\ % Si vide, n'affiche pas de titre de théorème
    }{%
        \textbf{Démonstration #1 :}\ % Si non vide, affiche le titre avec le premier argument
    }
    \begin{quote}
        #2
    \end{quote}
}
% ========== Autres blocs ==========

\newcommand{\newbloc}[2]{
    \textbf{#1}
    \begin{quote}
        #2
    \end{quote}
}

% ========== Notations ==========

% Style
\newcommand{\ds}{\displaystyle}
\newcommand{\llb}{\llbracket}
\newcommand{\rrb}{\rrbracket}

% Ensembles
\newcommand{\C}{\mathbb{C}}
\newcommand{\R}{\mathbb{R}}
\newcommand{\Q}{\mathbb{Q}}
\newcommand{\D}{\mathbb{D}}
\newcommand{\Z}{\mathbb{Z}}
\newcommand{\N}{\mathbb{N}}

% Limites
\let\oldlim\lim
\renewcommand{\lim}[1]{\mathop{\displaystyle\oldlim}\limits_{#1}}

% Opérateurs
\newcommand{\x}{\times}
\newcommand{\equival}{\Leftrightarrow}
\newcommand{\involve}{\Rightarrow}

% Ensembles
\renewcommand{\bar}[1]{\overline{#1}}
\renewcommand{\part}[1]{\mathcal{P}({#1})}  % Ensemle des parties de E
\newcommand{\rel}{\mathcal{R}} % Relation entre deux ensembles

% Vecteurs
\renewcommand{\Vec}[1]{\overrightarrow{#1}}
\newcommand*{\norme}[1]{\|#1\|}
\newcommand{\pdt}{\mathbin{\vcenter{\hbox{\scalebox{0.6}{\textbullet}}}}}
\newcommand{\vcoord}[2]{\begin{pmatrix} #1 \\ #2 \end{pmatrix}}
\newcommand{\Vcoord}[3]{\begin{pmatrix} #1 \\ #2 \\ #3 \end{pmatrix}}

% Fonctions
\newcommand{\aire}[1]{\text{aire}\left( #1 \right)}
\newcommand{\Int}{\ds\int}


% ========== Systèmes ==========
\newcommand{\sys}[2]{\begin{cases}#1\\#2\end{cases}}
\newcommand{\Sys}[3]{\begin{cases}#1\\#2\\#3\end{cases}}

% ========== Tableaux ==========
\newcommand{\boldhline}{\hline\noalign{\vskip 0pt}\hline}

% ========== Graphiques ==========

% Triange de Pascal
\newcommand{\binomial}[2]{\pgfmathparse{int((#1)!/((#2)!*((#1)-(#2))!))}\pgfmathresult}
\newcommand{\Pascal}[1]{
    \begin{tikzpicture}[scale=1, every node/.style={scale=1.2}]
        \foreach \n in {0,...,#1} {
                \foreach \k in {0,...,\n} {
                        \node at (\k-\n/2,-\n) {$\binomial{\n}{\k}$};
                        \draw (\k-\n/2,-\n) +(-0.5,-0.5) rectangle ++(0.5,0.5);
                    }
            }
    \end{tikzpicture}
}

\newcommand{\Pascalbinome}[1]{
    \begin{tikzpicture}[scale=1, every node/.style={scale=0.9}]
        \foreach \n in {0,...,#1} {
                \foreach \k in {0,...,\n} {
                        \pgfmathtruncatemacro{\symk}{min(\k,\n-\k)}
                        \node at (\k-\n/2,-\n) {$\ds\binom{\n}{\symk}$};
                        \draw (\k-\n/2,-\n) +(-0.5,-0.5) rectangle ++(0.5,0.5);
                    }
            }
    \end{tikzpicture}
}

\begin{document}

\maketitle


% Sections et sous-sections
\section{Généralités}
\subsection{Introduction}

\definitions{}{
	Une suite numérique $u$ est une fonction $u:n \mapsto u(n)$ définie pour tout entier naturel $n$ (ou tout entier
	naturel $n \geq k$, $k$ étant un entier naturel).
	\begin{itemize}
		\item  $u(n)$ ou $u_n$ s'appelle le terme de rang $n$ (ou le terme général de la suite).
		\item  $u$ désigne la suite elle-même, elle peut être noté aussi $(u_n)$.
		\item  $n$ est l'indice (ou le rang).
		\item  $u_{n-1}$ est le terme précédant $u_n$.
		\item  $u_{n+1}$ est le terme suivant $u_n$.
		\item  $u_0$ (ou parfois $u_1$) est le terme initial (ou le premier terme).
	\end{itemize}
}

\subsection{Différents modes de génération d'une suite}


\definition{}{
	Une suite est définie de façon explicite lorsqu'on peut calculer n'importe quel terme de la suite directement en fonction de $n$.
}


\exemple{}{
	Soit la suite $u$ définie par $u_n=12+2n$ pour tout entier naturel. On calcule ses premiers termes :

	$u_0=12+2\times0=12 \qquad u_1=12+2\times1=14 \qquad u_2=12+2\times2=16$

	$u_3=12+2\times3=18 \qquad u_4=12+2\times4=20 \qquad u_5=12+2\times5=22$
}

\definition{}{
	Lorsqu'une suite est définie par la donnée de son premier terme et d'une relation qui permet de calculer chaque terme
	en fonction du terme précédent, on dit que la suite est définie par récurrence. On donne l'expression de $u_{n+1}$ en
	fonction de $u_n$. Cette relation s'appelle relation (ou formule) de récurrence.
}

\exemple{}{

	Soit $F$ la suite définie par
	$ \begin{cases}
			F_0 = 0                     & \\
			F_1 = 1                     & \\
			F_{n+1} = F_{n-1} + F_{n-2} &
		\end{cases}
	$ pour tout $n \geq 2$.

	Cette suite s'appelle la suite de Fibonacci. Elle est définie par une relation de récurrence d'ordre 2, c'est à dire
	que chaque terme de la suite est la somme des deux termes qui le précèdent. Ses premiers termes sont :

	$F_2=1+0=1 \qquad F_3=1+1=2 \qquad F_4=2+1=3$

	$F_5=3+2=5 \qquad F_6=5+3=8 \qquad F_7=8+5=13$
}

\subsection{Représentation graphique d'un suite}

Une suite $u$ peut être représentée :
\begin{itemize}
	\item en plaçant les points de coordonnées $(n, u_n)$ dans un repère (on appelle cet ensemble nuage de points).
	\item en plaçant les réels $u_0$, $u_1$, $u_2\dots$ sur une droite graduée.
\end{itemize}

\exemple{}{
	On représente la suite $u$ définie pour tout entier naturel $n$ par $u_n=n^2-4n+2$.\\
	On a  $u_0=2$, $u_1=-1$, $u_2=-2$, $u_3=-1$ et $u_4=2$. \\

	\begin{minipage}{0.5\textwidth}
		\centering
		\begin{tikzpicture}
			\begin{axis}[
					axis lines=middle,
					xlabel={$n$},
					ylabel={$u_n$},
					xmin=-0.5,
					xmax=4.5,
					ymin=-2.5, ymax=2.5,
					xtick={0,1,2,3,4},
					ytick={-2,-1,0,1,2},
					xlabel style={below right},
					ylabel style={above left},
					tick label style={font=\footnotesize},
					grid=both,
					major grid style={line width=.2pt,draw=gray!50},
					minor tick num=0,
					extra y ticks={0},
					extra y tick style={grid=major}]
				\addplot[mark=, mark size=2, only marks] coordinates {(0,2)};
				\node[label={below right:$u_0$}] at (axis cs:0,2) {};
				\addplot[mark=, mark size=2, only marks] coordinates {(1,-1)};
				\node[label={below right:$u_1$}] at (axis cs:1,-1) {};
				\addplot[mark=, mark size=2, only marks] coordinates {(2,-2)};
				\node[label={above right:$u_2$}] at (axis cs:2,-2) {};
				\addplot[mark=, mark size=2, only marks] coordinates {(3,-1)};
				\node[label={below right:$u_3$}] at (axis cs:3,-1) {};
				\addplot[mark=*, mark size=2, only marks] coordinates {(4,2)};
				\node[label={below left:$u_4$}] at (axis cs:4,2) {};
			\end{axis}
		\end{tikzpicture}
	\end{minipage}%
	\begin{minipage}{0.5\textwidth}
		\centering
		\begin{tikzpicture}
			\draw[->] (-4,0) -- (4,0);
			\foreach \x in {-3,0,1,3}
			\draw (\x cm,3pt) -- (\x cm,-3pt);
			\foreach \x/\label in {-3/-3,-2/-2,0/0,1/1,3/3}
			\node[anchor=north] at (\x cm,-3pt) {$\label$};
			\draw[fill=black] (2,0) circle (2pt) node[above=3pt] {$u_0$};
			\draw[fill=black] (-1,0) circle (2pt) node[above=3pt] {$u_1$};
			\draw[fill=black] (-2,0) circle (2pt) node[above=3pt] {$u_2$};
			\draw[fill=black] (-1,0) circle (2pt) node[below=3pt] {$u_3$};
			\draw[fill=black] (2,0) circle (2pt) node[below=3pt] {$u_4$};
		\end{tikzpicture}
	\end{minipage}
}


\subsection{Sens de variation d'une suite}

\definition{}{
	Soit une suite $u$ définie sur $\N$.
	\begin{itemize}
		\item Dire que $u$ est strictement croissante signifie que pour tout $n\in\N$, on a $u_{n+1}>u_n$.
		\item Dire que $u$ est strictement décroissante signifie que pour tout $n\in\N$, on a $u_{n+1}<u_n$.
		\item Dire que $u$ est constante signifie que pour tout $n\in\N$, on a $u_{n+1}=u_n$.
	\end{itemize}
}

\remarques{
	\begin{itemize}
		\item Lorsqu'une suite est croissante ou décroissantes, on dit qu'elle est monotone.
		\item Pour étudier le sens de variation d'une suite, on peut :
		      \begin{enumerate}
			      \item calculer la différence $u_{n+1}-u_n$ et étudier son signe.
			      \item pour une suite positive, calculer le quotient $\dfrac{u_{n+1}}{u_n}$ et étudier sa position par rapport à $1$.
			      \item pour une suite définie de façon explicite par $u_n=f(n)$, utiliser le sens de variation de $f$. \\
		      \end{enumerate}
	\end{itemize}
}

\exemple{}{
	Soit $u$ définie par $u_n=\dfrac{1}{3^n}$ pour tout $n\in\N$.

	Pour tout $n\in\N$, $u_n>0$ donc on utilise la deuxième méthode.

	On calcule $\dfrac{u_{n+1}}{u_n}
		=\dfrac{\frac{1}{3^{n+1}}}{\frac{1}{3^n}}
		=\dfrac{1}{3^{n+1}}\times\dfrac{3^n}{1}
		=\dfrac{3^n}{3^{n+1}}
		=\underbrace{\frac
			{\overbrace{\cancel{3}\times\cancel{3}\times...\times\cancel{3}}^{n\text{ fois}}\times1}
			{\cancel{3}\times\cancel{3}\times...\times\cancel{3}\times3}}_{n+1\text{ fois}}
		=\dfrac{1}{3}$
}

\section{Suites arithmétiques et géométriques}

\subsection{Suites arithmétiques}

\definition{}{
	On dit que la suite $u$ est arithmétique si, à partir de son premier terme, chaque terme est obtenu en ajoutant au
	précédent un même nombre appelé raison. Pour tout $n\in\N$, $u_{n+1}=u_n+r$.
}

\remarque{Une suite est arithmétique si $u_{n+1}-u_n=r$ pour tout $n\in\N$.}

\newpage

\propriete{(formule explicite)}{
	Soit $u$ une suite arithmétique de raison $r$. Alors, pour tout $n\in\N$, $u_n=u_0+nr$. Plus généralement, pour tout
	entier $n$ et $p$ : $u_n=u_p+(n-p)r$
}

\propriete{(sens de variation)}{
	Soit $u$ une suite arithmétique de raison $r$.
	\begin{enumerate}[(i)]
		\item Si $r>0$, la suite $u$ est strictement croissante.
		\item Si $r<0$, la suite $u$ est strictement décroissante.
		\item Si $r=0$, la suite $u$ est constante.
	\end{enumerate}
}

\propriete{(représentation graphique)}{
	La représentation graphique d'une suite arithmétique est un nuage de points situé sur une droite d'équation $y=u_0+xr$.
}

\exemple{}{
	On représente la suite arithmétique de premier terme $u_0=1$ et de raison $r=2$ :\\ \\
	\begin{minipage}[t]{0.4\textwidth}
		\begin{tikzpicture}
			\begin{axis}[
					xlabel={$n$},
					ylabel={$u_n$},
					axis lines=middle,
					grid=both,
					minor tick num=1,
					enlargelimits={abs=0.3},
					ymin=0,
					ymax=11,
					xmin=0,
					xmax=5
				]
				\addplot+[ycomb, black, mark=*, mark size=1.5pt, only marks, samples at={0,1,2,3,4}] {1+2*x};
			\end{axis}
		\end{tikzpicture}
	\end{minipage}
	\hfill
	\begin{minipage}[t]{0.4\textwidth}
		\vspace{-100pt}
		\begin{tikzpicture}
			% Dessinez la droite graduée
			\draw[line width=1pt, ->] (0,0) -- (7.5,0);

			% Ajoutez les graduations
			\foreach \x in {0, 2, 4, 6} {
					\draw[line width=0.6pt] (\x, 0.1) -- (\x, -0.1);
				}

			% Ajoutez les étiquettes de graduation
			\foreach \x/\n in {1/1,2/2,3/3,4/4,5/5,6/6,7/7} {
					\node[below] at (\x, -0.15) {$\n$};
				}

			% Dessinez les points de la suite
			\foreach \x in {1,3,5,7} {
					\node[blue, fill, circle, inner sep=1pt] at (\x, 0) {};
				}

			% Ajoutez les étiquettes des points
			\foreach \x/\n in {1/u_0,3/u_1,5/u_2,7/u_3} {
					\node[blue, above] at (\x, 0) {$\n$};
				}
		\end{tikzpicture}
	\end{minipage}
}


\propriete{(somme de termes consécutifs)}{
	Soit $n$ un entier naturel non nul. La somme des $n$ premiers termes non nuls est $1+2+3+...+n=\dfrac{n(n+1)}{2}$.
}

\exemple{}{
	On calcule $S = 1+2+3+\dots+999 = \dfrac{999(999+1)}{2} = 499500$
}

\subsection{Suites géométriques}

\definition{}{
	On dit que la suite $u$ est géométrique si, à partir de son premier terme, chaque terme est obtenu en multipliant le
	précédent par un même nombre appelé raison. Pour tout $n\in\N$, $u_{n+1}=u_n\times q$.
}

\remarque{Une suite est géométrique si $\dfrac{u_{n+1}}{u_n}=q$ pour tout $n\in\N$.}


\propriete{(formule explicite)}{
	Soit $u$ une suite géométrique de raison $q$. Alors, pour tout $n\in\N$, $u_n=u_0\times q^n$. Plus généralement, pour
	tout entier $n$ et $p$ : $u_n=u_p\times q^{n-p}.$
}


\propriete{(sens de variation)}{
	Soit $u$ une suite géométrique de raison $q$ et de premier terme $u_0$ strictement positif.
	\begin{enumerate}[(i)]
		\item Si $q>1$, la suite $u$ est strictement croissante.
		\item Si $q=1$, la suite $u$ est constante, égal à $u_0$.
		\item Si $0<q<1$, la suite $u$ est strictement décroissante.
		\item Si $q=0$, à partir du rang $1$ : la suite $u$ est constante et égale à $0$.
		\item Si $q<0$, la suite $u$ n'est ni croissante ni décroissante.
	\end{enumerate}
}

\propriete{(représentation graphique)}{
	La représentation graphique d'une suite géométrique est un nuage de points situé sur la courbe d'une fonction exponentielle.
}

\exemple{}{

	On représente la suite géométrique de premier terme $u_0=1$ et de raison $q=1,5$ :\\ \\
	\begin{minipage}[t]{0.4\textwidth}
		\begin{tikzpicture}
			\begin{axis}[
					xlabel={$n$},
					ylabel={$u_n$},
					axis lines=middle,
					grid=both,
					minor tick num=1,
					enlargelimits={abs=0.3},
					ymin=0,
					ymax=11,
					xmin=0,
					xmax=5
				]
				\addplot+[ycomb, blue, mark=*, mark size=1.5pt, only marks, samples at={0,1,2,3,4, 5}] {1*1.5^x};
			\end{axis}
		\end{tikzpicture}
	\end{minipage}
	\hfill
	\begin{minipage}[t]{0.43\textwidth}
		\vspace{-100pt}
		\begin{tikzpicture}
			% Dessinez la droite graduée
			\draw[line width=1pt, ->] (0,0) -- (8,0);

			% Ajoutez les graduations
			\foreach \x in {0,1,...,7} {
					\draw[line width=0.6pt] (\x, 0.1) -- (\x, -0.1);
				}

			% Ajoutez les étiquettes de graduation
			\foreach \x/\n in {1/1,2/2,3/3,4/4,5/5,6/6,7/7} {
					\node[below] at (\x, -0.15) {$\n$};
				}

			% Dessinez les points de la suite
			\foreach \x in {1,1.5, 2.25, 3.375, 5.0625, 7.59375} {
					\node[fill, blue, circle, inner sep=1pt] at (\x, 0) {};
				}

			% Ajoutez les étiquettes des points
			\foreach \x/\n in {1/u_0, 1.5/u_1, 2.2/u_2, 3.375/u_3, 5.0625/u_4, 7.59375/u_5} {
					\node[blue, above] at (\x, 0) {$\n$};
				}
		\end{tikzpicture}
	\end{minipage}
}

\propriete{somme de termes consécutifs}{
Soit $n$ un entier naturel non nul et $q$ un réel différent de $1$. Alors $1+q+q^2+...+q^n=\dfrac{1-q^{n+1}}{1-q}$.
}

\exemple{}{
On calcule $S = 1+2+2^2+2^3+\dots+2^{10} = \dfrac{1-2^{11}}{1-2} = 2047$
}

\section{Suites minorées et majorées}

\definition{}{
	Soit $u$ une suite.
	\begin{itemize}
		\item On dit que $u$ est minorée s'il existe un réel $m$ tel que pour tout $n\in\N, u_n\geq m$.
		      On dit que $m$ est un minorant de la suite.
		\item On dit que $u$ est majorée s'il existe un réel $M$ tel que pour tout $n\in\N, u_n\leq m$.
		      On dit que $M$ est un majorant de la suite.
	\end{itemize}
}

\remarques{
	\begin{itemize}
		\item Une suite croissante est nécesairement minorée par son terme initial : $\forall n\in\N, u_n\geq u_0$.
		\item Une suite décroissante est nécesairement majorée par son terme initial : $\forall n\in\N, u_n\leq u_0$.
	\end{itemize}
}

\exemples{}{
	\begin{itemize}
		\item On pose $\forall n\in\N^*, u_n=\frac{-1}{n}$. \\
		      La suite $(u)$ est strictement croissante car $\forall n\in\N^*, n<n+1 \Rightarrow \frac{1}{n}>\frac{1}{n+1}\Rightarrow\frac{-1}{n}<\frac{-1}{n+1}$. \\
		      Donc $(u)$ est majorée par $0$ : $\forall n\in\N^*, \frac{-1}{n}\leq0$.
		\item On pose $u_n =
			      \begin{cases}
				      -n \text{ si $n$ pair} \\
				      n \text{ si $n$ impair}
			      \end{cases}$ pour tout $n\in\N$. La suite$(u)$ n'est ni minorée ni majorée.
	\end{itemize}
}

\section{Propositions héréditaires et démonstrations par récurrence}

\definition{}{
	Soit $P(n)$ un prédicat de $n$. On dit que $P$ est héréditaires si la proposition
	« $\forall n\in\N, \text{ } P(n)\Rightarrow P(n+1)$ » est vérifiée.
}

\exemple{}{
	\begin{itemize}
		\item Soit $P(n)$ : « $n^2>10$ » pour tout $n\in\N$. $P(0)$, $P(1)$, $P(2)$ et $P(3)$ sont fausses. \\
		      Ensuite, $P(4)$, $P(5)$, etc sont vraies et ceci pour tout $n\geq4$ (preuve : $f(x)=x^2$ est croissante sur $[0;+\infty[$ et $f(4)=16$).\\
		      Donc $P$ est héréditaires.
		\item Soit $P(n)$ : « $\frac{6}{n+1}>1,1$ » pour tout $n\in\N$. $P(0)$, $P(1)$, $P(2)$ $P(3)$ et $P(4)$ sont vraies.\\
		      Ensuite, $P(5)$, $P(6)$, etc sont fausses et ceci pour tout $n\geq5$ (preuve : si $n\geq5$, alors $n+1\geq6$ donc $\frac{6}{n+1}\leq\frac{6}{6}=1$).\\
		      Donc $P$ n'est pas héréditaires car $P(4)$ est vraie mais $P(5)$ est fausse ce qui contredit « $P(n)\Rightarrow P(n+1)$ ».
	\end{itemize}
}


\theoreme{(principe de récurrence)}{
	Soit $P(n)$ un prédicat de $n$. On suppose que $P(n)$ est héréditaire. Deux cas sont alors possibles :
	\begin{enumerate}[(i)]
		\item $P(n)$ est fausse pour tout $n\in\N$.
		\item il existe un entier $n_0$ tel que $P(n)$ soit vraie pour tout entier $n\geq n_0$.
	\end{enumerate}
}

\remarque{Faire une « démonstration par récurrence » consiste à prouver qu'une proposition $P$ est héréditaire et
	trouver un entier $n_0$ pour lequelle $P(n_0)$ est vraie. Le principe de récurrence permet alors de conclure que pour
	tout entier naturel $n\geq n_0$, $P(n)$ est vraie.}

\exemple{démonstration par récurrence}{
	Soit la suite
	$\left\{
		\begin{array}{l}
			u_0 = 3 \\
			u_n+1 = \frac{u_n}{2} + 1
		\end{array}
		\right. \text{.}$\\

	Nommons $P(n)$ : « $u_n\leq3$ ».
	\begin{itemize}
		\item \underline{Initialisation :}\\
		      $P(0)$ est vraie car $u_0=3$.
		\item \underline{Hérédité :}\\
		      Montrons que $P$ est héréditaire.\\
		      Soit $n$ un entier naturel. Supposons $P(n)$ :
		      \vspace{-4pt}
		      \begin{flalign*}
			       & \text{On a } u_n \leq 3                           \\
			       & \text{donc } \frac{u_n}{2} \leq 1,5               \\
			       & \text{donc } \frac{u_n}{2}+1 \leq 2,5             \\
			       & \text{donc } \frac{u_n}{2}+1 \leq 2,5 \leq 3      \\
			       & \text{donc } u_{n+1} \leq 3 \text{ donc $P(n+1)$}
		      \end{flalign*}
		\item \underline{Conclusion :}\\
		      Par principe de récurrence, comme $P$ est héréditaire et vraie au rang $0$, on déduit que pour tout $n\geq0$, $P(n)$ est vraie.\\
		      Autrement dit, pour tout $n\in\N$, $u_n\leq3$.
	\end{itemize}
}

\remarque{En math, supposer $\not=$ admettre. Lorsqu'on suppose, on attribut temporairement la valeur de vérité « vrai »
	à une proposition (on suppose souvent lors de démonstrations) tandis que lorsqu'on admet, on suppose de manière
	définitive (ex : on admet des théorèmes lorsqu'ils sont trop complexes pour être démontrés).}


\end{document}